\documentclass[12pt,oneside,article,draft]{memoir}
% !TEX root = ./CCC_Note.tex

\usepackage{amsmath}
\usepackage{amsthm}
\usepackage{amsfonts}
\usepackage{amssymb}
\usepackage{mathtools}
%\usepackage{datetime}
\usepackage[usenames,dvipsnames]{xcolor}
\usepackage[bookmarks=true,colorlinks=true, linkcolor=MidnightBlue, citecolor=cyan]{hyperref}
\usepackage[T1]{fontenc}
\usepackage[sc]{mathpazo}
\linespread{1.05}
\usepackage{mathrsfs}
\usepackage{euscript}
%\usepackage{MnSymbol}
\usepackage{paralist}
\usepackage{todonotes}
\usepackage{makecell}
\usepackage{booktabs}
\usepackage{tikz}
\usetikzlibrary{cd}
\usepackage{tensor}

\usetikzlibrary{decorations.markings,arrows.meta,calc,fit,quotes}
\hypersetup{final}

\DeclareMathOperator{\id}{id}
\DeclareMathOperator{\dom}{dom}
\DeclareMathOperator{\cod}{cod}
\DeclareMathOperator{\dvert}{Vert}
\DeclareMathOperator{\Lax}{Lax}
\DeclareMathOperator{\Hom}{Hom}
\DeclareMathOperator{\Ob}{Ob}
\DeclareMathOperator{\Tr}{Tr}


\theoremstyle{plain}
\newtheorem{theorem}{Theorem}[section]
\newtheorem*{theorem*}{Theorem}
\newtheorem{proposition}[theorem]{Proposition}
\newtheorem{corollary}[theorem]{Corollary}
\newtheorem{lemma}[theorem]{Lemma}
\newtheorem*{lemma*}{Lemma}

\theoremstyle{definition}
\newtheorem{definition}[theorem]{Definition}
\newtheorem{exercise}{Exercise}[section]

\theoremstyle{remark}
\newtheorem{example}[theorem]{Example}
\newtheorem{remark}[theorem]{Remark}

\newcommand{\prodb}{\mathbin{\Pi}}
\newcommand{\iso}{\cong}

\newcommand{\cat}[1]{\mathscr{#1}}
\newcommand{\Cat}[1]{\mathbf{#1}}
\newcommand{\fun}[1]{#1}
\newcommand{\Fun}[1]{\mathsf{#1}}
%\newcommand{\hom}{\mathrm{hom}}
\newcommand{\twocat}[1]{\mathcal{#1}}
\newcommand{\dblcat}[1]{\mathbb{#1}}
\newcommand{\Mon}{\Cat{Mon}}
\newcommand{\Prof}{\Cat{Prof}}
\newcommand{\MProf}{\Cat{MProf}}
\newcommand{\MonCat}{\Cat{MonCat}}
\newcommand{\SymMonCat}{\Cat{SymMonCat}}
\newcommand{\CompCat}{\Cat{CompCat}}
\newcommand{\TrCat}{\Cat{TrCat}}
\newcommand{\Set}{\Cat{Set}}
\newcommand{\Int}{\Fun{Int}}

\newcommand{\op}[1]{{#1}^{\text{op}}}
\newcommand{\vop}[1]{{#1}^{\text{vop}}}
\newcommand{\hop}[1]{{#1}^{\text{hop}}}

\newcommand{\Alg}{\mathrm{Alg}}
\newcommand{\Coalg}{\mathrm{Coalg}}
\newcommand{\RAlg}[1][]{\mathbb{R}_{#1}\text{-}\Alg}
\newcommand{\LCoalg}[1][]{\mathbb{L}_{#1}\text{-}\Coalg}
\newcommand{\LCoalgA}{\mathbb{L}_1\text{-}\Coalg}
\newcommand{\LCoalgB}{\mathbb{L}_2\text{-}\Coalg}

\newcommand{\twocell}[3][]{\arrow[draw=none,to path={(dom#2.center)--(cod#2.center)\tikztonodes}]{}[anchor=center,#1]{\Downarrow #3}}
\newcommand{\twocellalt}[3][]{\arrow[draw=none,to path={(dom#2.center)--(cod#2.center)\tikztonodes}]{}[anchor=center,#1]{#3}}
\newcommand{\twocellA}[2][]{\twocell[#1]{A}{#2}}
\newcommand{\twocellB}[2][]{\twocell[#1]{B}{#2}}
\newcommand{\twocellC}[2][]{\twocell[#1]{C}{#2}}
\newcommand{\twocellD}[2][]{\twocell[#1]{D}{#2}}
\newcommand{\twocellE}[2][]{\twocell[#1]{E}{#2}}
\newcommand{\twocellF}[2][]{\twocell[#1]{F}{#2}}



\tikzcdset{
	arrow style=tikz,
	diagrams={>={Classical TikZ Rightarrow[angle=63:4pt, line width=.6pt]}},
	arrows={semithick}
}

\tikzset{tick/.style={postaction={decorate,decoration={markings,mark=at position 0.5 with {\draw[-] (0,.4ex) -- (0,-.4ex);}}}}}
\tikzset{dom/.style={append after command={coordinate[alias=dom#1]}},
		domA/.style={dom=A}, domB/.style={dom=B},
		domC/.style={dom=C}, domD/.style={dom=D},
		domE/.style={dom=E}, domF/.style={dom=F}}
\tikzset{cod/.style={append after command={coordinate[alias=cod#1]}},
		codA/.style={cod=A}, codB/.style={cod=B},
		codC/.style={cod=C}, codD/.style={cod=D},
		codE/.style={cod=E}, codF/.style={cod=F}}


\tikzset{
	%label/.style={font=\everymath\expandafter{\the\everymath\scriptstyle}},
	wiring diagram/.style={
		every to/.style={out=0,in=180,draw},
		label/.style={
			font=\everymath\expandafter{\the\everymath\scriptstyle},
			inner sep=0pt,
			node distance=2pt and -2pt},
		semithick,
		node distance=1 and 1,
		decoration={markings, mark=at position .5 with {\arrow{stealth};}},
		ar/.style={postaction={decorate}},
		execute at begin picture={\tikzset{
			x=\bbx, y=\bby,
			every fit/.style={inner xsep=\bbx, inner ysep=\bby}}}
		},
	bbx/.store in=\bbx,
	bbx = 1.5cm,
	bby/.store in=\bby,
	bby = 1.75ex,
	bb port sep/.store in=\bbportsep,
	bb port sep=2,
	% bb wire sep/.store in=\bbwiresep,
	% bb wire sep=1.75ex,
	bb port length/.store in=\bbportlen,
	bb port length=4pt,
	bb min width/.store in=\bbminwidth,
	bb min width=1cm,
	bb rounded corners/.store in=\bbcorners,
	bb rounded corners=2pt,
	bb small/.style={bb port sep=1, bb port length=2.5pt, bbx=.4cm, bb min width=.4cm, bby=.7ex},
	bb/.code 2 args={
		\pgfmathsetlengthmacro{\bbheight}{\bbportsep * (max(#1,#2)+1) * \bby}
		\pgfkeysalso{draw,minimum height=\bbheight,minimum width=\bbminwidth,outer sep=0pt,
			rounded corners=\bbcorners,thick,
			prefix after command={\pgfextra{\let\fixname\tikzlastnode}},
			append after command={\pgfextra{\draw
				\ifnum #1=0{} \else foreach \i in {1,...,#1} {
					($(\fixname.north west)!{\i/(#1+1)}!(\fixname.south west)$) +(-\bbportlen,0) coordinate (\fixname_in\i) -- +(\bbportlen,0) coordinate (\fixname_in\i')}\fi
				\ifnum #2=0{} \else foreach \i in {1,...,#2} {
					($(\fixname.north east)!{\i/(#2+1)}!(\fixname.south east)$) +(-\bbportlen,0) coordinate (\fixname_out\i') -- +(\bbportlen,0) coordinate (\fixname_out\i)}\fi;
			}}}
	},
	bb name/.style={append after command={\pgfextra{\node[anchor=north] at (\fixname.north) {#1};}}}
}

\usetikzlibrary{arrows,calc,chains,matrix,positioning,scopes,snakes}


\newcommand{\vinp}[1]{\overline{\inp{#1}}}
\newcommand{\voutp}[1]{\overline{\outp{#1}}}
%\newcommand{\inp}[1]{#1^{\textnormal{in}}}
%\newcommand{\outp}[1]{#1^{\textnormal{out}}}
\newcommand{\inp}[1]{#1^-}
\newcommand{\outp}[1]{#1^+}

% \def\bhline{\Xhline{2\arrayrulewidth}}
% \def\bbhline{\Xhline{2.5\arrayrulewidth}}
\def\alg{{\text \textendash}\Cat{Alg}}
\def\XCat{\textnormal{$\Cat{X}$-$\Cat{Cat}$}}
\def\To{\xrightarrow}
\def\ul{\underline}
\def\List{\textnormal{List}}
\def\SList{\textnormal{SList}}
\def\SSList{\textnormal{SSList}}

\newcommand{\erase}[1]{{}}
\def\NN{\mathbb{N}}
\def\ss{\subseteq}
\def\boo{{\Ob\iso}}
\newcommand{\bo}{\mathsf{bo}}
\newcommand{\ff}{\mathsf{ff}}


\settrims{0pt}{0pt} % page and stock same size
\setlxvchars %calculate line length such that there are about 65 characters per line in \normalfont
\settypeblocksize{*}{36pc}{*} % {height}{width}{ratio}
\setlrmargins{*}{*}{1} % {spine}{edge}{ratio}
%\setulmargins{*}{*}{1} % {upper}{lower}{ratio}, hight of typeblock fixed
\setulmarginsandblock{1in}{1in}{*} % hight of typeblock computed
\setheadfoot{\onelineskip}{2\onelineskip} % {headheight}{footskip}
\setheaderspaces{*}{1.5\onelineskip}{*} % {headdrop}{headsep}{ratio}
\checkandfixthelayout


\setcounter{tocdepth}{2}
\setcounter{secnumdepth}{2}
%\chapterstyle{dash}
\pagestyle{companion}

\title{Enriched Traced Monoidal Categories are Lax Functors out of Compact Categories}
\author{
Dylan Rupel 
 \and 
David I. Spivak\thanks{Spivak and Schultz were supported by AFOSR grant FA9550-14-1-0031, ONR grant N000141310260, and NASA grant NNL14AA05C.}
 \and 
 Patrick Schultz${}^*$%\footnotemark[1]
 }



\begin{document}
\tightlists
\firmlists

\maketitle
\begin{abstract}
We give an alternate conception of string diagrams as 1-dimensional oriented cobordisms. The operad $\Cat{1-Cob}$ of cobordisms encodes the axioms of traced categories, in the sense that there is an equivalence of categories between $\Cat{1-Cob}$-algebras and enriched traced categories. We also prove a substantial generalization of this fact, which characterizes lax functors out of compact or traced categories.
\end{abstract}
Global todo's:
\begin{enumerate}
\item Decide when (if ever) to say $\Cat{Cob}$ and when (if ever) to say $\Cat{1-Cob}$. 
\end{enumerate}
\tableofcontents
\todo{Table of contents says ``Contents" twice.}
% !TEX root = ./CCC_Note.tex
\chapter{Introduction}

Traced symmetric monoidal categories, hereafter \emph{traced categories}, have been used to model processes with feedback (\cite{http://arxiv.org/pdf/1401.5113v1.pdf})  or operators with fixed points (\cite{http://arxiv.org/pdf/1107.6032.pdf}). A graphical calculus for traced categories was developed by Joyal, Street, and Verity (\cite{JoyalStreetVerity}), in which string diagrams of the form
%
%
%WITHOUT OUTER BOX:
%
%
%\begin{align}\label{dia:string diagram}
%\begin{tikzpicture}
%	%little box 1
%	\path(2,1.5);
%	\blackbox{(2,2)}{2}{1}{$X_1$}{.5}
%	%little box 2
%	\path(6,1.5);
%	\blackbox{(2,2)}{2}{2}{$X_2$}{.5}
%	%wires
%	\directarc{(4.25,2.5)}{(5.75,2.16667)} % X_1 -> X_2
%	\directarc{(0.35,2.833333)}{(1.75,2.83333)} % Y -> X_1
%	\fancyarc{(0.35,4)}{(5.75,2.83333)}{-35}{16} % Y -> X_2
%	\directarc{(8.25,2.8333)}{(9.65,2.83333)} % X_2 -> Y
%	\fancyarc{(1.75,2.16667)}{(8.25,2.16667)}{20}{-45} % X_2 -> X_1
%\end{tikzpicture}
%\end{align}
%
%
%WITH OUTER BOX:
%
%
\begin{align}\label{dia:string diagram}
\begin{tikzpicture}
   %big box
	\path(0,0);
	\blackbox{(10,5)}{2}{1}{$Y$}{.7}
	    %inner wires
	        \node at (.4,3.6) {\tiny $\inp{Y}_{a}$};
	        \node at (.4,1.9) {\tiny $\inp{Y}_{b}$};
	    %outer wire
	        \node at (9.6,2.75) {\tiny $\outp{Y}_a$};
	%little box 1
	\path(2,1.5);
	\blackbox{(2,2)}{2}{1}{$X_1$}{.5}
	    %tank info
	    %inner wires
	        \node at (1.75,3.03) {\tiny $\inp{X}_{1a}$};
	        \node at (1.75,2.35) {\tiny $\inp{X}_{1b}$};
	    %outer wire
	        \node at (4.33,2.68) {\tiny $\outp{X}_{1c}$};
	%little box 2
	\path(6,1.5);
	\blackbox{(2,2)}{2}{2}{$X_2$}{.5}
	    %tank info
	    %inner wires
	        \node at (5.75,3.03) {\tiny $\inp{X}_{2a}$};
	        \node at (5.75,2.35) {\tiny $\inp{X}_{2b}$};
	    %outer wires
	        \node at (8.33,3.03) {\tiny $\outp{X}_{2c}$};
	        \node at (8.33,2.35) {\tiny $\outp{X}_{2d}$};
	%wires
	\directarc{(4.25,2.5)}{(5.75,2.16667)} % X_1 -> X_2
	\directarc{(0.35,1.6667)}{(1.75,2.83333)} % Y -> X_1
	\fancyarc{(0.35,3.3333)}{(5.75,2.83333)}{-40}{25} % Y -> X_2
	\directarc{(8.25,2.8333)}{(9.65,2.5)} % X_2 -> Y
	\fancyarc{(1.75,2.16667)}{(8.25,2.16667)}{20}{-45} % X_2 -> X_1
\end{tikzpicture}
\end{align}
represent compositions, i.e., new morphisms are constructed from old by specifying which outputs will be fed back into which inputs. In fact, these generalize Penrose diagrams in $\Cat{Vect}$, and the word \emph{traced} originates in vector space terminology.  

But notice that the picture in (\ref{dia:string diagram}) has another interpretation, in terms of 1-dimensional cobordisms between oriented 0-manifolds. The box $X_1$ in the picture includes only the data of a pair of finite sets, $(\inp{X_1},\outp{X_1})=(\{1a,1b\},\{1c\})$. Thus each box encodes an oriented 0-manifold. A string diagram, in which boxes $X_1,\ldots,X_n$ are wired together inside a larger box $Y$, can be interpreted as an oriented cobordism from $X_1\sqcup\cdots\sqcup X_n$ to $Y$. This is a morphism in the multicategory $\Cat{Cob}$, underlying the symmetric monoidal category of oriented 1-cobordisms.

There is actually a bit more data in a string diagram for a traced category $\cat{C}$; namely, each wire is labeled by an object of $\cat{C}$. The strings in a string diagram must respect these labels. We will thus consider the multicategory $\Cat{Cob}/\cat{O}$ of oriented 1-dimensional cobordisms over a fixed set $\cat{O}$ of labels. 

In the table below, we record these two interpretations of a string diagram. Note the ``degree shift" between the second and third columns.
\begin{center}
% \begin{tabular}{| l | l | l |}
% \hline
% \multicolumn{3}{|c|}{Interpretations of string diagrams}\\\hline
% String diagram & Traced category $\cat{C}$ & $\Cat{Cob}/\cat{O}$\\\bhline
% Wire label set, $\cat{O}$&Objects, $\cat{O}:=\Ob(\cat{C})$&Label set, $\cat{O}$\\
% Box & Morphism & Object (oriented 0-mfd over $\cat{O}$)\\
% String diagram & Composition & Morphism (Cobordism over $\cat{O}$)\\
% Nesting & Axioms of TSMCs & Composition\\\hline
% \end{tabular}
\begin{tabular}{lll}
\toprule
\multicolumn{3}{c}{Interpretations of string diagrams} \\
\midrule
String diagram & Traced category $\cat{C}$ & $\Cat{Cob}/\cat{O}$ \\
\midrule
Wire label set, $\cat{O}$ & Objects, $\cat{O}:=\Ob(\cat{C})$ & Label set, $\cat{O}$ \\
Box & Morphism & Object (oriented 0-mfd over $\cat{O}$) \\
String diagram & Composition & Morphism (Cobordism over $\cat{O}$) \\
Nesting & Axioms of TSMCs & Composition \\
\bottomrule
\end{tabular}
\end{center}

The relationship between these interpretations is made precise in the following first main theorem, which will be proved in Section~\ref{3}. For any multicategory $\cat{M}$, we denote by $\cat{M}\alg$ the category of lax functors $\cat{M}\to\Cat{Set}$. Let $\Cat{TrCat}$ denote the category of traced categories and traced functors; these will be recalled in Section~\ref{sec:traced categories}.

\begin{theorem}\label{thm:traced as cob-alg}
Consider the functor $\op\Set\to\Cat{Cat}$, given by $\cat{O}\mapsto(\Cat{Cob}/\cat{O})\alg$, and let $(\Cat{Cob}/\bullet)\alg$ denote the total category of corresponding split Grothendieck fibration. Then there is an equivalence of categories
$$(\Cat{Cob}/\bullet)\alg\iso\Cat{TrCat}.$$
\end{theorem}

We sketch the equivalence, with a few minor abuses of notation, as follows. It suffices to fix a set $\cat{O}$ and find an equivalence, natural in $\cat{O}$, between the category of $\Cat{Cob}/\cat{O}$-algebras and the category of traced categories with \todo{Is this accurate?} generating objects $\cat{O}$. We will show that specifying a lax functor $P\colon\Cat{Cob}/\cat{O}\to\Cat{Set}$ requires the same data as specifying a traced category $\cat{C}$ with generating objects $\cat{O}$. 

First, for each box $(\inp{X},\outp{X})$ in a string diagram, both $P$ and $\cat{C}$ require a set, $P(\inp{X},\outp{X})$ and $\Hom_{\cat{C}}(\inp{X},\outp{X})$, respectively.  Second, for each string diagram, both $P$ and $\cat{C}$ require a function: an action on morphisms, in the case of $P$, and a formula for performing the required compositions, tensors, and traces, in the case of $\cat{C}$. The functoriality of $P$ corresponds to the fact that $\cat{C}$ satisfies the axioms of traced categories.

\section{Generalization: lax algebras on a traced or compact category}

It is most convenient to prove Theorem~\ref{thm:traced as cob-alg} by proving a much more general result, which relates lax algebras out of compact categories $\cat{C}$ to strong, bijective-on-objects functors out of $\cat{C}$. Let $\Cat{CompCat}$ denote the category of compact categories and strong monoidal functors between them, and let $\Cat{CompCat}_{\Ob\iso}$ denote the wide subcategory spanned by functors that are bijective on objects. For any symmetric monoidal category $\cat{V}$, we denote by $\Cat{CompCat}^{\cat{V}}_{\Ob\iso}\ss\Cat{CompCat}^{\cat{V}}$ the respective categories of $\cat{V}$-enriched compact categories.

\begin{theorem}\label{thm:compact lax and strong}
 Let $\cat{C}$ be a compact category \todo{Does $\cat{C}$ have to be $\cat{V}$-enriched?} and $\cat{V}$ a symmetric monoidal category. If $\Lax(\cat{C},\cat{V})$ denotes the category of lax monoidal functors out of $\cat{C}$, and $\cat{C}/\Cat{CompCat}^\cat{V}_{\Ob\iso}$ denotes the coslice category of $\cat{V}$-enriched compact categories and strong functors under $\cat{C}$, then there is an equivalence of categories
$$\Lax(\cat{C},\cat{V})\cong\cat{C}/\Cat{CompCat}^\cat{V}_{\Ob\iso}$$
\end{theorem}

By \cite{Abramsky}, $\Cat{Cob}/\cat{O}$ is the free compact category on the set $\cat{O}$ of objects. In this case Theorem~\ref{thm:compact lax and strong} amounts to the following equivalence of categories when $\cat{V}=\Cat{Set}$: 
$$\Lax(\Cat{Cob}/\cat{O},\Cat{Set})\iso\Cat{CompCat}_{\Ob\cong\cat{O}}.$$
It turns out that if our compact category $\cat{C}$ is Int of something, then Theorem~\ref{thm:compact lax and strong} lifts to a result about traced categories, recorded as Corollary~\ref{cor:thm:traced lax and strong}, from which Theorem~\ref{thm:traced as cob-alg} follows. 

\begin{corollary}\label{cor:thm:traced lax and strong}
Suppose that $\cat{D}$ is a traced category and its compact closure is $\cat{C}=Int\cat{D}$. Then there is an equivalence of categories
$$\Lax(\cat{C},\cat{V})\iso\cat{C}/\Cat{TrCat}^\cat{V}_{\Ob\iso}$$
This equivalence is natural in the compact category $\cat{C}$, using the factorization system on $\Cat{CompCat}$ from Lemma~\ref{lemma:factorization system}.
\end{corollary}

We prove the following lemma here, because there is no other place I've found to put it.\todo{Find a better place?}

\begin{lemma}\label{lemma:factorization system}

Let $\Cat{MonCat}$ denote the category of monoidal categories and strong functors. It admits an orthogonal factorization system $(\cat{L},\cat{R})$, where the morphisms in $\cat{L}$ are bijective on objects, and the morphisms in $\cat{R}$ are fully faithful.

\end{lemma}

\begin{proof}[Sketch of proof]

Let $(\cat{C},\otimes_{\cat{C}})$ and $(\cat{D},\otimes_{\cat{D}})$ be monoidal categories, and let $F\colon\cat{C}\to\cat{D}$ be a strong monoidal functor. Define a monoidal category $(\cat{F},\otimes_{\cat{F}})$ to act like $\cat{C}$ on objects and like $\cat{D}$ on morphisms. That is, $\Ob\cat{F}:=\Ob\cat{C}$, and $\Hom_{\cat{F}}(c_1,c_2):=\Hom_{\cat{D}}(Fc_1,Fc_2).$ On objects, put $c_1\otimes_{\cat{F}}c_2:=c_1\otimes_{\cat{C}}c_2$. Given morphisms $f_1\colon c_1\to c_1'$ and $f_2\colon c_2\to c_2'$ in $\cat{F}$, put $f_1\otimes_{\cat{F}}f_2:=f_1\otimes_{\cat{D}}f_2$, using the coherence isomorphisms.

There are induced strong monoidal functors 
$$(\cat{C},\otimes_{\cat{C}})\to(\cat{F},\otimes_{\cat{F}})\to(\cat{D},\otimes_{\cat{D}})$$
which compose to $F$, such that the former is bijective on objects and the latter is fully faithful. The proof of orthogonality is straightforward.\todo{None of this has been rigorously checked.}

\end{proof}


\section[Applications of the cobordism-algebra perspective]{Applications of the cobordism-algebra perspective in engineering design}

When designing or investigating a complex system, it is often useful to think in terms of interacting subsystems, put together to make a larger whole. This is often called \emph{compositionality}.

Thinking of this as processes wired together to make larger processes has and Rupel were originally motivated to formalize the operadic nature of compositionality. This led to the notion of wiring diagrams and their algebras, as discussed in previous work \cite{Spivak}, \cite{Rupel-Spivak}, \cite{Vagner-Spivak-Lerman}. 
 

When drawing 0-manifolds as boxes and cobordisms as string diagrams, composition of cobordisms become nested diagrams.
\begin{figure}[hb]
\activetikz{
	%left large box,
	\node at (.5,2.7){$X\xrightarrow{\Phi}Y\xrightarrow{\Psi}Z$};
	\node at (.5,1.15){\tiny $\alpha$};
	\node at (.2,2.05){\tiny $\beta$};
	\path(-1,0);\blackbox{(3,2.4)}{1}{2}{}{.5}
	%left small boxes (dashed),
	\path(0,1.4);\dashblackbox{(1,.8)}{1}{1}{}{.2}
	\path(0,.3);\dashblackbox{(1,.6)}{1}{2}{}{.2}
	%left outer arcs
	\directarc{(1.1,1.8)}{(1.75,1.6)}
	\directarc{(1.1,.5)}{(1.75,.8)}
	\directarc{(-.75,1.2)}{(-.1,.6)} %this
	\fancyarc{(-.1,1.8)}{(1.1,.7)}{10}{0}
	%left inner arcs, top
	\fancyarc{(.283,1.817)}{(.716,1.817)}{4}{5}
	\directarc{(.1,1.8)}{(.283,1.733)}
	\directarc{(.716,1.733)}{(.9,1.8)}
	%left inner arcs, bottom
	\directarc{(.1,.6)}{(.6,.75)}
	\directarc{(.85,.725)}{(.9,.7)}
	\directarc{(.55,.55)}{(.6,.7)}
	%\directarc{(.1,.5)}{(.2,.525)}
	\directarc{(.55,.5)}{(.9,.5)}
	%left tiny boxes,
	\path(0.333,1.65);\blackbox{(.333,.25)}{2}{2}{}{.1}
	\path(0.65,.65);\blackbox{(.15,.15)}{2}{1}{}{.1}
	\path(0.35,.45);\blackbox{(.15,.15)}{0}{2}{}{.1} %this
	%right large box,
	\node at (4.5,2.7){$X\xrightarrow{\Psi\circ\Phi}Z$};
	\node at (4.5,1.12){\tiny $\alpha$};
	\node at (4.2,2.05){\tiny $\beta$};
	\path(3,0);\blackbox{(3,2.4)}{1}{2}{}{.5}
	%right tiny boxes,
	\path(4.333,1.65);\blackbox{(.333,.25)}{2}{2}{}{.1}
	\path(4.65,.65);\blackbox{(.15,.15)}{2}{1}{}{.1}
	\path(4.25,.45);\blackbox{(.15,.15)}{0}{2}{}{.1}
	%right arcs, between
	\fancyarc{(4.283,1.733)}{(4.85,.725)}{10}{0}
	%right arcs, top
	\fancyarc{(4.283,1.817)}{(4.716,1.817)}{4}{5}
	\directarc{(4.716,1.733)}{(5.75,1.6)}
	%right arcs, bottom
	\directarc{(3.25,1.2)}{(4.6,.75)}
	\directarc{(4.45,.55)}{(4.6,.7)}
	\directarc{(4.45,.5)}{(5.75,.8)}
	%\fancyarc{(4.2,.525)}{(4.45,.5)}{5}{-5}
}
\end{figure}
A commutative square in $\Cat{Cob}$ corresponds to finding two different ways to chunk small boxes inside a big box. The functoriality of $P\colon\Cat{Cob}\to\Set$ can be thought of as ensuring that any two ways to chunk boxes gives the same result.

\section{Acknowledgments}

Thanks go to Steve Awodey and Ed Morehouse for suggesting we formally connect our operad-algebra picture to the traced one. 

\chapter{Wiring Diagrams and $\Cat{Cob}$}

In this section, we more carefully explain the equivalence between the category of traced categories and the category of cobordism algebras.

\section{Objects in $\Cat{Cob}$ as interfaces}

The objects in $\Cat{Cob}$ are signed sets $(\inp{X},\outp{X})$, each of which can be drawn as a box with input wires $\inp{X}$ drawn entering the box, on its left, and output wires $\outp{X}$ drawn exiting the box, on its right. We call the latter style \emph{an interface}.

\begin{figure}
\activetikz{
\draw (0,2) node {$-$};
\draw(0,1.5) node {$-$};
\draw(0,1) node {$-$};
\draw(0,.5) node {$+$};
\draw(0,0) node {$+$};
\path(5,0);
\blackbox{(2,2)}{3}{2}{$X$}{.5}
}
\caption{The signed set $(\ul{3},\ul{2})$, drawn in the usual style and as an interface.}
\end{figure}

\section{Morphisms in $\Cat{Cob}$ as wiring diagrams}

Wiring diagrams are a new way to visualize morphisms in 1-Cob.
\begin{center}
\activetikz{
    %big box
	\path(0,0);
	\blackbox{(10,5)}{2}{1}{$Y$}{.7}
	    %inner wires
	        \node at (.4,3.6) {\tiny $\inp{Y}_{a}$};
	        \node at (.4,1.9) {\tiny $\inp{Y}_{b}$};
	    %outer wire
	        \node at (9.6,2.75) {\tiny $\outp{Y}_a$};
	%little box 1
	\path(2,1.5);
	\blackbox{(2,2)}{2}{1}{$X_1$}{.5}
	    %tank info
	    %inner wires
	        \node at (1.75,3.03) {\tiny $\inp{X}_{1a}$};
	        \node at (1.75,2.35) {\tiny $\inp{X}_{1b}$};
	    %outer wire
	        \node at (4.33,2.68) {\tiny $\outp{X}_{1c}$};
	%little box 2
	\path(6,1.5);
	\blackbox{(2,2)}{2}{2}{$X_2$}{.5}
	    %tank info
	    %inner wires
	        \node at (5.75,3.03) {\tiny $\inp{X}_{2a}$};
	        \node at (5.75,2.35) {\tiny $\inp{X}_{2b}$};
	    %outer wires
	        \node at (8.33,3.03) {\tiny $\outp{X}_{2c}$};
	        \node at (8.33,2.35) {\tiny $\outp{X}_{2d}$};
	%wires
	\directarc{(4.25,2.5)}{(5.75,2.16667)} % X_1 -> X_2
	\directarc{(0.35,1.6667)}{(1.75,2.83333)} % Y -> X_1
	\fancyarc{(0.35,3.3333)}{(5.75,2.83333)}{-40}{25} % Y -> X_2
	\directarc{(8.25,2.8333)}{(9.65,2.5)} % X_2 -> Y
	\fancyarc{(1.75,2.16667)}{(8.25,2.16667)}{20}{-45} % X_2 -> X_1
}
\end{center}
~\todo{Put these two drawings on one line?}
\begin{center}
\activetikz{
%Left manifolds
\def\factor{.7}
\def\lspace{.7}
\def\bspace{5}
\draw (0,7*\factor) node {$\inp{X}_{1a}$};\draw (\lspace,7*\factor) node {$-$};
\draw(0,6*\factor) node {$\inp{X}_{1b}$};\draw (\lspace,6*\factor) node {$-$};
\draw(0,5*\factor) node {$\outp{X}_{1c}$};\draw (\lspace,5*\factor) node {$+$};
%skip 3 to separate manifolds
\draw(0,3*\factor) node {$\inp{X}_{2a}$};\draw (\lspace,3*\factor) node {$-$};
\draw(0,2*\factor) node {$\inp{X}_{2b}$};\draw (\lspace,2*\factor) node {$-$};
\draw(0,1*\factor) node {$\outp{X}_{2c}$};\draw (\lspace,1*\factor) node {$+$};
\draw(0,0*\factor) node {$\outp{X}_{2d}$};\draw (\lspace,0*\factor) node {$+$};
%Right manifold
\draw (\bspace,6*\factor) node {$\inp{Y}_a$};\draw (\bspace-\lspace,6*\factor) node {$-$};
\draw (\bspace,4*\factor) node {$\inp{Y}_b$};\draw (\bspace-\lspace,4*\factor) node {$-$};
\draw (\bspace,2*\factor) node {$\outp{Y}_a$};\draw (\bspace-\lspace,2*\factor) node {$+$};
%Arcs
\draw (1.3*\lspace,6*\factor) .. controls (3*\lspace,6*\factor) and (3*\lspace,0*\factor) .. (1.3*\lspace,0*\factor);
\draw (1.3*\lspace,5*\factor) .. controls (2.5*\lspace,5*\factor) and (2.5*\lspace,2*\factor) .. (1.3*\lspace,2*\factor);
\directarc{(1.3*\lspace,7*\factor)}{(\bspace-1.3*\lspace,4*\factor)}
\directarc{(1.3*\lspace,3*\factor)}{(\bspace-1.3*\lspace,6*\factor)}
\directarc{(1.3*\lspace,1*\factor)}{(\bspace-1.3*\lspace,2*\factor)}
}
\end{center}

\section{$\Cat{Cob}$-algebras and applications}

Traced categories are often used to model processes with feedback. The processes correspond to morphisms, each drawn as an interface, and these processes can be strung together in series or parallel, and with feedback, using string diagram notation. Theorem~\ref{thm:traced as cob-alg} says that we can also view string diagrams as morphisms in $\Cat{Cob}$. In this setting, the processes are modeled using an algebra on $\Cat{Cob}$.

References for applications of traced monoidal categories.

References for applications of $\Cat{Cob}$-algebras.

\chapter{Traced Categories}\label{sec:traced categories}

Let $(\cat{M},\otimes,I)$ be a symmetric monoidal category where for any $X,Y\in\Ob(\cat{M})$ we write $\gamma_{X,Y}:X\otimes Y\To{\sim} Y\otimes X$ for the distinguished symmetry isomorphisms. Recall that we have $\gamma_{Y,X}=\gamma_{X,Y}^{-1}$.  We define the {\em monoid of scalars in $\cat{M}$} to be the set $S_{\cat{M}}:=\Hom(I,I)$ with multiplication given by composition.  Note that $S_{\cat{M}}$ is commutative since $\cat{M}$ is symmetric.  There is an action of $S_{\cat{M}}$ on the set $\Hom_{\cat{M}}(X,Y)$ for each $(X,Y)\in\Ob(\cat{M}^{op}\times\cat{M})$, where $s\in S_{\cat{M}}$ acts on a morphism $f\colon X\to Y$ by sending it to the composite morphism
$$s\bullet f:X\to I\otimes X\To{s\otimes f}I\otimes Y\to Y.$$
We write $|X|:=\Hom(I,X)$ for the \emph{elements} of $X$.

The functor $|\cdot|=\Hom(I,\cdot)\colon\cat{M}\to\Set$ is a unital \todo{This is just a lax monoidal functor, right?} algebra on $\cat{M}$ with unit map
\[\eta:\{1\}\To{\id_{I}}S_{\cat{M}}\]
and multiplication map
\[\mu:\Hom(I,X)\otimes\Hom(I,X')\to\Hom(I\otimes I,X\otimes X')\iso\Hom(I,X\otimes X').\]

A (left) \emph{trace} on a symmetric monoidal category is a collection of functions 
\[\Tr^U_{X,Y}:\Hom(U\otimes X,U\otimes Y)\to\Hom(X,Y)\]
for $U,X,Y\in\Ob(\cat{M})$ satisfying the following axioms:
\begin{description}
 \item [Dinaturality I:] for every $f:U\otimes X\to V\otimes Y$ and $g:V\to U$ we have
 \[\Tr^U_{X,Y}\Big[\big(g\otimes Y\big)\circ f\Big]=\Tr^V_{X,Y}\Big[f\circ\big(g\otimes X\big)\Big];\] 
 \item [Naturality:] for every $f:U\otimes X\to U\otimes Y$, $g:X'\to X$, and $h:Y\to Y'$ we have
 \[\Tr^U_{X',Y'}\Big[\big(U\otimes h\big)\circ f\circ\big(U\otimes g\big)\Big]=h\circ\Tr^U_{X,Y}\big[f\big]\circ g;\]
 \item [Superposing:] for every $f:U\otimes X\to U\otimes Y$ and $g:W\to Z$ we have
 \[\Tr^U_{X,Y}\big[f\big]\otimes g=\Tr^U_{X\otimes W,Y\otimes Z}\big[f\otimes g\big];\]
 \item [Vanishing I:] for every $f:X\to Y$ we have
 \[\Tr^{I}_{X,Y}\big[f\big]=f;\]
 \item [Vanishing II:] for every $f:U\otimes V\otimes X\to U\otimes V\otimes Y$ we have
 \[\Tr^{U\otimes V}_{X,Y}\big[f\big]=\Tr^V_{X,Y}\Big[\Tr^U_{V\otimes X,V\otimes Y}\big[f\big]\Big];\]
 \item [Yanking:] for any $X\in\Ob(\cat{M})$ we have
 \[\Tr^X_{X,X}\big[\gamma_{X,X}\big].\]
\end{description}

For an object $X$ in a traced category $\cat{M}$ we write $\dim(X):=\Tr^X_{I,I}\big[\id_X\big]\in S_{\cat{M}}$ for the {\em dimension} of $X$.  Note also that for any endomorphism $f:X\to X$ there is a scalar $s_f=\Tr^X_{I,I}[f]\in S_{\cat{M}}$.

\begin{proposition}\label{prop:dinaturality}\mbox{}
Let $\cat{M}$ be a symmetric monoidal category.
\begin{enumerate}
 \item If $\cat{M}$ is traced then for every $f:X\to Y$ and $g:Y\to Z$ in $\cat{M}$, we have
 \[g\circ f=\Tr^Y_{X,Z}\Big[\big(f\otimes g\big)\circ\gamma_{Y,X}\Big].\]
 \item Consider the following axiom:
 \begin{description}
  \item [Dinaturality I':] for every $h:U\otimes V\otimes X\to U\otimes V\otimes Y$ we have
 \[\Tr^{U\otimes V}_{X,Y}\big[h\big]=\Tr^{V\otimes U}_{X,Y}\Big[\big(\gamma_{U,V}\otimes Y\big)\circ h\circ\big(\gamma_{V,U}\otimes X\big)\Big].\]
 \end{description}
 In the presence of the other five Axioms, the axioms Dinaturality~I and Dinaturality~I' are equivalent.
\end{enumerate}
\end{proposition}
\begin{proof}
 (1) can be shown from the Naturality and Yanking axioms of the trace as follows:
 \begin{align*}
  \Tr^Y_{X,Z}\Big[\big(f\otimes g\big)\circ\gamma_{Y,X}\Big]
  &=\Tr^Y_{X,Z}\Big[\big(Y\otimes g\big)\circ\big(f\otimes Y\big)\circ\gamma_{Y,X}\Big]=\Tr^Y_{X,Z}\Big[\big(Y\otimes g\big)\circ\gamma_{Y,Y}\circ\big(Y\otimes f\big)\Big]\\
  &=g\circ\Tr^Y_{Y,Y}\big[\gamma_{Y,Y}\big]\circ f=g\circ f.
 \end{align*}
 
 Dinaturality I' is an immediate consequence of Dinaturality I, indeed apply Dinaturality I with $f=h\circ\big(\gamma_{V,U}\otimes X\big)$ and $g=\gamma_{U,V}$.  For the other direction we apply the composition formula of (1) to the left side of the Dinaturality I equation to get
 \begin{align*}
  \Tr^U_{X,Y}\Big[\big(g\otimes Y\big)\circ f\Big]
  &=\Tr^U_{X,Y}\bigg[\Tr^{V\otimes Y}_{U\otimes X,U\otimes Y}\Big[\big(f\otimes g\otimes Y\big)\circ\gamma_{V\otimes Y,U\otimes X}\Big]\bigg]\\
 &=\Tr^{V\otimes Y\otimes U}_{X,Y}\Big[\big(f\otimes g\otimes Y\big)\circ\gamma_{V\otimes Y,U\otimes X}\Big]\\
  &=\Tr^{V\otimes Y\otimes U}_{X,Y}\Big[\big(f\otimes g\otimes Y\big)\circ\gamma_{V\otimes Y,U\otimes X}\Big]\circ\Tr^X_{X,X}\big[\gamma_{X,X}\big],
 \end{align*}
 where the last two equalities follow from Vanishing II and a trivial application of Yanking.  
 
 By Naturality we may bring the first trace into the second to get
 \begin{align*}
  &\Tr^X_{X,Y}\Bigg[\bigg(X\otimes \Tr^{V\otimes Y\otimes U}_{X,Y}\Big[\big(f\otimes g\otimes Y\big)\circ\gamma_{V\otimes Y,U\otimes X}\Big]\bigg)\circ\gamma_{X,X}\Bigg]\\
  &\quad=\Tr^X_{X,Y}\Bigg[\gamma_{Y,X}\circ\bigg(\Tr^{V\otimes Y\otimes U}_{X,Y}\Big[\big(f\otimes g\otimes Y\big)\circ\gamma_{V\otimes Y,U\otimes X}\Big]\otimes X\bigg)\Bigg]\\
  &\quad=\Tr^X_{X,Y}\bigg[\gamma_{Y,X}\circ\Tr^{V\otimes Y\otimes U}_{X\otimes X,Y\otimes X}\Big[\big(f\otimes g\otimes Y\otimes X\big)\circ\big(\gamma_{V\otimes Y,U\otimes X}\otimes X\big)\Big]\bigg]\\
  &\quad=\Tr^X_{X,Y}\bigg[\Tr^{V\otimes Y\otimes U}_{X\otimes X,X\otimes Y}\Big[\big(V\otimes Y\otimes U\otimes \gamma_{Y,X}\big)\circ\big(f\otimes g\otimes Y\otimes X\big)\circ\big(\gamma_{V\otimes Y,U\otimes X}\otimes X\big)\Big]\bigg]\\
  &\quad=\Tr^X_{X,Y}\bigg[\Tr^{V\otimes Y\otimes U}_{X\otimes X,X\otimes Y}\Big[\big(f\otimes g\otimes \gamma_{Y,X}\big)\circ\big(\gamma_{V\otimes Y,U\otimes X}\otimes X\big)\Big]\bigg]\\
  &\quad=\Tr^{V\otimes Y\otimes U\otimes X}_{X,Y}\Big[\big(f\otimes g\otimes \gamma_{Y,X}\big)\circ\big(\gamma_{V\otimes Y,U\otimes X}\otimes X\big)\Big],
 \end{align*}
 where we apply Superposing to get the second equality, Naturality to get the third, and Vanishing II for the last.
 
 Now we are ready to apply Dinaturality I' to get
 \begin{align*}
  \Tr^{U\otimes X\otimes V\otimes Y}_{X,Y}\Big[\big(\gamma_{V\otimes Y,U\otimes X}\otimes Y\big)\circ\big(f\otimes g\otimes \gamma_{Y,X}\big)\Big]
 \end{align*}
 from which we may apply, in reverse, an analogous sequence of equalities to that above to get the right hand side of the Dinaturality I equation.
 \erase{\begin{align*}%begin erase
  &=\Tr^Y_{X,Y}\bigg[\Tr^{U\otimes X\otimes V}_{Y\otimes X,Y\otimes Y}\Big[\big(\gamma_{V\otimes Y,U\otimes X}\otimes Y\big)\circ\big(f\otimes g\otimes \gamma_{Y,X}\big)\Big]\bigg]\\
  &=\Tr^Y_{X,Y}\bigg[\Tr^{U\otimes X\otimes V}_{Y\otimes X,Y\otimes Y}\Big[\big(\gamma_{V\otimes Y,U\otimes X}\otimes Y\big)\circ\big(f\otimes g\otimes X\otimes Y\big)\circ\big(U\otimes X\otimes V\otimes \gamma_{Y,X}\big)\Big]\bigg]\\
  &=\Tr^Y_{X,Y}\bigg[\Tr^{U\otimes X\otimes V}_{X\otimes Y,Y\otimes Y}\Big[\big(\gamma_{V\otimes Y,U\otimes X}\otimes Y\big)\circ\big(f\otimes g\otimes X\otimes Y\big)\Big]\circ\gamma_{Y,X}\bigg]\\
  &=\Tr^Y_{X,Y}\Bigg[\bigg(\Tr^{U\otimes X\otimes V}_{X,Y}\Big[\gamma_{V\otimes Y,U\otimes X}\circ\big(f\otimes g\otimes X\big)\Big]\otimes Y\bigg)\circ\gamma_{Y,X}\Bigg]\\
  &=\Tr^Y_{X,Y}\Bigg[\gamma_{Y,Y}\circ\bigg(Y\otimes\Tr^{U\otimes X\otimes V}_{X,Y}\Big[\gamma_{V\otimes Y,U\otimes X}\circ\big(f\otimes g\otimes X\big)\Big]\bigg)\Bigg]\\
  &=\Tr^Y_{Y,Y}\big[\gamma_{Y,Y}\big]\circ\Tr^{U\otimes X\otimes V}_{X,Y}\Big[\gamma_{V\otimes Y,U\otimes X}\circ\big(f\otimes g\otimes X\big)\Big]\\
  &=\Tr^{U\otimes X\otimes V}_{X,Y}\Big[\gamma_{V\otimes Y,U\otimes X}\circ\big(f\otimes g\otimes X\big)\Big]\\
  &=\Tr^{U\otimes X\otimes V}_{X,Y}\Big[\big(g\otimes X\otimes f\big)\circ\gamma_{U\otimes X,V\otimes X}\Big]\\
  &=\Tr^V_{X,Y}\bigg[\Tr^{U\otimes X}_{V\otimes X,V\otimes Y}\Big[\big(g\otimes X\otimes f\big)\circ\gamma_{U\otimes X, V\otimes X}\Big]\bigg]\\
  &=\Tr^V_{X,Y}\Big[f\circ\big(g\otimes X\big)\Big].
 \end{align*}}%end erase
 %%Another proof
 \erase{By Dinaturality I' this becomes%begin erase
 \[\Tr^U_{X,Y}\bigg[\Tr^{Y\otimes V}_{U\otimes X,U\otimes Y}\Big[\big(\gamma_{V,Y}\otimes U\otimes Y\big)\circ\big(f\otimes g\otimes Y\big)\circ\gamma_{V\otimes Y,U\otimes X}\circ\big(\gamma_{Y,V}\otimes U\otimes X\big)\Big]\bigg]\]
 which is equal via Vanishing II and Naturality to
 \begin{align*}
  &\Tr^U_{X,Y}\Bigg[\Tr^V_{U\otimes X,U\otimes Y}\bigg[\Tr^Y_{V\otimes U\otimes X,V\otimes U\otimes Y}\Big[\big(\gamma_{V,Y}\otimes U\otimes Y\big)\circ\big(f\otimes g\otimes Y\big)\circ\gamma_{V\otimes Y,U\otimes X}\circ\big(\gamma_{Y,V}\otimes U\otimes X\big)\Big]\bigg]\Bigg]\\
  &\quad=\Tr^{V\otimes U}_{X,Y}\bigg[\Tr^Y_{V\otimes U\otimes X,V\otimes U\otimes Y}\Big[\big(Y\otimes V\otimes g\otimes Y\big)\circ\big(\gamma_{V,Y}\otimes V\otimes Y\big)\circ\gamma_{V\otimes Y,V\otimes Y}\circ\big(\gamma_{Y,V}\otimes V\otimes Y\big)\circ\big(Y\otimes V\otimes f\big)\Big]\bigg]\\
  &\quad=\Tr^{V\otimes U}_{X,Y}\bigg[\big(V\otimes g\otimes Y\big)\circ\Tr^Y_{V\otimes V\otimes Y,V\otimes V\otimes Y}\Big[\big(\gamma_{V,Y}\otimes V\otimes Y\big)\circ\gamma_{V\otimes Y,V\otimes Y}\circ\big(\gamma_{Y,V}\otimes V\otimes Y\big)\Big]\circ\big(V\otimes f\big)\bigg]\\
  &\quad=\Tr^{V\otimes U}_{X,Y}\bigg[\big(V\otimes g\otimes Y\big)\circ\big(\gamma_{V,V}\otimes Y\big)\circ\big(V\otimes f\big)\bigg].
 \end{align*}
 Applying Dinaturality I' again we get
 \begin{align*}
  &\Tr^{U\otimes V}_{X,Y}\bigg[\big(\gamma_{V,U}\otimes Y\big)\circ\big(V\otimes g\otimes Y\big)\circ\big(\gamma_{V,V}\otimes Y\big)\circ\big(V\otimes f\big)\circ\big(\gamma_{U,V}\otimes X\big)\bigg]\\
  &=\Tr^{U\otimes V}_{X,Y}\bigg[\big(g\otimes f\big)\circ\big(\gamma_{U,V}\otimes X\big)\bigg]\\
  &=\Tr^{U\otimes V}_{X,Y}\bigg[\big(U\otimes f\big)\circ\big(\gamma_{U,U}\otimes X\big)\circ\big(U\otimes g\otimes X\big)\bigg]\\
  &=\Tr^{U\otimes V}_{X,Y}\bigg[\big(U\otimes f\big)\circ\Tr^X_{U\otimes U\otimes X,U\otimes U\otimes X}\Big[\big(\gamma_{U,X}\otimes U\otimes X\big)\circ\gamma_{U\otimes X, U\otimes X}\circ\big(\gamma_{X,U}\otimes U\otimes X\big)\Big]\circ\big(U\otimes g\otimes X\big)\bigg]
 \end{align*}
 Now using Naturality and Vanishing II we get
 \begin{align*}
  &\Tr^{U\otimes V}_{X,Y}\bigg[\Tr^X_{U\otimes V\otimes X,U\otimes V\otimes Y}\Big[\big(X\otimes U\otimes f\big)\circ\big(\gamma_{U,X}\otimes U\otimes X\big)\circ\gamma_{U\otimes X, U\otimes X}\circ\big(\gamma_{X,U}\otimes U\otimes X\big)\circ\big(X\otimes U\otimes g\otimes X\big)\Big]\bigg]\\
  &=\Tr^V_{X,Y}\Bigg[\Tr^U_{V\otimes X,V\otimes Y}\bigg[\Tr^X_{U\otimes V\otimes X,U\otimes V\otimes Y}\Big[\big(\gamma_{U,X}\otimes V\otimes Y\big)\circ\big(g\otimes X\otimes f\big)\circ\gamma_{U\otimes X, V\otimes X}\circ\big(\gamma_{X,U}\otimes V\otimes X\big)\Big]\bigg]\Bigg]\\
  &=\Tr^V_{X,Y}\bigg[\Tr^{X\otimes U}_{V\otimes X,V\otimes Y}\Big[\big(\gamma_{U,X}\otimes V\otimes Y\big)\circ\big(g\otimes X\otimes f\big)\circ\gamma_{U\otimes X, V\otimes X}\circ\big(\gamma_{X,U}\otimes V\otimes X\big)\Big]\bigg]
 \end{align*}
 which by Dinaturality I' is equal to
 \[\Tr^V_{X,Y}\bigg[\Tr^{U\otimes X}_{V\otimes X,V\otimes Y}\Big[\big(g\otimes X\otimes f\big)\circ\gamma_{U\otimes X, V\otimes X}\Big]\bigg],\]
 but this is exactly the composition formula applied to the right hand side of Dinaturality I.}%end erase
\end{proof}

\section{The Int construction}

For a traced category $\cat{M}$ let $\widetilde{\cat{M}}=\Int(\cat{M})$ denote the category with objects given by pairs $(\inp{X},\outp{X})$ where $\inp{X},\outp{X}\in \Ob(\cat{M})$ and morphisms given by 
\[\Hom_{\widetilde{\cat{M}}}\big((\inp{X},\outp{X}),(\inp{Y},\outp{Y})\big)=\Hom_{\cat{M}}(\inp{X}\otimes \outp{Y},\outp{X}\otimes \inp{Y}).\]
For morphisms $\Phi:(\inp{X},\outp{X})\to(\inp{Y},\outp{Y})$ and $\Psi:(\inp{Y},\outp{Y})\to(\inp{Z},\outp{Z})$ in $\widetilde{\cat{M}}$ we define their composition to be
\[\Psi\circ\Phi:=\Tr^{\outp{Y}}_{\inp{X}\otimes \outp{Z},\outp{X}\otimes \inp{Z}}\Big[\big(\gamma_{\outp{X},\outp{Y}}\otimes \inp{Z}\big)\circ\big(\outp{X}\otimes\Psi\big)\circ\big(\Phi\otimes \outp{Z}\big)\circ\big(\gamma_{\outp{Y},\inp{X}}\otimes \outp{Z}\big)\Big].\]
It is well known that $\widetilde{\cat{M}}$ is a compact  category whose tensor is given by
\[(\inp{X},\outp{X})\odot(\inp{Y},\outp{Y}):=(\inp{X}\otimes \inp{Y},\outp{X}\otimes \outp{Y})\]
with unit object $\tilde I:=(I,I)$ and duality $(\inp{X},\outp{X})^\vee:=(\outp{X},\inp{X})$.  The following is immediate from the definitions.

\begin{lemma}\todo{This lemma has a sign-error. Either change definition of $\Hom_{\widetilde{\cat{M}}}$ or lemma statement.}

Let $\cat{M}$ be a traced category.  Then for any object $(\inp{X},\outp{X})\in\Ob\big(\widetilde{\cat{M}}\big)$ there is a canonical bijection
\[|(\inp{X},\outp{X})|\iso\Hom_{\cat{M}}(\inp{X},\outp{X}).\]

\end{lemma}

\section{Free Constructions}

To continue we recall the constructions of various free monoidal structures on a category $\cat{C}$ following \cite{abramsky}.  These are defined inductively by successively including additional structure into the constructions.

The objects of the free monoidal category $F_M(\cat{C})$ are lists of objects in $\cat{C}$ with componentwise morphisms.  More formally the objects are pairs $(n,X)$ where $n\in\NN$ and $X$ is a map $X\colon[n]\to \Ob\cat{C}$ and a morphism $f:(n,X)\to(m,Y)$ exists if and only if $n=m$ in which case $f\in\prod_{i=1}^n\Hom_{\cat{C}}(X_i,Y_i)$ with compositions formed as expected.  The tensor product is given by concatenation of lists, i.e. $(n,X)\otimes(m,Y)=(n+m,X+Y)$, with tensor unit $(0,!)$ where $!$ is the unique function from the empty set.

The objects of the free symmetric monoidal category $F_{SM}(\cat{C})$ are again lists of objects in $\cat{C}$, however morphisms now come equipped with a permutation.  More formally, a morphism from $(n,X)$ to $(n,Y)$ is a pair $(f,\pi)$ where $f\in\prod_{i=1}^n\Hom_{\cat{C}}(X_i,Y_{\pi(i)})$ and $\pi\in S(n)$ is a permutation which by abuse of notation can be thought of as an isomorphism $\pi:Y_1\otimes\cdots\otimes Y_n\To{\sim} Y_{\pi(1)}\otimes\cdots\otimes Y_{\pi(n)}$.  The composition of morphisms is given by 
$$(\sigma,g)\circ(\pi,f):=\Big(\sigma\circ\pi,\otimes_{i=1}^n (g_{\pi(i)}\circ f_i)\Big)$$ 
and the tensor is given by 
$$(\pi,f)\otimes(\sigma,g):=(\pi\otimes\sigma,f\otimes g).$$  


\section{Definition of functors between $\Cat{TrCat}$ and $\Cat{Cob}/\cat{O}$-algebras}

Let $\cat{M}$ be a traced category with objects $\cat{O}$. We will define a $\Cat{Cob}/\cat{O}$-algebra $\cat{P}=R(\cat{M})\colon\Cat{Cob}/\cat{O}\to\Cat{Set}$ as follows. For an object $X\in\Ob(\Cat{Cob}/\cat{O})$, set 
$$\cat{P}(X):=\Hom_{\cat{M}}(\vinp{X},\voutp{X}).$$
We next consider morphisms.

Following Proposition~\ref{prop:set theoretic cob1} (borrow notation/setup from Abramsky instead) a morphism $\Phi\colon X\longrightarrow Y$ consists of a typed bijection 
$$\varphi\colon\inp{X}\sqcup \outp{Y}\xrightarrow{\iso}\outp{X}\sqcup \inp{Y},$$ 
together with a typed finite set $S$. Given an element $f\in\cat{P}(X)$ we must construct $\cat{P}(\Phi)(f)\in\cat{P}(Y)$. Let $\dim(\overline{S})=\textnormal{Tr}^{\overline{S}}_{I,I}\big[\id_{\overline{S}}\big]\in\cat{S}_\cat{M}$. Then we use the formula
$$\cat{P}(\Phi)(f):=
\textnormal{Tr}^{\voutp{X}}_{\vinp{Y},\voutp{Y}}\Big[\big(f\otimes\id_{\voutp{Y}}\big)\circ\overline{\varphi}\Big]
\otimes\dim(\overline{S}).	
$$

\begin{theorem}
 The category $\Cat{TrCat}$ is equivalent to the category of $\Cat{Cob}/\cat{O}$-algebras.
\end{theorem}
\begin{proof}
 
\end{proof}
\begin{corollary}
 Enriched setting?
\end{corollary}



% -*- root: CCC_Note.tex -*-
\chapter{Preliminaries}

Let $\cat{C}$ and $\cat{D}$ be monoidal categories. Recall that a functor $F\colon\cat{C}\to\cat{D}$ is called \emph{lax monoidal} if it is equipped with a morphism
\[
\begin{tikzcd}
	I_D \rar{\epsilon} & F(I_C)
\end{tikzcd}
\]
and a natural transformation
\[
\begin{tikzcd}
	F(X) \otimes_D F(Y) \rar{\mu_{X,Y}} & F(X\otimes_C Y)
\end{tikzcd}
\]
such that for all $X,Y,Z\in\cat{C}$, the diagram (suppressing associators)
\[
\begin{tikzcd}
	F(X)\otimes F(Y) \otimes F(Z)
		\rar{\id\otimes\mu}
		\dar[swap]{\mu\otimes\id}
	& F(X)\otimes F(Y\otimes Z)
		\dar{\mu} \\
	F(X\otimes Y)\otimes F(Z)
		\rar[swap]{\mu}
	& F(X\otimes Y\otimes Z)
\end{tikzcd}
\]
commutes, and for all $X\in\cat{C}$ the two diagrams
\[
\begin{tikzcd}
	I_D\otimes F(X)
		\dar[swap]{\epsilon\otimes\id}
	& F(X)
		\lar[swap]{l_{F(X)}}
		\dar{F(l_X)} \\
	F(I_C)\otimes F(X)
		\rar[swap]{\mu}
	& F(I_C\otimes X)
\end{tikzcd}
\qquad
\begin{tikzcd}
	F(X) \otimes I_D
		\dar[swap]{\id\otimes\epsilon}
	& F(X)
		\lar[swap]{r_{F(X)}}
		\dar{F(r_X)} \\
	F(X)\otimes F(I_C)
		\rar[swap]{\mu}
	& F(X\otimes I_C)
\end{tikzcd}
\]
commute. If $\epsilon$ and $\mu$ are isomorphisms, then $F$ is \emph{strong}.

If $\cat{C}$ and $\cat{D}$ are symmetric monoidal, then $F$ is a \emph{lax symmetric monoidal functor} if it is lax monoidal, and commutes with the symmetries, in the sense that the diagram
\[
\begin{tikzcd}
	F(X)\otimes F(Y)
		\rar{\sigma}
		\dar[swap]{\mu}
	& F(Y)\otimes F(X)
		\dar{\mu} \\
	F(X\otimes Y)
		\rar[swap]{F(\sigma)}
	& F(Y\otimes X)
\end{tikzcd}
\]
commutes.

If $F$ and $G$ are lax monoidal functors (possibly symmetric), then a natural transformation $\alpha\colon F\to G$ is called a \emph{monoidal transformation} if the diagrams
\[
\begin{tikzcd}
	F(X)\otimes F(Y)
		\rar{\alpha_X\otimes\alpha_Y}
		\dar[swap]{\mu}
	& G(X)\otimes G(Y)
		\dar{\mu} \\
	F(X\otimes Y)
		\rar[swap]{\alpha_{X\otimes Y}}
	& G(X\otimes Y)
\end{tikzcd}
\qquad
\begin{tikzcd}[column sep=tiny]
	{} & I_D \dlar[swap]{\epsilon} \drar{\epsilon} & \\
	F(I_C) \ar{rr}[swap]{\alpha_I} && G(I_C)
\end{tikzcd}
\]
commute.

Let $\SymMonCat$ denote the bicategory of symmetric monoidal categories and strong monoidal functors, and let $\Lax(\cat{C},\cat{D})$ denote the category of lax monoidal functors and monoidal transformations from $\cat{C}$ to $\cat{D}$. Let $\CompCat$ denote the full subcategory of $\SymMonCat$ spanned by the compact categories.

% \begin{theorem}
% 	Let $\cat{C}$ be a compact category. There is an equivalence of categories
% 	\[
% 		\Lax(\cat{C},\Set) \simeq (\cat{C}\backslash\CompCat)_{\text{boo}}
% 	\]
% 	between the lax functor category from $\cat{C}$ to $\Set$ equipped with the cartesian monoidal structure, and the full subcategory of the undercategory $\cat{C}\backslash\CompCat$ spanned by the bijective-on-objects functors.
% \end{theorem}
% \begin{proof}
% 	Fix a lax symmetric functor $F\colon\cat{C}\to\Set$. We can construct a compact category $\hat{F}$ and a strong bijective-on-objects functor $\tilde{F}\colon\cat{C}\to\hat{F}$ as follows:
% 	\begin{compactitem}
% 		\item The objects of $\hat{F}$ are the objects of $\cat{C}$.
% 		\item $\Hom_{\hat{F}}(A,B)=F(A^{\star}\otimes B)$.
% 		\item Composition $\Hom(A,B)\times\Hom(B,C)\to\Hom(A,C)$ is defined by
% 		\[
% 		\begin{tikzcd}[column sep=-2ex]
% 			{} & F(A^{\star}\otimes B\otimes B^{\star}\otimes C)
% 				\drar{F(\id\otimes\epsilon_B\otimes\id)} & \\
% 			F(A^{\star}\otimes B)\times F(B^{\star}\otimes C)
% 				\urar{\mu_F}
% 			&& F(A^{\star}\otimes B)
% 		\end{tikzcd}
% 		\]
% 		\item Identities $1\to\Hom(A,A)$ are defined by
% 		\[
% 		\begin{tikzcd}
% 			1 \rar{\epsilon_F} & F(I) \rar{F(\eta_A)} & F(A^{\star}\otimes A).
% 		\end{tikzcd}
% 		\]
% 		\item The tensor product
% 		\[
% 		\begin{tikzcd}
% 			\Hom(A,A')\times\Hom(B,B') \rar{\otimes}
% 			& \Hom(A\otimes B,A'\otimes B')
% 		\end{tikzcd}
% 		\]
% 		is defined by
% 		\[
% 		\begin{tikzcd}[column sep=-3ex]
% 			{} & F(A^{\star}\otimes A'\otimes B^{\star}\otimes B')
% 				\drar{F(\id\otimes\sigma\otimes\id)} & \\
% 			F(A^{\star}\otimes A')\times F(B^{\star}\otimes B')
% 				\urar{\mu_F}
% 			&& F(A^{\star}\otimes B^{\star}\otimes A'\otimes B')
% 		\end{tikzcd}
% 		\]
% 		\item $\tilde{F}$ is identity on objects, and for any $f\colon A\to B$ in $\cat{C}$, define $\tilde{F}(f)$ by
% 		\[
% 		\begin{tikzcd}
% 			1 \rar{\epsilon_F}
% 			& F(I) \rar{F(\eta_A)}
% 			& F(A^{\star}\otimes A) \rar{F(\id\otimes f)}
% 			& F(A^{\star}\otimes B)
% 		\end{tikzcd}
% 		\]
% 		\item The associator and symmetry isomorphisms of $\hat{F}$ are given by the image under $\tilde{F}$ of those in $\cat{C}$.
% 	\end{compactitem}

% 	In the other direction, suppose we are given a compact category $\hat{F}$ and a strong bijective-on-objects functor $\tilde{F}\colon\cat{C}\to\hat{F}$. Define $F$ by
% 	\begin{compactitem}
% 		\item $F(A)=\Hom_{\hat{F}}(I,A)$.
% 		\item For $f\colon A\to B$ in $\cat{C}$, define $F(f)\colon\Hom(I,A)\to\Hom(I,B)$ by post-composition with $\tilde{F}(f)$.
% 		\item $\epsilon\colon 1\to F(I)$ is defined by $\id_I\in\Hom_{\hat{F}}(I,I)$.
% 		\item $\mu\colon F(A)\times F(B)\to F(A\otimes B)$ is defined by
% 		\[
% 		\begin{tikzcd}[column sep=-3ex]
% 			{} & \Hom(I\otimes I,A\otimes B)
% 				\drar & \\
% 			\Hom(I,A)\times\Hom(I,B)
% 				\urar{\otimes_{\hat{F}}}
% 			&& \Hom(I,A\otimes B)
% 		\end{tikzcd}
% 		\]
% 	\end{compactitem}
% \end{proof}

\chapter{Profunctors}

Let $\cat{C}$ and $\cat{D}$ be categories. Recall that a profunctor $M$ from $\cat{C}$ to $\cat{D}$, written
\[
\begin{tikzcd}
	\cat{C} \ar[r,tick,"M"] & \cat{D},
\end{tikzcd}
\]
is defined to be a functor $M\colon\op{\cat{C}}\times\cat{D}\to\Set$. We can think of a profunctor as a sort of graded bimodule: for each object $c\in\cat{C}$ and $d\in\cat{D}$ there is a set $M(c,d)$ of elements in the bimodule, and given an element $m\in M(c,d)$ and morphisms $f\colon c'\to c$ in $\cat{C}$ and $g\colon d\to d'$ in $\cat{D}$, there are elements $g\cdot m\in M(c,d')$ and $m\cdot f\in F(c',d)$, such that $(g\cdot m)\cdot f=g\cdot(m\cdot f)$, and $g'\cdot(g\cdot m)=(g'\circ g)\cdot m$ and $(m\cdot f)\cdot f'=m\cdot(f\circ f')$ whenever they make sense.

If $F\colon\cat{C}'\to\cat{C}$ and $G\colon\cat{D}'\to\cat{D}$ are functors, and $M$ is a profunctor as before, then there is a profunctor $M(F,G)$ from $\cat{C}'$ to $\cat{D}'$, defined to be the composite
\[
\begin{tikzcd}
	\op{\cat{C}'}\times\cat{D}' \ar[r,"\op{F}\times G"]
		&[1.5em] \op{\cat{C}}\times\cat{D} \ar[r,"M"]
		& \Set.
\end{tikzcd}
\]
In other words, for any objects $c\in\Cat{C}'$ and $d\in\Cat{D}'$, the profunctor $M(F,G)$ has elements $M(Fc,Gd)$, and if $m\in M(Fc,Gd)$ and $g\colon d\to d'$ is a morphism in $\cat{D}'$, then the element $m\cdot g$ in $M(F,G)$ is defined by the element $m\cdot G(g)$ in $M$, and similarly for the $\cat{C}'$ action.

Given two profunctors
\[
\begin{tikzcd}
	\cat{C} \ar[r,tick,shift left,"M"] \ar[r,tick,shift right,"N"'] & \cat{D}
\end{tikzcd}
\]
define a profunctor morphism $\phi\colon M\Rightarrow N$ to be a natural transformation. In other words, for each $c\in\cat{C}$ and $d\in\cat{D}$ there is a function $\phi_{c,d}\colon M(c,d)\to N(c,d)$ such that $\phi(f\cdot m \cdot g)=f\cdot\phi(m)\cdot g$ whenever it makes sense.

There is a tensor product of profunctors: given two profunctors
\[
\begin{tikzcd}
	\cat{C} \ar[r,tick,"M"] & \cat{D} \ar[r,tick,"N"] & \cat{E}
\end{tikzcd}
\]
define the profunctor $M\otimes N$ such that for objects $c\in\cat{C}$ and $e\in\cat{E}$, $(M\otimes N)(c,e)$ is the coequalizer of the diagram
\[
\begin{tikzcd}
	\displaystyle\coprod_{d_1,d_2\in\cat{D}} M(c,d_1)\times\cat{D}(d_1,d_2)\times N(d_2,e)
		\ar[r,shift left] \ar[r,shift right]
	& \displaystyle\coprod_{d\in\cat{D}} M(c,d)\times N(d,e)
\end{tikzcd}
\]
where the two maps are given by the right action of $\cat{D}$ on $M$ and by the left action of $\cat{D}$ on $N$. We can write elements of $(M\otimes N)(c,e)$ as tensors $m\otimes n$, where $m\in M(c,d)$ and $n\in N(d,e)$ for some $d\in\cat{D}$. The coequalizer then implies that $(m\cdot f)\otimes n=m\otimes(f\cdot n)$ whenever the equation makes sense.

For any category $\cat{C}$, there is a profunctor $\Hom_{\cat{C}}\colon\op{\cat{C}}\times\cat{C}\to\Set$, and these hom profunctors act as units for the tensor product. Precisely, if $M$ is as above, there are canonical isomorphisms $\Hom_{\cat{C}}\otimes M \iso M \iso M\otimes\Hom_{\cat{D}}$.

Given a category $\cat{C}$, there is a monoidal category $\Prof(\cat{C},\cat{C})$ of profunctors from $\cat{C}$ to itself and morphisms of profunctors, with the tensor product given above and $\Hom_{\cat{C}}$ as the monoidal unit. We would now like to investigate monoids in this monoidal category.

Suppose $M\in\Prof(\cat{C},\cat{C})$ has a monoid structure. The unit is a profunctor morphism $i\colon\Hom_{\cat{C}}\to M$. So for any $f\colon c\to d$ in $\cat{C}$ there is an element $i(f)\in M(c,d)$, such that $f\cdot i(g)\cdot h = i(f\circ g\circ h)$ whenever this makes sense. The multiplication $M\otimes M\to M$ is an operation assigning to any elements $m_1\in M(c,d)$ and $m_2\in M(d,e)$ an element $m_2\bullet m_1\in M(c,e)$, which is associative, and satisfies the following equations whenever they make sense:
\begin{gather*}
	(f\cdot m_2)\bullet(m_1\cdot h) = f\cdot(m_2\bullet m_1)\cdot h \\
	(m_3\cdot g)\bullet m_1 = m_3\bullet(g\cdot m_1) \\
	m\bullet i(f) = m\cdot f \quad\text{and}\quad i(g)\bullet m = g\cdot m
\end{gather*}

\begin{lemma}
	There is an equivalence of categories $\Mon(\Prof(\cat{C},\cat{C}))\iso (\cat{C}/\Cat{Cat})_{\text{b.o.o.}}$ between the category of monoids in $\Prof(\cat{C},\cat{C})$ and the full subcategory of the coslice category $\cat{C}/\Cat{Cat}$ spanned by the bijective-on-objects functors.
\end{lemma}
\begin{proof}
	Simple to check. The unit provides the identities and the functor from $\cat{C}$, while the multiplication provides the composition.
\end{proof}

Now suppose $\cat{C}$ and $\cat{D}$ are monoidal categories. We will write
\[
	a_{c,d,e}\colon (c\otimes d)\otimes e \to c\otimes(d\otimes e), 
		\quad \lambda_c\colon I\otimes c\to c,
		\quad \rho_c\colon c\otimes I \to c 
\]
for the associator and left and right unitor isomorphisms, respectively, leaving it to context to make clear whether we are in $\cat{C}$ or $\cat{D}$.

A \emph{monoidal profunctor} $M$ from $\cat{C}$ to $\cat{D}$ is an ordinary profunctor such that the functor $M\colon \op{\cat{C}}\times\cat{D}\to\Set$ is equipped with a lax-monoidal structure, with the cartesian monoidal structure on $\Set$. In the bimodule notation, this means that there is an associative operation assigning to any elements $m_1\in M(c_1,c'_1)$ and $m_2\in M(c_2,c'_2)$ an element $m_1\boxtimes m_2\in M(c_1\otimes c_2,c'_1\otimes c'_2)$ such that
\[
	(f_1\cdot m_1\cdot g_1)\boxtimes(f_2\cdot m_2\cdot g_2) = (f_1\otimes f_2)\cdot(m_1\boxtimes m_2)\cdot(g_1\otimes g_2),
\]
as well as a distinguished element $I_M\in M(I,I)$ such that $\lambda_d\cdot(I_M\boxtimes m)\cdot\lambda^{-1}_c = m = \rho_d\cdot(m\boxtimes I_M)\cdot\rho^{-1}_c$ for any $m\in M(c,d)$.

A monoidal profunctor morphism $\phi\colon M\to N$ is simply a monoidal transformation. Spelling this out in bimodule notation, $\phi$ is an ordinary morphism of profunctors such that $\phi(m_1\boxtimes m_2)=\phi(m_1)\boxtimes\phi(m_2)$ and $\phi(I_M)=I_N$. We will denote the category of monoidal profunctors from $\cat{C}$ to $\cat{D}$ and monoidal profunctor morphisms as $\MProf(\cat{C},\cat{D})$.

A unit for a monoidal profunctor $M\in\MProf(\cat{C},\cat{C})$ is a unit $i\colon\Hom_{\cat{C}}\to M$ in $\Prof(\cat{C},\cat{C})$ such that, additionally, $i(\id_{I_{\cat{C}}})=I_M$ and $i(f\otimes g)=i(f)\boxtimes i(g)$ for any morphisms $f$ and $g$ in $\cat{C}$. Similarly, a multiplication on $M$ is as above, with the additional conditions
\begin{gather*}
	I_M\bullet I_M=I_M \\
	(m_1\boxtimes m'_1)\bullet(m_2\boxtimes m'_2) = (m_1\bullet m_2)\boxtimes(m'_1\bullet m'_2)
\end{gather*}
for any $m_1\in M(c,d)$, $m'_1\in M(c',d')$, $m_2\in M(d,e)$, and $m'_2\in M(d'e')$.

\begin{lemma}
	Let $\cat{C}$ be a monoidal category. There is an equivalence of categories $\Mon(\MProf(\cat{C},\cat{C}))\iso (\cat{C}/\Cat{MonCat})_{\text{b.o.o.}}$ between the category of monoids in $\MProf(\cat{C},\cat{C})$ and the full subcategory of the coslice category $\cat{C}/\Cat{MonCat}$ spanned by the bijective-on-objects functors.
\end{lemma}

\chapter{Compact closed categories}

Let $\cat{C}$ be a compact closed category.

\begin{proposition}
	There are functors
	\[
	\begin{tikzcd}
		\MProf(1,\cat{C}) \ar[r,shift left,"F"]
		& \MProf(\cat{C},\cat{C}) \ar[l,shift left,"U"]
	\end{tikzcd}
	\]
\end{proposition}
\begin{proof}
	For any $M\colon\cat{C}\to\Set$, define $FM\colon\op{\cat{C}}\times\cat{C}\to\Set$ by $FM(A,B)=M(A^*\otimes B)$. In the other direction, for $N\colon\op{\cat{C}}\times\cat{C}\to\Set$, define $UN(A)=N(1,A)$.
\end{proof}

\begin{proposition}
	Let $N\in\MProf(\cat{C},\cat{C})$ be a monoidal profunctor equipped with a unit $\eta\colon\Hom_{\cat{C}}\to N$. Then $N$ has a canonical multiplication $\mu\colon N\otimes N\to N$ making $N$ a monoid in $\MProf(\cat{C},\cat{C})$.
\end{proposition}
\begin{proof}
	We can define a multiplication on $N$ by the following formula: given any $n_1\in N(c,d)$ and $n_2\in N(d,e)$,
	\[
		n_2\bullet n_1 = (\epsilon_d\otimes\id_e)\cdot(n_1\boxtimes i(\id_{d^*})\boxtimes n_2)\cdot(\id_c\otimes \eta_d).
	\]
	We first check the equation $n\bullet i(f)=n\cdot f$ for any $n\in N(d,e)$ and $f\colon c\to d$:
	\begin{align*}
		n\bullet i(f) &= (\epsilon_d\otimes\id_e)\cdot\bigl(i(f)\boxtimes i(\id_{d^*})\boxtimes n\bigr)\cdot(\id_c\otimes \eta_d) \\
		&= (\epsilon_d\otimes\id_e)\cdot\bigl(i(f\otimes \id_{d^*})\boxtimes n\bigr)\cdot(\id_c\otimes \eta_d) \\
		&= \bigl((\epsilon_d\cdot i(f\otimes \id_{d^*}))\boxtimes (\id_e\cdot n)\bigr)\cdot(\id_c\otimes \eta_d) \\
		&= \bigl(i(\epsilon_d\circ (f\otimes \id_{d^*}))\boxtimes n\bigr)\cdot(\id_c\otimes \eta_d) \\
		&= \bigl(i(\id_I)\boxtimes n\bigr)\cdot\bigl(((\epsilon_d\circ (f\otimes \id_{d^*}))\otimes\id_d)\circ(\id_c\otimes \eta_d)\bigr) \\
		&= \bigl(I_N\boxtimes n\bigr)\cdot\bigl((\epsilon_d\otimes\id_d)\circ(\id_d\otimes\eta_d)\circ(f\otimes\id_I)\bigr) \\
		&= n\cdot f
	\end{align*}
\end{proof}

\begin{proposition}
	For any $M\in\MProf(1,\cat{C})$, $FM\in\Prof(\cat{C},\cat{C})$ has a canonical unit.
\end{proposition}

\begin{corollary}
	The functor $F$ factors canonically through the category $\Mon(\MProf(\cat{C},\cat{C}))$ of monoid objects.
\end{corollary}

\begin{proposition}
	The functors $F$ and $U$ induce an equivalence of categories $\MProf(1,\cat{C})\simeq\Mon(\MProf(\cat{C},\cat{C}))$.
\end{proposition}

\begin{corollary}
	There is an equivalence of categories $\Lax(\cat{C},\Set)\simeq(\cat{C}/\Cat{CompCat})_{\text{b.o.o.}}$.
\end{corollary}

\chapter{Traced Monoidal Categories}

Recall from~\cite{JoyalStreet}
\begin{compactitem}
	\item Let $\cat{D}$ be a traced symmetric monoidal category, and $F\colon\cat{C}\to\cat{D}$ a fully faithful symmetric monoidal functor. Then $\cat{C}$ has a unique trace for which $F$ is a traced functor.
	\item Any compact category has a canonical trace, defining a functor $U\colon\Cat{CompCat}\to\Cat{TrCat}$.
	\item The Int construction $\Int\colon\Cat{TrCat}\to\Cat{CompCat}$ is left 2-adjoint to $U$. For any traced symmetric monoidal category $\cat{C}$, the unit $\cat{C}\to\Int(\cat{C})$ is fully faithful.
\end{compactitem}

\begin{lemma}
	Let $\cat{D}$ be a traced symmetric monoidal category, and $F\colon\cat{C}\to\cat{D}$ a fully faithful symmetric monoidal functor. Then, using the unique trace on $\cat{C}$ making $F$ a traced functor, the functor $\Int(\cat{C})\to\cat{D}$ which is adjunct to $F$ is also fully faithful.
\end{lemma}

\begin{proposition}
	Let $\cat{C}$ be a traced symmetric monoidal category. Then the Int construction provides an equivalence of categories
	\[
		(\cat{C}/\Cat{TrCat})_{\text{b.o.o.}} \simeq (\Int(\cat{C})/\Cat{CompCat})_{\text{b.o.o.}}
	\]
\end{proposition}

\end{document} 