\documentclass[12pt,oneside,article,draft]{memoir}
% !TEX root = ./CCC_Note.tex

\usepackage{amsmath}
\usepackage{amsthm}
\usepackage{amsfonts}
\usepackage{amssymb}
\usepackage{mathtools}
%\usepackage{datetime}
\usepackage[T1]{fontenc}
\usepackage[sc]{mathpazo}
\linespread{1.05}
\usepackage{mathrsfs}
\usepackage{euscript}
%\usepackage{MnSymbol}
\usepackage{paralist}
\usepackage{todonotes}
\usepackage{makecell}
\usepackage{booktabs}
\usepackage{tikz}
\usetikzlibrary{cd}
\usepackage{hyperref}
\usepackage{tensor}

\usetikzlibrary{decorations.markings,arrows.meta,calc,fit,quotes}
\hypersetup{final}

\DeclareMathOperator{\id}{id}
\DeclareMathOperator{\dom}{dom}
\DeclareMathOperator{\cod}{cod}
\DeclareMathOperator{\dvert}{Vert}
\DeclareMathOperator{\Lax}{Lax}
\DeclareMathOperator{\Hom}{Hom}
\DeclareMathOperator{\Prof}{Prof}
\DeclareMathOperator{\MProf}{MProf}
\DeclareMathOperator{\Int}{Int}
\DeclareMathOperator{\Ob}{Ob}
\DeclareMathOperator{\Tr}{Tr}


\theoremstyle{plain}
\newtheorem{theorem}{Theorem}[section]
\newtheorem*{theorem*}{Theorem}
\newtheorem{proposition}[theorem]{Proposition}
\newtheorem{corollary}[theorem]{Corollary}
\newtheorem{lemma}[theorem]{Lemma}
\newtheorem*{lemma*}{Lemma}

\theoremstyle{definition}
\newtheorem{definition}[theorem]{Definition}
\newtheorem{exercise}{Exercise}[section]

\theoremstyle{remark}
\newtheorem{example}[theorem]{Example}
\newtheorem{remark}[theorem]{Remark}

\newcommand{\prodb}{\mathbin{\Pi}}
\newcommand{\iso}{\cong}

\newcommand{\cat}[1]{\mathscr{#1}}
\newcommand{\Cat}[1]{\mathbf{#1}}
%\newcommand{\hom}{\mathrm{hom}}
\newcommand{\twocat}[1]{\mathcal{#1}}
\newcommand{\dblcat}[1]{\mathbb{#1}}
\newcommand{\btwo}{\mathbf{2}}
\newcommand{\FF}{\mathbb{F}\Cat{F}}
\newcommand{\FFD}{\FF(\dblcat{D})}
\newcommand{\Mon}{\Cat{Mon}}
\newcommand{\DMon}{\mathbb{M}\Cat{on}}
\newcommand{\Comon}{\Cat{Comon}}
\newcommand{\DComon}{\mathbb{C}\Cat{omon}}
\newcommand{\Bimon}{\Cat{Bimon}}
\newcommand{\Sq}{\mathbb{S}\Cat{q}}
\newcommand{\Span}{\mathbb{S}\Cat{pan}}
\newcommand{\Hor}{\twocat{H}or}
\newcommand{\LAdj}{\dblcat{L}\Cat{Adj}}
\newcommand{\RAdj}{\dblcat{R}\Cat{Adj}}
\newcommand{\MAdjC}{\Cat{MAdj}}
\newcommand{\MAdj}{\dblcat{M}\Cat{Adj}}
\newcommand{\EAdj}{\dblcat{E}\Cat{Adj}}
\newcommand{\SymMonCat}{\Cat{SymMonCat}}
\newcommand{\CompCat}{\Cat{CompCat}}
\newcommand{\Set}{\Cat{Set}}

\newcommand{\op}[1]{{#1}^{\text{op}}}
\newcommand{\vop}[1]{{#1}^{\text{vop}}}
\newcommand{\hop}[1]{{#1}^{\text{hop}}}

\newcommand{\Alg}{\mathrm{Alg}}
\newcommand{\Coalg}{\mathrm{Coalg}}
\newcommand{\RAlg}[1][]{\mathbb{R}_{#1}\text{-}\Alg}
\newcommand{\LCoalg}[1][]{\mathbb{L}_{#1}\text{-}\Coalg}
\newcommand{\LCoalgA}{\mathbb{L}_1\text{-}\Coalg}
\newcommand{\LCoalgB}{\mathbb{L}_2\text{-}\Coalg}

\newcommand{\twocell}[3][]{\arrow[draw=none,to path={(dom#2.center)--(cod#2.center)\tikztonodes}]{}[anchor=center,#1]{\Downarrow #3}}
\newcommand{\twocellalt}[3][]{\arrow[draw=none,to path={(dom#2.center)--(cod#2.center)\tikztonodes}]{}[anchor=center,#1]{#3}}
\newcommand{\twocellA}[2][]{\twocell[#1]{A}{#2}}
\newcommand{\twocellB}[2][]{\twocell[#1]{B}{#2}}
\newcommand{\twocellC}[2][]{\twocell[#1]{C}{#2}}
\newcommand{\twocellD}[2][]{\twocell[#1]{D}{#2}}
\newcommand{\twocellE}[2][]{\twocell[#1]{E}{#2}}
\newcommand{\twocellF}[2][]{\twocell[#1]{F}{#2}}

\tikzcdset{
	arrow style=tikz,
	diagrams={>={Classical TikZ Rightarrow[angle=63:4pt, line width=.6pt]}},
	arrows={semithick}
}

\tikzset{tick/.style={postaction={decorate,decoration={markings,mark=at position 0.5 with {\draw[-] (0,.4ex) -- (0,-.4ex);}}}}}
\tikzset{dom/.style={append after command={coordinate[alias=dom#1]}},
		domA/.style={dom=A}, domB/.style={dom=B},
		domC/.style={dom=C}, domD/.style={dom=D},
		domE/.style={dom=E}, domF/.style={dom=F}}
\tikzset{cod/.style={append after command={coordinate[alias=cod#1]}},
		codA/.style={cod=A}, codB/.style={cod=B},
		codC/.style={cod=C}, codD/.style={cod=D},
		codE/.style={cod=E}, codF/.style={cod=F}}


\tikzset{
	%label/.style={font=\everymath\expandafter{\the\everymath\scriptstyle}},
	wiring diagram/.style={
		every to/.style={out=0,in=180,draw},
		label/.style={
			font=\everymath\expandafter{\the\everymath\scriptstyle},
			inner sep=0pt,
			node distance=1pt and -2pt},
		semithick,
		node distance=1 and 1,
		decoration={markings, mark=at position .5 with {\arrow{stealth};}},
		ar/.style={postaction={decorate}},
		execute at begin picture={\tikzset{
			x=\bbx, y=\bby,
			every fit/.style={inner xsep=\bbx, inner ysep=\bby}}}
		},
	bbx/.store in=\bbx,
	bbx = 2cm,
	bby/.store in=\bby,
	bby = 1.75ex,
	bb port sep/.store in=\bbportsep,
	bb port sep=2,
	% bb wire sep/.store in=\bbwiresep,
	% bb wire sep=1.75ex,
	bb port length/.store in=\bbportlen,
	bb port length=4pt,
	bb min width/.store in=\bbminwidth,
	bb min width=.8cm,
	bb small/.style={bb port sep=1, bb port length=2.5pt, bbx=.4cm, bb min width=.4cm, bby=.7ex},
	bb/.code 2 args={
		\pgfmathsetlengthmacro{\bbheight}{\bbportsep * (max(#1,#2)+1) * \bby}
		\pgfkeysalso{draw,minimum height=\bbheight,minimum width=\bbminwidth,outer sep=0pt,
			rounded corners=2pt,thick,
			prefix after command={\pgfextra{\let\fixname\tikzlastnode}},
			append after command={\pgfextra{\draw
				\ifnum #1=0{} \else foreach \i in {1,...,#1} {
					($(\fixname.north west)!{\i/(#1+1)}!(\fixname.south west)$) +(-\bbportlen,0) coordinate (\fixname_in\i) -- +(\bbportlen,0) coordinate (\fixname_in\i')}\fi
				\ifnum #2=0{} \else foreach \i in {1,...,#2} {
					($(\fixname.north east)!{\i/(#2+1)}!(\fixname.south east)$) +(-\bbportlen,0) coordinate (\fixname_out\i') -- +(\bbportlen,0) coordinate (\fixname_out\i)}\fi;
			}}}
	},
	bb name/.style={append after command={\pgfextra{\node[anchor=north] at (\fixname.north) {#1};}}}
}

\usetikzlibrary{arrows,calc,chains,matrix,positioning,scopes,snakes}

%Begin tikz macros
\def\blackbox#1#2#3#4#5{%(width,height), number inputs, number outputs, label, arrow length
  \pgfgetlastxy{\llx}{\lly}%assumes path has been set to a point representing the lower left corner of the box
  \path #1;
  \pgfgetlastxy{\w}{\h}
  \pgfmathsetlengthmacro{\urx}{\llx+\w}
  \pgfmathsetlengthmacro{\ury}{\lly+\h}
  \draw (\llx,\lly) rectangle (\urx,\ury);
  \pgfmathsetlengthmacro{\xave}{(\llx+\urx)/2}
  \pgfmathsetlengthmacro{\yave}{\ury-8}
  \node at (\xave,\yave) {#4};
  \pgfmathsetlengthmacro{\ydiff}{\ury-\lly}
  \pgfmathsetlengthmacro{\lstep}{\ydiff/(#2+1)}
  \pgfmathsetlengthmacro{\rstep}{\ydiff/(#3+1)}
  \ifnum #2=0{}\else{ 
   \foreach \l in {1,...,#2}{
    \draw [->] ($(\llx,\lly)+(-#5/2,0)+\l*(0,\lstep)$) -- ($(\llx,\lly)+(#5/2,0)+\l*(0,\lstep)$);}}\fi
  \ifnum #3=0{}\else{
   \foreach \r in {1,...,#3}{
    \draw [->] ($(\urx,\ury)+(-#5/2,0)-\r*(0,\rstep)$) -- ($(\urx,\ury)+(#5/2,0)-\r*(0,\rstep)$);}}\fi
}

\def\blackboxinners#1#2#3#4#5{%(width,height), number inputs, number outputs, label, arrow length
  \pgfgetlastxy{\llx}{\lly}%assumes path has been set to a point representing the lower left corner of the box
  \path #1;
  \pgfgetlastxy{\w}{\h}
  \pgfmathsetlengthmacro{\urx}{\llx+\w}
  \pgfmathsetlengthmacro{\ury}{\lly+\h}
  \draw (\llx,\lly) rectangle (\urx,\ury);
  \pgfmathsetlengthmacro{\xave}{(\llx+\urx)/2}
  \pgfmathsetlengthmacro{\yave}{\ury-8}
  \node at (\xave,\yave) {#4};
  \pgfmathsetlengthmacro{\ydiff}{\ury-\lly}
  \pgfmathsetlengthmacro{\lstep}{\ydiff/(#2+1)}
  \pgfmathsetlengthmacro{\rstep}{\ydiff/(#3+1)}
  \ifnum #2=0{}\else{ 
   \foreach \l in {1,...,#2}{
    \pgfmathsetlengthmacro{\newx}{\llx+#5*28.45274/2}
    \pgfmathsetlengthmacro{\newy}{\lly+\l*\lstep}
    \node at ($(\newx,\newy)+(-1.5,\l*12-\l*\lstep)$) {\tiny$(\pgfmathparse{\newx/28.45274}\pgfmathresult cm,\pgfmathparse{\newy/28.45274}\pgfmathresult cm)$};
    \draw [->] ($(\llx,\lly)+(-#5/2,0)+\l*(0,\lstep)$) -- ($(\llx,\lly)+(#5/2,0)+\l*(0,\lstep)$);}}\fi
  \ifnum #3=0{}\else{
   \foreach \r in {1,...,#3}{
    \pgfmathsetlengthmacro{\newx}{\urx-#5*28.45274/2}
    \pgfmathsetlengthmacro{\newy}{\ury-\r*\rstep}
    \node at ($(\newx,\newy)+(1.5,-\r*12+\r*\rstep)$) {\tiny $(\pgfmathparse{\newx/28.45274}\pgfmathresult cm,\pgfmathparse{\newy/28.45274}\pgfmathresult cm)$};
    \draw [->] ($(\urx,\ury)+(-#5/2,0)-\r*(0,\rstep)$) -- ($(\urx,\ury)+(#5/2,0)-\r*(0,\rstep)$);}}\fi
}

\def\blackboxouters#1#2#3#4#5{%(width,height), number inputs, number outputs, label, arrow length
  \pgfgetlastxy{\llx}{\lly}%assumes path has been set to a point representing the lower left corner of the box
  \path #1;
  \pgfgetlastxy{\w}{\h}
  \pgfmathsetlengthmacro{\urx}{\llx+\w}
  \pgfmathsetlengthmacro{\ury}{\lly+\h}
  \draw (\llx,\lly) rectangle (\urx,\ury);
  \pgfmathsetlengthmacro{\xave}{(\llx+\urx)/2}
  \pgfmathsetlengthmacro{\yave}{\ury-8}
  \node at (\xave,\yave) {#4};
  \pgfmathsetlengthmacro{\ydiff}{\ury-\lly}
  \pgfmathsetlengthmacro{\lstep}{\ydiff/(#2+1)}
  \pgfmathsetlengthmacro{\rstep}{\ydiff/(#3+1)}
  \ifnum #2=0{}\else{ 
   \foreach \l in {1,...,#2}{
    \pgfmathsetlengthmacro{\newx}{\llx-#5*28.45274/2}
    \pgfmathsetlengthmacro{\newy}{\lly+\l*\lstep}
    \node at ($(\newx,\newy)+(-1.5,\l*12-\l*\lstep)$) {\tiny$(\pgfmathparse{\newx/28.45274}\pgfmathresult cm,\pgfmathparse{\newy/28.45274}\pgfmathresult cm)$};
    \draw [->] ($(\llx,\lly)+(-#5/2,0)+\l*(0,\lstep)$) -- ($(\llx,\lly)+(#5/2,0)+\l*(0,\lstep)$);}}\fi
  \ifnum #3=0{}\else{
   \foreach \r in {1,...,#3}{
    \pgfmathsetlengthmacro{\newx}{\urx+#5*28.45274/2}
    \pgfmathsetlengthmacro{\newy}{\ury-\r*\rstep}
    \node at ($(\newx,\newy)+(1.5,-\r*12+\r*\rstep)$) {\tiny $(\pgfmathparse{\newx/28.45274}\pgfmathresult cm,\pgfmathparse{\newy/28.45274}\pgfmathresult cm)$};
    \draw [->] ($(\urx,\ury)+(-#5/2,0)-\r*(0,\rstep)$) -- ($(\urx,\ury)+(#5/2,0)-\r*(0,\rstep)$);}}\fi
}


\def\dashblackbox#1#2#3#4#5{%(width,height), number inputs, number outputs, label, arrow length
  \pgfgetlastxy{\llx}{\lly}%assumes path has been set to a point representing the lower left corner of the box
  \path #1;
  \pgfgetlastxy{\w}{\h}
  \pgfmathsetlengthmacro{\urx}{\llx+\w}
  \pgfmathsetlengthmacro{\ury}{\lly+\h}
  \draw [dashed] (\llx,\lly) rectangle (\urx,\ury);
  \pgfmathsetlengthmacro{\xave}{(\llx+\urx)/2}
  \pgfmathsetlengthmacro{\yave}{\ury-8}
  \node at (\xave,\yave) {#4};
  \pgfmathsetlengthmacro{\ydiff}{\ury-\lly}
  \pgfmathsetlengthmacro{\lstep}{\ydiff/(#2+1)}
  \pgfmathsetlengthmacro{\rstep}{\ydiff/(#3+1)}
  \ifnum #2=0{}\else{ 
   \foreach \l in {1,...,#2}{
    \draw [->] ($(\llx,\lly)+(-#5/2,0)+\l*(0,\lstep)$) -- ($(\llx,\lly)+(#5/2,0)+\l*(0,\lstep)$);}}\fi
  \ifnum #3=0{}\else{
   \foreach \r in {1,...,#3}{
    \draw [->] ($(\urx,\ury)+(-#5/2,0)-\r*(0,\rstep)$) -- ($(\urx,\ury)+(#5/2,0)-\r*(0,\rstep)$);}}\fi
}

\def\directarc#1#2{%left coordinate, right coordinate
  \path #1;
  \pgfgetlastxy{\lx}{\ly}
  \path #2;
  \pgfgetlastxy{\rx}{\ry}
  \pgfmathsetlengthmacro{\xave}{(\lx+\rx)/2}
  \draw #1 .. controls (\xave,\ly) and (\xave,\ry) .. #2;
}

\def\loopright#1#2#3{%upper coordinate, lower coordinate, stretch width
  \path #1;
  \pgfgetlastxy{\ux}{\uy}
  \path #2;
  \pgfgetlastxy{\lx}{\ly}
  \pgfmathsetlengthmacro{\maxx}{max(\ux,\lx)}
  \pgfmathsetlengthmacro{\farx}{\maxx+#3}
  \draw #1 .. controls (\farx,\uy) and (\farx,\ly) .. #2;
}

\def\loopleft#1#2#3{%upper coordinate, lower coordinate, stretch width
  \path #1;
  \pgfgetlastxy{\ux}{\uy}
  \path #2;
  \pgfgetlastxy{\lx}{\ly}
  \pgfmathsetlengthmacro{\minx}{min(\ux,\lx)}
  \pgfmathsetlengthmacro{\farx}{\minx-#3}
  \draw #1 .. controls (\farx,\uy) and (\farx,\ly) .. #2;
}

\def\fancyarc#1#2#3#4{%upper coordinate, lower coordinate, stretch width, max height adjust
  \path #1;
  \pgfgetlastxy{\ux}{\uy}
  \path #2;
  \pgfgetlastxy{\lx}{\ly}
  \pgfmathsetlengthmacro{\xave}{(\lx+\ux)/2}
%  \node at (\lx,\ly+20){\tiny $\pgfmathparse{\lx/28.45274}\pgfmathresult cm$,\hsp$\pgfmathparse{\ux/28.45274}\pgfmathresult cm$};
%  \node at (\xave,\ly+50){\tiny $\pgfmathparse{\xave/28.45274}\pgfmathresult cm$};
  \pgfmathsetlengthmacro{\yave}{(\ly+\uy)/2+#4}
  \loopleft{#1}{(\xave,\yave)}{#3}
  \loopright{#2}{(\xave,\yave)}{#3}
}

\def\delaynode#1{%coordinates
   \filldraw[black] #1 circle (2pt);
}

\def\activetikz#1{$$\begin{tikzpicture}#1\end{tikzpicture}$$}
\def\inactivetikz#1{\begin{center}\fbox{Tikz picture exists but is not being displayed}\end{center}}
%End tikz macros

\newcommand{\vinp}[1]{\overline{\inp{#1}}}
\newcommand{\voutp}[1]{\overline{\outp{#1}}}
%\newcommand{\inp}[1]{#1^{\textnormal{in}}}
%\newcommand{\outp}[1]{#1^{\textnormal{out}}}
\newcommand{\inp}[1]{#1^-}
\newcommand{\outp}[1]{#1^+}

% \def\bhline{\Xhline{2\arrayrulewidth}}
% \def\bbhline{\Xhline{2.5\arrayrulewidth}}
\def\alg{{\text \textendash}\Cat{Alg}}
\def\To{\xrightarrow}
\def\ul{\underline}
\def\List{\textnormal{List}}

\newcommand{\erase}[1]{{}}
\def\NN{\mathbb{N}}
\def\ss{\subseteq}


\settrims{0pt}{0pt} % page and stock same size
\setlxvchars %calculate line length such that there are about 65 characters per line in \normalfont
\settypeblocksize{*}{36pc}{*} % {height}{width}{ratio}
\setlrmargins{*}{*}{1} % {spine}{edge}{ratio}
%\setulmargins{*}{*}{1} % {upper}{lower}{ratio}, hight of typeblock fixed
\setulmarginsandblock{1in}{1in}{*} % hight of typeblock computed
\setheadfoot{\onelineskip}{2\onelineskip} % {headheight}{footskip}
\setheaderspaces{*}{1.5\onelineskip}{*} % {headdrop}{headsep}{ratio}
\checkandfixthelayout


\setcounter{tocdepth}{2}
\setcounter{secnumdepth}{2}
%\chapterstyle{dash}
\pagestyle{companion}

\title{Enriched Traced Monoidal Categories are Lax Functors out of Compact Categories}
\author{
Dylan Rupel 
 \and 
David I. Spivak\thanks{Spivak and Schultz were supported by AFOSR grant FA9550-14-1-0031, ONR grant N000141310260, and NASA grant NNL14AA05C.}
 \and 
 Patrick Schultz${}^*$%\footnotemark[1]
 }



\begin{document}
\tightlists
\firmlists

\maketitle
\begin{abstract}
We give an alternate conception of string diagrams as 1-dimensional oriented cobordisms. The operad $\Cat{1-Cob}$ of cobordisms encodes the axioms of traced categories, in the sense that there is an equivalence of categories between $\Cat{1-Cob}$-algebras and enriched traced categories. We also prove a substantial generalization of this fact, which characterizes lax functors out of compact or traced categories.
\end{abstract}
To Do:
\begin{enumerate}
\item Put global todo's here.
\end{enumerate}
\tableofcontents
\chapter{Introduction}

\section{Overview of string diagrams in the literature. 
TSMC's and Penrose diagrams for tensor calculus.
}
Traced symmetric monoidal categories are often used to model processes with feedback or operators with fixed points (Ponto, Shulman). A graphical calculus for TSMCs was developed by Joyal, Street, and Verity, in which string diagrams
\begin{center}Put in diagram here\end{center}
represent compositions, i.e., new morphisms from old. In fact, these generalize Penrose diagrams in $\Cat{Vect}$, and the word \emph{traced} originates in vector space terminology.  

\section{Dual interpretation of wiring diagrams as pictures of objects and morphisms in the category $\Cat{Cob}$ of oriented $0$-manifolds and as diagrams for computing with morphisms, compositions, and traces in a TSMC.  Labeled wires.
}

\section{Table of equivalences, shifted data}

\section{Nesting properties and self-similarity in applications}

\section{First main theorem}

\begin{theorem}
 The category $\Cat{TSMC}$ is equivalent to the category of $\Cat{Cob}/\cat{O}$-algebras.
\end{theorem}

\section{Generalization}

\begin{theorem}
 Let $\cat{C}$ be a compact closed category and $\cat{V}$ a symmetric monodical category.  The category $\Lax(\cat{C},\cat{V})$ of lax monodical functors is equivalent to the coslice category $\cat{C}/\cat{V}-\Cat{CompCat}$ spanned by bijective on objects functors.
\end{theorem}
\begin{corollary}
 $\Lax(\Int(\cat{T}),\cat{V})=\cat{V}-\Cat{TSMC}_{\cat{T}/}$
\end{corollary}

\chapter{Wiring Diagrams and $1-\Cat{Cob}$}
\section{Set-theoretic formulation of $1-\Cat{Cob}$, as free compact closed category on one object, as $\Int$ of the free TSMC on one object ($\Cat{Bij}$)}
\subsection{Many object case/generalization $\Cat{Cob}/\cat{O}$}

\section{Drawings of morphisms in $1-\Cat{Cob}$ as wiring diagrams, new way to visualize these}

\section{$1-\Cat{Cob}$-algebras and applications}

\section{Definition of functors between $\Cat{TSMC}$ and $\Cat{Cob}/\cat{O}$-algebras}

Let $\cat{M}$ be a traced symmetric monoidal category with objects $\cat{O}$. We will define a $\Cat{Cob}/\cat{O}$-algebra $\cat{P}=R(\cat{M})\colon\Cat{Cob}/\cat{O}\to\Cat{Set}$ as follows. For an object $X\in\Ob(\Cat{Cob}/\cat{O})$, set 
$$\cat{P}(X):=\Hom_{\cat{M}}(\vinp{X},\voutp{X}).$$
We next consider morphisms.

Following Proposition~\ref{prop:set theoretic cob1} (borrow notation/setup from Abramsky instead) a morphism $\Phi\colon X\longrightarrow Y$ consists of a typed bijection 
$$\varphi\colon\inp{X}\sqcup \outp{Y}\xrightarrow{\iso}\outp{X}\sqcup \inp{Y},$$ 
together with a typed finite set $S$. Given an element $f\in\cat{P}(X)$ we must construct $\cat{P}(\Phi)(f)\in\cat{P}(Y)$. Let $\dim(\overline{S})=\textnormal{Tr}^{\overline{S}}_{I,I}\big[\id_{\overline{S}}\big]\in\cat{S}_\cat{M}$. Then we use the formula
$$\cat{P}(\Phi)(f):=
\textnormal{Tr}^{\overline{X_+}}_{\overline{Y_-},\overline{Y_+}}\Big[\big(f\otimes\id_{\overline{Y_+}}\big)\circ\overline{\varphi}\Big]
\otimes\dim(\overline{S}).	
$$

\begin{theorem}
 The category $\Cat{TSMC}$ is equivalent to the category of $\Cat{Cob}/\cat{O}$-algebras.
\end{theorem}
\begin{proof}
 
\end{proof}
\begin{corollary}
 Enriched setting?
\end{corollary}



% -*- root: CCC_Note.tex -*-
\chapter{Preliminaries}

Let $\cat{C}$ and $\cat{D}$ be monoidal categories. Recall that a functor $F\colon\cat{C}\to\cat{D}$ is called \emph{lax monoidal} if it is equipped with a morphism
\[
\begin{tikzcd}
	I_D \rar{\epsilon} & F(I_C)
\end{tikzcd}
\]
and a natural transformation
\[
\begin{tikzcd}
	F(X) \otimes_D F(Y) \rar{\mu_{X,Y}} & F(X\otimes_C Y)
\end{tikzcd}
\]
such that for all $X,Y,Z\in\cat{C}$, the diagram (suppressing associators)
\[
\begin{tikzcd}
	F(X)\otimes F(Y) \otimes F(Z)
		\rar{\id\otimes\mu}
		\dar[swap]{\mu\otimes\id}
	& F(X)\otimes F(Y\otimes Z)
		\dar{\mu} \\
	F(X\otimes Y)\otimes F(Z)
		\rar[swap]{\mu}
	& F(X\otimes Y\otimes Z)
\end{tikzcd}
\]
commutes, and for all $X\in\cat{C}$ the two diagrams
\[
\begin{tikzcd}
	I_D\otimes F(X)
		\dar[swap]{\epsilon\otimes\id}
	& F(X)
		\lar[swap]{l_{F(X)}}
		\dar{F(l_X)} \\
	F(I_C)\otimes F(X)
		\rar[swap]{\mu}
	& F(I_C\otimes X)
\end{tikzcd}
\qquad
\begin{tikzcd}
	F(X) \otimes I_D
		\dar[swap]{\id\otimes\epsilon}
	& F(X)
		\lar[swap]{r_{F(X)}}
		\dar{F(r_X)} \\
	F(X)\otimes F(I_C)
		\rar[swap]{\mu}
	& F(X\otimes I_C)
\end{tikzcd}
\]
commute. If $\epsilon$ and $\mu$ are isomorphisms, then $F$ is \emph{strong}.

If $\cat{C}$ and $\cat{D}$ are symmetric monoidal, then $F$ is a \emph{lax symmetric monoidal functor} if it is lax monoidal, and commutes with the symmetries, in the sense that the diagram
\[
\begin{tikzcd}
	F(X)\otimes F(Y)
		\rar{\sigma}
		\dar[swap]{\mu}
	& F(Y)\otimes F(X)
		\dar{\mu} \\
	F(X\otimes Y)
		\rar[swap]{F(\sigma)}
	& F(Y\otimes X)
\end{tikzcd}
\]
commutes.

If $F$ and $G$ are lax monoidal functors (possibly symmetric), then a natural transformation $\alpha\colon F\to G$ is called a \emph{monoidal transformation} if the diagrams
\[
\begin{tikzcd}
	F(X)\otimes F(Y)
		\rar{\alpha_X\otimes\alpha_Y}
		\dar[swap]{\mu}
	& G(X)\otimes G(Y)
		\dar{\mu} \\
	F(X\otimes Y)
		\rar[swap]{\alpha_{X\otimes Y}}
	& G(X\otimes Y)
\end{tikzcd}
\qquad
\begin{tikzcd}[column sep=tiny]
	{} & I_D \dlar[swap]{\epsilon} \drar{\epsilon} & \\
	F(I_C) \ar{rr}[swap]{\alpha_I} && G(I_C)
\end{tikzcd}
\]
commute.

Let $\SymMonCat$ denote the bicategory of symmetric monoidal categories and strong monoidal functors, and let $\Lax(\cat{C},\cat{D})$ denote the category of lax monoidal functors and monoidal transformations from $\cat{C}$ to $\cat{D}$. Let $\CompCat$ denote the full subcategory of $\SymMonCat$ spanned by the compact categories.

% \begin{theorem}
% 	Let $\cat{C}$ be a compact category. There is an equivalence of categories
% 	\[
% 		\Lax(\cat{C},\Set) \simeq (\cat{C}\backslash\CompCat)_{\text{boo}}
% 	\]
% 	between the lax functor category from $\cat{C}$ to $\Set$ equipped with the cartesian monoidal structure, and the full subcategory of the undercategory $\cat{C}\backslash\CompCat$ spanned by the bijective-on-objects functors.
% \end{theorem}
% \begin{proof}
% 	Fix a lax symmetric functor $F\colon\cat{C}\to\Set$. We can construct a compact category $\hat{F}$ and a strong bijective-on-objects functor $\tilde{F}\colon\cat{C}\to\hat{F}$ as follows:
% 	\begin{compactitem}
% 		\item The objects of $\hat{F}$ are the objects of $\cat{C}$.
% 		\item $\Hom_{\hat{F}}(A,B)=F(A^{\star}\otimes B)$.
% 		\item Composition $\Hom(A,B)\times\Hom(B,C)\to\Hom(A,C)$ is defined by
% 		\[
% 		\begin{tikzcd}[column sep=-2ex]
% 			{} & F(A^{\star}\otimes B\otimes B^{\star}\otimes C)
% 				\drar{F(\id\otimes\epsilon_B\otimes\id)} & \\
% 			F(A^{\star}\otimes B)\times F(B^{\star}\otimes C)
% 				\urar{\mu_F}
% 			&& F(A^{\star}\otimes B)
% 		\end{tikzcd}
% 		\]
% 		\item Identities $1\to\Hom(A,A)$ are defined by
% 		\[
% 		\begin{tikzcd}
% 			1 \rar{\epsilon_F} & F(I) \rar{F(\eta_A)} & F(A^{\star}\otimes A).
% 		\end{tikzcd}
% 		\]
% 		\item The tensor product
% 		\[
% 		\begin{tikzcd}
% 			\Hom(A,A')\times\Hom(B,B') \rar{\otimes}
% 			& \Hom(A\otimes B,A'\otimes B')
% 		\end{tikzcd}
% 		\]
% 		is defined by
% 		\[
% 		\begin{tikzcd}[column sep=-3ex]
% 			{} & F(A^{\star}\otimes A'\otimes B^{\star}\otimes B')
% 				\drar{F(\id\otimes\sigma\otimes\id)} & \\
% 			F(A^{\star}\otimes A')\times F(B^{\star}\otimes B')
% 				\urar{\mu_F}
% 			&& F(A^{\star}\otimes B^{\star}\otimes A'\otimes B')
% 		\end{tikzcd}
% 		\]
% 		\item $\tilde{F}$ is identity on objects, and for any $f\colon A\to B$ in $\cat{C}$, define $\tilde{F}(f)$ by
% 		\[
% 		\begin{tikzcd}
% 			1 \rar{\epsilon_F}
% 			& F(I) \rar{F(\eta_A)}
% 			& F(A^{\star}\otimes A) \rar{F(\id\otimes f)}
% 			& F(A^{\star}\otimes B)
% 		\end{tikzcd}
% 		\]
% 		\item The associator and symmetry isomorphisms of $\hat{F}$ are given by the image under $\tilde{F}$ of those in $\cat{C}$.
% 	\end{compactitem}

% 	In the other direction, suppose we are given a compact category $\hat{F}$ and a strong bijective-on-objects functor $\tilde{F}\colon\cat{C}\to\hat{F}$. Define $F$ by
% 	\begin{compactitem}
% 		\item $F(A)=\Hom_{\hat{F}}(I,A)$.
% 		\item For $f\colon A\to B$ in $\cat{C}$, define $F(f)\colon\Hom(I,A)\to\Hom(I,B)$ by post-composition with $\tilde{F}(f)$.
% 		\item $\epsilon\colon 1\to F(I)$ is defined by $\id_I\in\Hom_{\hat{F}}(I,I)$.
% 		\item $\mu\colon F(A)\times F(B)\to F(A\otimes B)$ is defined by
% 		\[
% 		\begin{tikzcd}[column sep=-3ex]
% 			{} & \Hom(I\otimes I,A\otimes B)
% 				\drar & \\
% 			\Hom(I,A)\times\Hom(I,B)
% 				\urar{\otimes_{\hat{F}}}
% 			&& \Hom(I,A\otimes B)
% 		\end{tikzcd}
% 		\]
% 	\end{compactitem}
% \end{proof}

\chapter{Profunctors}

Let $\cat{C}$ and $\cat{D}$ be categories. Recall that a profunctor $M$ from $\cat{C}$ to $\cat{D}$, written
\[
\begin{tikzcd}
	\cat{C} \ar[r,tick,"M"] & \cat{D},
\end{tikzcd}
\]
is defined to be a functor $M\colon\op{\cat{C}}\times\cat{D}\to\Set$. We can think of a profunctor as a sort of graded bimodule: for each object $c\in\cat{C}$ and $d\in\cat{D}$ there is a set $M(c,d)$ of elements in the bimodule, and given an element $m\in M(c,d)$ and morphisms $f\colon c'\to c$ in $\cat{C}$ and $g\colon d\to d'$ in $\cat{D}$, there are elements $g\cdot m\in M(c,d')$ and $m\cdot f\in F(c',d)$, such that $(g\cdot m)\cdot f=g\cdot(m\cdot f)$, and $g'\cdot(g\cdot m)=(g'\circ g)\cdot m$ and $(m\cdot f)\cdot f'=m\cdot(f\circ f')$ whenever they make sense.

If $F\colon\cat{C}'\to\cat{C}$ and $G\colon\cat{D}'\to\cat{D}$ are functors, and $M$ is a profunctor as before, then there is a profunctor $M(F,G)$ from $\cat{C}'$ to $\cat{D}'$, defined to be the composite
\[
\begin{tikzcd}
	\op{\cat{C}'}\times\cat{D}' \ar[r,"\op{F}\times G"]
		&[1.5em] \op{\cat{C}}\times\cat{D} \ar[r,"M"]
		& \Set.
\end{tikzcd}
\]
In other words, for any objects $c\in\Cat{C}'$ and $d\in\Cat{D}'$, the profunctor $M(F,G)$ has elements $M(Fc,Gd)$, and if $m\in M(Fc,Gd)$ and $g\colon d\to d'$ is a morphism in $\cat{D}'$, then the element $m\cdot g$ in $M(F,G)$ is defined by the element $m\cdot G(g)$ in $M$, and similarly for the $\cat{C}'$ action.

Given two profunctors
\[
\begin{tikzcd}
	\cat{C} \ar[r,tick,shift left,"M"] \ar[r,tick,shift right,"N"'] & \cat{D}
\end{tikzcd}
\]
define a profunctor morphism $\phi\colon M\Rightarrow N$ to be a natural transformation. In other words, for each $c\in\cat{C}$ and $d\in\cat{D}$ there is a function $\phi_{c,d}\colon M(c,d)\to N(c,d)$ such that $\phi(f\cdot m \cdot g)=f\cdot\phi(m)\cdot g$ whenever it makes sense.

There is a tensor product of profunctors: given two profunctors
\[
\begin{tikzcd}
	\cat{C} \ar[r,tick,"M"] & \cat{D} \ar[r,tick,"N"] & \cat{E}
\end{tikzcd}
\]
define the profunctor $M\otimes N$ such that for objects $c\in\cat{C}$ and $e\in\cat{E}$, $(M\otimes N)(c,e)$ is the coequalizer of the diagram
\[
\begin{tikzcd}
	\displaystyle\coprod_{d_1,d_2\in\cat{D}} M(c,d_1)\times\cat{D}(d_1,d_2)\times N(d_2,e)
		\ar[r,shift left] \ar[r,shift right]
	& \displaystyle\coprod_{d\in\cat{D}} M(c,d)\times N(d,e)
\end{tikzcd}
\]
where the two maps are given by the right action of $\cat{D}$ on $M$ and by the left action of $\cat{D}$ on $N$. We can write elements of $(M\otimes N)(c,e)$ as tensors $m\otimes n$, where $m\in M(c,d)$ and $n\in N(d,e)$ for some $d\in\cat{D}$. The coequalizer then implies that $(m\cdot f)\otimes n=m\otimes(f\cdot n)$ whenever the equation makes sense.

For any category $\cat{C}$, there is a profunctor $\Hom_{\cat{C}}\colon\op{\cat{C}}\times\cat{C}\to\Set$, and these hom profunctors act as units for the tensor product. Precisely, if $M$ is as above, there are canonical isomorphisms $\Hom_{\cat{C}}\otimes M \iso M \iso M\otimes\Hom_{\cat{D}}$.

Given a category $\cat{C}$, there is a monoidal category $\Prof(\cat{C},\cat{C})$ of profunctors from $\cat{C}$ to itself and morphisms of profunctors, with the tensor product given above and $\Hom_{\cat{C}}$ as the monoidal unit. We would now like to investigate monoids in this monoidal category.

Suppose $M\in\Prof(\cat{C},\cat{C})$ has a monoid structure. The unit is a profunctor morphism $i\colon\Hom_{\cat{C}}\to M$. So for any $f\colon c\to d$ in $\cat{C}$ there is an element $i(f)\in M(c,d)$, such that $f\cdot i(g)\cdot h = i(f\circ g\circ h)$ whenever this makes sense. The multiplication $M\otimes M\to M$ is an operation assigning to any elements $m_1\in M(c,d)$ and $m_2\in M(d,e)$ an element $m_2\bullet m_1\in M(c,e)$, which is associative, and satisfies the following equations whenever they make sense:
\begin{gather*}
	(f\cdot m_2)\bullet(m_1\cdot h) = f\cdot(m_2\bullet m_1)\cdot h \\
	(m_3\cdot g)\bullet m_1 = m_3\bullet(g\cdot m_1) \\
	m\bullet i(f) = m\cdot f \quad\text{and}\quad i(g)\bullet m = g\cdot m
\end{gather*}

\begin{lemma}
	There is an equivalence of categories $\Mon(\Prof(\cat{C},\cat{C}))\iso (\cat{C}/\Cat{Cat})_{\text{b.o.o.}}$ between the category of monoids in $\Prof(\cat{C},\cat{C})$ and the full subcategory of the coslice category $\cat{C}/\Cat{Cat}$ spanned by the bijective-on-objects functors.
\end{lemma}
\begin{proof}
	Simple to check. The unit provides the identities and the functor from $\cat{C}$, while the multiplication provides the composition.
\end{proof}

Now suppose $\cat{C}$ and $\cat{D}$ are symmetric monoidal categories. We will write
\begin{gather*}
	a_{c,d,e}\colon (c\otimes d)\otimes e \to c\otimes(d\otimes e), \\
		\lambda_c\colon I\otimes c\to c,
		\qquad \rho_c\colon c\otimes I \to c, \\
		\sigma_{c,d}\colon c\otimes d\to d\otimes c
\end{gather*}
for the associator, left and right unitor, and symmetry isomorphisms, respectively, leaving it to context to make clear whether we are in $\cat{C}$ or $\cat{D}$.

A \emph{monoidal profunctor} $M$ from $\cat{C}$ to $\cat{D}$ is an ordinary profunctor such that the functor $M\colon \op{\cat{C}}\times\cat{D}\to\Set$ is equipped with a lax-monoidal structure, with the cartesian monoidal structure on $\Set$. In the bimodule notation, this means that there is an associative operation assigning to any elements $m_1\in M(c_1,c'_1)$ and $m_2\in M(c_2,c'_2)$ an element $m_1\boxtimes m_2\in M(c_1\otimes c_2,c'_1\otimes c'_2)$ such that
\[
	(f_1\cdot m_1\cdot g_1)\boxtimes(f_2\cdot m_2\cdot g_2) = (f_1\otimes f_2)\cdot(m_1\boxtimes m_2)\cdot(g_1\otimes g_2),
\]
as well as a distinguished element $I_M\in M(I,I)$ such that $\lambda_d\cdot(I_M\boxtimes m)\cdot\lambda^{-1}_c = m = \rho_d\cdot(m\boxtimes I_M)\cdot\rho^{-1}_c$ for any $m\in M(c,d)$. If moreover $m_2\boxtimes m_1 = \sigma_{c'_1,c'_2}\cdot(m_1\boxtimes m_2)\cdot\sigma_{c_1,c_2}^{-1}$, we say $M$ is \emph{symmetric monoidal}.

A monoidal profunctor morphism $\phi\colon M\to N$ is simply a monoidal transformation. Spelling this out in bimodule notation, $\phi$ is an ordinary morphism of profunctors such that $\phi(m_1\boxtimes m_2)=\phi(m_1)\boxtimes\phi(m_2)$ and $\phi(I_M)=I_N$. We will denote the category of monoidal profunctors from $\cat{C}$ to $\cat{D}$ and monoidal profunctor morphisms as $\MProf(\cat{C},\cat{D})$.

A unit for a monoidal profunctor $M\in\MProf(\cat{C},\cat{C})$ is a unit $i\colon\Hom_{\cat{C}}\to M$ in $\Prof(\cat{C},\cat{C})$ such that, additionally, $i(\id_{I_{\cat{C}}})=I_M$ and $i(f\otimes g)=i(f)\boxtimes i(g)$ for any morphisms $f$ and $g$ in $\cat{C}$. Similarly, a multiplication on $M$ is as above, with the additional conditions
\begin{gather*}
	I_M\bullet I_M=I_M \\
	(m_1\boxtimes m'_1)\bullet(m_2\boxtimes m'_2) = (m_1\bullet m_2)\boxtimes(m'_1\bullet m'_2)
\end{gather*}
for any $m_1\in M(c,d)$, $m'_1\in M(c',d')$, $m_2\in M(d,e)$, and $m'_2\in M(d'e')$.

\begin{lemma}
	Let $\cat{C}$ be a monoidal category. There is an equivalence of categories $\Mon(\MProf(\cat{C},\cat{C}))\iso (\cat{C}/\Cat{MonCat})_{\text{b.o.o.}}$ between the category of monoids in $\MProf(\cat{C},\cat{C})$ and the full subcategory of the coslice category $\cat{C}/\Cat{MonCat}$ spanned by the bijective-on-objects functors.
\end{lemma}

\chapter{Compact closed categories}

Let $\cat{C}$ be a compact closed category.

\begin{proposition}
	There are functors
	\[
	\begin{tikzcd}
		\MProf(1,\cat{C}) \ar[r,shift left,"F"]
		& \MProf(\cat{C},\cat{C}) \ar[l,shift left,"U"]
	\end{tikzcd}
	\]
\end{proposition}
\begin{proof}
	For any $M\colon\cat{C}\to\Set$, define $FM\colon\op{\cat{C}}\times\cat{C}\to\Set$ by $FM(A,B)=M(A^*\otimes B)$. In the other direction, for $N\colon\op{\cat{C}}\times\cat{C}\to\Set$, define $UN(A)=N(1,A)$.
\end{proof}

\begin{proposition}
	Let $N\in\MProf(\cat{C},\cat{C})$ be a monoidal profunctor equipped with a unit $\eta\colon\Hom_{\cat{C}}\to N$. Then $N$ has a canonical multiplication $\mu\colon N\otimes N\to N$ making $N$ a monoid in $\MProf(\cat{C},\cat{C})$.
\end{proposition}
\begin{proof}
	We can define a multiplication on $N$ by the following formula: given any $n_1\in N(c,d)$ and $n_2\in N(d,e)$,
	\[
		n_2\bullet n_1 = \bigl(\lambda_e\circ(\epsilon_d\otimes\id_e)\bigr)\cdot(n_1\boxtimes i(\id_{d^*})\boxtimes n_2)\cdot\bigl((\id_c\otimes \eta_d)\circ\rho_c^{-1}\bigr).
	\]
	We first check the equation $n\bullet i(f)=n\cdot f$ for any $n\in N(d,e)$ and $f\colon c\to d$:
	\begin{align*}
		n\bullet i(f) &= \bigl(\lambda_e\circ(\epsilon_d\otimes\id_e)\bigr)\cdot(i(f)\boxtimes i(\id_{d^*})\boxtimes n)\cdot\bigl((\id_c\otimes \eta_d)\circ\rho_c^{-1}\bigr) \\
		&= \bigl(\lambda_e\circ(\epsilon_d\otimes\id_e)\bigr)\cdot(i(f\otimes\id_{d^*})\boxtimes n)\cdot\bigl((\id_c\otimes \eta_d)\circ\rho_c^{-1}\bigr) \\
		&= \lambda_e\cdot\bigl((\epsilon_d\cdot i(f\otimes \id_{d^*}))\boxtimes (\id_e\cdot n)\bigr)\cdot\bigl((\id_c\otimes \eta_d)\circ\rho_c^{-1}\bigr) \\
		&= \lambda_e\cdot\bigl(i(\epsilon_d\circ (f\otimes \id_{d^*}))\boxtimes (n\cdot\id_d)\bigr)\cdot\bigl((\id_c\otimes \eta_d)\circ\rho_c^{-1}\bigr) \\
		&= \lambda_e\cdot\bigl(i(\id_I)\boxtimes n\bigr)\cdot\bigl(((\epsilon_d\circ (f\otimes \id_{d^*}))\otimes\id_d)\circ(\id_c\otimes \eta_d)\circ\rho_c^{-1}\bigr) \\
		&= \lambda_e\cdot\bigl(I_N\boxtimes n\bigr)\cdot\bigl((\epsilon_d\otimes\id_d)\circ(\id_d\otimes\eta_d)\circ(f\otimes\id_I)\circ\rho_c^{-1}\bigr) \\
		&= \lambda_e\cdot\bigl(I_N\boxtimes n\bigr)\cdot\bigl(\lambda_d^{-1}\circ\rho_d\circ(f\otimes\id_I)\circ\rho_c^{-1}\bigr) \\
		&= \bigl(\lambda_e\cdot(I_N\boxtimes n)\cdot\lambda_d^{-1}\bigr)\cdot\bigl(\rho_d\circ(f\otimes\id_I)\circ\rho_c^{-1}\bigr) \\
		&= n\cdot f.
	\end{align*}

	The equation $i(f)\bullet n=f\cdot n$ follows similarly, and the associativity of $\bullet$ is a straightforward verification.

	Finally we check the equation $(n_2\boxtimes n'_2)\bullet(n_1\boxtimes n'_1)=(n_2\bullet n_1)\boxtimes(n'_2\bullet n'_1)$ for any $n_1\in N(c,d)$, $n'_1\in N(c',d')$, $n_2\in N(d,e)$, and $n'_2\in N(d',e')$, after which the remaining equations follow directly.

	\begin{align*}
		&(n_2\boxtimes n'_2)\bullet(n_1\boxtimes n'_1) \\
		&= \bigl(\lambda_{e\otimes e'}\circ(\epsilon_{d\otimes d'}\otimes\id_{e\otimes e'})\bigr) \\
		&\qquad \cdot\left((n_1\boxtimes n'_1)\boxtimes i(\id_{d^*\otimes d'^{*}})\boxtimes(n_2\boxtimes n'_2)\right) \\
		&\qquad \cdot\left((\id_{c\otimes c'}\otimes\eta_{d\otimes d'})\otimes\rho_{c\otimes c'}^{-1}\right)\\
		&= \bigl((\lambda_e\otimes\lambda_e')\circ(\epsilon_d\otimes\id_{e\otimes I\otimes e'})\circ(\id_d\otimes\sigma_{I,d^*\otimes e}\otimes\id_{e'})\bigr) \\
		%&= \bigl(\lambda_e\circ(\epsilon_e\otimes\lambda_e)\circ(\id_d\otimes\sigma_{I,d^*}\otimes\id_e)\bigr)\otimes\id_{e'} \\
		&\qquad \cdot\Bigl[n_1\boxtimes\bigl(\epsilon_{d'}\cdot(n'_1\boxtimes i(\id_{d'^{*}}))\bigr)\boxtimes\bigl((i(\id_{d*})\boxtimes n_2)\cdot\eta_d\bigr)\boxtimes n'_2\Bigr] \\
		&\qquad \cdot\bigl((\id_c\otimes\sigma_{I,c'\otimes d'^*}\otimes\eta_{d'})\circ(\rho_c^{-1}\otimes\rho_{c'}^{-1})\bigr) \\
		%&\qquad \cdot\id_c\otimes\bigl((\id_{c'}\otimes\sigma_{I,d'^{*}}\otimes\id_{d'})\circ(\rho_{c'}^{-1}\otimes\eta_{d'})\circ\rho_{c'}^{-1}\bigr) \\
		&= \bigl(\lambda_e\circ(\epsilon_e\otimes\lambda_e)\circ(\id_d\otimes\sigma_{I,d^*}\otimes\id_e)\circ(\id_d\otimes\sigma_{d^*\otimes e,I})\bigr)\otimes\id_{e'} \\
		&= \bigl((\lambda_e\otimes\lambda_e')\circ(\epsilon_d\otimes\id_{e\otimes I}\otimes\id_{e'})\bigr) \\
		&\qquad \cdot\Bigl[n_1\boxtimes\bigl((i(\id_{d*})\boxtimes n_2)\cdot\eta_d\bigr)\boxtimes\bigl(\epsilon_{d'}\cdot(n'_1\boxtimes i(\id_{d'^{*}}))\bigr)\boxtimes n'_2\Bigr] \\
		&\qquad \cdot\id_c\otimes\bigl((\sigma_{c'\otimes d'^*,I}\otimes\id_{d'})\circ(\id_{c'}\otimes\sigma_{I,d'^{*}}\otimes\id_{d'})\circ(\rho_{c'}^{-1}\otimes\eta_{d'})\circ\rho_{c'}^{-1}\bigr) \\
		&= \bigl(\lambda_e\circ(\epsilon_d\otimes\rho_e)\bigr)\otimes\id_{e'} \\
		&\qquad \cdot\Bigl[n_1\boxtimes\bigl((i(\id_{d*})\boxtimes n_2)\cdot\eta_d\bigr)\boxtimes\bigl(\epsilon_{d'}\cdot(n'_1\boxtimes i(\id_{d'^{*}}))\bigr)\boxtimes n'_2\Bigr] \\
		&\qquad \cdot\id_c\otimes\bigl((\lambda_{c'}^{-1}\otimes\eta_{d'})\circ\rho_{c'}^{-1}\bigr)
	\end{align*}
\end{proof}

\begin{proposition}
	For any $M\in\MProf(1,\cat{C})$, $FM\in\Prof(\cat{C},\cat{C})$ has a canonical unit.
\end{proposition}

\begin{corollary}
	The functor $F$ factors canonically through the category $\Mon(\MProf(\cat{C},\cat{C}))$ of monoid objects.
\end{corollary}

\begin{proposition}
	The functors $F$ and $U$ induce an equivalence of categories $\MProf(1,\cat{C})\simeq\Mon(\MProf(\cat{C},\cat{C}))$.
\end{proposition}

\begin{corollary}
	There is an equivalence of categories $\Lax(\cat{C},\Set)\simeq(\cat{C}/\Cat{CompCat})_{\text{b.o.o.}}$.
\end{corollary}

\chapter{Traced Monoidal Categories}

Recall from~\cite{JoyalStreet}
\begin{compactitem}
	\item Let $\cat{D}$ be a traced symmetric monoidal category, and $F\colon\cat{C}\to\cat{D}$ a fully faithful symmetric monoidal functor. Then $\cat{C}$ has a unique trace for which $F$ is a traced functor.
	\item Any compact category has a canonical trace, defining a functor $U\colon\Cat{CompCat}\to\Cat{TrCat}$.
	\item The Int construction $\Int\colon\Cat{TrCat}\to\Cat{CompCat}$ is left 2-adjoint to $U$. For any traced symmetric monoidal category $\cat{C}$, the unit $\cat{C}\to\Int(\cat{C})$ is fully faithful.
\end{compactitem}

\begin{lemma}
	Let $\cat{D}$ be a traced symmetric monoidal category, and $F\colon\cat{C}\to\cat{D}$ a fully faithful symmetric monoidal functor. Then, using the unique trace on $\cat{C}$ making $F$ a traced functor, the functor $\Int(\cat{C})\to\cat{D}$ which is adjunct to $F$ is also fully faithful.
\end{lemma}

\begin{proposition}
	Let $\cat{C}$ be a traced symmetric monoidal category. Then the Int construction provides an equivalence of categories
	\[
		(\cat{C}/\Cat{TrCat})_{\text{b.o.o.}} \simeq (\Int(\cat{C})/\Cat{CompCat})_{\text{b.o.o.}}
	\]
\end{proposition}

\end{document} 