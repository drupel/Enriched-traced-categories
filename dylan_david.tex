% !TEX root = ./CCC_Note.tex
\chapter{Introduction}

Traced symmetric monoidal categories have been used to model processes with feedback (\url{http://arxiv.org/pdf/1401.5113v1.pdf})  or operators with fixed points (\url{http://arxiv.org/pdf/1107.6032.pdf}). A graphical calculus for TSMCs was developed by Joyal, Street, and Verity, in which string diagrams of the form
\begin{center}\missingfigure[figwidth=3in]{String diagram with labeled wires}\end{center}
represent compositions, i.e., new morphisms are constructed from old by specifying which outputs will be fed back into which inputs. In fact, these generalize Penrose diagrams in $\Cat{Vect}$, and the word \emph{traced} originates in vector space terminology.  

But notice that the above picture has another interpretation, in terms of 1-dimensional cobordisms between oriented 0-manifolds. The box $X$ in the picture includes only the data of two finite sets $(\inp{X},\outp{X})$, here $(blah,blah)$. Thus each box encodes a signed set, which can be interpreted as an oriented 0-manifold. A string diagram, in which boxes $X_1,\ldots,X_n$ are wired together inside a larger box $Y$, can be interpreted as a cobordism from $X_1\sqcup\cdots\sqcup X_n$ to $Y$. 

There is actually a bit more data in a string diagram for a TSMC $\cat{C}$; namely, each wire is labeled by an object of $\cat{C}$. We will thus consider the category $\Cat{Cob}/L$ of oriented 1-dimensional cobordisms over a fixed set $L$ of labels. 

We record these two interpretations of string diagrams in the table below. Note the ``degree shift" between the second and third columns.
\begin{center}
\begin{tabular}{| l | l | l |}
\hline
\multicolumn{3}{|c|}{Interpretations of string diagrams}\\\hline
String diagram & Traced category $\cat{C}$ & $\Cat{Cob}/L$\\\bhline
Wire label set, $L$&Objects, $L:=\Ob(\cat{C})$&Label set, $L$\\
Box & Morphism & Object (oriented 0-mfd over $L$)\\
String diagram & Composition & Morphism (Cobordism over $L$)\\
Nesting & Axioms of TSMCs & Composition\\\hline
\end{tabular}
\end{center}

The relationship between these interpretations is made precise in the following first main theorem, which will be proved in Section~\ref{**}.
\begin{theorem}
Consider the functor $\op\Set\to\Cat{Cat}$, given by $L\mapsto(\Cat{Cob}/L)\alg$, and let $(\Cat{Cob}/\bullet)\alg\to\Set$ denote the corresponding split fibration. Then there is an equivalence of categories
$$\Cat{TSMC}\simeq(\Cat{Cob}/\bullet)\alg.$$
\end{theorem}

We sketch the connection, with a few minor abuses of notation, as follows. Recall that a $(\Cat{Cob}/L)$-algebra is a lax functor $P\colon\Cat{Cob}/L\to\Cat{Set}$. Once a set $L$ is established, specifying a traced category $\cat{C}$ with objects $L$ requires the same data as specifying a lax functor $P$. First, for each box $(\inp{X},\outp{X})$ in a string diagram, both $\cat{C}$ and $P$ require a set, $\Hom_{\cat{C}}(\inp{X},\outp{X})$ and $P(\inp{X},\outp{X})$, respectively.  Second, for each string diagram, both $\cat{C}$ and $P$ require a function: a formula for composition and trace, in the case of $\cat{C}$, and an action on morphisms, in the case of $P$. The fact that $\cat{C}$ satisfies the axioms of traced symmetric monoidal categories corresponds to checking the functoriality of $P$.

\section{Nesting properties and self-similarity in applications}

When drawing 0-manifolds as boxes and cobordisms as string diagrams, composition of cobordisms become nested diagrams.
\missingfigure{Draw nested diagrams}



\section{Generalization}

\begin{theorem}
 Let $\cat{C}$ be a compact closed category and $\cat{V}$ a symmetric monodical category.  The category $\Lax(\cat{C},\cat{V})$ of lax monodical functors is equivalent to the coslice category $\cat{C}/\cat{V}-\Cat{CompCat}$ spanned by bijective on objects functors.
\end{theorem}
\begin{corollary}
 $\Lax(\Int(\cat{T}),\cat{V})=\cat{V}-\Cat{TSMC}_{\cat{T}/}$
\end{corollary}

\chapter{Wiring Diagrams and $\Cat{1-Cob}$}
\section{Set-theoretic formulation of $\Cat{1-Cob}$, as free compact closed category on one object, as $\Int$ of the free TSMC on one object ($\Cat{Bij}$)}
\subsection{Many object case/generalization $\Cat{Cob}/\cat{O}$}

\section{Morphisms in $\Cat{1-Cob}$ as wiring diagrams}

Wiring diagrams are a new way to visualize morphisms in 1-Cob.

\missingfigure{Draw, perhaps side-by-side, a cobordism in the usual style and in our style.}

\section{$\Cat{1-Cob}$-algebras and applications}

\section{Definition of functors between $\Cat{TSMC}$ and $\Cat{Cob}/\cat{O}$-algebras}

Let $\cat{M}$ be a traced symmetric monoidal category with objects $\cat{O}$. We will define a $\Cat{Cob}/\cat{O}$-algebra $\cat{P}=R(\cat{M})\colon\Cat{Cob}/\cat{O}\to\Cat{Set}$ as follows. For an object $X\in\Ob(\Cat{Cob}/\cat{O})$, set 
$$\cat{P}(X):=\Hom_{\cat{M}}(\vinp{X},\voutp{X}).$$
We next consider morphisms.

Following Proposition~\ref{prop:set theoretic cob1} (borrow notation/setup from Abramsky instead) a morphism $\Phi\colon X\longrightarrow Y$ consists of a typed bijection 
$$\varphi\colon\inp{X}\sqcup \outp{Y}\xrightarrow{\iso}\outp{X}\sqcup \inp{Y},$$ 
together with a typed finite set $S$. Given an element $f\in\cat{P}(X)$ we must construct $\cat{P}(\Phi)(f)\in\cat{P}(Y)$. Let $\dim(\overline{S})=\textnormal{Tr}^{\overline{S}}_{I,I}\big[\id_{\overline{S}}\big]\in\cat{S}_\cat{M}$. Then we use the formula
$$\cat{P}(\Phi)(f):=
\textnormal{Tr}^{\overline{X_+}}_{\overline{Y_-},\overline{Y_+}}\Big[\big(f\otimes\id_{\overline{Y_+}}\big)\circ\overline{\varphi}\Big]
\otimes\dim(\overline{S}).	
$$

\begin{theorem}
 The category $\Cat{TSMC}$ is equivalent to the category of $\Cat{Cob}/\cat{O}$-algebras.
\end{theorem}
\begin{proof}
 
\end{proof}
\begin{corollary}
 Enriched setting?
\end{corollary}


