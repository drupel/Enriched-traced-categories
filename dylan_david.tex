% !TEX root = ./CCC_Note.tex
\chapter{Introduction}

Traced symmetric monoidal categories, hereafter \emph{traced categories}, have been used to model processes with feedback (\cite{http://arxiv.org/pdf/1401.5113v1.pdf})  or operators with fixed points (\cite{http://arxiv.org/pdf/1107.6032.pdf}). A graphical calculus for traced categories was developed by Joyal, Street, and Verity (\cite{JoyalStreetVerity}), in which string diagrams of the form
%
%
%WITHOUT OUTER BOX:
%
%
%\begin{align}\label{dia:string diagram}
%\begin{tikzpicture}
%	%little box 1
%	\path(2,1.5);
%	\blackbox{(2,2)}{2}{1}{$X_1$}{.5}
%	%little box 2
%	\path(6,1.5);
%	\blackbox{(2,2)}{2}{2}{$X_2$}{.5}
%	%wires
%	\directarc{(4.25,2.5)}{(5.75,2.16667)} % X_1 -> X_2
%	\directarc{(0.35,2.833333)}{(1.75,2.83333)} % Y -> X_1
%	\fancyarc{(0.35,4)}{(5.75,2.83333)}{-35}{16} % Y -> X_2
%	\directarc{(8.25,2.8333)}{(9.65,2.83333)} % X_2 -> Y
%	\fancyarc{(1.75,2.16667)}{(8.25,2.16667)}{20}{-45} % X_2 -> X_1
%\end{tikzpicture}
%\end{align}
%
%
%WITH OUTER BOX:
%
%
\begin{align}\label{dia:string diagram}
\begin{tikzpicture}
   %big box
	\path(0,0);
	\dashblackbox{(10,5)}{2}{1}{$Y$}{.7}
	    %inner wires
	        \node at (.4,3.6) {\tiny $\inp{Y}_{a}$};
	        \node at (.4,1.9) {\tiny $\inp{Y}_{b}$};
	    %outer wire
	        \node at (9.6,2.75) {\tiny $\outp{Y}_a$};
	%little box 1
	\path(2,1.5);
	\blackbox{(2,2)}{2}{1}{$X_1$}{.5}
	    %tank info
	    %inner wires
	        \node at (1.75,3.03) {\tiny $\inp{X}_{1a}$};
	        \node at (1.75,2.35) {\tiny $\inp{X}_{1b}$};
	    %outer wire
	        \node at (4.33,2.68) {\tiny $\outp{X}_{1c}$};
	%little box 2
	\path(6,1.5);
	\blackbox{(2,2)}{2}{2}{$X_2$}{.5}
	    %tank info
	    %inner wires
	        \node at (5.75,3.03) {\tiny $\inp{X}_{2a}$};
	        \node at (5.75,2.35) {\tiny $\inp{X}_{2b}$};
	    %outer wires
	        \node at (8.33,3.03) {\tiny $\outp{X}_{2c}$};
	        \node at (8.33,2.35) {\tiny $\outp{X}_{2d}$};
	%wires
	\directarc{(4.25,2.5)}{(5.75,2.16667)} % X_1 -> X_2
	\directarc{(0.35,1.6667)}{(1.75,2.83333)} % Y -> X_1
	\fancyarc{(0.35,3.3333)}{(5.75,2.83333)}{-40}{25} % Y -> X_2
	\directarc{(8.25,2.8333)}{(9.65,2.5)} % X_2 -> Y
	\fancyarc{(1.75,2.16667)}{(8.25,2.16667)}{20}{-45} % X_2 -> X_1
\end{tikzpicture}
\end{align}
represent compositions, i.e., new morphisms are constructed from old by specifying which outputs will be fed back into which inputs. In fact, these generalize Penrose diagrams in $\Cat{Vect}$, and the word \emph{traced} originates in vector space terminology.  

A string diagram usually does not explicitly include the outer box $Y$. If we include it, as in (\ref{dia:string diagram}), the resulting \emph{wiring diagram} can be given another interpretation: it represents a 1-dimensional cobordism between oriented 0-manifolds. For example, the box $X_1$ in the picture includes only the data of a pair of finite sets, $(\inp{X_1},\outp{X_1})=(\{1a,1b\},\{1c\})$, which can be interpreted as an oriented 0-manifold.  And the wiring diagram itself, in which boxes $X_1,\ldots,X_n$ are wired together inside a larger box $Y$, can be interpreted as an oriented cobordism from $X_1\sqcup\cdots\sqcup X_n$ to $Y$. This is a morphism in the multicategory $\Cat{Cob}$, underlying the symmetric monoidal category of oriented 1-cobordisms.

There is actually a bit more data in a string (or wiring) diagram for a traced category $\cat{C}$. Namely, each input and output of a box must be labeled by an object of $\cat{C}$, and the strings must respect these labels. We will thus consider the multicategory $\Cat{Cob}/\cat{O}$ of oriented 1-dimensional cobordisms over a fixed set $\cat{O}$ of labels. 

In the table below, we record these two interpretations of a string diagram. Note the "degree shift" between the second and third columns.
\begin{center}
% \begin{tabular}{| l | l | l |}
% \hline
% \multicolumn{3}{|c|}{Interpretations of string diagrams}\\\hline
% String diagram & Traced category $\cat{C}$ & $\Cat{Cob}/\cat{O}$\\\bhline
% Wire label set, $\cat{O}$&Objects, $\cat{O}:=\Ob(\cat{C})$&Label set, $\cat{O}$\\
% Box & Morphism & Object (oriented 0-mfd over $\cat{O}$)\\
% String diagram & Composition & Morphism (Cobordism over $\cat{O}$)\\
% Nesting & Axioms of TSMCs & Composition\\\hline
% \end{tabular}
\begin{tabular}{lll}
\toprule
\multicolumn{3}{c}{Interpretations of string diagrams} \\
\midrule
String diagram & Traced category $\cat{C}$ & $\Cat{Cob}/\cat{O}$ \\
\midrule
Wire label set, $\cat{O}$ & Objects, $\cat{O}:=\Ob(\cat{C})$ & Label set, $\cat{O}$ \\
Box & Morphism & Object (oriented 0-mfd over $\cat{O}$) \\
String diagram & Composition & Morphism (Cobordism over $\cat{O}$) \\
Nesting & Axioms of TSMCs & Composition \\
\bottomrule
\end{tabular}
\end{center}

The relationship between these interpretations is made precise in the following first main theorem, which will be proved in Section~\ref{sec:first equivalence}. For any multicategory $\cat{M}$, we denote by $\cat{M}\alg$ the category of lax functors $\cat{M}\to\Cat{Set}$. Let $\Cat{TrCat}$ denote the category of traced categories and traced functors; these notions will be recalled in Section~\ref{sec:define traced}.

\begin{theorem}\label{thm:traced as cob-alg}
Consider the functor $\op\Set\to\Cat{Cat}$, given by $\cat{O}\mapsto(\Cat{Cob}/\cat{O})\alg$, and let $(\Cat{Cob}/\bullet)\alg$ denote the total category of corresponding Grothendieck fibration. Then there is an equivalence of categories
$$(\Cat{Cob}/\bullet)\alg\iso\Cat{TrCat}.$$
\end{theorem}

We sketch the equivalence, with a few minor abuses of notation, as follows. It suffices to fix a set $\cat{O}$ and find an equivalence, natural in $\cat{O}$, between the category of $\Cat{Cob}/\cat{O}$-algebras and the category of traced categories with \todo{Is this accurate?} generating objects $\cat{O}$. We will show that the same data are required, and the same conditions are satisfied, whether one is specifying a lax functor $P\colon\Cat{Cob}/\cat{O}\to\Cat{Set}$ or a traced category $\cat{C}$ with objects generated by $\cat{O}$. 

First, for each box $(\inp{X},\outp{X})$ in a string diagram, both $P$ and $\cat{C}$ require a set, $P(\inp{X},\outp{X})$ and $\Hom_{\cat{C}}(\inp{X},\outp{X})$, respectively.  Second, for each string diagram, both $P$ and $\cat{C}$ require a function: an action on morphisms, in the case of $P$, and a formula for performing the required compositions, tensors, and traces, in the case of $\cat{C}$. The condition that $P$ is functorial corresponds to the fact that $\cat{C}$ satisfies the axioms of traced categories.

\section{Generalization: lax algebras on a traced or compact category}

It is most convenient to prove Theorem~\ref{thm:traced as cob-alg} by proving a much more general result, which relates the category of lax functors out of a compact category $\cat{C}$ to the category of strong, bijective-on-objects functors out of $\cat{C}$. Let $\Cat{CompCat}$ denote the category of compact categories and strong monoidal functors between them, and let $\Cat{CompCat}_{\Ob\iso}$ denote the wide subcategory spanned by functors whose object part is a bijection. 

\begin{theorem}\label{thm:compact lax and strong}
 Let $\cat{C}$ be a compact category. If $\Lax(\cat{C},\Cat{Set})$ denotes the category of lax monoidal set-valued functors out of $\cat{C}$, and $\cat{C}/\Cat{CompCat}_{\Ob\iso}$ denotes the coslice category of compact categories and strong bijective-on-objects functors under $\cat{C}$, then there is an equivalence of categories
\begin{align}\label{dia:compact lax strong}
\Lax(\cat{C},\Cat{Set})\cong\cat{C}/\Cat{CompCat}_{\Ob\iso}
\end{align}
\end{theorem}

By \cite{Abramsky}, $\Cat{Cob}/\cat{O}$ is the free compact category on the set $\cat{O}$ of objects. In this case Theorem~\ref{thm:compact lax and strong} amounts to the following equivalence of categories: 
$$\Lax(\Cat{Cob}/\cat{O},\Cat{Set})\iso\Cat{CompCat}_{\Ob\cong\cat{O}}.$$
It turns out that if our compact category $\cat{C}$ is Int of something, then Theorem~\ref{thm:compact lax and strong} lifts to a result about traced categories, recorded as Corollary~\ref{cor:thm:traced lax and strong}, from which Theorem~\ref{thm:traced as cob-alg} follows. 

\begin{corollary}\label{cor:thm:traced lax and strong}
Suppose that $\cat{D}$ is a traced category and its compact closure is $\cat{C}=Int\cat{D}$. Then there is an equivalence of categories
\begin{align}\label{dia:traced lax strong}
\Lax(\cat{C},\Cat{Set})\iso\cat{C}/\Cat{TrCat}_{\Ob\iso}
\end{align}
This equivalence is natural in the compact category $\cat{C}$, using the factorization system on $\Cat{CompCat}$ from Lemma~\ref{lemma:factorization system}.
\end{corollary}

\begin{remark}

Let $\cat{V}$ denote a symmetric monoidal category. The categories of $\cat{V}$-enriched compact or traced categories have not, to the best of our knowledge, been defined in the literature. Theorem~\ref{thm:compact lax and strong} and Corollary~\ref{cor:thm:traced lax and strong} could be used to motivate a definition of these categories by replacing $\Cat{Set}$ with $\cat{V}$ in (\ref{dia:compact lax strong}) and (\ref{dia:traced lax strong}). That is, one could define a $\cat{V}$-enriched compact category to be a set $\cat{O}$ and a lax functor $\Cat{Cob}/\cat{O}\to\cat{V}$. 

There are other ways to define $\cat{V}$-enriched traced and compact categories, e.g., using the theory of framed bicategories. The above definition agrees, though we do not prove this result in the present paper.\todo{True?}

\end{remark}

\begin{lemma}\label{lemma:factorization system}

Let $\Cat{XCat}$, denote the category of $\Cat{X}$-strong functors between $\Cat{X}$-categories,  where $\Cat{X}$ is monoidal, traced, or compact. Then $\Cat{XCat}$ admits an orthogonal factorization system $(\cat{L},\cat{R})$, where the morphisms in $\cat{L}$ are bijective-on-objects functors, and the morphisms in $\cat{R}$ are fully faithful functors.

\end{lemma}

\begin{proof}[Sketch of proof]

Let $(\cat{C},\otimes_{\cat{C}})$ and $(\cat{D},\otimes_{\cat{D}})$ be $\Cat{X}$-categories, and let $F\colon\cat{C}\to\cat{D}$ be a strong $\Cat{X}$-functor. Define a $\Cat{X}$-category $(\cat{F},\otimes_{\cat{F}})$ to act like $\cat{C}$ on objects and like $\cat{D}$ on morphisms. That is, $\Ob\cat{F}:=\Ob\cat{C}$, and $\Hom_{\cat{F}}(c_1,c_2):=\Hom_{\cat{D}}(Fc_1,Fc_2).$ On objects, put $c_1\otimes_{\cat{F}}c_2:=c_1\otimes_{\cat{C}}c_2$. Given morphisms $f_1\colon c_1\to c_1'$ and $f_2\colon c_2\to c_2'$ in $\cat{F}$, put $f_1\otimes_{\cat{F}}f_2:=f_1\otimes_{\cat{D}}f_2$, using the coherence isomorphisms.

There are induced strong $\Cat{X}$-functors 
$$(\cat{C},\otimes_{\cat{C}})\to(\cat{F},\otimes_{\cat{F}})\to(\cat{D},\otimes_{\cat{D}})$$
which compose to $F$, such that the former is bijective on objects and the latter is fully faithful. The proof of orthogonality is straightforward.\todo{None of this has been rigorously checked.}

\end{proof}


\section[Applications of the cobordism-algebra perspective]{Applications of the cobordism-algebra perspective in engineering design}

When designing or investigating a complex system, it is often useful to think in terms of interacting subsystems, put together to make a larger whole. This is often called \emph{compositionality}.

Thinking of this as processes wired together to make larger processes has and Rupel were originally motivated to formalize the operadic nature of compositionality. This led to the notion of wiring diagrams and their algebras, as discussed in previous work \cite{Spivak}, \cite{Rupel-Spivak}, \cite{Vagner-Spivak-Lerman}. 
 

When drawing 0-manifolds as boxes and cobordisms as string diagrams, composition of cobordisms become nested diagrams.
\begin{figure}[hb]
\activetikz{
	%left large box,
	\node at (.5,2.7){$X\xrightarrow{\Phi}Y\xrightarrow{\Psi}Z$};
	\node at (.5,1.15){\tiny $\alpha$};
	\node at (.2,2.05){\tiny $\beta$};
	\path(-1,0);\blackbox{(3,2.4)}{1}{2}{}{.5}
	%left small boxes (dashed),
	\path(0,1.4);\dashblackbox{(1,.8)}{1}{1}{}{.2}
	\path(0,.3);\dashblackbox{(1,.6)}{1}{2}{}{.2}
	%left outer arcs
	\directarc{(1.1,1.8)}{(1.75,1.6)}
	\directarc{(1.1,.5)}{(1.75,.8)}
	\directarc{(-.75,1.2)}{(-.1,.6)} %this
	\fancyarc{(-.1,1.8)}{(1.1,.7)}{10}{0}
	%left inner arcs, top
	\fancyarc{(.283,1.817)}{(.716,1.817)}{4}{5}
	\directarc{(.1,1.8)}{(.283,1.733)}
	\directarc{(.716,1.733)}{(.9,1.8)}
	%left inner arcs, bottom
	\directarc{(.1,.6)}{(.6,.75)}
	\directarc{(.85,.725)}{(.9,.7)}
	\directarc{(.55,.55)}{(.6,.7)}
	%\directarc{(.1,.5)}{(.2,.525)}
	\directarc{(.55,.5)}{(.9,.5)}
	%left tiny boxes,
	\path(0.333,1.65);\blackbox{(.333,.25)}{2}{2}{}{.1}
	\path(0.65,.65);\blackbox{(.15,.15)}{2}{1}{}{.1}
	\path(0.35,.45);\blackbox{(.15,.15)}{0}{2}{}{.1} %this
	%right large box,
	\node at (4.5,2.7){$X\xrightarrow{\Psi\circ\Phi}Z$};
	\node at (4.5,1.12){\tiny $\alpha$};
	\node at (4.2,2.05){\tiny $\beta$};
	\path(3,0);\blackbox{(3,2.4)}{1}{2}{}{.5}
	%right tiny boxes,
	\path(4.333,1.65);\blackbox{(.333,.25)}{2}{2}{}{.1}
	\path(4.65,.65);\blackbox{(.15,.15)}{2}{1}{}{.1}
	\path(4.25,.45);\blackbox{(.15,.15)}{0}{2}{}{.1}
	%right arcs, between
	\fancyarc{(4.283,1.733)}{(4.85,.725)}{10}{0}
	%right arcs, top
	\fancyarc{(4.283,1.817)}{(4.716,1.817)}{4}{5}
	\directarc{(4.716,1.733)}{(5.75,1.6)}
	%right arcs, bottom
	\directarc{(3.25,1.2)}{(4.6,.75)}
	\directarc{(4.45,.55)}{(4.6,.7)}
	\directarc{(4.45,.5)}{(5.75,.8)}
	%\fancyarc{(4.2,.525)}{(4.45,.5)}{5}{-5}
}
\end{figure}
A commutative square in $\Cat{Cob}$ corresponds to finding two different ways to chunk small boxes inside a big box. The functoriality of $P\colon\Cat{Cob}\to\Set$ can be thought of as ensuring that any two ways to chunk boxes gives the same result.

\section{Acknowledgments}

Thanks go to Steve Awodey and Ed Morehouse for suggesting we formally connect our operad-algebra picture to the traced one. 

\chapter{Wiring Diagrams and $\Cat{Cob}$}

In this section, we more carefully explain the equivalence between the category of traced categories and the category of cobordism algebras.

\section{Objects in $\Cat{Cob}$ as interfaces}

The objects in $\Cat{Cob}$ are signed sets $(\inp{X},\outp{X})$, each of which can be drawn as a box with input wires $\inp{X}$ drawn entering the box, on its left, and output wires $\outp{X}$ drawn exiting the box, on its right. We call the latter style \emph{an interface}.

\begin{figure}
\activetikz{
\draw (0,2) node {$-$};
\draw(0,1.5) node {$-$};
\draw(0,1) node {$-$};
\draw(0,.5) node {$+$};
\draw(0,0) node {$+$};
\path(5,0);
\blackbox{(2,2)}{3}{2}{$X$}{.5}
}
\caption{The signed set $(\ul{3},\ul{2})$, drawn in the usual style and as an interface.}
\end{figure}

\section{Morphisms in $\Cat{Cob}$ as wiring diagrams}

Wiring diagrams are a new way to visualize morphisms in 1-Cob.
\begin{center}
\activetikz{
    %big box
	\path(0,0);
	\blackbox{(10,5)}{2}{1}{$Y$}{.7}
	    %inner wires
	        \node at (.4,3.6) {\tiny $\inp{Y}_{a}$};
	        \node at (.4,1.9) {\tiny $\inp{Y}_{b}$};
	    %outer wire
	        \node at (9.6,2.75) {\tiny $\outp{Y}_a$};
	%little box 1
	\path(2,1.5);
	\blackbox{(2,2)}{2}{1}{$X_1$}{.5}
	    %tank info
	    %inner wires
	        \node at (1.75,3.03) {\tiny $\inp{X}_{1a}$};
	        \node at (1.75,2.35) {\tiny $\inp{X}_{1b}$};
	    %outer wire
	        \node at (4.33,2.68) {\tiny $\outp{X}_{1c}$};
	%little box 2
	\path(6,1.5);
	\blackbox{(2,2)}{2}{2}{$X_2$}{.5}
	    %tank info
	    %inner wires
	        \node at (5.75,3.03) {\tiny $\inp{X}_{2a}$};
	        \node at (5.75,2.35) {\tiny $\inp{X}_{2b}$};
	    %outer wires
	        \node at (8.33,3.03) {\tiny $\outp{X}_{2c}$};
	        \node at (8.33,2.35) {\tiny $\outp{X}_{2d}$};
	%wires
	\directarc{(4.25,2.5)}{(5.75,2.16667)} % X_1 -> X_2
	\directarc{(0.35,1.6667)}{(1.75,2.83333)} % Y -> X_1
	\fancyarc{(0.35,3.3333)}{(5.75,2.83333)}{-40}{25} % Y -> X_2
	\directarc{(8.25,2.8333)}{(9.65,2.5)} % X_2 -> Y
	\fancyarc{(1.75,2.16667)}{(8.25,2.16667)}{20}{-45} % X_2 -> X_1
}
\end{center}
~\todo{Put these two drawings on one line?}
\begin{center}
\activetikz{
%Left manifolds
\def\factor{.7}
\def\lspace{.7}
\def\bspace{5}
\draw (0,7*\factor) node {$\inp{X}_{1a}$};\draw (\lspace,7*\factor) node {$-$};
\draw(0,6*\factor) node {$\inp{X}_{1b}$};\draw (\lspace,6*\factor) node {$-$};
\draw(0,5*\factor) node {$\outp{X}_{1c}$};\draw (\lspace,5*\factor) node {$+$};
%skip 3 to separate manifolds
\draw(0,3*\factor) node {$\inp{X}_{2a}$};\draw (\lspace,3*\factor) node {$-$};
\draw(0,2*\factor) node {$\inp{X}_{2b}$};\draw (\lspace,2*\factor) node {$-$};
\draw(0,1*\factor) node {$\outp{X}_{2c}$};\draw (\lspace,1*\factor) node {$+$};
\draw(0,0*\factor) node {$\outp{X}_{2d}$};\draw (\lspace,0*\factor) node {$+$};
%Right manifold
\draw (\bspace,6*\factor) node {$\inp{Y}_a$};\draw (\bspace-\lspace,6*\factor) node {$-$};
\draw (\bspace,4*\factor) node {$\inp{Y}_b$};\draw (\bspace-\lspace,4*\factor) node {$-$};
\draw (\bspace,2*\factor) node {$\outp{Y}_a$};\draw (\bspace-\lspace,2*\factor) node {$+$};
%Arcs
\draw (1.3*\lspace,6*\factor) .. controls (3*\lspace,6*\factor) and (3*\lspace,0*\factor) .. (1.3*\lspace,0*\factor);
\draw (1.3*\lspace,5*\factor) .. controls (2.5*\lspace,5*\factor) and (2.5*\lspace,2*\factor) .. (1.3*\lspace,2*\factor);
\directarc{(1.3*\lspace,7*\factor)}{(\bspace-1.3*\lspace,4*\factor)}
\directarc{(1.3*\lspace,3*\factor)}{(\bspace-1.3*\lspace,6*\factor)}
\directarc{(1.3*\lspace,1*\factor)}{(\bspace-1.3*\lspace,2*\factor)}
}
\end{center}

\section{$\Cat{Cob}$-algebras and applications}

Traced categories are often used to model processes with feedback. The processes correspond to morphisms, each drawn as an interface, and these processes can be strung together in series or parallel, and with feedback, using string diagram notation. Theorem~\ref{thm:traced as cob-alg} says that we can also view string diagrams as morphisms in $\Cat{Cob}$. In this setting, the processes are modeled using an algebra on $\Cat{Cob}$.

References for applications of traced monoidal categories.

References for applications of $\Cat{Cob}$-algebras.

\chapter{Traced Categories}\label{sec:traced categories}

Let $(\cat{M},\otimes,I)$ be a symmetric monoidal category where for any $X,Y\in\Ob(\cat{M})$ we write $\gamma_{X,Y}:X\otimes Y\To{\sim} Y\otimes X$ for the distinguished symmetry isomorphisms. Recall that we have $\gamma_{Y,X}=\gamma_{X,Y}^{-1}$.  We define the {\em monoid of scalars in $\cat{M}$} to be the set $S_{\cat{M}}:=\Hom(I,I)$ with multiplication given by composition.  Note that $S_{\cat{M}}$ is commutative since $\cat{M}$ is symmetric.  There is an action of $S_{\cat{M}}$ on the set $\Hom_{\cat{M}}(X,Y)$ for each $(X,Y)\in\Ob(\cat{M}^{op}\times\cat{M})$, where $s\in S_{\cat{M}}$ acts on a morphism $f\colon X\to Y$ by sending it to the composite morphism
$$s\bullet f:X\to I\otimes X\To{s\otimes f}I\otimes Y\to Y.$$
We write $|X|:=\Hom(I,X)$ for the \emph{elements} of $X$.

The functor $|\cdot|=\Hom(I,\cdot)\colon\cat{M}\to\Set$ is a unital \todo{This is just a lax monoidal functor, right?} algebra on $\cat{M}$ with unit map
\[\eta:\{1\}\To{\id_{I}}S_{\cat{M}}\]
and multiplication map
\[\mu:\Hom(I,X)\otimes\Hom(I,X')\to\Hom(I\otimes I,X\otimes X')\iso\Hom(I,X\otimes X').\]

\section{Definition of traced category}\label{sec:define traced}

A (left) \emph{trace} on a symmetric monoidal category is a collection of functions 
\[\Tr^U_{X,Y}:\Hom(U\otimes X,U\otimes Y)\to\Hom(X,Y)\]
for $U,X,Y\in\Ob(\cat{M})$ satisfying the following axioms:
\begin{description}
 \item [Dinaturality I:] for every $f:U\otimes X\to V\otimes Y$ and $g:V\to U$ we have
 \[\Tr^U_{X,Y}\Big[\big(g\otimes Y\big)\circ f\Big]=\Tr^V_{X,Y}\Big[f\circ\big(g\otimes X\big)\Big];\] 
 \item [Naturality:] for every $f:U\otimes X\to U\otimes Y$, $g:X'\to X$, and $h:Y\to Y'$ we have
 \[\Tr^U_{X',Y'}\Big[\big(U\otimes h\big)\circ f\circ\big(U\otimes g\big)\Big]=h\circ\Tr^U_{X,Y}\big[f\big]\circ g;\]
 \item [Superposing:] for every $f:U\otimes X\to U\otimes Y$ and $g:W\to Z$ we have
 \[\Tr^U_{X,Y}\big[f\big]\otimes g=\Tr^U_{X\otimes W,Y\otimes Z}\big[f\otimes g\big];\]
 \item [Vanishing I:] for every $f:X\to Y$ we have
 \[\Tr^{I}_{X,Y}\big[f\big]=f;\]
 \item [Vanishing II:] for every $f:U\otimes V\otimes X\to U\otimes V\otimes Y$ we have
 \[\Tr^{U\otimes V}_{X,Y}\big[f\big]=\Tr^V_{X,Y}\Big[\Tr^U_{V\otimes X,V\otimes Y}\big[f\big]\Big];\]
 \item [Yanking:] for any $X\in\Ob(\cat{M})$ we have
 \[\Tr^X_{X,X}\big[\gamma_{X,X}\big].\]
\end{description}

For an object $X$ in a traced category $\cat{M}$ we write $\dim(X):=\Tr^X_{I,I}\big[\id_X\big]\in S_{\cat{M}}$ for the {\em dimension} of $X$.  Note also that for any endomorphism $f:X\to X$ there is a scalar $s_f=\Tr^X_{I,I}[f]\in S_{\cat{M}}$.

\begin{proposition}\label{prop:dinaturality}\mbox{}
Let $\cat{M}$ be a symmetric monoidal category.
\begin{enumerate}
 \item If $\cat{M}$ is traced then for every $f:X\to Y$ and $g:Y\to Z$ in $\cat{M}$, we have
 \[g\circ f=\Tr^Y_{X,Z}\Big[\big(f\otimes g\big)\circ\gamma_{Y,X}\Big].\]
 \item Consider the following axiom:
 \begin{description}
  \item [Dinaturality I':] for every $h:U\otimes V\otimes X\to U\otimes V\otimes Y$ we have
 \[\Tr^{U\otimes V}_{X,Y}\big[h\big]=\Tr^{V\otimes U}_{X,Y}\Big[\big(\gamma_{U,V}\otimes Y\big)\circ h\circ\big(\gamma_{V,U}\otimes X\big)\Big].\]
 \end{description}
 In the presence of the other five Axioms, the axioms Dinaturality~I and Dinaturality~I' are equivalent.
\end{enumerate}
\end{proposition}
\begin{proof}
 (1) can be shown from the Naturality and Yanking axioms of the trace as follows:
 \begin{align*}
  \Tr^Y_{X,Z}\Big[\big(f\otimes g\big)\circ\gamma_{Y,X}\Big]
  &=\Tr^Y_{X,Z}\Big[\big(Y\otimes g\big)\circ\big(f\otimes Y\big)\circ\gamma_{Y,X}\Big]=\Tr^Y_{X,Z}\Big[\big(Y\otimes g\big)\circ\gamma_{Y,Y}\circ\big(Y\otimes f\big)\Big]\\
  &=g\circ\Tr^Y_{Y,Y}\big[\gamma_{Y,Y}\big]\circ f=g\circ f.
 \end{align*}
 
 Dinaturality I' is an immediate consequence of Dinaturality I, indeed apply Dinaturality I with $f=h\circ\big(\gamma_{V,U}\otimes X\big)$ and $g=\gamma_{U,V}$.  For the other direction we apply the composition formula of (1) to the left side of the Dinaturality I equation to get
 \begin{align*}
  \Tr^U_{X,Y}\Big[\big(g\otimes Y\big)\circ f\Big]
  &=\Tr^U_{X,Y}\bigg[\Tr^{V\otimes Y}_{U\otimes X,U\otimes Y}\Big[\big(f\otimes g\otimes Y\big)\circ\gamma_{V\otimes Y,U\otimes X}\Big]\bigg]\\
 &=\Tr^{V\otimes Y\otimes U}_{X,Y}\Big[\big(f\otimes g\otimes Y\big)\circ\gamma_{V\otimes Y,U\otimes X}\Big]\\
  &=\Tr^{V\otimes Y\otimes U}_{X,Y}\Big[\big(f\otimes g\otimes Y\big)\circ\gamma_{V\otimes Y,U\otimes X}\Big]\circ\Tr^X_{X,X}\big[\gamma_{X,X}\big],
 \end{align*}
 where the last two equalities follow from Vanishing II and a trivial application of Yanking.  
 
 By Naturality we may bring the first trace into the second to get
 \begin{align*}
  &\Tr^X_{X,Y}\Bigg[\bigg(X\otimes \Tr^{V\otimes Y\otimes U}_{X,Y}\Big[\big(f\otimes g\otimes Y\big)\circ\gamma_{V\otimes Y,U\otimes X}\Big]\bigg)\circ\gamma_{X,X}\Bigg]\\
  &\quad=\Tr^X_{X,Y}\Bigg[\gamma_{Y,X}\circ\bigg(\Tr^{V\otimes Y\otimes U}_{X,Y}\Big[\big(f\otimes g\otimes Y\big)\circ\gamma_{V\otimes Y,U\otimes X}\Big]\otimes X\bigg)\Bigg]\\
  &\quad=\Tr^X_{X,Y}\bigg[\gamma_{Y,X}\circ\Tr^{V\otimes Y\otimes U}_{X\otimes X,Y\otimes X}\Big[\big(f\otimes g\otimes Y\otimes X\big)\circ\big(\gamma_{V\otimes Y,U\otimes X}\otimes X\big)\Big]\bigg]\\
  &\quad=\Tr^X_{X,Y}\bigg[\Tr^{V\otimes Y\otimes U}_{X\otimes X,X\otimes Y}\Big[\big(V\otimes Y\otimes U\otimes \gamma_{Y,X}\big)\circ\big(f\otimes g\otimes Y\otimes X\big)\circ\big(\gamma_{V\otimes Y,U\otimes X}\otimes X\big)\Big]\bigg]\\
  &\quad=\Tr^X_{X,Y}\bigg[\Tr^{V\otimes Y\otimes U}_{X\otimes X,X\otimes Y}\Big[\big(f\otimes g\otimes \gamma_{Y,X}\big)\circ\big(\gamma_{V\otimes Y,U\otimes X}\otimes X\big)\Big]\bigg]\\
  &\quad=\Tr^{V\otimes Y\otimes U\otimes X}_{X,Y}\Big[\big(f\otimes g\otimes \gamma_{Y,X}\big)\circ\big(\gamma_{V\otimes Y,U\otimes X}\otimes X\big)\Big],
 \end{align*}
 where we apply Superposing to get the second equality, Naturality to get the third, and Vanishing II for the last.
 
 Now we are ready to apply Dinaturality I' to get
 \begin{align*}
  \Tr^{U\otimes X\otimes V\otimes Y}_{X,Y}\Big[\big(\gamma_{V\otimes Y,U\otimes X}\otimes Y\big)\circ\big(f\otimes g\otimes \gamma_{Y,X}\big)\Big]
 \end{align*}
 from which we may apply, in reverse, an analogous sequence of equalities to that above to get the right hand side of the Dinaturality I equation.
 \erase{\begin{align*}%begin erase
  &=\Tr^Y_{X,Y}\bigg[\Tr^{U\otimes X\otimes V}_{Y\otimes X,Y\otimes Y}\Big[\big(\gamma_{V\otimes Y,U\otimes X}\otimes Y\big)\circ\big(f\otimes g\otimes \gamma_{Y,X}\big)\Big]\bigg]\\
  &=\Tr^Y_{X,Y}\bigg[\Tr^{U\otimes X\otimes V}_{Y\otimes X,Y\otimes Y}\Big[\big(\gamma_{V\otimes Y,U\otimes X}\otimes Y\big)\circ\big(f\otimes g\otimes X\otimes Y\big)\circ\big(U\otimes X\otimes V\otimes \gamma_{Y,X}\big)\Big]\bigg]\\
  &=\Tr^Y_{X,Y}\bigg[\Tr^{U\otimes X\otimes V}_{X\otimes Y,Y\otimes Y}\Big[\big(\gamma_{V\otimes Y,U\otimes X}\otimes Y\big)\circ\big(f\otimes g\otimes X\otimes Y\big)\Big]\circ\gamma_{Y,X}\bigg]\\
  &=\Tr^Y_{X,Y}\Bigg[\bigg(\Tr^{U\otimes X\otimes V}_{X,Y}\Big[\gamma_{V\otimes Y,U\otimes X}\circ\big(f\otimes g\otimes X\big)\Big]\otimes Y\bigg)\circ\gamma_{Y,X}\Bigg]\\
  &=\Tr^Y_{X,Y}\Bigg[\gamma_{Y,Y}\circ\bigg(Y\otimes\Tr^{U\otimes X\otimes V}_{X,Y}\Big[\gamma_{V\otimes Y,U\otimes X}\circ\big(f\otimes g\otimes X\big)\Big]\bigg)\Bigg]\\
  &=\Tr^Y_{Y,Y}\big[\gamma_{Y,Y}\big]\circ\Tr^{U\otimes X\otimes V}_{X,Y}\Big[\gamma_{V\otimes Y,U\otimes X}\circ\big(f\otimes g\otimes X\big)\Big]\\
  &=\Tr^{U\otimes X\otimes V}_{X,Y}\Big[\gamma_{V\otimes Y,U\otimes X}\circ\big(f\otimes g\otimes X\big)\Big]\\
  &=\Tr^{U\otimes X\otimes V}_{X,Y}\Big[\big(g\otimes X\otimes f\big)\circ\gamma_{U\otimes X,V\otimes X}\Big]\\
  &=\Tr^V_{X,Y}\bigg[\Tr^{U\otimes X}_{V\otimes X,V\otimes Y}\Big[\big(g\otimes X\otimes f\big)\circ\gamma_{U\otimes X, V\otimes X}\Big]\bigg]\\
  &=\Tr^V_{X,Y}\Big[f\circ\big(g\otimes X\big)\Big].
 \end{align*}}%end erase
 %%Another proof
 \erase{By Dinaturality I' this becomes%begin erase
 \[\Tr^U_{X,Y}\bigg[\Tr^{Y\otimes V}_{U\otimes X,U\otimes Y}\Big[\big(\gamma_{V,Y}\otimes U\otimes Y\big)\circ\big(f\otimes g\otimes Y\big)\circ\gamma_{V\otimes Y,U\otimes X}\circ\big(\gamma_{Y,V}\otimes U\otimes X\big)\Big]\bigg]\]
 which is equal via Vanishing II and Naturality to
 \begin{align*}
  &\Tr^U_{X,Y}\Bigg[\Tr^V_{U\otimes X,U\otimes Y}\bigg[\Tr^Y_{V\otimes U\otimes X,V\otimes U\otimes Y}\Big[\big(\gamma_{V,Y}\otimes U\otimes Y\big)\circ\big(f\otimes g\otimes Y\big)\circ\gamma_{V\otimes Y,U\otimes X}\circ\big(\gamma_{Y,V}\otimes U\otimes X\big)\Big]\bigg]\Bigg]\\
  &\quad=\Tr^{V\otimes U}_{X,Y}\bigg[\Tr^Y_{V\otimes U\otimes X,V\otimes U\otimes Y}\Big[\big(Y\otimes V\otimes g\otimes Y\big)\circ\big(\gamma_{V,Y}\otimes V\otimes Y\big)\circ\gamma_{V\otimes Y,V\otimes Y}\circ\big(\gamma_{Y,V}\otimes V\otimes Y\big)\circ\big(Y\otimes V\otimes f\big)\Big]\bigg]\\
  &\quad=\Tr^{V\otimes U}_{X,Y}\bigg[\big(V\otimes g\otimes Y\big)\circ\Tr^Y_{V\otimes V\otimes Y,V\otimes V\otimes Y}\Big[\big(\gamma_{V,Y}\otimes V\otimes Y\big)\circ\gamma_{V\otimes Y,V\otimes Y}\circ\big(\gamma_{Y,V}\otimes V\otimes Y\big)\Big]\circ\big(V\otimes f\big)\bigg]\\
  &\quad=\Tr^{V\otimes U}_{X,Y}\bigg[\big(V\otimes g\otimes Y\big)\circ\big(\gamma_{V,V}\otimes Y\big)\circ\big(V\otimes f\big)\bigg].
 \end{align*}
 Applying Dinaturality I' again we get
 \begin{align*}
  &\Tr^{U\otimes V}_{X,Y}\bigg[\big(\gamma_{V,U}\otimes Y\big)\circ\big(V\otimes g\otimes Y\big)\circ\big(\gamma_{V,V}\otimes Y\big)\circ\big(V\otimes f\big)\circ\big(\gamma_{U,V}\otimes X\big)\bigg]\\
  &=\Tr^{U\otimes V}_{X,Y}\bigg[\big(g\otimes f\big)\circ\big(\gamma_{U,V}\otimes X\big)\bigg]\\
  &=\Tr^{U\otimes V}_{X,Y}\bigg[\big(U\otimes f\big)\circ\big(\gamma_{U,U}\otimes X\big)\circ\big(U\otimes g\otimes X\big)\bigg]\\
  &=\Tr^{U\otimes V}_{X,Y}\bigg[\big(U\otimes f\big)\circ\Tr^X_{U\otimes U\otimes X,U\otimes U\otimes X}\Big[\big(\gamma_{U,X}\otimes U\otimes X\big)\circ\gamma_{U\otimes X, U\otimes X}\circ\big(\gamma_{X,U}\otimes U\otimes X\big)\Big]\circ\big(U\otimes g\otimes X\big)\bigg]
 \end{align*}
 Now using Naturality and Vanishing II we get
 \begin{align*}
  &\Tr^{U\otimes V}_{X,Y}\bigg[\Tr^X_{U\otimes V\otimes X,U\otimes V\otimes Y}\Big[\big(X\otimes U\otimes f\big)\circ\big(\gamma_{U,X}\otimes U\otimes X\big)\circ\gamma_{U\otimes X, U\otimes X}\circ\big(\gamma_{X,U}\otimes U\otimes X\big)\circ\big(X\otimes U\otimes g\otimes X\big)\Big]\bigg]\\
  &=\Tr^V_{X,Y}\Bigg[\Tr^U_{V\otimes X,V\otimes Y}\bigg[\Tr^X_{U\otimes V\otimes X,U\otimes V\otimes Y}\Big[\big(\gamma_{U,X}\otimes V\otimes Y\big)\circ\big(g\otimes X\otimes f\big)\circ\gamma_{U\otimes X, V\otimes X}\circ\big(\gamma_{X,U}\otimes V\otimes X\big)\Big]\bigg]\Bigg]\\
  &=\Tr^V_{X,Y}\bigg[\Tr^{X\otimes U}_{V\otimes X,V\otimes Y}\Big[\big(\gamma_{U,X}\otimes V\otimes Y\big)\circ\big(g\otimes X\otimes f\big)\circ\gamma_{U\otimes X, V\otimes X}\circ\big(\gamma_{X,U}\otimes V\otimes X\big)\Big]\bigg]
 \end{align*}
 which by Dinaturality I' is equal to
 \[\Tr^V_{X,Y}\bigg[\Tr^{U\otimes X}_{V\otimes X,V\otimes Y}\Big[\big(g\otimes X\otimes f\big)\circ\gamma_{U\otimes X, V\otimes X}\Big]\bigg],\]
 but this is exactly the composition formula applied to the right hand side of Dinaturality I.}%end erase
\end{proof}

\begin{definition}

Let $(\cat{M},\otimes,\tensor*[^{\cat{M}}]{\Tr}{})$ and $(\cat{N},\odot,\tensor*[^{\cat{N}}]{\Tr}{})$ be traced categories. A \emph{traced functor} is a strong monoidal functor $F\colon\cat{M}\to\cat{N}$, such that for any objects $X,Y,U\in\cat{M}$, the following diagram commutes:
$$
\begin{tikzcd}
	\Hom_{\cat{M}}(X\otimes U,Y\otimes U)
		\rar{F}
		\dar[swap]{\tensor*[^{\cat{M}}]{\Tr}{^U_{X,Y}}}
	& \Hom_{\cat{N}}(FX\odot FU,FY\odot FU)
		\dar{\tensor*[^{\cat{N}}]{\Tr}{^{FU}_{FX,FY}}} \\
	\Hom_{\cat{M}}(X,Y)
		\rar[swap]{F}
	& \Hom_{\cat{N}}(FX,FY)
\end{tikzcd}
$$
We generally suppress the upper left subscripts from the trace symbols, if no confusion is likely to arise.

\end{definition}

\section{The Int construction}

For a traced category $\cat{M}$ let $\widetilde{\cat{M}}=\Int(\cat{M})$ denote the category with objects given by pairs $(\inp{X},\outp{X})$ where $\inp{X},\outp{X}\in \Ob(\cat{M})$ and morphisms given by 
\[\Hom_{\widetilde{\cat{M}}}\big((\inp{X},\outp{X}),(\inp{Y},\outp{Y})\big)=\Hom_{\cat{M}}(\inp{X}\otimes \outp{Y},\outp{X}\otimes \inp{Y}).\]
For morphisms $\Phi:(\inp{X},\outp{X})\to(\inp{Y},\outp{Y})$ and $\Psi:(\inp{Y},\outp{Y})\to(\inp{Z},\outp{Z})$ in $\widetilde{\cat{M}}$ we define their composition to be
\[\Psi\circ\Phi:=\Tr^{\outp{Y}}_{\inp{X}\otimes \outp{Z},\outp{X}\otimes \inp{Z}}\Big[\big(\gamma_{\outp{X},\outp{Y}}\otimes \inp{Z}\big)\circ\big(\outp{X}\otimes\Psi\big)\circ\big(\Phi\otimes \outp{Z}\big)\circ\big(\gamma_{\outp{Y},\inp{X}}\otimes \outp{Z}\big)\Big].\]
It is well known that $\widetilde{\cat{M}}$ is a compact  category whose tensor is given by
\[(\inp{X},\outp{X})\odot(\inp{Y},\outp{Y}):=(\inp{X}\otimes \inp{Y},\outp{X}\otimes \outp{Y})\]
with unit object $\tilde I:=(I,I)$ and duality $(\inp{X},\outp{X})^\vee:=(\outp{X},\inp{X})$.  The following is immediate from the definitions.

\begin{lemma}\todo{This lemma has a sign-error. Either change definition of $\Hom_{\widetilde{\cat{M}}}$ or lemma statement.}

Let $\cat{M}$ be a traced category.  Then for any object $(\inp{X},\outp{X})\in\Ob\big(\widetilde{\cat{M}}\big)$ there is a canonical bijection
\[|(\inp{X},\outp{X})|\iso\Hom_{\cat{M}}(\inp{X},\outp{X}).\]

\end{lemma}

\section{Free Constructions}

To continue we recall the constructions of various free monoidal structures on a category $\cat{C}$ following \cite{abramsky}.  These are defined inductively by successively including additional structure into the constructions.

The objects of the free monoidal category $F_M(\cat{C})$ are lists of objects in $\cat{C}$ with componentwise morphisms.  More formally the objects are pairs $(n,X)$ where $n\in\NN$ and $X$ is a map $X\colon[n]\to \Ob\cat{C}$ and a morphism $f:(n,X)\to(m,Y)$ exists if and only if $n=m$ in which case $f\in\prod_{i=1}^n\Hom_{\cat{C}}(X_i,Y_i)$ with compositions formed as expected.  The tensor product is given by concatenation of lists, i.e. $(n,X)\otimes(m,Y)=(n+m,X+Y)$, with tensor unit $(0,!)$ where $!$ is the unique function from the empty set.

The objects of the free symmetric monoidal category $F_{SM}(\cat{C})$ are again lists of objects in $\cat{C}$, however morphisms now come equipped with a permutation.  More formally, a morphism from $(n,X)$ to $(n,Y)$ is a pair $(f,\pi)$ where $f\in\prod_{i=1}^n\Hom_{\cat{C}}(X_i,Y_{\pi(i)})$ and $\pi\in S(n)$ is a permutation which by abuse of notation can be thought of as an isomorphism $\pi:Y_1\otimes\cdots\otimes Y_n\To{\sim} Y_{\pi(1)}\otimes\cdots\otimes Y_{\pi(n)}$.  The composition of morphisms is given by 
$$(\sigma,g)\circ(\pi,f):=\Big(\sigma\circ\pi,\otimes_{i=1}^n (g_{\pi(i)}\circ f_i)\Big)$$ 
and the tensor is given by 
$$(\pi,f)\otimes(\sigma,g):=(\pi\otimes\sigma,f\otimes g).$$  


\section{The equivalence of $\Cat{TrCat}_{\Ob=\cat{O}}$ and $(\Cat{Cob}/\cat{O})\alg$}\label{sec:first equivalence}

Let $\cat{M}$ be a traced category with objects $\cat{O}$. We will define a $\Cat{Cob}/\cat{O}$-algebra $\cat{P}=R(\cat{M})\colon\Cat{Cob}/\cat{O}\to\Cat{Set}$ as follows. For an object $X\in\Ob(\Cat{Cob}/\cat{O})$, set 
$$\cat{P}(X):=\Hom_{\cat{M}}(\vinp{X},\voutp{X}).$$
We next consider morphisms.

Following Proposition~\ref{prop:set theoretic cob1} (borrow notation/setup from Abramsky instead) a morphism $\Phi\colon X\longrightarrow Y$ consists of a typed bijection 
$$\varphi\colon\inp{X}\sqcup \outp{Y}\xrightarrow{\iso}\outp{X}\sqcup \inp{Y},$$ 
together with a typed finite set $S$. Given an element $f\in\cat{P}(X)$ we must construct $\cat{P}(\Phi)(f)\in\cat{P}(Y)$. Let $\dim(\overline{S})=\textnormal{Tr}^{\overline{S}}_{I,I}\big[\id_{\overline{S}}\big]\in\cat{S}_\cat{M}$. Then we use the formula
$$\cat{P}(\Phi)(f):=
\textnormal{Tr}^{\voutp{X}}_{\vinp{Y},\voutp{Y}}\Big[\big(f\otimes\id_{\voutp{Y}}\big)\circ\overline{\varphi}\Big]
\otimes\dim(\overline{S}).	
$$

\begin{theorem}
 For any set $\cat{O}$, the category $\Cat{TrCat}_{\Ob=\cat{O}}$ is equivalent to the category of $\Cat{Cob}/\cat{O}$-algebras.
\end{theorem}
\begin{proof}
 
\end{proof}

\begin{corollary}
 Enriched setting?\todo{So far, we have not found references for enriched traced or compact categories, so this part may be moot.}
\end{corollary}


