% -*- root: CCC_Note.tex -*-
\chapter{Introduction}

Traced symmetric monoidal categories have been used to model processes with feedback or operators with fixed points (Ponto, Shulman). A graphical calculus for TSMCs was developed by Joyal, Street, and Verity, in which string diagrams of the form
\begin{center}\missingfigure[figwidth=3in]{String diagram with labeled wires}\end{center}
represent compositions, i.e., new morphisms are constructed from old by specifying which outputs will be fed back into which inputs. In fact, these generalize Penrose diagrams in $\Cat{Vect}$, and the word \emph{traced} originates in vector space terminology.  

But notice that the above picture has another interpretation, in terms of 1-dimensional cobordisms between oriented 0-manifolds. A box in the picture includes only the data of two finite sets $(\inp{X},\outp{X})$, drawn as input wires on the left and output wires on the right, which can be interpreted as an oriented 0-manifold. A string diagram then consists of boxes $X_1,\ldots,X_n$ wired together inside a larger box $Y$, and can be interpreted as a cobordism from $X_1\sqcup\cdots\sqcup X_n$ to $Y$. 

There is actually a bit more data in a string diagram for a TSMC $\cat{C}$; namely, each wire is labeled by an element of $\cat{O}:=\Ob(\cat{C})$. The category of cobordisms is thus taken relative to a fixed set $\cat{O}$, i.e. 0-manifolds and cobordisms are coherently labeled by objects of $\cat{C}$.

We record these interpretations of string diagrams below.

\begin{center}
\begin{tabular}{| l | l | l |}
\hline
\multicolumn{3}{|c|}{Interpretations of string diagrams}\\\hline\hline
String diagram & Traced category $\cat{C}$ & $\Cat{Cob}/\cat{O}$\\\hline
Boxes & Morphisms & Labeled 0-manifolds\\
Diagram & Pasting diagram & Coherent cobordism\\
Nesting & Axioms of TSMCs & Composition law\\\hline
\end{tabular}
\end{center}

The relationship between these interpretations is made precise in the following first main theorem.
\begin{theorem}
 The category $\Cat{TSMC}$ is equivalent to the category of $\Cat{Cob}/\cat{O}$-algebras.
\end{theorem}

This precise connection between objects in $\Cat{Cob}/\cat{O}$ and morphisms in the traced category $\cat{C}$ is made by a $\Cat{Cob}/\cat{O}$-algebra, i.e. a lax functor $P\colon\Cat{Cob}/\cat{O}\to\Cat{Set}$. For each object box $(\inp{X},\outp{X})$ in a string diagram, there is a set of morphisms in $\cat{C}$, roughly $\Hom_{\cat{C}}(\inp{X},\outp{X})$, which can fill the box. The functor $P$ assigns to each morphism in $\Cat{Cob}/\cat{O}$ a pasting diagram, which encodes a collection of compositions, monoidal products, and traces in $\cat{C}$.  Finally the functor $P$ provides a method for evaluating the pasting diagram on a set of morphisms of $\cat{C}$ filling the inner boxes to obtain a morphism filling the outer box, this encodes all of the axioms which specify how compositions, monoidal products, and traces must interact in $\cat{C}$. This process is intimately connected to the composition law in $\Cat{Cob}/\cat{O}$ which thus corresponds roughly to the set of axioms for the trace in $\cat{C}$.

\section{Nesting properties and self-similarity in applications}





\section{Generalization}

\begin{theorem}
 Let $\cat{C}$ be a compact closed category and $\cat{V}$ a symmetric monodical category.  The category $\Lax(\cat{C},\cat{V})$ of lax monodical functors is equivalent to the coslice category $\cat{C}/\cat{V}-\Cat{CompCat}$ spanned by bijective on objects functors.
\end{theorem}
\begin{corollary}
 $\Lax(\Int(\cat{T}),\cat{V})=\cat{V}-\Cat{TSMC}_{\cat{T}/}$
\end{corollary}

\chapter{Wiring Diagrams and $1-\Cat{Cob}$}
\section{Set-theoretic formulation of $1-\Cat{Cob}$, as free compact closed category on one object, as $\Int$ of the free TSMC on one object ($\Cat{Bij}$)}
\subsection{Many object case/generalization $\Cat{Cob}/\cat{O}$}

\section{Drawings of morphisms in $1-\Cat{Cob}$ as wiring diagrams, new way to visualize these}

\section{$1-\Cat{Cob}$-algebras and applications}

\section{Definition of functors between $\Cat{TSMC}$ and $\Cat{Cob}/\cat{O}$-algebras}

Let $\cat{M}$ be a traced symmetric monoidal category with objects $\cat{O}$. We will define a $\Cat{Cob}/\cat{O}$-algebra $\cat{P}=R(\cat{M})\colon\Cat{Cob}/\cat{O}\to\Cat{Set}$ as follows. For an object $X\in\Ob(\Cat{Cob}/\cat{O})$, set 
$$\cat{P}(X):=\Hom_{\cat{M}}(\vinp{X},\voutp{X}).$$
We next consider morphisms.

Following Proposition~\ref{prop:set theoretic cob1} (borrow notation/setup from Abramsky instead) a morphism $\Phi\colon X\longrightarrow Y$ consists of a typed bijection 
$$\varphi\colon\inp{X}\sqcup \outp{Y}\xrightarrow{\iso}\outp{X}\sqcup \inp{Y},$$ 
together with a typed finite set $S$. Given an element $f\in\cat{P}(X)$ we must construct $\cat{P}(\Phi)(f)\in\cat{P}(Y)$. Let $\dim(\overline{S})=\textnormal{Tr}^{\overline{S}}_{I,I}\big[\id_{\overline{S}}\big]\in\cat{S}_\cat{M}$. Then we use the formula
$$\cat{P}(\Phi)(f):=
\textnormal{Tr}^{\overline{X_+}}_{\overline{Y_-},\overline{Y_+}}\Big[\big(f\otimes\id_{\overline{Y_+}}\big)\circ\overline{\varphi}\Big]
\otimes\dim(\overline{S}).	
$$

\begin{theorem}
 The category $\Cat{TSMC}$ is equivalent to the category of $\Cat{Cob}/\cat{O}$-algebras.
\end{theorem}
\begin{proof}
 
\end{proof}
\begin{corollary}
 Enriched setting?
\end{corollary}


