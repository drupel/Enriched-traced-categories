% !TEX root = ./TracedCobAlg.tex
\usepackage{mathtools}
\usepackage{amsthm}
\usepackage{amssymb}
\usepackage{fixltx2e}
\usepackage[T1]{fontenc}
\usepackage{newpxtext}
\usepackage[varg,bigdelims,varbb]{newpxmath}%[varbb] causes problems with hypersetup{final}
\linespread{1.05}
\usepackage[cal=euler,scr=rsfso]{mathalfa}
\usepackage{bm}
\usepackage{inputenc}
\usepackage{microtype}
\usepackage[usenames,dvipsnames]{xcolor}
\usepackage{paralist}
\usepackage{booktabs}
\usepackage{tikz}
\usepackage{todonotes}
\usepackage[bookmarks=true,colorlinks=true, linkcolor=MidnightBlue, citecolor=cyan]{hyperref}

\usetikzlibrary{decorations.markings,arrows.meta,calc,fit,quotes,cd}
\hypersetup{final} 

\DeclareMathOperator{\id}{id}
\DeclareMathOperator{\dom}{dom}
\DeclareMathOperator{\cod}{cod}
\DeclareMathOperator{\dvert}{Vert}
\DeclareMathOperator{\Lax}{Lax}
\DeclareMathOperator{\Hom}{Hom}
\DeclareMathOperator{\Mor}{Mor}
\DeclareMathOperator{\Ob}{Ob}
\DeclareMathOperator{\MOb}{|\cdot|}
\DeclareMathOperator{\Tr}{Tr}


\theoremstyle{plain}
\newtheorem{theorem}{Theorem}[section]
\newtheorem*{theorem*}{Theorem}

\newtheorem{proposition}[theorem]{Proposition}
\newtheorem{corollary}[theorem]{Corollary}
\newtheorem{lemma}[theorem]{Lemma}
\newtheorem*{lemma*}{Lemma}

\newtheorem*{namedthm}{\namedthmname}
\newcounter{namedthm}
\makeatletter
\newenvironment{named}[1]
  {\def\namedthmname{#1}%
   \refstepcounter{namedthm}%
   \namedthm\def\@currentlabel{#1}}
  {\endnamedthm}
\makeatother

\theoremstyle{definition}
\newtheorem{definition}[theorem]{Definition}
\newtheorem{exercise}{Exercise}[section]

\theoremstyle{remark}
\newtheorem{example}[theorem]{Example}
\newtheorem{remark}[theorem]{Remark}
\newtheorem{warning}[theorem]{Warning}


\newcommand{\prodb}{\mathbin{\Pi}}
\newcommand{\iso}{\cong}
\renewcommand{\equiv}{\simeq}

\newcommand{\cat}[1]{\mathscr{#1}} %category variable
\newcommand{\ncat}[1]{\mathbf{#1}} %named category
\newcommand{\ccat}[1]{\mathcal{#1}} % 2-category variable
%\newcommand{\dcat}[1]{\pmb{\mathbb{#1}}} %double category variable
\newcommand{\dcat}[1]{\mathbb{#1}} %double category variable
\newcommand{\fun}[1]{#1}               %functor variable
\newcommand{\nfun}[1]{\mathbf{#1}}  %named functor
\newcommand{\nncat}[2]{\bm{\ccat{#1}}\ncat{#2}} %named 2-category
\newcommand{\ndcat}[2]{\dcat{#1}\ncat{#2}}

\newcommand{\MTCfont}[1]{\mathsf{#1}} %Use this font for labeling traced, compact, monoidal, etc.
\newcommand{\FM}{\nfun{F}_{\MTCfont{M}}}
\newcommand{\UM}{\nfun{U}_{\MTCfont{M}}}
\newcommand{\FT}{\nfun{F}_{\MTCfont{T}}}
\newcommand{\UT}{\nfun{U}_{\MTCfont{T}}}
\newcommand{\FC}{\nfun{F}_{\MTCfont{C}}}
\newcommand{\UC}{\nfun{U}_{\MTCfont{C}}}
\newcommand{\TM}{\nfun{T}_{\MTCfont{M}}}
\newcommand{\TT}{\nfun{T}_{\MTCfont{T}}}
\newcommand{\TC}{\nfun{T}_{\MTCfont{C}}}
\newcommand{\UCM}{\nfun{U}_{\MTCfont{CM}}}
\newcommand{\UCT}{\nfun{U}_{\MTCfont{CT}}}
\newcommand{\UTM}{\nfun{U}_{\MTCfont{TM}}}
\newcommand{\FMC}{\nfun{F}_{\MTCfont{MC}}}

\newcommand{\Set}{\ncat{Set}} % 1-cat Set
\newcommand{\Span}{\nncat{S}{pan}} % bicat Span
\newcommand{\dSpan}{\dcat{S}\ncat{pan}} % double cat Span
\newcommand{\Cat}{\ncat{Cat}} % 1-cat Cat
\newcommand{\CCat}{\nncat{C}{at}} % F-cat Cat
\newcommand{\Prof}{\nncat{P}{rof}} % bicat Prof(unctors)
\newcommand{\dProf}{\ndcat{P}{rof}} % double cat Prof(unctors)
\newcommand{\MonCat}{\ncat{MonCat}}
\newcommand{\MMonCat}{\nncat{M}{onCat}} % 2-cat MonCat
\newcommand{\MMonCatLax}{\MMonCat_{\mathrm{lax}}}
%\newcommand{\SSymMonCat}{\nncat{S}{ymMonCat}} % 2-cat SymMonCat
\newcommand{\SSymMonCat}{\MMonCat}
\newcommand{\MonProf}{\nncat{M}{onProf}} % bicat MProf (monoidal Prof)
\newcommand{\dMonProf}{\ndcat{M}{onProf}} % double cat MProf (monoidal Prof)
\newcommand{\MProf}{\nncat{M}{onProf}} % just to get old version to compile -- can delete
\newcommand{\dMProf}{\ndcat{M}{Prof}} % just to get old version to compile -- can delete
\newcommand{\FrMonCat}{\ncat{FMonCat}}
\newcommand{\FFrMonCat}{\nncat{F}{MonCat}}
\newcommand{\FrMonProf}{\nncat{F}{MonProf}}
\newcommand{\dFrMonProf}{\ndcat{F}{MonProf}}
\newcommand{\dTrProf}{\ndcat{T}{rProf}}
\newcommand{\dFrTrProf}{\ndcat{F}{TrProf}}
\newcommand{\CompCat}{\ncat{CompCat}} % 1-cat CompCat
\newcommand{\CCompCat}{\nncat{C}{ompCat}} % 2-cat CompCat
\newcommand{\CompProf}{\nncat{C}{ompProf}} % bicat CompProf
\newcommand{\TrProf}{\nncat{T}{rProf}} % bicat CompProf
\newcommand{\dCompProf}{\ndcat{C}{ompProf}} % bicat CompProf
\newcommand{\FrCompCat}{\ncat{FCompCat}}
\newcommand{\FFrCompCat}{\nncat{F}{CompCat}}
\newcommand{\FrCompProf}{\nncat{F}{CompProf}}
\newcommand{\dFrCompProf}{\ndcat{F}{CompProf}}
\newcommand{\dFrMonCompProf}{\ndcat{F}{MCompProf}}
\newcommand{\TrCat}{\ncat{TrCat}} % 1-cat TrCat
\newcommand{\TTrCat}{\nncat{T}{rCat}} % 2-cat TrCat
\newcommand{\TTrFrObCat}{\nncat{T}{rFrObCat}}
% \def\2Cat{\tn{$2$-$\ncat{Cat}$}}


\newcommand{\Int}{\nfun{Int}}
\newcommand{\List}{\nfun{List}}
\newcommand{\TrFrObCat}{\nfun{TrFrObCat}}
\newcommand{\TrFr}{\nfun{TrFr}}
\newcommand{\CompFr}{\nfun{CompFr}}
\newcommand{\End}{\ncat{End}}
\newcommand{\Ptd}{\ncat{Ptd}} % 1-cat Pointed endo-bimodules
\newcommand{\PPt}{\bm{\ccat{P}}\ncat{t}} % 2-cat Pointed endo-bimodules
\newcommand{\Mon}{\ncat{Mon}} % 1-cat Mon(oids/ads)
\newcommand{\MMon}{\bm{\ccat{M}}\ncat{on}} % 2-cat Mon(oids/ads)
\newcommand{\Mod}{\bm{\ccat{M}}\ncat{od}} % bicat cat Mod(ules)
\newcommand{\dMod}{\dcat{M}\ncat{od}} % double cat Mod(ules)
\newcommand{\Ver}{\ncat{Vert}} % vertical 1-cat
\newcommand{\VVer}{\bm{\ccat{V}}\ncat{ert}} % vertical 2-cat
\newcommand{\HHor}{\bm{\ccat{H}}\ncat{or}} % horizontal bicat
\newcommand{\Psh}{\ncat{Psh}}
\newcommand{\CPsh}{\ncat{CPsh}}
\newcommand{\Emb}{\mathrm{Emb}} % set of embeddings

\newcommand{\Col}[1]{\langle#1\rangle}

\newcommand{\Cob}{\ncat{Cob}}
\newcommand{\LCob}[1]{\Cob_{/#1}}
\newcommand{\CobSetAlg}{\int^{\Set}(\LCob{\bullet})\alg}
\newcommand{\CobKlsAlg}{\int^{\cat{O}\in\Set_{\TM}}(\LCob{\cat{O}})\alg}
\newcommand{\op}[1]{{#1}^{\text{op}}}
\newcommand{\vop}[1]{{#1}^{\text{vop}}}
\newcommand{\hop}[1]{{#1}^{\text{hop}}}

\newcommand{\Alg}{\mathrm{Alg}}
\newcommand{\Coalg}{\mathrm{Coalg}}
\newcommand{\RAlg}[1][]{\mathbb{R}_{#1}\text{-}\Alg}
\newcommand{\LCoalg}[1][]{\mathbb{L}_{#1}\text{-}\Coalg}
\newcommand{\LCoalgA}{\mathbb{L}_1\text{-}\Coalg}
\newcommand{\LCoalgB}{\mathbb{L}_2\text{-}\Coalg}
\newcommand{\Bimod}{\mathrm{Bimod}}

\newcommand{\tickar}{\begin{tikzcd}[baseline=-0.5ex,cramped,sep=small,ampersand replacement=\&]{}\ar[r,tick]\&{}\end{tikzcd}}

\newcommand{\twocell}[3][]{\arrow[draw=none,to path={(dom#2.center)--(cod#2.center)\tikztonodes}]{}[anchor=center,#1]{\Downarrow #3}}
\newcommand{\twocellalt}[3][]{\arrow[draw=none,to path={(dom#2.center)--(cod#2.center)\tikztonodes}]{}[anchor=center,#1]{#3}}
\newcommand{\twocellA}[2][]{\twocell[#1]{A}{#2}}
\newcommand{\twocellB}[2][]{\twocell[#1]{B}{#2}}
\newcommand{\twocellC}[2][]{\twocell[#1]{C}{#2}}
\newcommand{\twocellD}[2][]{\twocell[#1]{D}{#2}}
\newcommand{\twocellE}[2][]{\twocell[#1]{E}{#2}}
\newcommand{\twocellF}[2][]{\twocell[#1]{F}{#2}}



\tikzcdset{
	arrow style=tikz,
	diagrams={>={Classical TikZ Rightarrow[angle=63:4pt, line width=.6pt]}},
	arrows={semithick}
}

\tikzset{tick/.style={postaction={decorate,decoration={markings,mark=at position 0.5 with {\draw[-] (0,.4ex) -- (0,-.4ex);}}}}}
\tikzset{dom/.style={append after command={coordinate[alias=dom#1]}},
		domA/.style={dom=A}, domB/.style={dom=B},
		domC/.style={dom=C}, domD/.style={dom=D},
		domE/.style={dom=E}, domF/.style={dom=F}}
\tikzset{cod/.style={append after command={coordinate[alias=cod#1]}},
		codA/.style={cod=A}, codB/.style={cod=B},
		codC/.style={cod=C}, codD/.style={cod=D},
		codE/.style={cod=E}, codF/.style={cod=F}}


\tikzset{
	wiring diagram/.style={
		every to/.style={out=0,in=180,draw},
		label/.style={
			font=\everymath\expandafter{\the\everymath\scriptstyle},
			inner sep=0pt,
			node distance=2pt and -2pt},
		semithick,
		node distance=1 and 1,
		decoration={markings, mark=at position .5 with {\arrow{stealth};}},
		ar/.style={postaction={decorate}},
		execute at begin picture={\tikzset{
			x=\bbx, y=\bby,
			every fit/.style={inner xsep=\bbx, inner ysep=\bby}}}
		},
	bbx/.store in=\bbx,
	bbx = 1.5cm,
	bby/.store in=\bby,
	bby = 1.75ex,
	bb port sep/.store in=\bbportsep,
	bb port sep=2,
	% bb wire sep/.store in=\bbwiresep,
	% bb wire sep=1.75ex,
	bb port length/.store in=\bbportlen,
	bb port length=4pt,
	bb min width/.store in=\bbminwidth,
	bb min width=1cm,
	bb rounded corners/.store in=\bbcorners,
	bb rounded corners=2pt,
	bb small/.style={bb port sep=1, bb port length=2.5pt, bbx=.4cm, bb min width=.4cm, bby=.7ex},
	bb/.code 2 args={
		\pgfmathsetlengthmacro{\bbheight}{\bbportsep * (max(#1,#2)+1) * \bby}
		\pgfkeysalso{draw,minimum height=\bbheight,minimum width=\bbminwidth,outer sep=0pt,
			rounded corners=\bbcorners,thick,
			prefix after command={\pgfextra{\let\fixname\tikzlastnode}},
			append after command={\pgfextra{\draw
				\ifnum #1=0{} \else foreach \i in {1,...,#1} {
					($(\fixname.north west)!{\i/(#1+1)}!(\fixname.south west)$) +(-\bbportlen,0) coordinate (\fixname_in\i) -- +(\bbportlen,0) coordinate (\fixname_in\i')}\fi
				\ifnum #2=0{} \else foreach \i in {1,...,#2} {
					($(\fixname.north east)!{\i/(#2+1)}!(\fixname.south east)$) +(-\bbportlen,0) coordinate (\fixname_out\i') -- +(\bbportlen,0) coordinate (\fixname_out\i)}\fi;
			}}}
	},
	bb name/.style={append after command={\pgfextra{\node[anchor=north] at (\fixname.north) {#1};}}}
}



\newcommand{\vinp}[1]{\overline{\inp{#1}}}
\newcommand{\voutp}[1]{\overline{\outp{#1}}}
%\newcommand{\inp}[1]{#1^{\textnormal{in}}}
%\newcommand{\outp}[1]{#1^{\textnormal{out}}}
\newcommand{\inp}[1]{#1^-}
\newcommand{\outp}[1]{#1^+}

% \def\bhline{\Xhline{2\arrayrulewidth}}
% \def\bbhline{\Xhline{2.5\arrayrulewidth}}
\def\alg{{\text \textendash}\ncat{Alg}}
\def\XCat{\ncat{XCat}}
\def\ul{\underline}
\def\List{\textnormal{List}}
\def\SList{\textnormal{SList}}
\def\SSList{\textnormal{SSList}}

\newcommand{\erase}[1]{{}}
\def\NN{\mathbb{N}}
\def\ss{\subseteq}
\newcommand{\bo}{\mathsf{bo}}
\newcommand{\ff}{\mathsf{ff}}

%sets:
\newcommand{\feeddd}[3]{{\tensor*[^{#2}_{\color{white}{!}}]{{|#1|}}{^{#3}}}}%the color thing is to get overlines to be the same height.
\newcommand{\feeddc}[3]{{\tensor*[^{#2}]{{|#1|}}{_{#3}}}}
\newcommand{\feedcd}[3]{{\tensor*[_{#2}]{{|#1|}}{^{#3}}}}
\newcommand{\feedcc}[3]{{\tensor*[^{\color{white}{!}}_{#2}]{{|#1|}}{_{#3}}}}
%maps
\newcommand{\feeddb}[2]{{\tensor*[^{#2}]{{#1}}{}}}
\newcommand{\feedbc}[2]{{\tensor*{{#1}}{^~_{#2}}}}
\newcommand{\feedcb}[2]{{\tensor*[^~_{#2}]{{#1}}{}}}
\newcommand{\feedbd}[2]{{\tensor*{{#1}}{^{#2}}}}
%horrible maps
\newcommand{\feedda}[3]{{\tensor*[^{#2}_{\color{white}{!}}]{{#1}}{^{#2}_{#3}}}}
\newcommand{\feedca}[3]{{\tensor*[_{#2}]{{#1}}{_{#2}^{#3}}}}
\newcommand{\feedad}[3]{{\tensor*[^{#2}_{#3}]{{#1}}{^{#2}}}}
\newcommand{\feedac}[3]{{\tensor*[_{#2}^{#3}]{{#1}}{_{#2}}}}


\newcommand{\into}{\hookrightarrow}
\newcommand{\onto}{\to\!\!\!\!\!\to}
\def\tn{\textnormal}
\newcommand{\To}[1]{\xrightarrow{#1}}
\newcommand{\Too}[1]{\To{\ \ #1\ \ }}
\newcommand{\from}{\leftarrow}
\newcommand{\tto}{\twoheadrightarrow}
\newcommand{\inj}{\hookrightarrow}
\newcommand{\cev}[1]{\reflectbox{\ensuremath{\vec{\reflectbox{\ensuremath{#1}}}}}}
\newtheorem{claim}[subsubsection]{Claim}
\def\bfe{{\textbf{e}}}
\def\bfo{{\textbf{o}}}
\newcommand{\start}[1]{\textbf{Start here: #1}\\}
\newcommand{\too}{\longrightarrow}

\allowdisplaybreaks
