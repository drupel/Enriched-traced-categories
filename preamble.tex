% -*- root: CCC Note.tex -*-

\usepackage{amsmath}
\usepackage{amsthm}
\usepackage{amsfonts}
\usepackage{amssymb}
\usepackage{mathtools}
%\usepackage{datetime}
\usepackage[T1]{fontenc}
\usepackage[sc]{mathpazo}
\linespread{1.05}
\usepackage{mathrsfs}
\usepackage{euscript}
\usepackage{MnSymbol}
\usepackage{paralist}
\usepackage{tikz}
\usetikzlibrary{cd}

\usetikzlibrary{decorations.markings,arrows.meta}

\DeclareMathOperator{\id}{id}
\DeclareMathOperator{\dom}{dom}
\DeclareMathOperator{\cod}{cod}
\DeclareMathOperator{\dvert}{Vert}
\DeclareMathOperator{\Lax}{Lax}
\DeclareMathOperator{\Hom}{Hom}
\DeclareMathOperator{\Prof}{Prof}
\DeclareMathOperator{\MProf}{MProf}
\DeclareMathOperator{\Int}{Int}
\DeclareMathOperator{\Ob}{Ob}


\theoremstyle{plain}
\newtheorem{theorem}{Theorem}[chapter]
\newtheorem*{theorem*}{Theorem}
\newtheorem{proposition}[theorem]{Proposition}
\newtheorem{corollary}[theorem]{Corollary}
\newtheorem{lemma}[theorem]{Lemma}
\newtheorem*{lemma*}{Lemma}

\theoremstyle{definition}
\newtheorem{definition}[theorem]{Definition}
\newtheorem{exercise}{Exercise}[chapter]

\theoremstyle{remark}
\newtheorem{example}[theorem]{Example}
\newtheorem{remark}[theorem]{Remark}

\newcommand{\prodb}{\mathbin{\Pi}}
\newcommand{\iso}{\cong}

\newcommand{\cat}[1]{\mathscr{#1}}
\newcommand{\Cat}[1]{\mathbf{#1}}
%\newcommand{\hom}{\mathrm{hom}}
\newcommand{\twocat}[1]{\mathcal{#1}}
\newcommand{\dblcat}[1]{\mathbb{#1}}
\newcommand{\btwo}{\mathbf{2}}
\newcommand{\FF}{\mathbb{F}\Cat{F}}
\newcommand{\FFD}{\FF(\dblcat{D})}
\newcommand{\Mon}{\Cat{Mon}}
\newcommand{\DMon}{\mathbb{M}\Cat{on}}
\newcommand{\Comon}{\Cat{Comon}}
\newcommand{\DComon}{\mathbb{C}\Cat{omon}}
\newcommand{\Bimon}{\Cat{Bimon}}
\newcommand{\Sq}{\mathbb{S}\Cat{q}}
\newcommand{\Span}{\mathbb{S}\Cat{pan}}
\newcommand{\Hor}{\twocat{H}or}
\newcommand{\LAdj}{\dblcat{L}\Cat{Adj}}
\newcommand{\RAdj}{\dblcat{R}\Cat{Adj}}
\newcommand{\MAdjC}{\Cat{MAdj}}
\newcommand{\MAdj}{\dblcat{M}\Cat{Adj}}
\newcommand{\EAdj}{\dblcat{E}\Cat{Adj}}
\newcommand{\SymMonCat}{\Cat{SymMonCat}}
\newcommand{\CompCat}{\Cat{CompCat}}
\newcommand{\Set}{\Cat{Set}}

\newcommand{\op}[1]{{#1}^{\text{op}}}
\newcommand{\vop}[1]{{#1}^{\text{vop}}}
\newcommand{\hop}[1]{{#1}^{\text{hop}}}

\newcommand{\Alg}{\mathrm{Alg}}
\newcommand{\Coalg}{\mathrm{Coalg}}
\newcommand{\RAlg}[1][]{\mathbb{R}_{#1}\text{-}\Alg}
\newcommand{\LCoalg}[1][]{\mathbb{L}_{#1}\text{-}\Coalg}
\newcommand{\LCoalgA}{\mathbb{L}_1\text{-}\Coalg}
\newcommand{\LCoalgB}{\mathbb{L}_2\text{-}\Coalg}

\newcommand{\twocell}[3][]{\arrow[draw=none,to path={(dom#2.center)--(cod#2.center)\tikztonodes}]{}[anchor=center,#1]{\Downarrow #3}}
\newcommand{\twocellalt}[3][]{\arrow[draw=none,to path={(dom#2.center)--(cod#2.center)\tikztonodes}]{}[anchor=center,#1]{#3}}
\newcommand{\twocellA}[2][]{\twocell[#1]{A}{#2}}
\newcommand{\twocellB}[2][]{\twocell[#1]{B}{#2}}
\newcommand{\twocellC}[2][]{\twocell[#1]{C}{#2}}
\newcommand{\twocellD}[2][]{\twocell[#1]{D}{#2}}
\newcommand{\twocellE}[2][]{\twocell[#1]{E}{#2}}
\newcommand{\twocellF}[2][]{\twocell[#1]{F}{#2}}

\tikzcdset{
	arrow style=tikz,
	diagrams={>={Classical TikZ Rightarrow[angle=63:4pt, line width=.6pt]}},
	arrows={semithick}
}

\tikzset{tick/.style={postaction={decorate,decoration={markings,mark=at position 0.5 with {\draw[-] (0,.4ex) -- (0,-.4ex);}}}}}
\tikzset{dom/.style={append after command={coordinate[alias=dom#1]}},
		domA/.style={dom=A}, domB/.style={dom=B},
		domC/.style={dom=C}, domD/.style={dom=D},
		domE/.style={dom=E}, domF/.style={dom=F}}
\tikzset{cod/.style={append after command={coordinate[alias=cod#1]}},
		codA/.style={cod=A}, codB/.style={cod=B},
		codC/.style={cod=C}, codD/.style={cod=D},
		codE/.style={cod=E}, codF/.style={cod=F}}


\newcommand{\vinp}[1]{\overline{\inp{#1}}}%Delete me later
\newcommand{\voutp}[1]{\overline{\outp{#1}}}%Delete me later
\newcommand{\inp}[1]{{#1_-}}%Delete me later
\newcommand{\outp}[1]{{#1_+}}%Delete me later

