\documentclass{amsart}

\usepackage{savesym}
\usepackage{txfonts,stmaryrd}
\savesymbol{lrcorner}\savesymbol{ulcorner}
\usepackage{amssymb, amscd,setspace,mathtools,makecell}
\usepackage{enumerate}
\usepackage[usenames,dvipsnames]{xcolor}
\usepackage[bookmarks=true,colorlinks=true, linkcolor=MidnightBlue, citecolor=cyan]{hyperref}
\usepackage{lmodern}
\usepackage{graphicx,float}
\restoresymbol{txfonts}{lrcorner}\restoresymbol{txfonts}{ulcorner}


\usepackage{tikz}
\usetikzlibrary{arrows,calc,chains,matrix,positioning,scopes,snakes}

%Begin tikz macros
\def\blackbox#1#2#3#4#5{%(width,height), number inputs, number outputs, label, arrow length
  \pgfgetlastxy{\llx}{\lly}%assumes path has been set to a point representing the lower left corner of the box
  \path #1;
  \pgfgetlastxy{\w}{\h}
  \pgfmathsetlengthmacro{\urx}{\llx+\w}
  \pgfmathsetlengthmacro{\ury}{\lly+\h}
  \draw (\llx,\lly) rectangle (\urx,\ury);
  \pgfmathsetlengthmacro{\xave}{(\llx+\urx)/2}
  \pgfmathsetlengthmacro{\yave}{\ury-8}
  \node at (\xave,\yave) {#4};
  \pgfmathsetlengthmacro{\ydiff}{\ury-\lly}
  \pgfmathsetlengthmacro{\lstep}{\ydiff/(#2+1)}
  \pgfmathsetlengthmacro{\rstep}{\ydiff/(#3+1)}
  \ifnum #2=0{}\else{ 
   \foreach \l in {1,...,#2}{
    \draw [->] ($(\llx,\lly)+(-#5/2,0)+\l*(0,\lstep)$) -- ($(\llx,\lly)+(#5/2,0)+\l*(0,\lstep)$);}}\fi
  \ifnum #3=0{}\else{
   \foreach \r in {1,...,#3}{
    \draw [->] ($(\urx,\ury)+(-#5/2,0)-\r*(0,\rstep)$) -- ($(\urx,\ury)+(#5/2,0)-\r*(0,\rstep)$);}}\fi
}

\def\blackboxinners#1#2#3#4#5{%(width,height), number inputs, number outputs, label, arrow length
  \pgfgetlastxy{\llx}{\lly}%assumes path has been set to a point representing the lower left corner of the box
  \path #1;
  \pgfgetlastxy{\w}{\h}
  \pgfmathsetlengthmacro{\urx}{\llx+\w}
  \pgfmathsetlengthmacro{\ury}{\lly+\h}
  \draw (\llx,\lly) rectangle (\urx,\ury);
  \pgfmathsetlengthmacro{\xave}{(\llx+\urx)/2}
  \pgfmathsetlengthmacro{\yave}{\ury-8}
  \node at (\xave,\yave) {#4};
  \pgfmathsetlengthmacro{\ydiff}{\ury-\lly}
  \pgfmathsetlengthmacro{\lstep}{\ydiff/(#2+1)}
  \pgfmathsetlengthmacro{\rstep}{\ydiff/(#3+1)}
  \ifnum #2=0{}\else{ 
   \foreach \l in {1,...,#2}{
    \pgfmathsetlengthmacro{\newx}{\llx+#5*28.45274/2}
    \pgfmathsetlengthmacro{\newy}{\lly+\l*\lstep}
    \node at ($(\newx,\newy)+(-1.5,\l*12-\l*\lstep)$) {\tiny$(\pgfmathparse{\newx/28.45274}\pgfmathresult cm,\pgfmathparse{\newy/28.45274}\pgfmathresult cm)$};
    \draw [->] ($(\llx,\lly)+(-#5/2,0)+\l*(0,\lstep)$) -- ($(\llx,\lly)+(#5/2,0)+\l*(0,\lstep)$);}}\fi
  \ifnum #3=0{}\else{
   \foreach \r in {1,...,#3}{
    \pgfmathsetlengthmacro{\newx}{\urx-#5*28.45274/2}
    \pgfmathsetlengthmacro{\newy}{\ury-\r*\rstep}
    \node at ($(\newx,\newy)+(1.5,-\r*12+\r*\rstep)$) {\tiny $(\pgfmathparse{\newx/28.45274}\pgfmathresult cm,\pgfmathparse{\newy/28.45274}\pgfmathresult cm)$};
    \draw [->] ($(\urx,\ury)+(-#5/2,0)-\r*(0,\rstep)$) -- ($(\urx,\ury)+(#5/2,0)-\r*(0,\rstep)$);}}\fi
}

\def\blackboxouters#1#2#3#4#5{%(width,height), number inputs, number outputs, label, arrow length
  \pgfgetlastxy{\llx}{\lly}%assumes path has been set to a point representing the lower left corner of the box
  \path #1;
  \pgfgetlastxy{\w}{\h}
  \pgfmathsetlengthmacro{\urx}{\llx+\w}
  \pgfmathsetlengthmacro{\ury}{\lly+\h}
  \draw (\llx,\lly) rectangle (\urx,\ury);
  \pgfmathsetlengthmacro{\xave}{(\llx+\urx)/2}
  \pgfmathsetlengthmacro{\yave}{\ury-8}
  \node at (\xave,\yave) {#4};
  \pgfmathsetlengthmacro{\ydiff}{\ury-\lly}
  \pgfmathsetlengthmacro{\lstep}{\ydiff/(#2+1)}
  \pgfmathsetlengthmacro{\rstep}{\ydiff/(#3+1)}
  \ifnum #2=0{}\else{ 
   \foreach \l in {1,...,#2}{
    \pgfmathsetlengthmacro{\newx}{\llx-#5*28.45274/2}
    \pgfmathsetlengthmacro{\newy}{\lly+\l*\lstep}
    \node at ($(\newx,\newy)+(-1.5,\l*12-\l*\lstep)$) {\tiny$(\pgfmathparse{\newx/28.45274}\pgfmathresult cm,\pgfmathparse{\newy/28.45274}\pgfmathresult cm)$};
    \draw [->] ($(\llx,\lly)+(-#5/2,0)+\l*(0,\lstep)$) -- ($(\llx,\lly)+(#5/2,0)+\l*(0,\lstep)$);}}\fi
  \ifnum #3=0{}\else{
   \foreach \r in {1,...,#3}{
    \pgfmathsetlengthmacro{\newx}{\urx+#5*28.45274/2}
    \pgfmathsetlengthmacro{\newy}{\ury-\r*\rstep}
    \node at ($(\newx,\newy)+(1.5,-\r*12+\r*\rstep)$) {\tiny $(\pgfmathparse{\newx/28.45274}\pgfmathresult cm,\pgfmathparse{\newy/28.45274}\pgfmathresult cm)$};
    \draw [->] ($(\urx,\ury)+(-#5/2,0)-\r*(0,\rstep)$) -- ($(\urx,\ury)+(#5/2,0)-\r*(0,\rstep)$);}}\fi
}

\def\dashbox#1#2#3#4#5{%(width,height), number inputs, number outputs, label, arrow length
  \pgfgetlastxy{\llx}{\lly}%assumes path has been set to a point representing the lower left corner of the box
  \path #1;
  \pgfgetlastxy{\w}{\h}
  \pgfmathsetlengthmacro{\urx}{\llx+\w}
  \pgfmathsetlengthmacro{\ury}{\lly+\h}
  \draw [dashed] (\llx,\lly) rectangle (\urx,\ury);
  \pgfmathsetlengthmacro{\xave}{(\llx+\urx)/2}
  \pgfmathsetlengthmacro{\yave}{\ury-8}
  \node at (\xave,\yave) {#4};
  \pgfmathsetlengthmacro{\ydiff}{\ury-\lly}
  \pgfmathsetlengthmacro{\lstep}{\ydiff/(#2+1)}
  \pgfmathsetlengthmacro{\rstep}{\ydiff/(#3+1)}
  \ifnum #2=0{}\else{ 
   \foreach \l in {1,...,#2}{
    \draw [->] ($(\llx,\lly)+(-#5/2,0)+\l*(0,\lstep)$) -- ($(\llx,\lly)+(#5/2,0)+\l*(0,\lstep)$);}}\fi
  \ifnum #3=0{}\else{
   \foreach \r in {1,...,#3}{
    \draw [->] ($(\urx,\ury)+(-#5/2,0)-\r*(0,\rstep)$) -- ($(\urx,\ury)+(#5/2,0)-\r*(0,\rstep)$);}}\fi
}

\def\fillbox#1#2#3#4#5#6{%(width,height), number inputs, number outputs, label, arrow length, fill strength
  \pgfgetlastxy{\llx}{\lly}%assumes path has been set to a point representing the lower left corner of the box
  \path #1;
  \pgfgetlastxy{\w}{\h}
  \pgfmathsetlengthmacro{\urx}{\llx+\w}
  \pgfmathsetlengthmacro{\ury}{\lly+\h}
  \filldraw [fill=gray!#6] (\llx,\lly) rectangle (\urx,\ury);
  \pgfmathsetlengthmacro{\xave}{(\llx+\urx)/2}
  \pgfmathsetlengthmacro{\yave}{\ury-8}
  \node at (\xave,\yave) {#4};
  \pgfmathsetlengthmacro{\ydiff}{\ury-\lly}
  \pgfmathsetlengthmacro{\lstep}{\ydiff/(#2+1)}
  \pgfmathsetlengthmacro{\rstep}{\ydiff/(#3+1)}
  \ifnum #2=0{}\else{ 
   \foreach \l in {1,...,#2}{
    \draw [->] ($(\llx,\lly)+(-#5/2,0)+\l*(0,\lstep)$) -- ($(\llx,\lly)+(#5/2,0)+\l*(0,\lstep)$);}}\fi
  \ifnum #3=0{}\else{
   \foreach \r in {1,...,#3}{
    \draw [->] ($(\urx,\ury)+(-#5/2,0)-\r*(0,\rstep)$) -- ($(\urx,\ury)+(#5/2,0)-\r*(0,\rstep)$);}}\fi
}


\def\delaynode#1#2#3{%(x-coord,y-coord), size, arrow length
  \path #1;
  \pgfgetlastxy{\lx}{\ly}
  \pgfmathsetlengthmacro{\rx}{\lx+#2*3+#3}
  \pgfmathsetlengthmacro{\ry}{\ly}
  \filldraw (\lx,\ly) circle (#2 pt);
  \draw [->] (\lx,\ly) -- (\rx,\ry);
}

\def\delaynodeouters#1#2#3{%(x-coord,y-coord), size, arrow length
  \path #1;
  \pgfgetlastxy{\lx}{\ly}
  \pgfmathsetlengthmacro{\rx}{\lx+#2*3+#3}
  \pgfmathsetlengthmacro{\ry}{\ly}
  \filldraw (\lx,\ly) circle (#2 pt);
  \draw [->] (\lx,\ly) -- (\rx,\ry);
   \node at (\rx+15,\ry+15){\tiny $(\pgfmathparse{\rx/28.45274}\pgfmathresult cm,\pgfmathparse{\ry/28.45274}\pgfmathresult cm)$};
}


\def\directarc#1#2{%left endpoint, right endpoint
  \path #1;
  \pgfgetlastxy{\lx}{\ly}
  \path #2;
  \pgfgetlastxy{\rx}{\ry}
  \pgfmathsetlengthmacro{\xave}{(\lx+\rx)/2}
  \draw #1 .. controls (\xave,\ly) and (\xave,\ry) .. #2;
}

\def\backarc#1#2{%left coordinate, right coordinate
  \path #1;
  \pgfgetlastxy{\lx}{\ly}
  \path #2;
  \pgfgetlastxy{\rx}{\ry}
  \pgfmathsetlengthmacro{\xave}{(\lx+\rx)/2}
  \draw #1 .. controls (\xave,\ry) and (\xave,\ly) .. #2;
}

\def\loopright#1#2#3{%upper coordinate, lower coordinate, stretch width
  \path #1;
  \pgfgetlastxy{\ux}{\uy}
  \path #2;
  \pgfgetlastxy{\lx}{\ly}
  \pgfmathsetlengthmacro{\maxx}{max(\ux,\lx)}
  \pgfmathsetlengthmacro{\farx}{\maxx+#3}
  \draw #1 .. controls (\farx,\uy) and (\farx,\ly) .. #2;
}

\def\loopleft#1#2#3{%upper coordinate, lower coordinate, stretch width
  \path #1;
  \pgfgetlastxy{\ux}{\uy}
  \path #2;
  \pgfgetlastxy{\lx}{\ly}
  \pgfmathsetlengthmacro{\minx}{min(\ux,\lx)}
  \pgfmathsetlengthmacro{\farx}{\minx-#3}
  \draw #1 .. controls (\farx,\uy) and (\farx,\ly) .. #2;
}

\def\fancyarc#1#2#3#4{%upper coordinate, lower coordinate, stretch width, max height adjust
  \path #1;
  \pgfgetlastxy{\ux}{\uy}
  \path #2;
  \pgfgetlastxy{\lx}{\ly}
  \pgfmathsetlengthmacro{\xave}{(\lx+\ux)/2}
%  \node at (\lx,\ly+20){\tiny $\pgfmathparse{\lx/28.45274}\pgfmathresult cm$,\hsp$\pgfmathparse{\ux/28.45274}\pgfmathresult cm$};
%  \node at (\xave,\ly+50){\tiny $\pgfmathparse{\xave/28.45274}\pgfmathresult cm$};
  \pgfmathsetlengthmacro{\yave}{(\ly+\uy)/2+#4}
  \loopleft{#1}{(\xave,\yave)}{#3}
  \loopright{#2}{(\xave,\yave)}{#3}
}

\def\activetikz#1{$$\begin{tikzpicture}#1\end{tikzpicture}$$}
\def\inactivetikz#1{}
%End tikz macros


\input xy
\xyoption{all} \xyoption{poly} \xyoption{knot}\xyoption{curve}
\usepackage{xy,color}
\newcommand{\comment}[1]{}

\newcommand{\longnote}[2][4.9in]{\fcolorbox{black}{yellow}{\parbox{#1}{\color{black} #2}}}
\newcommand{\shortnote}[1]{\fcolorbox{black}{yellow}{\color{black} #1}}
\newcommand{\start}[1]{\shortnote{Start here: #1.}}
\newcommand{\q}[1]{\begin{question}#1\end{question}}
\newcommand{\g}[1]{\begin{guess}#1\end{guess}}

\def\tn{\textnormal}
\def\mf{\mathfrak}
\def\mc{\mathcal}

\def\ZZ{{\mathbb Z}}
\def\QQ{{\mathbb Q}}
\def\RR{{\mathbb R}}
\def\CC{{\mathbb C}}
\def\AA{{\mathbb A}}
\def\PP{{\mathbb P}}
\def\NN{{\mathbb N}}
\def\SS{{\mathbb S}}
\def\HH{{\mathbb H}}

\def\acts{\lefttorightarrow}
\def\Hom{\tn{Hom}}
\def\iHom{\Rightarrow}
\def\Aut{\tn{Aut}}
\def\Sub{\tn{Sub}}
\def\Mor{\tn{Mor}}
\def\Fun{\tn{Fun}}
\def\Path{\tn{Path}}
\def\im{\tn{im}}
\def\Ob{\tn{Ob}}
\def\Trace{\tn{Tr}}
\def\Op{\tn{Op}}
\def\SEL*{\tn{SEL*}}
\def\Res{\tn{Res}}
\def\hsp{\hspace{.3in}}
\newcommand{\hsps}[1]{{\hspace{2mm} #1\hspace{2mm}}}
\newcommand{\tin}[1]{\text{\tiny #1}}

\def\singleton{{\{*\}}}
\newcommand{\boxtitle}[1]{\begin{center}#1\end{center}}
\def\Loop{{\mcL oop}}
\def\LoopSchema{{\parbox{.5in}{\fbox{\xymatrix{\LMO{s}\ar@(l,u)[]^f}}}}}
\def\Wks{{\mcW ks}}
\def\lcone{^\triangleleft}
\def\rcone{^\triangleright}
\def\to{\rightarrow}
\def\from{\leftarrow}
\def\down{\downarrrow}
\def\Down{\Downarrow}
\def\tensor{\otimes}
\def\Up{\Uparrow}
\def\taking{\colon}
\def\pls{``\!+\!"}
\newcommand{\pathto}[1]{\stackrel{#1}\leadsto}
\def\inj{\hookrightarrow}
\def\surj{\twoheadrightarrow}
\def\surjj{\longtwoheadrightarrow}
\def\pfunc{\rightharpoonup}
\def\Pfunc{\xrightharpoonup}
\def\too{\longrightarrow}
\def\fromm{\longleftarrow}
\def\tooo{\longlongrightarrow}
\def\tto{\rightrightarrows}
\def\ttto{\equiv\!\!>}
\newcommand{\xyto}[2]{\xymatrix@=1pt{\ar[rr]^-{#1}&\hspace{#2}&}}
\newcommand{\xyequals}[1]{\xymatrix@=1pt{\ar@{=}[rr]&\hspace{#1}&}}
\def\ss{\subseteq}
\def\superset{\supseteq}
\def\iso{\cong}
\def\down{\downarrow}
\def\|{{\;|\;}}
\def\m1{{-1}}
\def\op{^\tn{op}}
\def\la{\langle}
\def\ra{\rangle}
\def\wt{\widetilde}
\def\wh{\widehat}
\def\we{\simeq}
\def\ol{\overline}
\def\ul{\underline}
\def\vect{\overrightarrow}
\def\qeq{\mathop{=}^?}

\def\rr{\raggedright}

%\newcommand{\LMO}[1]{\bullet^{#1}}
%\newcommand{\LTO}[1]{\bullet^{\tn{#1}}}
\newcommand{\LMO}[1]{\stackrel{#1}{\bullet}}
\newcommand{\LTO}[1]{\stackrel{\tt{#1}}{\bullet}}
\newcommand{\LA}[2]{\ar[#1]^-{\tn {#2}}}
\newcommand{\LAL}[2]{\ar[#1]_-{\tn {#2}}}
\newcommand{\obox}[3]{\stackrel{#1}{\fbox{\parbox{#2}{#3}}}}
\newcommand{\labox}[2]{\obox{#1}{1.6in}{#2}}
\newcommand{\mebox}[2]{\obox{#1}{1in}{#2}}
\newcommand{\smbox}[2]{\stackrel{#1}{\fbox{#2}}}
\newcommand{\fakebox}[1]{\tn{$\ulcorner$#1$\urcorner$}}
\newcommand{\sq}[4]{\xymatrix{#1\ar[r]\ar[d]&#2\ar[d]\\#3\ar[r]&#4}}
\newcommand{\namecat}[1]{\begin{center}$#1:=$\end{center}}


\def\ullimit{\ar@{}[rd]|(.3)*+{\lrcorner}}
\def\urlimit{\ar@{}[ld]|(.3)*+{\llcorner}}
\def\lllimit{\ar@{}[ru]|(.3)*+{\urcorner}}
\def\lrlimit{\ar@{}[lu]|(.25)*+{\ulcorner}}
\def\ulhlimit{\ar@{}[rd]|(.3)*+{\diamond}}
\def\urhlimit{\ar@{}[ld]|(.3)*+{\diamond}}
\def\llhlimit{\ar@{}[ru]|(.3)*+{\diamond}}
\def\lrhlimit{\ar@{}[lu]|(.3)*+{\diamond}}
\newcommand{\clabel}[1]{\ar@{}[rd]|(.5)*+{#1}}
\newcommand{\TriRight}[7]{\xymatrix{#1\ar[dr]_{#2}\ar[rr]^{#3}&&#4\ar[dl]^{#5}\\&#6\ar@{}[u] |{\Longrightarrow}\ar@{}[u]|>>>>{#7}}}
\newcommand{\TriLeft}[7]{\xymatrix{#1\ar[dr]_{#2}\ar[rr]^{#3}&&#4\ar[dl]^{#5}\\&#6\ar@{}[u] |{\Longleftarrow}\ar@{}[u]|>>>>{#7}}}
\newcommand{\TriIso}[7]{\xymatrix{#1\ar[dr]_{#2}\ar[rr]^{#3}&&#4\ar[dl]^{#5}\\&#6\ar@{}[u] |{\Longleftrightarrow}\ar@{}[u]|>>>>{#7}}}


\newcommand{\arr}[1]{\ar@<.5ex>[#1]\ar@<-.5ex>[#1]}
\newcommand{\arrr}[1]{\ar@<.7ex>[#1]\ar@<0ex>[#1]\ar@<-.7ex>[#1]}
\newcommand{\arrrr}[1]{\ar@<.9ex>[#1]\ar@<.3ex>[#1]\ar@<-.3ex>[#1]\ar@<-.9ex>[#1]}
\newcommand{\arrrrr}[1]{\ar@<1ex>[#1]\ar@<.5ex>[#1]\ar[#1]\ar@<-.5ex>[#1]\ar@<-1ex>[#1]}

\newcommand{\To}[1]{\xrightarrow{#1}}
\newcommand{\Too}[1]{\xrightarrow{\ \ #1\ \ }}
\newcommand{\From}[1]{\xleftarrow{#1}}
\newcommand{\Fromm}[1]{\xleftarrow{\ \ #1\ \ }}
\def\qeq{\stackrel{?}{=}}

\newcommand{\Adjoint}[4]{\xymatrix@1{{#2}\ar@<.5ex>[r]^-{#1} &{#3} \ar@<.5ex>[l]^-{#4}}}

\def\id{\tn{id}}
\def\dom{\tn{dom}}
\def\cod{\tn{cod}}
\def\Top{{\bf Top}}
\def\Kls{{\bf Kls}}
\def\Cat{{\bf Cat}}
\def\Oprd{{\bf Oprd}}
\def\LH{{\bf LH}}
\def\Monad{{\bf Monad}}
\def\Mon{{\bf Mon}}
\def\CMon{{\bf CMon}}
\def\cpo{{\bf cpo}}
\def\Vect{\text{Vect}}
\def\Rep{{\bf Rep}}
\def\HCat{{\bf HCat}}
\def\Cnst{{\bf Cnst}}
\def\Str{\tn{Str}}
\def\List{\tn{List}}
\def\Exc{\tn{Exc}}
\def\Sets{{\bf Sets}}
\def\Cob{{\bf Cob}}
\def\Grph{{\bf Grph}}
\def\SGrph{{\bf SGrph}}
\def\Rel{\mcR\tn{el}}
\def\JRel{J\mcR\tn{el}}
\def\Inst{{\bf Inst}}
\def\Type{{\bf Type}}
\def\Set{{\bf Set}}
\def\TFS{{\bf TFS}}
\def\Qry{{\bf Qry}}
\def\set{{\text \textendash}{\bf Set}}
\def\sets{{\text \textendash}{\bf Alg}}
\def\alg{{\text \textendash}{\bf Alg}}
\def\rel{{\text \textendash}{\bf Rel}}
\def\inst{{{\text \textendash}\bf \Inst}}
\def\sSet{{\bf sSet}}
\def\sSets{{\bf sSets}}
\def\Grpd{{\bf Grpd}}
\def\Pre{{\bf Pre}}
\def\Shv{{\bf Shv}}
\def\Rings{{\bf Rings}}
\def\bD{{\bf \Delta}}
\def\dispInt{\parbox{.1in}{$\int$}}
\def\bhline{\Xhline{2\arrayrulewidth}}
\def\bbhline{\Xhline{2.5\arrayrulewidth}}


\def\Comp{\tn{Comp}}
\def\Supp{\tn{Supp}}
\def\Dmnd{\tn{Dmnd}}


\def\colim{\mathop{\tn{colim}}}
\def\hocolim{\mathop{\tn{hocolim}}}
\def\undsc{\rule{2mm}{0.4pt}}


\def\mcA{\mc{A}}
\def\mcB{\mc{B}}
\def\mcC{\mc{C}}
\def\mcD{\mc{D}}
\def\mcE{\mc{E}}
\def\mcF{\mc{F}}
\def\mcG{\mc{G}}
\def\mcH{\mc{H}}
\def\mcI{\mc{I}}
\def\mcJ{\mc{J}}
\def\mcK{\mc{K}}
\def\mcL{\mc{L}}
\def\mcM{\mc{M}}
\def\mcN{\mc{N}}
\def\mcO{\mc{O}}
\def\mcP{\mc{P}}
\def\mcQ{\mc{Q}}
\def\mcR{\mc{R}}
\def\mcS{\mc{S}}
\def\mcT{\mc{T}}
\def\mcU{\mc{U}}
\def\mcV{\mc{V}}
\def\mcW{\mc{W}}
\def\mcX{\mc{X}}
\def\mcY{\mc{Y}}
\def\mcZ{\mc{Z}}

\def\bfS{{\bf S}}\def\bfSs{{\bf Ss}}
\def\bfT{{\bf T}}\def\bfTs{{\bf Ts}}
\def\bfW{{\bf W}}

\def\tnN{\tn{N}}


\def\bE{\bar{E}}
\def\bF{\bar{F}}
\def\bG{\bar{G}}
\def\bH{\bar{H}}
\def\bX{\bar{X}}
\def\bY{\bar{Y}}

\newcommand{\subsub}[1]{\setcounter{subsubsection}{\value{theorem}}\subsubsection{#1}\addtocounter{theorem}{1}}

\def\Finm{{\bf Fin_{m}}}
\def\Bag{\tn{Bag}}
\newcommand{\back}[1]{\stackrel{\from}{#1}\!}
%\newcommand{\kls}[1]{{\text \textendash}\wt{\bf Kls}({#1})}
\def\Dist{\text{Dist}}
\def\Dst{{\bf Dst}}
\def\WkFlw{{\bf WkFlw}}
\def\monOb{{\blacktriangle}}
\def\Infl{{\bf Infl}}
\def\Tur{\tn{Tur}}
\def\Halt{\{\text{Halt}\}}
\def\Tape{{T\!ape}}
\def\Prb{{\bf Prb}}
\def\Prbs{{\wt{\bf Prb}}}
\def\El{{\bf El}}
\def\Gr{{\bf Gr}}
\def\DT{{\bf DT}}
\def\DB{{\bf DB}}
\def\Tables{{\bf Tables}}
\def\Sch{{\bf Sch}}
\def\Fin{{\bf Fin}}
\def\PrO{{\bf PrO}}
\def\PrOs{{\bf PrOs}}
\def\JLat{{\bf JLat}}
\def\JLats{{\bf JLats}}
\def\P{{\bf P}}
\def\SC{{\bf SC}}
\def\ND{{\bf ND}}
\def\Poset{{\bf Poset}}
\def\Bool{\tn{Bool}}
\newcommand{\labelDisp}[2]{\begin{align}\label{#1}\text{#2}\end{align}}

\newcommand{\inp}[1]{#1_-}
\newcommand{\outp}[1]{#1_+}
\newcommand{\vset}[1]{#1_{\tt type}}
\newcommand{\loc}[1]{{\tt loc}(#1)}
\newcommand{\extr}[1]{{\tt ext}(#1)}
\newcommand{\vLst}[1]{\ol{#1}}
\newcommand{\Strm}[1]{{\tn{Strm}(#1)}}
\newcommand{\SP}[2]{{\tn{SP}(#1,#2)}}
\newcommand{\LP}[2]{{\tn{LP}(#1,#2)}}
\newcommand{\LPP}[2]{{\tn{LP}'(#1,#2)}}
\newcommand{\Hist}{\tn{Hist}}
\newcommand{\xleadsto}[1]{\stackrel{#1}{\leadsto}}
\newcommand{\strst}[1]{\big|_{[1,#1]}} %"stream restriction"
\newcommand{\Del}[1]{DN_{#1}}
\newcommand{\Dem}[1]{{Dm_{#1}}}
\newcommand{\Sup}[1]{{Sp_{#1}}}
%\newcommand{\Con}[1]{{Con_{#1}}}
\newcommand{\inDem}[1]{{in\Dem{#1}}}
\newcommand{\inSup}[1]{{in\Sup{#1}}}
\newcommand{\vinSup}[1]{\vLst{\inSup{#1}}}
\newcommand{\vinDem}[1]{\vLst{\inDem{#1}}}
\newcommand{\vSup}[1]{\vLst{\Sup{#1}}}
\newcommand{\vDem}[1]{\vLst{\Dem{#1}}}
\newcommand{\vDel}[1]{\vLst{\Del{#1}}}
\newcommand{\vinp}[1]{\vLst{\inp{#1}}}
\newcommand{\vintr}[1]{\vLst{\loc{#1}}}
\newcommand{\voutp}[1]{\vLst{\outp{#1}}}
\def\zipwith{\;\raisebox{3pt}{${}_\varcurlyvee$}\;}
\def\tbzipwith{\!\;\raisebox{2pt}{${}_\varcurlyvee$}\;\!}
\newcommand{\ffootnote}[2]{\hspace{#1}\footnote{#2}}



\def\lin{\ell\tn{In}}
\def\lout{\ell\tn{Out}}
\def\gin{g\tn{In}}
\def\gout{g\tn{Out}}
\def\min{m^{in}}
\def\mout{m^{out}}
\def\sin{s^{in}}
\def\sout{s^{out}}
\newcommand{\disc}[1]{{\ul{#1}}}

\newcommand{\Wir}[1]{\bfW_{#1}}
\def\SMC{{\bf SMC}}
\def\TSMC{{\bf TSMC}}
\newcommand{\FS}[1]{\tn{FS}_{/\mathcal{#1}}}
\def\TFSO{\text{\bf TSMC}_{\Ob(\FS{O})}}
\newcommand{\Bij}[1]{\mcB_{\mathcal{#1}}}
\def\Int{\tn{Int}}

%\newcommand{\strm}[1]{{\left(#1\right)^\NN}}

\makeatletter\let\c@figure\c@equation\makeatother %Aligns figure numbering and equation numbering.
\newtheorem{theorem}[subsubsection]{Theorem}
\newtheorem{lemma}[subsubsection]{Lemma}
\newtheorem{proposition}[subsubsection]{Proposition}
\newtheorem{corollary}[subsubsection]{Corollary}
\newtheorem{fact}[subsubsection]{Fact}

\theoremstyle{remark}
\newtheorem{remark}[subsubsection]{Remark}
\newtheorem{example}[subsubsection]{Example}
\newtheorem{application}[subsubsection]{Application}
\newtheorem{warning}[subsubsection]{Warning}
\newtheorem{question}[subsubsection]{Question}
\newtheorem{guess}[subsubsection]{Guess}
\newtheorem{answer}[subsubsection]{Answer}
\newtheorem{claim}[subsubsection]{Claim}

\theoremstyle{definition}
\newtheorem{definition}[subsubsection]{Definition}
\newtheorem{notation}[subsubsection]{Notation}
\newtheorem{conjecture}[subsubsection]{Conjecture}
\newtheorem{postulate}[subsubsection]{Postulate}
\newtheorem{construction}[subsubsection]{Construction}
\newtheorem{ann}[subsubsection]{Announcement}
\newenvironment{announcement}{\begin{ann}}{\hspace*{\fill}$\lozenge$\end{ann}}


\setcounter{tocdepth}{1}
\setcounter{secnumdepth}{2}


%Standard geometry (DO NOT CHANGE) seems to be: \newgeometry{left=1.76in,right=1.76in,top=1.6in,bottom=1.3in}

\usepackage[paperwidth=8.5in,paperheight=11in,text={5.7in,8in},centering]{geometry}
%\newgeometry{left=1.55in,right=1.55in,top=1.6in,bottom=1.3in}
\usepackage{lscape}

\begin{document}

\title{String diagrams for traced monoidal categories as Cob(1)-algebras}

\author{Dylan Rupel}
\address{Northeastern University\\360 Huntington Ave.\\Boston, MA 02115}
\email{dylanrupel@gmail.com}

\author{David I. Spivak}
\address{Massachusetts Institute of Technology\\77 Massachusetts Ave.\\Cambridge, MA 02139}
\email{dspivak@gmail.com}

\thanks{Spivak acknowledges support by ONR grant N000141310260 and AFOSR grant FA9550-14-1-0031.}


\maketitle

\tableofcontents

By manifold, we always mean manifold with boundary; we must say {\em closed manifold} if we mean to say that the boundary is empty. When a manifold $Y$ is oriented, we denote the same manifold with opposite orientation by $Y^\vee$.

Recall that an {\em oriented finite 0-manifold} consists of a pair of finite sets $\inp{X},\outp{X}$, where $\inp{X}$ has negative orientation and $\outp{X}$ has positive orientation. Given a manifold $\mcO$, an {\em oriented 0-manifold over $\mcO$} is an oriented $0$-manifold $X$ together with a function $\tau_X\taking\inp{X}\sqcup \outp{X}\to\mcO$. 

If $X$ and $Y$ are oriented $0$-manifolds, an {\em oriented cobordism from $X$ to $Y$} consists of an oriented manifold $W$ whose boundary $\partial W$ is identified with $X\sqcup Y^\vee$. We will always be working with oriented manifolds and cobordisms, so we often speak of them as simply manifolds and cobordisms. 

Given a cobordism $W$ from $X$ to $Y$ and a cobordism $W'$ from $Y$ to $Z$, we can glue them together along a thickening of $Y$ to obtain a cobordism from $X$ to $Z$. If $X$ and $Y$ are manifolds over $\mcO$, when we speak of a cobordism between them, we mean that our manifold $W$ is over $\mcO$.

We denote the category whose objects are oriented $0$-manifolds and oriented cobordisms between them by $\Cob$ (because we will only have occasion to speak of $0$-manifolds in this paper), and the category of 0-manifolds and cobordisms over $\mcO$ by $\Cob/\mcO$.

\begin{lemma}\label{lemma:cob as bij}

Let $X,Y\in\Ob(\Cob/\mcO)$ be oriented 0-manifolds over $\mcO$. The set $\Hom_{\Cob/\mcO}(X,Y)$ of cobordisms from $X$ to $Y$ can be identified with the set of pairs $(f,n)$ where 
$$f\taking\inp{X}\sqcup \outp{Y}\Too{\iso}\outp{X}\sqcup \inp{Y}$$
is a bijection over $\mcO$, and $n\in\Ob(\Fin/\mcO)$ is a finite set over $\mcO$.

\end{lemma}

\begin{proof}

**

\end{proof}

\begin{warning}

Note that for every injection of finite sets $f\taking m\inj n$ we have a function $Tr^{n-m}_{m,m}\taking |S_n|\to|S_m|$ between the underlying sets of the respective permutation groups. But this {\em does not} extend to a group homomorphism $S_n\to S_m$. Indeed, consider the case $m=2, n=4$ and the permutations $\sigma=(1 4)(2 3)$ and $\tau=(1)(2)(3 4)$. These are drawn below
$$\xymatrix@R=10pt{
\bullet\ar@{-}[dddr]&\bullet\ar@{-}[r]&\bullet\\
\bullet\ar@{-}[dr]&\bullet\ar@{-}[r]&\bullet\\
\bullet\ar@{-}[ur]&\bullet\ar@{-}[dr]&\bullet\\
\bullet\ar@{-}[uuur]&\bullet\ar@{-}[ur]&\bullet
}
$$
Then $Tr^{n-m}_{m,m}(\sigma)=(1)(2)$ and $Tr^{n-m}_{m,m}(\tau)=(1)(2)$, whereas $Tr^{n-m}_{m,m}(\tau\circ\sigma)=(1 2)$

\end{warning}

\begin{lemma}

We have a functor $Perm\taking Inj^{op}\to\Set$, given on morphisms by $Tr$.

\end{lemma}

\begin{proof}

**

\end{proof}

Suppose given a typed bijection $\phi\taking\inp{X}\sqcup \outp{Y}\To{\iso}\outp{X}\sqcup \inp{Y}$. We'll define $\phi_X^X, \phi_Y^X, \phi_X^Y,$ and $\phi_Y^Y$ below; roughly they are ``$X$'s feed $X$'s", ``$X$'s feed $Y$'s", ``$Y$'s feed $X$'s", and ``$Y$'s feed $Y$'s", respectively. They are given by the following fiber products in $\TFS$.
$$
\xymatrix@=35pt{
\phi_X^X\ar[r]\ar[d]\ullimit&\inp{X}\ar[d]^{\phi\big|_{\inp{X}}}&\phi_X^Y\ar[l]\ar[d]\urlimit\\
\outp{X}\ar[r]&\outp{X}\sqcup \inp{Y}&\inp{Y}\ar[l]\\
\phi_Y^X\ar[r]\ar[u]\lllimit&\outp{Y}\ar[u]_{\phi\big|_{\outp{Y}}}&\phi_Y^Y\ar[l]\ar[u]\lrlimit
}
$$
Note that every arrow above is an injection and we have the following four bijections, 
\begin{align*}
\phi_X^X\sqcup \phi_Y^X\iso\outp{X},
&\hsp
\phi_X^Y\sqcup \phi_Y^Y\iso\inp{Y},
\\
\phi_X^X\sqcup \phi_X^Y\iso\inp{X},
&\hsp
\phi_Y^X\sqcup \phi_Y^Y\iso\outp{Y}.
\end{align*}

\begin{lemma}\label{le:composed wires}

Suppose given $\phi\taking X\to Y$ and $\psi\taking Y\to Z$ in $\Cob/\mcO$, and form the diagram below. 
$$\xymatrix{
\phi_X^X\ar[r]\ar[d]&\inp{X}&\phi_X^Y\ar[l]\ar[d]\\
\outp{X}&\psi_Y^Y\ar[r]\ar[d]&\inp{Y}&\psi_Y^Z\ar[l]\ar[d]\\
\phi_Y^X\ar[u]\ar[r]&\outp{Y}&\phi_Y^Y\ar[u]\ar[l]&\inp{Z}\\
&\psi_Z^Y\ar[u]\ar[r]&\outp{Z}&\psi_Z^Z\ar[l]\ar[u]
}
$$
Let $P_Y$ denote the colimit of the central square above (as shown to the left below), and then define four sets, $P_X^X, P_X^Z, P_Z^X,$ and $P_Z^Z$ as the pullbacks shown in the diagram to the right below:
$$\xymatrix{
\psi_Y^Y\ar[dr]\ar[rr]\ar[dd]&&\inp{Y}\ar[dl]\\
&P_Y\\
\outp{Y}\ar[ur]&&\phi_Y^Y\ar[lu]\ar[ll]\ar[uu]
}
\hspace{1in}
\xymatrix{
P_X^X\ar[r]\ar[d]\ullimit&\phi_X^Y\ar[d]&P_X^Z\ar[l]\ar[d]\urlimit\\
\phi_Y^X\ar[r]&P_Y&\psi_Y^Z\ar[l]\\
P_Z^X\ar[u]\ar[r]\lllimit&\psi_Z^Y\ar[u]&P_Z^Z\ar[u]\ar[l]\lrlimit
}
$$
Then we know the following about the composite morphism $\psi\circ\phi\taking X\to Z$: 
\begin{itemize}
\item $(\psi\circ\phi)_Z^X=P_Z^X$,
\item $(\psi\circ\phi)_X^Z=P_X^Z$, 
\item $(\psi\circ\phi)_X^X=\phi_X^X\sqcup P_X^X$, 
\item $(\psi\circ\phi)_Z^Z=\psi_Z^Z\sqcup P_Z^Z$, and
\item The induced map $P_X^X\sqcup P_X^Z\sqcup P_Z^X\sqcup P_Z^Z\too P_Y$ is injective, and, letting $P_L\ss P_Y$ be the complement of its image, we have
$$\Loop(\psi\circ\phi)=\Loop(\phi)\sqcup \Loop(\psi)\sqcup P_L.$$
\end{itemize}

\end{lemma}

\begin{proof}

**

\end{proof}

\begin{proposition}

The category $\Cob/\mcO$ is a compact closed category. 

\end{proposition}

\begin{proof}

**

\end{proof}


\begin{lemma}

Given a function $f\taking\mcO\to\mcO'$, there is an induced strong monoidal functor $f_!\taking\Cob/\mcO\to\Cob/\mcO'$.

This defines a functor 
$$\Cob/-\taking\Set\to\SMC_{st}$$
from the category of sets to the category of SMCs and strong monoidal functors.

\end{lemma}

\begin{proof}

**

\end{proof}

\begin{definition}

Define {\em the category of wiring diagram algebras}, denoted $(\Cob/\bullet)\alg$, to have objects 
$$\Ob((\Cob/\bullet)\alg):=\{(\mcO,\mcP)\|\mcO\in\Ob(\Set)\tn{ and } \mcP\in\Ob\big((\Cob/\mcO)\alg\big)\},$$
and to have morphisms
$$\Hom_{(\Cob/\bullet)\alg}((\mcO,\mcP),(\mcO',\mcP')):=\{(f,r)\|f\taking\mcO\to\mcO', r\taking\mcP\to(\Cob/f)^*(\mcP')\}.$$

\end{definition}

\begin{theorem}

Let $(\Cob/\bullet)\alg$ be the category of wiring diagram algebras, and let $\TSMC$ denote the category of traced symmetric monoidal categories (and lax monoidal functors between them). There is an adjunction
$$\xymatrix{L\taking(\Cob/\bullet)\alg\ar@<.5ex>[r]&\TSMC:\!R\ar@<.5ex>[l]}$$

\end{theorem}

\begin{proof}

Let $\mcM$ be a traced monoidal category with objects $\mcO$. We will define a $\Cob/\mcO$-algebra $\mcP=R(\mcM)\taking\Cob/\mcO\to\Set$ as follows. On an object $[\inp{X},\outp{X}]\in\Ob(\Cob/\mcO)$, put 
$$\mcP[\inp{X},\outp{X}]:=\Hom_{\mcM}(\vinp{X},\voutp{X}).$$
We next consider morphisms.


A morphism $\Phi\taking[\inp{X},\outp{X}]\too[\inp{Y},\outp{Y}]$ consists of a typed bijection 
$$\phi\taking\inp{X}\sqcup \outp{Y}\To{\iso}\outp{X}\sqcup \inp{Y},$$ 
together with a typed finite set $N$. Given an element $f\in\mcP[\inp{X},\outp{X}]$ we must construct $\mcP(\Phi)(f)\in\mcP[\inp{Y},\outp{Y}]$. Let $\dim(\ol{S})=\Trace^{\ol{S}}_{I,I}(\id_{\ol{S}})$. Then we use the formula
$$\mcP(\Phi)(f):=\Trace^{\ol{\phi_X^X}}_{\ol{Y_-},\ol{Y_+}}(f\tensor\id_{\ol{\phi_Y^Y}})\tensor\dim(\ol{S})$$

Given also $\Psi\taking[\inp{Y},\outp{Y}]\too[\inp{Z},\outp{Z}]$, we need to show that the following equation holds: 
$$\mcP(\Psi)\circ\mcP(\Phi)(f)=^?\mcP(\Psi\circ\Phi)(f).$$
These are:
\begin{align}
\mcP(\Psi)\circ\mcP(\Phi)(f)&=
\Trace^{\ol{\psi_Y^Y}}_{\ol{Z_-},\ol{Z_+}}
\Big(
\Trace^{\ol{\phi_X^X}}_{\ol{Y_-},\ol{Y_+}}(f\tensor\id_{\ol{\phi_Y^Y}})\tensor\dim(\ol{S})\tensor\id_{\ol{\psi_Z^Z}}
\Big)
\tensor\dim(\ol{T})
\\\nonumber\\
\mcP(\Psi\circ\Phi)(f)&=
\Trace^{\ol{(\psi\circ\phi)_X^X}}_{\ol{Z_-},\ol{Z_+}}(f\tensor\id_{\ol{(\psi\circ\phi)_Z^Z}})\tensor\dim(\ol{S\sqcup T\sqcup ``newloops"})
\end{align}

Possibly useful identities for (1):
\[\Trace^{\ol{\psi_Y^Y}}_{\ol{Z_-},\ol{Z_+}}
\Big(
\Trace^{\ol{\phi_X^X}}_{\ol{Y_-},\ol{Y_+}}(f\tensor\id_{\ol{\phi_Y^Y}})\tensor\id_{\ol{\psi_Z^Z}}
\Big)=\Trace^{\ol{\psi_Y^Y}}_{\ol{\psi_Y^Z},\ol{\psi_Z^Y}}
\Big(
\Trace^{\ol{\phi_X^X}}_{\ol{Y_-},\ol{Y_+}}(f\tensor\id_{\ol{\phi_Y^Y}})
\Big)\tensor\id_{\ol{\psi_Z^Z}}\]
\[\Trace^{\ol{\psi_Y^Y}}_{\ol{\psi_Y^Z},\ol{\psi_Z^Y}}
\Big(\Trace^{\ol{\phi_X^X}}_{\ol{\inp{Y}},\ol{\outp{Y}}}(f\tensor\id_{\ol{\phi_Y^Y}})\Big)=\Trace^{\ol{\psi_Y^Y}\otimes\ol{\phi_X^X}}_{\ol{\psi_Y^Z},\ol{\psi_Z^Y}}
(f\tensor\id_{\ol{\phi_Y^Y}})\]
\[\Trace^{\ol{\psi_Y^Y}\otimes\ol{\phi_X^X}}_{\ol{\psi_Y^Z},\ol{\psi_Z^Y}}
(f\tensor\id_{\ol{\phi_Y^Y}})\tensor\id_{\ol{\psi_Z^Z}}=\Trace^{\ol{\psi_Y^Y}\otimes\ol{\phi_X^X}}_{\ol{\inp{Z}},\ol{\outp{Z}}}
(f\tensor\id_{\ol{\phi_Y^Y}}\tensor\id_{\ol{\psi_Z^Z}})\]
\shortnote{*****}
\[\Trace^{\ol{\psi_Y^Y}}_{\ol{\psi_Y^Z},\ol{\psi_Z^Y}}
\Big(\Trace^{\ol{\phi_X^X}}_{\ol{\inp{Y}},\ol{\outp{Y}}}(f\tensor\id_{\ol{\phi_Y^Y}})\Big)=\Trace^{\ol{\psi_Y^Y}\otimes\ol{\phi_X^X}}_{\ol{\psi_Y^Z},\ol{\psi_Z^Y}}
(f\tensor\id_{\ol{\phi_Y^Y}})\cong\Trace^{\ol{\phi_X^X}\otimes\ol{\psi_Y^Y}}_{\ol{\psi_Y^Z},\ol{\psi_Z^Y}}
(f\tensor\id_{\ol{\phi_Y^Y}})=\Trace^{\ol{\phi_X^X}}_{\ol{\psi_Y^Z},\ol{\psi_Z^Y}}\Big(\Trace^{\ol{\psi_Y^Y}}_{\ol{\psi_Y^Z}\otimes\ol{\phi_X^X},\ol{\psi_Z^Y}\otimes\ol{\phi_X^X}}
(f\tensor\id_{\ol{\phi_Y^Y}})\Big)\]
\shortnote{*****}

Possibly useful identities for (2):
\[\Trace^{\ol{(\psi\circ\phi)_X^X}}_{\ol{\inp{Z}},\ol{\outp{Z}}}(f\tensor\id_{\ol{(\psi\circ\phi)_Z^Z}})=\Trace^{\ol{P_X^X}\otimes\ol{\phi_X^X}}_{\ol{\inp{Z}},\ol{\outp{Z}}}(f\tensor\id_{\ol{(\psi\circ\phi)_Z^Z}})\]
\[\Trace^{\ol{P_X^X}\otimes\ol{\phi_X^X}}_{\ol{\inp{Z}},\ol{\outp{Z}}}(f\tensor\id_{\ol{(\psi\circ\phi)_Z^Z}})=\Trace^{\ol{P_X^X}\otimes\ol{\phi_X^X}}_{\ol{\inp{Z}},\ol{\outp{Z}}}(f\tensor\id_{\ol{\psi_Z^Z}}\otimes\id_{\ol{P_Z^Z}})\]
\shortnote{The result will follow if we can show:}
\[\Trace^{\ol{\psi_Y^Y}\otimes\ol{\phi_X^X}}_{\ol{\inp{Z}},\ol{\outp{Z}}}
(f\tensor\id_{\ol{\phi_Y^Y}}\tensor\id_{\ol{\psi_Z^Z}})=\Trace^{\ol{\phi_X^X}\otimes\ol{P_X^X}}_{\ol{\inp{Z}},\ol{\outp{Z}}}(f\tensor\id_{\ol{\psi_Z^Z}}\otimes\id_{\ol{P_Z^Z}})\otimes\dim(\ol{``newloops"})\]
Using that 
\[\inp{Z}=(\psi\circ\phi)_X^Z\sqcup (\psi\circ\phi)_Z^Z=P_X^Z\sqcup P_Z^Z\sqcup \psi_Z^Z\]
\[\outp{Z}=(\psi\circ\phi)_Z^X\sqcup (\psi\circ\phi)_Z^Z=P_Z^X\sqcup P_Z^Z\sqcup \psi_Z^Z,\]
where each of the final equalities above follow from Lemma~\ref{le:composed wires}, and the superposition axiom of the trace we get
\[\Trace^{\ol{\psi_Y^Y}\otimes\ol{\phi_X^X}}_{\ol{\inp{Z}},\ol{\outp{Z}}}
(f\tensor\id_{\ol{\phi_Y^Y}}\tensor\id_{\ol{\psi_Z^Z}})=\Trace^{\ol{\psi_Y^Y}\otimes\ol{\phi_X^X}}_{\ol{P_X^Z}\otimes\ol{P_Z^Z},\ol{P_Z^X}\otimes\ol{P_Z^Z}}
(f\tensor\id_{\ol{\phi_Y^Y}})\tensor\id_{\ol{\psi_Z^Z}}\]
\shortnote{how to eliminate $\ol{P_Z^Z}$ above?} $\phi_Y^Y$ should decompose as those wires which map to $P_L$ and those wires that map to its complement in $P_Y$, with similar decompositions for $\psi_Y^Y$, this should allow to split off the needed ``newloops'' factor.  A similar statement should hold for the elements of $\phi_Y^Y$ and $\psi_Y^Y$ that map to $P_Z^Z$.  But what about the remaining elements, i.e. those mapping to $P_X^X\sqcup P_X^Z\sqcup P_Z^X$?

and
\[\Trace^{\ol{\phi_X^X}\otimes\ol{P_X^X}}_{\ol{\inp{Z}},\ol{\outp{Z}}}(f\tensor\id_{\ol{\psi_Z^Z}}\otimes\id_{\ol{P_Z^Z}})=\Trace^{\ol{\phi_X^X}\otimes\ol{P_X^X}}_{\ol{P_X^Z},\ol{P_Z^X}}(f)\tensor\id_{\ol{\psi_Z^Z}}\otimes\id_{\ol{P_Z^Z}}.\]

By dinaturality in the upper factor for the map $r\otimes\id_{\ol{\phi_X^X}}:\ol{\psi_Y^Y}\otimes\ol{\phi_X^X}\to\ol{P_X^X}\otimes\ol{\phi_X^X}$....

\bigskip

Here's a picture that may be useful.
\activetikz{
	%outer box Z
	\path(0,0);\blackbox{(10,8)}{3}{3}{$Z$}{.8}
	%mid box Y
	\path(2.5,1.5);\blackbox{(5,5)}{4}{4}{$Y$}{.6}
	%inner box X
	\path(4,3);\blackbox{(2,2)}{3}{3}{$X$}{.4}
	%outer morphism
	% Z feeds Z
	\fancyarc{(.4,6)}{(9.6,6)}{-30}{30}\node at (5,7.1){\tiny$\psi_Z^Z$};
	% Z feeds Y
	\directarc{(.4,4)}{(2.2,5.5)}\node at (1.5,5.1){\tiny$\psi_Y^Z$};
	\directarc{(.4,2)}{(2.2,4.5)}\node at (1.5,4.1){\tiny$\psi_Y^Z$};
	% Y feeds Z
	\directarc{(7.8,5.5)}{(9.6,4)}\node at (8.5,5.1){\tiny$\psi_Z^Y$};
	\directarc{(7.8,4.5)}{(9.6,2)}\node at (8.5,4.1){\tiny$\psi_Z^Y$};
	% Y feeds Y
	\fancyarc{(2.2,3.5)}{(7.8,3.5)}{40}{-90}\node at (5,.8){\tiny$\psi_Y^Y$};
	\fancyarc{(2.2,2.5)}{(7.8,2.5)}{20}{-50}\node at (5,.4){\tiny$\psi_Y^Y$};
	%circles
	\draw(5,1.1) circle(5pt);\node at (5.2,1.1){\tiny$\psi_{loop}$};
	%inner morphism
	%Y feeds Y
	\directarc{(2.8,5.5)}{(7.2,5.5)}\node at (5,5.5){\tiny$\phi_Y^Y$};
	\directarc{(2.8,2.5)}{(7.2,2.5)}\node at (5,2.5){\tiny$\phi_Y^Y$};
	%Y feeds X
	\directarc{(2.8,4.5)}{(3.8,4.5)}\node at (3.5,4.5){\tiny$\phi_X^Y$};
	\directarc{(2.8,3.5)}{(3.8,4)}\node at (3.5,3.8){\tiny$\phi_X^Y$};
	%X feeds Y
	\directarc{(6.2,4.5)}{(7.2,4.5)}\node at (6.5,4.5){\tiny$\phi_Y^X$};
	\directarc{(6.2,4)}{(7.2,3.5)}\node at (6.5,3.8){\tiny$\phi_Y^X$};
	%X feeds X 
	\fancyarc{(3.8,3.5)}{(6.2,3.5)}{10}{-23}\node at (5,2.7){\tiny$\phi_X^X$};
	%circles
	\draw(5,2) circle(5pt);\node at (5.2,2){\tiny$\phi_{loop}$};
}

\activetikz{
	%outer box Z
	\path(0,0);\blackbox{(10,8)}{3}{3}{$Z$}{.8}
	%inner box X
	\path(4,1.5);\blackbox{(2,2)}{3}{3}{$X$}{.4}
	% Z feeds Z
	\directarc{(.4,6)}{(9.6,6)}
	\directarc{(.4,4)}{(9.6,4)}
	% Z feeds X
	\directarc{(.4,2)}{(3.8,3)}
	% X feeds Z
	\directarc{(6.2,3)}{(9.6,2)}
	% X feeds X
	\fancyarc{(3.8,2.5)}{(6.2,2.5)}{20}{-60}
	\fancyarc{(3.8,2)}{(6.2,2)}{10}{-30}
	%circles
	\draw(2,1) circle(5pt);
	\draw(2.5,1.2) circle(5pt);
	\draw(3,.7) circle(5pt);
}

\end{proof}

\section{An alternate composition formula}

The following composition formula may or may not be the same as the one above. 

Suppose given oriented $0$-manifolds $X,Y,X$ and oriented cobordisms $\phi\taking X\to Y$ and $\psi\taking Y\to Z$. Let's reconsider how to think of $\phi$. Above we formulate it as a bijection $\phi\taking\inp{X}\sqcup\outp{Y}\To{\iso}\outp{X}\sqcup\inp{Y}$, which tells us a kind of supply-meets-demand story, together with a typed set of loops. Below we will understand wiring diagrams in a different way.

A newmap $X\to Y$ consists of a set $P$ together with four typed functions, arranged as to the left below:
$$\xymatrix{
&\inp{X}\ar[d]\\
\outp{X}\ar[r]&P&\outp{Y}\ar[l]\\
&\inp{Y}\ar[u]
}
\hspace{.8in}
\xymatrix{
P^X_X\ar[r]\ar[d]\ullimit&\inp{X}\ar[d]&P^X_Y\ar[l]\ar[d]\urlimit\\
\outp{X}\ar[r]&P&\outp{Y}\ar[l]\\
P^Y_X\ar[r]\ar[u]\lllimit&\inp{Y}\ar[u]&P_Y^Y\ar[u]\ar[l]\lrlimit
}
$$
with the property that if one forms fiber products as to the right above, then each of: the top horizontal, the bottom horizontal, the left vertical, and the right vertical pairs of morphisms is a coproduct. That is, 
$$
P^X_X\sqcup P^Y_X=\inp{X},\;\;\;\;\; P^X_Y\sqcup P^Y_Y=\outp{Y},\;\;\;\;\tn{etc}
$$
Then I think the following may be true. Given composable newmaps as below
$$\xymatrix{
&\inp{X}\ar[d]\\
\outp{X}\ar[r]&P&\outp{Y}\ar[l]\ar[d]\\
&\inp{Y}\ar[u]\ar[r]&Q&\outp{Z}\ar[l]\\
&&\inp{Z}\ar[u]
}
$$
form the colimit $P\sqcup_YQ$ of the diagram to the left below, and recover a questionable-composite as to the right.
$$
\xymatrix{
P\ar[dr]&&\outp{Y}\ar[ll]\ar[dd]\ar[dl]\\
&P\sqcup_YQ\\
\inp{Y}\ar[ur]\ar[uu]\ar[rr]&&Q\ar[ul]
}
\hspace{.8in}
\xymatrix{
&\inp{X}\ar[d]\\
\outp{X}\ar[r]&P\sqcup_YQ&\outp{Z}\ar[l]\\
&\inp{Z}\ar[u]
}
$$

\section{Yanking Compositions of Wiring Diagrams}
We have some technique ,``yanking", by which we trade $\phi,\psi$ in for $\phi',\psi'$ such that there's an equivalence and two nice facts:
$$\xymatrix@R=10pt{
&Y\ar[dr]^\psi\ar@{}[dd]|{\Updownarrow}\\
X\ar[ur]^\phi\ar[dr]_{\phi'}&&Z\\
&Y'\ar[ur]_{\psi'}
}
\hsp\tn{such that}
\hspace{.7in}
\xymatrix{
\emptyset\ar[r]\ar[d]\ullimit&{\phi'}_Y^Y\ar[d]\\
{\psi'}_Y^Y\ar[r]&\inp{Y'}
}
\hspace{.7in}
\xymatrix{
\emptyset\ar[r]\ar[d]\ullimit&{\phi'}_Y^Y\ar[d]\\
{\psi'}_Y^Y\ar[r]&\outp{Y'}
}
$$

Then we have $(\psi'\circ\phi')_X^X={\phi'}_X^X+{\psi'}_Y^Y$, ${P'}_X^X={\psi'}_Y^Y$, ${P'}_Z^Z={\phi'}_Y^Y$.  And so compatibility of trace and compositions is immediate.

\end{document}