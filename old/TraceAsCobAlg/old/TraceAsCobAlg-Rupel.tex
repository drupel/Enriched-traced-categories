\documentclass{amsart}

\usepackage{savesym}
\usepackage{txfonts,stmaryrd,accents}
\savesymbol{lrcorner}\savesymbol{ulcorner}
\usepackage{amssymb, amscd,setspace,mathtools,makecell}
\usepackage{enumerate}
\usepackage[usenames,dvipsnames]{xcolor}
\usepackage[bookmarks=true,colorlinks=true, linkcolor=MidnightBlue, citecolor=cyan]{hyperref}
\usepackage{lmodern}
\usepackage{graphicx,float}
\restoresymbol{txfonts}{lrcorner}\restoresymbol{txfonts}{ulcorner}
\usepackage{tensor}

\usepackage{tikz}
\usetikzlibrary{arrows,calc,chains,matrix,positioning,scopes,snakes}
%
%%Begin tikz macros
%\def\blackbox#1#2#3#4#5{%(width,height), number inputs, number outputs, label, arrow length
%  \pgfgetlastxy{\llx}{\lly}%assumes path has been set to a point representing the lower left corner of the box
%  \path #1;
%  \pgfgetlastxy{\w}{\h}
%  \pgfmathsetlengthmacro{\urx}{\llx+\w}
%  \pgfmathsetlengthmacro{\ury}{\lly+\h}
%  \draw (\llx,\lly) rectangle (\urx,\ury);
%  \pgfmathsetlengthmacro{\xave}{(\llx+\urx)/2}
%  \pgfmathsetlengthmacro{\yave}{\ury-8}
%  \node at (\xave,\yave) {#4};
%  \pgfmathsetlengthmacro{\ydiff}{\ury-\lly}
%  \pgfmathsetlengthmacro{\lstep}{\ydiff/(#2+1)}
%  \pgfmathsetlengthmacro{\rstep}{\ydiff/(#3+1)}
%  \ifnum #2=0{}\else{ 
%   \foreach \l in {1,...,#2}{
%    \draw [->] ($(\llx,\lly)+(-#5/2,0)+\l*(0,\lstep)$) -- ($(\llx,\lly)+(#5/2,0)+\l*(0,\lstep)$);}}\fi
%  \ifnum #3=0{}\else{
%   \foreach \r in {1,...,#3}{
%    \draw [->] ($(\urx,\ury)+(-#5/2,0)-\r*(0,\rstep)$) -- ($(\urx,\ury)+(#5/2,0)-\r*(0,\rstep)$);}}\fi
%}
%
%\def\blackboxinners#1#2#3#4#5{%(width,height), number inputs, number outputs, label, arrow length
%  \pgfgetlastxy{\llx}{\lly}%assumes path has been set to a point representing the lower left corner of the box
%  \path #1;
%  \pgfgetlastxy{\w}{\h}
%  \pgfmathsetlengthmacro{\urx}{\llx+\w}
%  \pgfmathsetlengthmacro{\ury}{\lly+\h}
%  \draw (\llx,\lly) rectangle (\urx,\ury);
%  \pgfmathsetlengthmacro{\xave}{(\llx+\urx)/2}
%  \pgfmathsetlengthmacro{\yave}{\ury-8}
%  \node at (\xave,\yave) {#4};
%  \pgfmathsetlengthmacro{\ydiff}{\ury-\lly}
%  \pgfmathsetlengthmacro{\lstep}{\ydiff/(#2+1)}
%  \pgfmathsetlengthmacro{\rstep}{\ydiff/(#3+1)}
%  \ifnum #2=0{}\else{ 
%   \foreach \l in {1,...,#2}{
%    \pgfmathsetlengthmacro{\newx}{\llx+#5*28.45274/2}
%    \pgfmathsetlengthmacro{\newy}{\lly+\l*\lstep}
%    \node at ($(\newx,\newy)+(-1.5,\l*12-\l*\lstep)$) {\tiny$(\pgfmathparse{\newx/28.45274}\pgfmathresult cm,\pgfmathparse{\newy/28.45274}\pgfmathresult cm)$};
%    \draw [->] ($(\llx,\lly)+(-#5/2,0)+\l*(0,\lstep)$) -- ($(\llx,\lly)+(#5/2,0)+\l*(0,\lstep)$);}}\fi
%  \ifnum #3=0{}\else{
%   \foreach \r in {1,...,#3}{
%    \pgfmathsetlengthmacro{\newx}{\urx-#5*28.45274/2}
%    \pgfmathsetlengthmacro{\newy}{\ury-\r*\rstep}
%    \node at ($(\newx,\newy)+(1.5,-\r*12+\r*\rstep)$) {\tiny $(\pgfmathparse{\newx/28.45274}\pgfmathresult cm,\pgfmathparse{\newy/28.45274}\pgfmathresult cm)$};
%    \draw [->] ($(\urx,\ury)+(-#5/2,0)-\r*(0,\rstep)$) -- ($(\urx,\ury)+(#5/2,0)-\r*(0,\rstep)$);}}\fi
%}
%
%\def\blackboxouters#1#2#3#4#5{%(width,height), number inputs, number outputs, label, arrow length
%  \pgfgetlastxy{\llx}{\lly}%assumes path has been set to a point representing the lower left corner of the box
%  \path #1;
%  \pgfgetlastxy{\w}{\h}
%  \pgfmathsetlengthmacro{\urx}{\llx+\w}
%  \pgfmathsetlengthmacro{\ury}{\lly+\h}
%  \draw (\llx,\lly) rectangle (\urx,\ury);
%  \pgfmathsetlengthmacro{\xave}{(\llx+\urx)/2}
%  \pgfmathsetlengthmacro{\yave}{\ury-8}
%  \node at (\xave,\yave) {#4};
%  \pgfmathsetlengthmacro{\ydiff}{\ury-\lly}
%  \pgfmathsetlengthmacro{\lstep}{\ydiff/(#2+1)}
%  \pgfmathsetlengthmacro{\rstep}{\ydiff/(#3+1)}
%  \ifnum #2=0{}\else{ 
%   \foreach \l in {1,...,#2}{
%    \pgfmathsetlengthmacro{\newx}{\llx-#5*28.45274/2}
%    \pgfmathsetlengthmacro{\newy}{\lly+\l*\lstep}
%    \node at ($(\newx,\newy)+(-1.5,\l*12-\l*\lstep)$) {\tiny$(\pgfmathparse{\newx/28.45274}\pgfmathresult cm,\pgfmathparse{\newy/28.45274}\pgfmathresult cm)$};
%    \draw [->] ($(\llx,\lly)+(-#5/2,0)+\l*(0,\lstep)$) -- ($(\llx,\lly)+(#5/2,0)+\l*(0,\lstep)$);}}\fi
%  \ifnum #3=0{}\else{
%   \foreach \r in {1,...,#3}{
%    \pgfmathsetlengthmacro{\newx}{\urx+#5*28.45274/2}
%    \pgfmathsetlengthmacro{\newy}{\ury-\r*\rstep}
%    \node at ($(\newx,\newy)+(1.5,-\r*12+\r*\rstep)$) {\tiny $(\pgfmathparse{\newx/28.45274}\pgfmathresult cm,\pgfmathparse{\newy/28.45274}\pgfmathresult cm)$};
%    \draw [->] ($(\urx,\ury)+(-#5/2,0)-\r*(0,\rstep)$) -- ($(\urx,\ury)+(#5/2,0)-\r*(0,\rstep)$);}}\fi
%}
%
%\def\dashbox#1#2#3#4#5{%(width,height), number inputs, number outputs, label, arrow length
%  \pgfgetlastxy{\llx}{\lly}%assumes path has been set to a point representing the lower left corner of the box
%  \path #1;
%  \pgfgetlastxy{\w}{\h}
%  \pgfmathsetlengthmacro{\urx}{\llx+\w}
%  \pgfmathsetlengthmacro{\ury}{\lly+\h}
%  \draw [dashed] (\llx,\lly) rectangle (\urx,\ury);
%  \pgfmathsetlengthmacro{\xave}{(\llx+\urx)/2}
%  \pgfmathsetlengthmacro{\yave}{\ury-8}
%  \node at (\xave,\yave) {#4};
%  \pgfmathsetlengthmacro{\ydiff}{\ury-\lly}
%  \pgfmathsetlengthmacro{\lstep}{\ydiff/(#2+1)}
%  \pgfmathsetlengthmacro{\rstep}{\ydiff/(#3+1)}
%  \ifnum #2=0{}\else{ 
%   \foreach \l in {1,...,#2}{
%    \draw [->] ($(\llx,\lly)+(-#5/2,0)+\l*(0,\lstep)$) -- ($(\llx,\lly)+(#5/2,0)+\l*(0,\lstep)$);}}\fi
%  \ifnum #3=0{}\else{
%   \foreach \r in {1,...,#3}{
%    \draw [->] ($(\urx,\ury)+(-#5/2,0)-\r*(0,\rstep)$) -- ($(\urx,\ury)+(#5/2,0)-\r*(0,\rstep)$);}}\fi
%}
%
%\def\fillbox#1#2#3#4#5#6{%(width,height), number inputs, number outputs, label, arrow length, fill strength
%  \pgfgetlastxy{\llx}{\lly}%assumes path has been set to a point representing the lower left corner of the box
%  \path #1;
%  \pgfgetlastxy{\w}{\h}
%  \pgfmathsetlengthmacro{\urx}{\llx+\w}
%  \pgfmathsetlengthmacro{\ury}{\lly+\h}
%  \filldraw [fill=gray!#6] (\llx,\lly) rectangle (\urx,\ury);
%  \pgfmathsetlengthmacro{\xave}{(\llx+\urx)/2}
%  \pgfmathsetlengthmacro{\yave}{\ury-8}
%  \node at (\xave,\yave) {#4};
%  \pgfmathsetlengthmacro{\ydiff}{\ury-\lly}
%  \pgfmathsetlengthmacro{\lstep}{\ydiff/(#2+1)}
%  \pgfmathsetlengthmacro{\rstep}{\ydiff/(#3+1)}
%  \ifnum #2=0{}\else{ 
%   \foreach \l in {1,...,#2}{
%    \draw [->] ($(\llx,\lly)+(-#5/2,0)+\l*(0,\lstep)$) -- ($(\llx,\lly)+(#5/2,0)+\l*(0,\lstep)$);}}\fi
%  \ifnum #3=0{}\else{
%   \foreach \r in {1,...,#3}{
%    \draw [->] ($(\urx,\ury)+(-#5/2,0)-\r*(0,\rstep)$) -- ($(\urx,\ury)+(#5/2,0)-\r*(0,\rstep)$);}}\fi
%}
%
%
%\def\delaynode#1#2#3{%(x-coord,y-coord), size, arrow length
%  \path #1;
%  \pgfgetlastxy{\lx}{\ly}
%  \pgfmathsetlengthmacro{\rx}{\lx+#2*3+#3}
%  \pgfmathsetlengthmacro{\ry}{\ly}
%  \filldraw (\lx,\ly) circle (#2 pt);
%  \draw [->] (\lx,\ly) -- (\rx,\ry);
%}
%
%\def\delaynodeouters#1#2#3{%(x-coord,y-coord), size, arrow length
%  \path #1;
%  \pgfgetlastxy{\lx}{\ly}
%  \pgfmathsetlengthmacro{\rx}{\lx+#2*3+#3}
%  \pgfmathsetlengthmacro{\ry}{\ly}
%  \filldraw (\lx,\ly) circle (#2 pt);
%  \draw [->] (\lx,\ly) -- (\rx,\ry);
%   \node at (\rx+15,\ry+15){\tiny $(\pgfmathparse{\rx/28.45274}\pgfmathresult cm,\pgfmathparse{\ry/28.45274}\pgfmathresult cm)$};
%}
%
%
%\def\directarc#1#2{%left endpoint, right endpoint
%  \path #1;
%  \pgfgetlastxy{\lx}{\ly}
%  \path #2;
%  \pgfgetlastxy{\rx}{\ry}
%  \pgfmathsetlengthmacro{\xave}{(\lx+\rx)/2}
%  \draw #1 .. controls (\xave,\ly) and (\xave,\ry) .. #2;
%}
%
%\def\backarc#1#2{%left coordinate, right coordinate
%  \path #1;
%  \pgfgetlastxy{\lx}{\ly}
%  \path #2;
%  \pgfgetlastxy{\rx}{\ry}
%  \pgfmathsetlengthmacro{\xave}{(\lx+\rx)/2}
%  \draw #1 .. controls (\xave,\ry) and (\xave,\ly) .. #2;
%}
%
%\def\loopright#1#2#3{%upper coordinate, lower coordinate, stretch width
%  \path #1;
%  \pgfgetlastxy{\ux}{\uy}
%  \path #2;
%  \pgfgetlastxy{\lx}{\ly}
%  \pgfmathsetlengthmacro{\maxx}{max(\ux,\lx)}
%  \pgfmathsetlengthmacro{\farx}{\maxx+#3}
%  \draw #1 .. controls (\farx,\uy) and (\farx,\ly) .. #2;
%}
%
%\def\loopleft#1#2#3{%upper coordinate, lower coordinate, stretch width
%  \path #1;
%  \pgfgetlastxy{\ux}{\uy}
%  \path #2;
%  \pgfgetlastxy{\lx}{\ly}
%  \pgfmathsetlengthmacro{\minx}{min(\ux,\lx)}
%  \pgfmathsetlengthmacro{\farx}{\minx-#3}
%  \draw #1 .. controls (\farx,\uy) and (\farx,\ly) .. #2;
%}
%
%\def\fancyarc#1#2#3#4{%upper coordinate, lower coordinate, stretch width, max height adjust
%  \path #1;
%  \pgfgetlastxy{\ux}{\uy}
%  \path #2;
%  \pgfgetlastxy{\lx}{\ly}
%  \pgfmathsetlengthmacro{\xave}{(\lx+\ux)/2}
%%  \node at (\lx,\ly+20){\tiny $\pgfmathparse{\lx/28.45274}\pgfmathresult cm$,\hsp$\pgfmathparse{\ux/28.45274}\pgfmathresult cm$};
%%  \node at (\xave,\ly+50){\tiny $\pgfmathparse{\xave/28.45274}\pgfmathresult cm$};
%  \pgfmathsetlengthmacro{\yave}{(\ly+\uy)/2+#4}
%  \loopleft{#1}{(\xave,\yave)}{#3}
%  \loopright{#2}{(\xave,\yave)}{#3}
%}
%
%\def\activetikz#1{$$\begin{tikzpicture}#1\end{tikzpicture}$$}
%\def\inactivetikz#1{}
%%End tikz macros


\input xy
\xyoption{all} \xyoption{poly} \xyoption{knot}\xyoption{curve}
\usepackage{xy,color}
\newcommand{\comment}[1]{}

\newcommand{\longnote}[2][4.9in]{\fcolorbox{black}{yellow}{\parbox{#1}{\color{black} #2}}}
\newcommand{\shortnote}[1]{\fcolorbox{black}{yellow}{\color{black} #1}}
\newcommand{\start}[1]{\shortnote{Start here: #1.}}
\newcommand{\q}[1]{\begin{question}#1\end{question}}
\newcommand{\g}[1]{\begin{guess}#1\end{guess}}
\newcommand{\erase}[1]{{}}

\def\tn{\textnormal}
\def\mf{\mathfrak}
\def\mc{\mathcal}

\def\ZZ{{\mathbb Z}}
\def\QQ{{\mathbb Q}}
\def\RR{{\mathbb R}}
\def\CC{{\mathbb C}}
\def\AA{{\mathbb A}}
\def\PP{{\mathbb P}}
\def\NN{{\mathbb N}}
\def\SS{{\mathbb S}}
\def\HH{{\mathbb H}}

\def\acts{\lefttorightarrow}
\def\Hom{\tn{Hom}}
\def\iHom{\Rightarrow}
\def\Aut{\tn{Aut}}
\def\Sub{\tn{Sub}}
\def\Mor{\tn{Mor}}
\def\Fun{\tn{Fun}}
\def\Path{\tn{Path}}
\def\im{\tn{im}}
\def\Ob{\tn{Ob}}
\def\dim{\tn{dim}}
\def\Trace{\tn{Tr}}
\def\Op{\tn{Op}}
\def\SEL*{\tn{SEL*}}
\def\Res{\tn{Res}}
\def\hsp{\hspace{.3in}}
\newcommand{\hsps}[1]{{\hspace{2mm} #1\hspace{2mm}}}
\newcommand{\tin}[1]{\text{\tiny #1}}

\def\singleton{{\{*\}}}
\newcommand{\boxtitle}[1]{\begin{center}#1\end{center}}
\def\Loop{{\mcL oop}}
\def\LoopSchema{{\parbox{.5in}{\fbox{\xymatrix{\LMO{s}\ar@(l,u)[]^f}}}}}
\def\Wks{{\mcW ks}}
\def\lcone{^\triangleleft}
\def\rcone{^\triangleright}
\def\to{\rightarrow}
\def\from{\leftarrow}
\def\down{\downarrrow}
\def\Down{\Downarrow}
\def\Up{\Uparrow}
\def\taking{\colon}
\def\pls{``\!+\!"}
\newcommand{\pathto}[1]{\stackrel{#1}\leadsto}
\def\inj{\hookrightarrow}
\def\surj{\twoheadrightarrow}
\def\surjj{\longtwoheadrightarrow}
\def\pfunc{\rightharpoonup}
\def\Pfunc{\xrightharpoonup}
\def\too{\longrightarrow}
\def\fromm{\longleftarrow}
\def\tooo{\longlongrightarrow}
\def\tto{\rightrightarrows}
\def\ttto{\equiv\!\!>}
\newcommand{\xyto}[2]{\xymatrix@=1pt{\ar[rr]^-{#1}&\hspace{#2}&}}
\newcommand{\xyequals}[1]{\xymatrix@=1pt{\ar@{=}[rr]&\hspace{#1}&}}
\def\ss{\subseteq}
\def\superset{\supseteq}
\def\iso{\cong}
\def\down{\downarrow}
\def\|{{\;|\;}}
\def\m1{{-1}}
\def\op{^\tn{op}}
\def\la{\langle}
\def\ra{\rangle}
\def\wt{\widetilde}
\def\wh{\widehat}
\def\we{\simeq}
\def\ol{\overline}
\def\ul{\underline}
\def\vect{\overrightarrow}
\def\qeq{\mathop{=}^?}

\def\rr{\raggedright}

%\newcommand{\LMO}[1]{\bullet^{#1}}
%\newcommand{\LTO}[1]{\bullet^{\tn{#1}}}
\newcommand{\LMO}[1]{\stackrel{#1}{\bullet}}
\newcommand{\LTO}[1]{\stackrel{\tt{#1}}{\bullet}}
\newcommand{\LA}[2]{\ar[#1]^-{\tn {#2}}}
\newcommand{\LAL}[2]{\ar[#1]_-{\tn {#2}}}
\newcommand{\obox}[3]{\stackrel{#1}{\fbox{\parbox{#2}{#3}}}}
\newcommand{\labox}[2]{\obox{#1}{1.6in}{#2}}
\newcommand{\mebox}[2]{\obox{#1}{1in}{#2}}
\newcommand{\smbox}[2]{\stackrel{#1}{\fbox{#2}}}
\newcommand{\fakebox}[1]{\tn{$\ulcorner$#1$\urcorner$}}
\newcommand{\sq}[4]{\xymatrix{#1\ar[r]\ar[d]&#2\ar[d]\\#3\ar[r]&#4}}
\newcommand{\namecat}[1]{\begin{center}$#1:=$\end{center}}


\def\ullimit{\ar@{}[rd]|(.3)*+{\lrcorner}}
\def\urlimit{\ar@{}[ld]|(.3)*+{\llcorner}}
\def\lllimit{\ar@{}[ru]|(.3)*+{\urcorner}}
\def\lrlimit{\ar@{}[lu]|(.25)*+{\ulcorner}}
\def\ulhlimit{\ar@{}[rd]|(.3)*+{\diamond}}
\def\urhlimit{\ar@{}[ld]|(.3)*+{\diamond}}
\def\llhlimit{\ar@{}[ru]|(.3)*+{\diamond}}
\def\lrhlimit{\ar@{}[lu]|(.3)*+{\diamond}}
\newcommand{\clabel}[1]{\ar@{}[rd]|(.5)*+{#1}}
\newcommand{\TriRight}[7]{\xymatrix{#1\ar[dr]_{#2}\ar[rr]^{#3}&&#4\ar[dl]^{#5}\\&#6\ar@{}[u] |{\Longrightarrow}\ar@{}[u]|>>>>{#7}}}
\newcommand{\TriLeft}[7]{\xymatrix{#1\ar[dr]_{#2}\ar[rr]^{#3}&&#4\ar[dl]^{#5}\\&#6\ar@{}[u] |{\Longleftarrow}\ar@{}[u]|>>>>{#7}}}
\newcommand{\TriIso}[7]{\xymatrix{#1\ar[dr]_{#2}\ar[rr]^{#3}&&#4\ar[dl]^{#5}\\&#6\ar@{}[u] |{\Longleftrightarrow}\ar@{}[u]|>>>>{#7}}}


\newcommand{\arr}[1]{\ar@<.5ex>[#1]\ar@<-.5ex>[#1]}
\newcommand{\arrr}[1]{\ar@<.7ex>[#1]\ar@<0ex>[#1]\ar@<-.7ex>[#1]}
\newcommand{\arrrr}[1]{\ar@<.9ex>[#1]\ar@<.3ex>[#1]\ar@<-.3ex>[#1]\ar@<-.9ex>[#1]}
\newcommand{\arrrrr}[1]{\ar@<1ex>[#1]\ar@<.5ex>[#1]\ar[#1]\ar@<-.5ex>[#1]\ar@<-1ex>[#1]}

\newcommand{\into}{\hookrightarrow}
\newcommand{\onto}{\to\!\!\!\!\!\to}
\newcommand{\To}[1]{\xrightarrow{#1}}
\newcommand{\Too}[1]{\xrightarrow{\ \ #1\ \ }}
\newcommand{\From}[1]{\xleftarrow{#1}}
\newcommand{\Fromm}[1]{\xleftarrow{\ \ #1\ \ }}
\def\qeq{\stackrel{?}{=}}
\newcommand{\cev}[1]{\reflectbox{\ensuremath{\vec{\reflectbox{\ensuremath{#1}}}}}}

\newcommand{\Adjoint}[4]{\xymatrix@1{{#2}\ar@<.5ex>[r]^-{#1} &{#3} \ar@<.5ex>[l]^-{#4}}}

\def\id{\tn{id}}
\def\dom{\tn{dom}}
\def\cod{\tn{cod}}
\def\Top{{\bf Top}}
\def\Kls{{\bf Kls}}
\def\Cat{{\bf Cat}}
\def\Oprd{{\bf Oprd}}
\def\LH{{\bf LH}}
\def\Monad{{\bf Monad}}
\def\Mon{{\bf Mon}}
\def\CMon{{\bf CMon}}
\def\cpo{{\bf cpo}}
\def\Vect{\text{Vect}}
\def\Rep{{\bf Rep}}
\def\HCat{{\bf HCat}}
\def\Cnst{{\bf Cnst}}
\def\Str{\tn{Str}}
\def\List{\tn{List}}
\def\Exc{\tn{Exc}}
\def\Sets{{\bf Sets}}
\def\Ord{{\bf Ord}}
\def\Cob{{\bf Cob}}
\def\Grph{{\bf Grph}}
\def\SGrph{{\bf SGrph}}
\def\Rel{\mcR\tn{el}}
\def\JRel{J\mcR\tn{el}}
\def\Inst{{\bf Inst}}
\def\Type{{\bf Type}}
\def\Set{{\bf Set}}
\def\TFS{{\bf TFS}}
\def\Qry{{\bf Qry}}
\def\set{{\text \textendash}{\bf Set}}
\def\sets{{\text \textendash}{\bf Alg}}
\def\alg{{\text \textendash}{\bf Alg}}
\def\rel{{\text \textendash}{\bf Rel}}
\def\inst{{{\text \textendash}\bf \Inst}}
\def\sSet{{\bf sSet}}
\def\sSets{{\bf sSets}}
\def\Grp{{\bf Grp}}
\def\Grpd{{\bf Grpd}}
\def\Pre{{\bf Pre}}
\def\Shv{{\bf Shv}}
\def\Rings{{\bf Rings}}
\def\bD{{\bf \Delta}}
\def\dispInt{\parbox{.1in}{$\int$}}
\def\bhline{\Xhline{2\arrayrulewidth}}
\def\bbhline{\Xhline{2.5\arrayrulewidth}}


\def\Comp{\tn{Comp}}
\def\Supp{\tn{Supp}}
\def\Dmnd{\tn{Dmnd}}


\def\colim{\mathop{\tn{colim}}}
\def\hocolim{\mathop{\tn{hocolim}}}
\def\undsc{\rule{2mm}{0.4pt}}


\def\mcA{\mc{A}}
\def\mcB{\mc{B}}
\def\mcC{\mc{C}}
\def\mcD{\mc{D}}
\def\mcE{\mc{E}}
\def\mcF{\mc{F}}
\def\mcG{\mc{G}}
\def\mcH{\mc{H}}
\def\mcI{\mc{I}}
\def\mcJ{\mc{J}}
\def\mcK{\mc{K}}
\def\mcL{\mc{L}}
\def\mcM{\mc{M}}
\def\mcN{\mc{N}}
\def\mcO{\mc{O}}
\def\mcP{\mc{P}}
\def\mcQ{\mc{Q}}
\def\mcR{\mc{R}}
\def\mcS{\mc{S}}
\def\mcT{\mc{T}}
\def\mcU{\mc{U}}
\def\mcV{\mc{V}}
\def\mcW{\mc{W}}
\def\mcX{\mc{X}}
\def\mcY{\mc{Y}}
\def\mcZ{\mc{Z}}

\def\bfe{{\bf e}}
\def\bfo{{\bf o}}
\def\bfS{{\bf S}}\def\bfSs{{\bf Ss}}
\def\bfT{{\bf T}}\def\bfTs{{\bf Ts}}
\def\bfW{{\bf W}}

\def\tnN{\tn{N}}


\def\bE{\bar{E}}
\def\bF{\bar{F}}
\def\bG{\bar{G}}
\def\bH{\bar{H}}
\def\bX{\bar{X}}
\def\bY{\bar{Y}}

\newcommand{\subsub}[1]{\setcounter{subsubsection}{\value{theorem}}\subsubsection{#1}\addtocounter{theorem}{1}}

\def\Finm{{\bf Fin_{m}}}
\def\Bag{\tn{Bag}}
\newcommand{\back}[1]{\stackrel{\from}{#1}\!}
%\newcommand{\kls}[1]{{\text \textendash}\wt{\bf Kls}({#1})}
\def\Dist{\text{Dist}}
\def\Dst{{\bf Dst}}
\def\WkFlw{{\bf WkFlw}}
\def\monOb{{\blacktriangle}}
\def\Infl{{\bf Infl}}
\def\Tur{\tn{Tur}}
\def\Halt{\{\text{Halt}\}}
\def\Tape{{T\!ape}}
\def\Prb{{\bf Prb}}
\def\Prbs{{\wt{\bf Prb}}}
\def\El{{\bf El}}
\def\Gr{{\bf Gr}}
\def\DT{{\bf DT}}
\def\DB{{\bf DB}}
\def\Tables{{\bf Tables}}
\def\Sch{{\bf Sch}}
\def\Fin{{\bf Fin}}
\def\PrO{{\bf PrO}}
\def\PrOs{{\bf PrOs}}
\def\JLat{{\bf JLat}}
\def\JLats{{\bf JLats}}
\def\P{{\bf P}}
\def\SC{{\bf SC}}
\def\ND{{\bf ND}}
\def\Poset{{\bf Poset}}
\def\Bool{\tn{Bool}}
\newcommand{\labelDisp}[2]{\begin{align}\label{#1}\text{#2}\end{align}}

\newcommand{\inp}[1]{{#1_-}}
\newcommand{\outp}[1]{{#1_+}}
\newcommand{\vset}[1]{#1_{\tt type}}
\newcommand{\loc}[1]{{\tt loc}(#1)}
\newcommand{\extr}[1]{{\tt ext}(#1)}
\newcommand{\domn}[1]{{\accentset{\bullet}{#1}}}
\newcommand{\codomn}[1]{{\underaccent{\bullet}{#1}}}
\newcommand{\outpm}[1]{{{\scriptstyle\bullet}#1}}
\newcommand{\inpm}[1]{{#1{\scriptstyle\bullet}}}
%\newcommand{\domn}[1]{{{\scriptstyle\bullet}#1}}
%\newcommand{\codomn}[1]{{#1{\scriptstyle\bullet}}}
%\newcommand{\inpm}[1]{{\underset{\bullet}{#1}}}
%\newcommand{\outpm}[1]{{\overset{\bullet}{#1}}}

%sets:
\newcommand{\feeddd}[3]{{\tensor*[^{#2}_{\color{white}{!}}]{{#1}}{^{#3}}}}%the color thing is to get overlines to be the same height.
\newcommand{\feeddc}[3]{{\tensor*[^{#2}]{{#1}}{_{#3}}}}
\newcommand{\feedcd}[3]{{\tensor*[_{#2}]{{#1}}{^{#3}}}}
\newcommand{\feedcc}[3]{{\tensor*[^{\color{white}{!}}_{#2}]{{#1}}{_{#3}}}}
%maps
\newcommand{\feeddb}[2]{{\tensor*[^{#2}]{{#1}}{}}}
\newcommand{\feedbc}[2]{{\tensor*{{#1}}{^~_{#2}}}}
\newcommand{\feedcb}[2]{{\tensor*[^~_{#2}]{{#1}}{}}}
\newcommand{\feedbd}[2]{{\tensor*{{#1}}{^{#2}}}}
%horrible maps
\newcommand{\feedda}[3]{{\tensor*[^{#2}_{\color{white}{!}}]{{#1}}{^{#2}_{#3}}}}
\newcommand{\feedca}[3]{{\tensor*[_{#2}]{{#1}}{_{#2}^{#3}}}}
\newcommand{\feedad}[3]{{\tensor*[^{#2}_{#3}]{{#1}}{^{#2}}}}
\newcommand{\feedac}[3]{{\tensor*[_{#2}^{#3}]{{#1}}{_{#2}}}}



\newcommand{\vLst}[1]{\ol{#1}}
\newcommand{\Strm}[1]{{\tn{Strm}(#1)}}
\newcommand{\SP}[2]{{\tn{SP}(#1,#2)}}
\newcommand{\LP}[2]{{\tn{LP}(#1,#2)}}
\newcommand{\LPP}[2]{{\tn{LP}'(#1,#2)}}
\newcommand{\Hist}{\tn{Hist}}
\newcommand{\xleadsto}[1]{\stackrel{#1}{\leadsto}}
\newcommand{\strst}[1]{\big|_{[1,#1]}} %"stream restriction"
\newcommand{\Del}[1]{DN_{#1}}
\newcommand{\Dem}[1]{{Dm_{#1}}}
\newcommand{\Sup}[1]{{Sp_{#1}}}
%\newcommand{\Con}[1]{{Con_{#1}}}
\newcommand{\inDem}[1]{{in\Dem{#1}}}
\newcommand{\inSup}[1]{{in\Sup{#1}}}
\newcommand{\vinSup}[1]{\vLst{\inSup{#1}}}
\newcommand{\vinDem}[1]{\vLst{\inDem{#1}}}
\newcommand{\vSup}[1]{\vLst{\Sup{#1}}}
\newcommand{\vDem}[1]{\vLst{\Dem{#1}}}
\newcommand{\vDel}[1]{\vLst{\Del{#1}}}
\newcommand{\vinp}[1]{\vLst{\inp{#1}}}
\newcommand{\vintr}[1]{\vLst{\loc{#1}}}
\newcommand{\voutp}[1]{\vLst{\outp{#1}}}
\def\zipwith{\;\raisebox{3pt}{${}_\varcurlyvee$}\;}
\def\tbzipwith{\!\;\raisebox{2pt}{${}_\varcurlyvee$}\;\!}
\newcommand{\ffootnote}[2]{\hspace{#1}\footnote{#2}}



\def\lin{\ell\tn{In}}
\def\lout{\ell\tn{Out}}
\def\gin{g\tn{In}}
\def\gout{g\tn{Out}}
\def\min{m^{in}}
\def\mout{m^{out}}
\def\sin{s^{in}}
\def\sout{s^{out}}
\newcommand{\disc}[1]{{\ul{#1}}}

\newcommand{\Wir}[1]{\bfW_{#1}}
\def\SMC{{\bf SMC}}
\def\TSMC{{\bf TSMC}}
\newcommand{\FS}[1]{\tn{FS}_{/\mathcal{#1}}}
\def\TFSO{\text{\bf TSMC}_{\Ob(\FS{O})}}
\newcommand{\Bij}[1]{\mcB_{\mathcal{#1}}}
\def\Int{\tn{Int}}

%\newcommand{\strm}[1]{{\left(#1\right)^\NN}}

\makeatletter\let\c@figure\c@equation\makeatother %Aligns figure numbering and equation numbering.
\newtheorem{theorem}[subsection]{Theorem}
\newtheorem{lemma}[subsection]{Lemma}
\newtheorem{proposition}[subsection]{Proposition}
\newtheorem{corollary}[subsection]{Corollary}
\newtheorem{fact}[subsection]{Fact}

\theoremstyle{remark}
\newtheorem{remark}[subsection]{Remark}
\newtheorem{example}[subsection]{Example}
\newtheorem{application}[subsection]{Application}
\newtheorem{warning}[subsection]{Warning}
\newtheorem{question}[subsection]{Question}
\newtheorem{guess}[subsection]{Guess}
\newtheorem{answer}[subsection]{Answer}
\newtheorem{claim}[subsection]{Claim}

\theoremstyle{definition}
\newtheorem{definition}[subsection]{Definition}
\newtheorem{notation}[subsection]{Notation}
\newtheorem{conjecture}[subsection]{Conjecture}
\newtheorem{postulate}[subsection]{Postulate}
\newtheorem{construction}[subsection]{Construction}
\newtheorem{ann}[subsection]{Announcement}
\newenvironment{announcement}{\begin{ann}}{\hspace*{\fill}$\lozenge$\end{ann}}


\setcounter{tocdepth}{1}
\setcounter{secnumdepth}{2}


%Standard geometry (DO NOT CHANGE) seems to be: \newgeometry{left=1.76in,right=1.76in,top=1.6in,bottom=1.3in}

\usepackage[paperwidth=8.5in,paperheight=11in,text={5.7in,8in},centering]{geometry}
%\newgeometry{left=1.55in,right=1.55in,top=1.6in,bottom=1.3in}
\usepackage{lscape}

\begin{document}

\title{String diagrams for traced monoidal categories as Cob(1)-algebras}

\author{Dylan Rupel}
\address{Northeastern University\\360 Huntington Ave.\\Boston, MA 02115}
\email{dylanrupel@gmail.com}

\author{David I. Spivak}
\address{Massachusetts Institute of Technology\\77 Massachusetts Ave.\\Cambridge, MA 02139}
\email{dspivak@gmail.com}

\thanks{Spivak acknowledges support by ONR grant N000141310260 and AFOSR grant FA9550-14-1-0031.}


\maketitle

\tableofcontents


%%%%%%%%%%%%%%%%%%%%%
\section{Symmetric Monoidal Categories}

 Let $(\mcM,\otimes,I_\mcM)$ be a symmetric monoidal category. We define the {\em monoid of scalars in $\mcM$}, denoted $S_\mcM$, to be the set $\Hom(I_\mcM,I_\mcM)$, with multiplication given by composition.  Note that $S_\mcM$ is commutative since $\mcM$ is symmetric.  There is an action of $S_\mcM$ on the set $\Hom_\mcM(a,b)$ for each $(a,b)\in\Ob(\mcM^{op}\times\mcM)$, where $s\in S_\mcM$ acts on a morphism $f\taking a\to b$ by sending it to the composite morphism $f^s$ given by
$$a\to a\otimes I_\mcM\To{f\otimes s}b\otimes I_\mcM\to b.$$
We write $|X|:=\Hom(I_\mcM,X)$ for the \emph{elements} of $X$.

The functor $|\cdot|=\Hom(I_\mcM,\cdot)\taking\mcM\to\Set$ is a unital algebra on $\mcM$ with unit map
\[\eta:\{1\}\To{\id_{I_\mcM}}\mcS_M\]
and multiplication map
\[\mu:\Hom_\mcM(I_\mcM,X)\otimes\Hom_\mcM(I_\mcM,X')\to\Hom(I_\mcM\otimes I_\mcM,X\otimes X')\iso\Hom(I_\mcM,X\otimes X').\]

For a symmetric monoidal category $\mcM$ let $\widetilde\mcM=\Int(\mcM)$ denote the category with objects given by pairs $(X_-,X_+)$ where $X_-,X_+\in \Ob(\mcM)$ and morphisms given by 
\[\Hom_{\widetilde{M}}\big((X_-,X_+),(Y_-,Y_+)\big)=\Hom_\mcM(X_-\otimes Y_+,X_+\otimes Y_-).\]
It is well known that $\widetilde{M}$ is a compact closed category whose tensor is given by
\[(X_-,X_+)\boxtimes(Y_-,Y_+):=(X_-\otimes Y_-,X_+\otimes Y_+)\]
with unit object $I_{\widetilde\mcM}:=(I_\mcM,I_\mcM)$ and duality $(X_-,X_+)^\vee:=(X_+,X_-)$.  The following is immediate from the definitions.

\begin{lemma}

Let $\mcM$ be a symmetric monoidal category.  Then for any object $(X_-,X_+)\in\Ob\big(\widetilde\mcM\big)$ there is a canonical bijection
\[|(X_-,X_+)|\iso\Hom_\mcM(X_-,X_+).\]

\end{lemma}

Let $(\mcN,\boxtimes,I_\mcN)$ be a compact closed category.  An algebra on $\mcN$ has the additional structure of a bialgebra...


%%%%%%%%%%%%
\section{1-Cobordisms}

By manifold, we always mean manifold with boundary and only finitely many connected components.  When a manifold $Y$ is oriented, we denote the same manifold with opposite orientation by $Y^\vee$.

An {\em oriented 0-manifold} $X$ comes with a canonical decomposition $X=\inp{X}\sqcup\outp{X}$, where $\inp{X}$ consists of negatively oriented points and $\outp{X}$ consists of positively oriented points.  Given any manifold $\mcO$, an {\em oriented 0-manifold over $\mcO$} is an oriented $0$-manifold $X$ together with a function $\tau_X\taking X\to\mcO$. 

For oriented $0$-manifolds $X$ and $Y$ an {\em oriented $1$-cobordism} $\Phi:X\to Y$ consists of an oriented $1$-manifold $\Phi$ whose boundary $\partial \Phi$ is identified with $X\sqcup Y^\vee$. We will always be working with oriented manifolds and oriented cobordisms, so we often speak of them as simply manifolds and cobordisms. 

Given a $1$-cobordism $\Phi:X\to Y$ and a $1$-cobordism $\Psi:Y\to Z$, we compute $\Psi\circ\Phi:X\to Z$ by gluing the 1-manifolds $\Phi$ and $\Psi$ together along a thickening of $Y$ to obtain a $1$-cobordism from $X$ to $Z$. We denote by $\Cob$ the category where objects are oriented $0$-manifolds and where morphisms $\Hom_\Cob(X,Y)$ are oriented $1$-cobordisms from $X$ to $Y$.  

If $X$ and $Y$ are manifolds over $\mcO$, when we speak of a cobordism between them we mean that our manifold $W$ is over $\mcO$.  The category of 0-manifolds and cobordisms over $\mcO$ is written as $\Cob/\mcO$.

The categories $\Cob$ and $\Cob/\mcO$ have natural compact closed structures where the tensor product is given by disjoint union and duality is given by reversing orientations.  However it will be useful for us to give an alternate description of the compact closed structure of these categories by realizing them via the ``Int construction" of Joyal, Street, and Verity \cite{joyal-street-verity}.

Let $\mcB'$ be the category whose objects are finite sets and whose morphisms are bijections. Consider the category $\mcB$ with $\Ob(\mcB)=\Ob(\mcB')$ but with 
$$\Hom_{\mcB}(n,m)=\Hom_{\mcB'}(n,m)\times\Ob(\Fin),$$
where composition in the second variable is given by disjoint union of sets. There is a wide, full functor $\mcB\to\mcB'$.

\begin{proposition}

The categories $\mcB$ and $\mcB'$ are traced symmetric monoidal categories, where the tensor product, denoted in each case by $\oplus$ is given by disjoint union of finite sets, and the tensor unit is $\emptyset$. The trace in each case is given by removing elements from cycle decompositions, $Rem$ and $Rem'$ above.

For any set $\mcO$ we also have traced symmetric monoidal categories $\mcB/\mcO$ and $\mcB'/\mcO$ of bijections over $\mcO$ with labeled loops. The scalar monoids of these are $S_{\mcB/\mcO}\iso\Fin/\mcO$ and the terminal monoid $S_{\mcB'/\mcO}\iso\singleton$ respectively. 

\end{proposition}

\begin{proof}

**

\end{proof}

\begin{proposition}

The category $\Cob/\mcO$ of cobordisms over $\mcO$ is equivalent to the Int construction applied to $\mcB/\mcO$ above,
$$\Cob/\mcO\simeq\Int(\mcB/\mcO).$$

\end{proposition}

\begin{proof}



\end{proof}

This provides another useful description of the category $\Cob/\mcO$.

\begin{proposition}\label{prop:set theoretic cob1}
We have the following combinatorial description of morphisms and compositions in the category $\Cob/\mcO$.

\begin{description}
\item[Morphisms]
A $1$-cobordism $\Phi\taking X\to Y$ in $\Cob/\mcO$ can be identified with a pair $(F,\varphi)$, where $F$ is a typed finite set, called the set of {\em components of $\Phi$} (as a set $F$ identifies with a basis of $H_0(\Phi)$), and $\varphi\taking X\sqcup Y\to F$ is a typed function, satisfying the following condition, which we call the {\em bijective inclusions property}:
	\begin{quote}\tn{(Bijective inclusions property:)}
	The restrictions $\inpm{\varphi}:=\varphi\big|_{\inp{X}\sqcup\outp{Y}}$ and $\outpm{\varphi}:=\varphi\big|_{\outp{X}\sqcup\inp{Y}}$ are monomorphisms, and they are isomorphic as such; i.e. there exists a typed bijection 
	$$\varphi'\taking\inp{X}\sqcup\outp{Y}\Too{\iso}\outp{X}\sqcup\inp{Y}$$
	such that the following diagram commutes:
	$$\xymatrix@C=12pt{
	\inp{X}\sqcup\outp{Y}\ar[rr]^{\varphi'}\ar@{_(->}[dr]_{\inpm{\varphi}}&&\outp{X}\sqcup\inp{Y}\ar@{^(->}[dl]^{\outpm{\varphi}}\\
	&F
	}
	$$
	Note that $\varphi'$ is unique if it exists, so it can be recovered from $\varphi$.
	\end{quote}
It is often convenient to denote the restrictions $\domn{\varphi}=\varphi\big|_{X}$ and $\codomn{\varphi}=\varphi\big|_{Y}$, which together define the morphism $\varphi$ as a co-relation,
$$X\Too{\domn{\varphi}}F\Fromm{\codomn{\varphi}}Y.$$

The two typed functions $\inpm{\varphi}$ and $\outpm{\varphi}$ have the same image in $F$; we denote the complement of this image as $\mcL(\Phi)$ called the \emph{loops of $\Phi$} (as a set $\mcL(\Phi)$ identifies with a minimal generating set for $\pi_1(\Phi)$). Thus an equivalent (but less useful) definition for a morphism $\Phi:X\to Y$ in $\Cob/\mcO$ is: a set $\mcL$ and a typed bijection $\inp{X}\sqcup\outp{Y}\To{\iso}\outp{X}\sqcup\inp{Y}$. 
\item [Composition] Consider $1$-cobordisms $\Phi\taking X\to Y$ and $\Psi\taking Y\to Z$ which are identified with pairs $(F,\phi)$ and $(G,\psi)$ respectively.  Define $H$ via the pushout of corelations as below:
$$
\xymatrix{
X\ar[r]^{\domn{\varphi}}&F\ar[r]&H\urlimit\\
&Y\ar[r]^{\domn{\psi}}\ar[u]_{\codomn{\varphi}}&G\ar[u]\\
&&Z\ar[u]_{\codomn{\psi}}
}
$$
Then the $1$-cobordism $\Psi\circ\Phi$ is identified with the pair $\big(H,\domn{\varphi}\sqcup\codomn{\psi}\big)$.
\end{description}

\end{proposition}

\begin{proof}

**

$$\xymatrix{
\inp{X}\ar[rr]\ar[d]&&\inp{X}\sqcup\outp{Z}\ar@{-->}@/_3pc/[dddd]&&\outp{Z}\ar[ll]\ar[d]\\
\inp{X}\sqcup\outp{Y}\ar[dd]\ar[dr]&\outp{Y}\ar[l]\ar[dr]&&\inp{Y}\ar[r]\ar[dl]&\inp{Y}\sqcup\outp{Z}\ar[dd]\ar[dl]\\
&F&Y\ar[l]\ar[r]&G\\
\outp{X}\sqcup\inp{Y}\ar[ur]&\inp{Y}\ar[l]\ar[ur]&&\outp{Y}\ar[ul]\ar[r]&\outp{Y}\sqcup\inp{Z}\ar[ul]\\
\outp{X}\ar[u]\ar[rr]&&\outp{X}\sqcup\inp{Z}&&\inp{Z}\ar[ll]\ar[u]
}
$$
$$\xymatrix{
\inp{X}\ar[rr]\ar[d]&&\inp{X}\sqcup\outp{Z}\ar[dd]\ar@{-->}@/_3pc/[dddd]&&\outp{Z}\ar[ll]\ar[d]\\
\inp{X}\sqcup\outp{Y}\ar[dd]\ar[drr]&\outp{Y}\ar[l]&&\inp{Y}\ar[r]&\inp{Y}\sqcup\outp{Z}\ar[dd]\ar[dll]\\
&&F\sqcup_{Y}G\\
\outp{X}\sqcup\inp{Y}\ar[urr]&\inp{Y}\ar[l]&&\outp{Y}\ar[r]&\outp{Y}\sqcup\inp{Z}\ar[ull]\\
\outp{X}\ar[u]\ar[rr]&&\outp{X}\sqcup\inp{Z}\ar[uu]&&\inp{Z}\ar[ll]\ar[u]
}
$$
The two $\inp{Y}$'s are identified over $F\sqcup_{Y}G$ and so are the two $\outp{Y}$'s. We can show that the dotted arrow is an isomorphism by induction on the cardinality of $Y$.

\end{proof}

%%%%%%%%%%
\subsection{Feeds}
The contravariant variable is on top, the covariant variable is on the bottom; the feeding variable is to the left, the fed variable is to the right. So given $\Phi\taking X\to Y$, we have the following.

$$\xymatrix@=30pt{
\feeddd{F}{X}{X}\ar[r]\ar[d]\ullimit&\inp{X}\ar[d]^{\feedbd{\varphi}{X}}&\feedcd{F}{Y}{X}\ar[l]\ar[d]\urlimit\\
\outp{X}\ar[r]^{\feeddb{\varphi}{X}}&F&\inp{Y}\ar[l]^{\feedcb{\varphi}{Y}}\\
\feeddc{F}{X}{Y}\ar[u]\ar[r]\lllimit&\outp{Y}\ar[u]^{\feedbc{\varphi}{Y}}&\feedcc{F}{Y}{Y}\ar[l]\ar[u]\lrlimit
}
$$
%It's not too bad of an abuse to redraw this as follows:
%$$\xymatrix@=30pt{
%\feeddd{\varphi}{X}{X}\ar[r]\ar[d]\ullimit&\feedbd{\varphi}{X}\ar[d]&\feedcd{\varphi}{Y}{X}\ar[l]\ar[d]\urlimit\\
%\feeddb{\varphi}{X}\ar[r]&|\varphi|&\feedcb{\varphi}{Y}\ar[l]\\
%\feeddc{\varphi}{X}{Y}\ar[u]\ar[r]\lllimit&\feedbc{\varphi}{Y}\ar[u]&\feedcc{\varphi}{Y}{Y}\ar[l]\ar[u]\lrlimit
%}
%$$
We denote the obvious four bijections as follows:
\begin{align*}
\feedca{\varphi}{Y}{X}&\taking\feedcd{F}{Y}{X}\sqcup\feedcc{F}{Y}{Y}\Too{\iso}\inp{Y}\\
\feedda{\varphi}{X}{Y}&\taking\feeddd{F}{X}{X}\sqcup\feeddc{F}{X}{Y}\Too{\iso}\outp{X}\\
\feedad{\varphi}{X}{Y}&\taking\feeddd{F}{X}{X}\sqcup\feedcd{F}{Y}{X}\Too{\iso}\inp{X}\\
\feedac{\varphi}{Y}{X}&\taking\feeddc{F}{X}{Y}\sqcup\feedcc{F}{Y}{Y}\Too{\iso}\outp{Y}
\end{align*}

\begin{proposition}[Composition wires]

Let $\Phi\taking X\to Y$ and $\Psi\taking Y\to Z$ be $1$-cobordisms which are equivalent to $(F,\varphi)$ and $(G,\psi)$ as above. Suppose $\Psi\circ\Phi$ is equivalent to $\big(H,\domn{\varphi}\sqcup\codomn{\psi}\big)$.  Let $P$ be the colimit of the following diagram:
$$\xymatrix{
\feeddd{F}{X}{X}\ar[r]\ar[d]&\inp{X}&\feedcd{F}{Y}{X}\ar[l]\ar[d]\\
\outp{X}&\feeddd{G}{Y}{Y}\ar[r]\ar[d]&\inp{Y}&\feedcd{G}{Z}{Y}\ar[l]\ar[d]\\
\feeddc{F}{X}{Y}\ar[u]\ar[r]&\outp{Y}&\feedcc{F}{Y}{Y}\ar[u]\ar[l]&\inp{Z}\\
&\feeddc{G}{Y}{Z}\ar[u]\ar[r]&\outp{Z}&\feedcc{G}{Z}{Z}\ar[l]\ar[u]
}
$$
Then we have typed functions as to the left below.
$$
\xymatrix{
&\inp{X}\ar[d]\\
\outp{X}\ar[r]&P&\inp{Z}\ar[l]\\
&\outp{Z}\ar[u]
}
\hspace{.8in}
\xymatrix{
\feeddd{H}{X}{X}\ar[r]\ar[d]\ullimit&\inp{X}\ar[d]&\feedcd{H}{Z}{X}\ar[l]\ar[d]\urlimit\\
\outp{X}\ar[r]&P&\inp{Z}\ar[l]\\
\feeddc{H}{X}{Z}\ar[r]\ar[u]\lllimit&\outp{Z}\ar[u]&\feedcc{H}{Z}{Z}\ar[u]\ar[l]\lrlimit
}
$$
The pullbacks in the righthand diagram are the indicated components of the composition $\Psi\circ\Phi$.

\end{proposition}

\begin{proof}

**

\end{proof}

By construction there are natural inclusions $\feeddd{F}{X}{X}\subset\feeddd{H}{X}{X}$ and $\feedcc{G}{Z}{Z}\subset\feedcc{H}{Z}{Z}$, we will write $\feeddd{\ol{H}}{X}{X}$ and $\feedcc{\ol{H}}{Z}{Z}$ for the respective complements.

\begin{lemma}\label{le:limit-colimit theorem}

Suppose given a diagram of finite sets of the form to the left:
$$\xymatrix{
&&S\ar[d]^f&&&&S\ar[dd]^{f'}\\
&A\ar[r]^h\ar[d]_i&B\\
T\ar[r]_g&C&D\ar[l]^j\ar[u]_k&\hspace{.3in}&T\ar[rr]_{g'}&&P
}
$$
where $h,i,j,k$ are injective. Suppose $P$ is the colimit of the $A,B,C,D$ square and let $f',g'$ be the induced functions.

Then the pullback $X\times_PY$ of the second diagram can be identified with a countable coproduct of finite limits, each of which factors through the left-hand diagram, where this coproduct is indexed by the set $\ZZ-\{0\}$. 

\end{lemma}

\begin{proof}

To each positive integer $n\in\ZZ_{\geq 1}$ we associate the bipartite graph $Q_{n}$,
\begin{equation}\label{eq:Q_n positive}
\xymatrix{
&&&&&&&S\ar[d]\\
&A\ar[d]\ar[r]&B&A\ar[d]\ar[r]&B&\cdots&A\ar[d]\ar[r]&B\\
T\ar[r]&C&D\ar[u]\ar[r]&C&D\ar[u]\ar[r]&\cdots\ar[r]&C
}
\end{equation}
where $A$ is repeated $n$ times (above without elipses, $n=3$). Similarly, to each negative integer $n\in\ZZ_{\leq -1}$ we associate the diagram $Q_{n}$,
\begin{equation}\label{eq:Q_n negative}
\xymatrix{
&&&&&&&S\ar[d]\\
&&B&A\ar[d]\ar[l]&B&A\ar[d]\ar[l]&\cdots&B\\
T\ar[r]&C&D\ar[l]\ar[u]&C&D\ar[l]\ar[u]&C&\cdots\ar[l]&D\ar[l]\ar[u]
}
\end{equation}
where $D$ is repeated $-n$ times (above, without elipses, $n=-3$).  For an element $n\in\ZZ_{\neq 0}$, let $L_n\in\Ob(\Fin)$ denote the limit of the diagram $Q_n$. We claim that there is an bijection
$$
\coprod_{n\in\ZZ_{\neq 0}}L_n\Too{\iso}S\times_PT.
$$

Indeed, $P=A\sqcup B\sqcup C\sqcup D/\sim$ where the equivalence relation $\sim$ is generated by the following:
\begin{itemize}
 \item $a\sim h(a)$ and $a\sim i(a)$ for $a\in A$;
 \item $d\sim j(d)$ and $d\sim k(d)$ for $d\in D$.
\end{itemize}
In particular, it follows from injectivity of the maps $h,i,j,k$ that for $b\in B$ and $c\in C$ we have $b\sim c$ if and only if there exists a sequence having one of the following forms:
\begin{itemize}
 \item $(b,a_1,c_1,d_1,b_1,\ldots,a_{n-1},c_{n-1},d_{n-1},b_{n-1},a_n,c)$ such that $h(a_t)=b_{t-1}$, $i(a_t)=c_t$ for $1\le t\le n$ and $j(d_t)=c_t$, $k(d_t)=b_t$ for $1\le t\le n-1$ where we take $b_0=b$, $c_n=c$;
 \item $(b,d_1,c_1,a_1,b_1,\ldots,d_{n-1},c_{n-1},a_{n-1},b_{n-1},d_n,c)$ such that $h(a_t)=b_t$, $i(a_t)=c_t$ for $1\le t\le n-1$ and $j(d_t)=c_t$, $k(d_t)=b_{t-1}$ for $1\le t\le n$ where we take $b_0=b$, $c_n=c$.
\end{itemize}
 
 But an element of $S\times_P T$ is a pair $(s,t)$ where $f'(s)=g'(t)$ in $P$, i.e. where $f(s)\sim g(t)$.  It only remains to realize that the sequence above realizing this equivalence exactly corresponds to an element of some $L_n$.
\end{proof}

\begin{corollary}

The composition formula induces a surjective function
$$\coprod_Y \big(\Hom_\Cob(X,Y)\times\Hom_\Cob(Y,Z)\big)\to\Hom_\Cob(X,Z)$$
which identifies an equivalence relation on the left-hand side. This equivalence is generated by
\begin{enumerate}
\item[(E1).]
\begin{equation*}
 \begin{tikzpicture}
  \node at (-4.25,1.5) {$Y$};
  \draw[fill=black] (-4.25,1) circle (1pt);
  \node at (-4.35,1.1) {$\scriptstyle+$};
  \draw[fill=black] (-4.25,0) circle (1pt);
  \node at (-4.35,0.1) {$\scriptstyle-$};
  \draw[fill=black] (-4.25,-1) circle (1pt);
  \node at (-4.35,-0.9) {$\scriptstyle+$};
  \draw (-4.25,1) arc (90:-90:0.5);
  \draw (-4.25,0) arc (90:270:0.5);
  %
  \node at (-3,0) {$\sim$};
  %
  \node at (-2,1.5) {$Y'$};
  \draw[fill=black] (-2,0) circle (1pt);
  \node at (-2.1,0.1) {$\scriptstyle+$};
  %%
  \node at (0,0) {and};
  %%
  \node at (2.5,1.5) {$Y$};
  \draw[fill=black] (2.5,1) circle (1pt);
  \node at (2.4,1.1) {$\scriptstyle-$};
  \draw[fill=black] (2.5,0) circle (1pt);
  \node at (2.4,0.1) {$\scriptstyle+$};
  \draw[fill=black] (2.5,-1) circle (1pt);
  \node at (2.4,-0.9) {$\scriptstyle-$};
  \draw (2.5,1) arc (90:-90:0.5);
  \draw (2.5,0) arc (90:270:0.5);
  %
  \node at (3.75,0) {$\sim$};
  %
  \node at (4.75,1.5) {$Y'$};
  \draw[fill=black] (4.75,0) circle (1pt);
  \node at (4.65,0.1) {$\scriptstyle-$};
 \end{tikzpicture}
\end{equation*}
\item[(E2).] 
\begin{equation*}
 \begin{tikzpicture}
  \node at (-3.25,1) {$Z$};
  \draw[fill=black] (-3.25,0.5) circle (1pt);
  \node at (-3.35,0.6) {$\scriptstyle+$};
  \draw[fill=black] (-3.25,-0.5) circle (1pt);
  \node at (-3.35,-0.4) {$\scriptstyle-$};
  \node at (-1.25,1) {$Y$};
  \draw[fill=black] (-1.25,0.5) circle (1pt);
  \node at (-1.35,0.6) {$\scriptstyle+$};
  \draw[fill=black] (-1.25,-0.5) circle (1pt);
  \node at (-1.35,-0.4) {$\scriptstyle-$};
  \draw (-1.25,0.5) arc (90:-90:0.5);
  \draw (-3.25,0.5) -- (-1.25,0.5);
  \draw (-3.25,-0.5) -- (-1.25,-0.5);
  %
  \node at (0,0) {$\sim$};
  %
  \node at (1,1) {$Z$};
  \draw[fill=black] (1,0.5) circle (1pt);
  \node at (0.9,0.6) {$\scriptstyle+$};
  \draw[fill=black] (1,-0.5) circle (1pt);
  \node at (0.9,-0.4) {$\scriptstyle-$};
  \draw (1,0.5) arc (90:-90:0.5);
  \node at (3,1) {$Y'$};
 \end{tikzpicture}
\end{equation*}
\item[(E3).] 
\begin{equation*}
 \begin{tikzpicture}
  \node at (-3,1) {$Y$};
  \draw[fill=black] (-3,0.5) circle (1pt);
  \node at (-3.1,0.6) {$\scriptstyle+$};
  \draw[fill=black] (-3,-0.5) circle (1pt);
  \node at (-3.1,-0.4) {$\scriptstyle-$};
  \draw (-3,0.5) arc (90:270:0.5);
  \node at (-1,1) {$X$};
  \draw[fill=black] (-1,0.5) circle (1pt);
  \node at (-1.1,0.6) {$\scriptstyle+$};
  \draw[fill=black] (-1,-0.5) circle (1pt);
  \node at (-1.1,-0.4) {$\scriptstyle-$};
  \draw (-3,0.5) -- (-1,0.5);
  \draw (-3,-0.5) -- (-1,-0.5);
  %
  \node at (0,0) {$\sim$};
  %
  \node at (1,1) {$Y'$};
  \node at (3,1) {$X$};
  \draw[fill=black] (3,0.5) circle (1pt);
  \node at (2.9,0.6) {$\scriptstyle+$};
  \draw[fill=black] (3,-0.5) circle (1pt);
  \node at (2.9,-0.4) {$\scriptstyle-$};
  \draw (3,0.5) arc (90:270:0.5);
 \end{tikzpicture}
\end{equation*}
\item[(E4).] 
\begin{equation*}
 \begin{tikzpicture}
  \node at (-3.25,1) {$Z$};
  \node at (-1.25,1) {$Y$};
  \draw[fill=black] (-1.25,0.5) circle (1pt);
  \node at (-1.35,0.6) {$\scriptstyle+$};
  \draw[fill=black] (-1.25,-0.5) circle (1pt);
  \node at (-1.35,-0.4) {$\scriptstyle-$};
  \draw (-1.25,0.5) arc (90:-90:0.5);
  \draw (-1.25,0.5) arc (90:270:0.5);
  %
  \node at (0,0) {$\sim$};
  %
  \node at (1,1) {$Z$};
  \draw (2,0.25) circle (8pt);
  \node at (3,1) {$Y'$};
 \end{tikzpicture}
\end{equation*}
\item[(E5).] 
\begin{equation*}
 \begin{tikzpicture}
  \node at (-5.25,1) {$Z$};
  \node at (-3.25,1) {$Y$};
  \draw (-2.25,0.25) circle (8pt);
  \node at (-1.25,1) {$X$};
  %
  \node at (0,0) {$\sim$};
  %
  \node at (1,1) {$Z$};
  \draw (2,0.25) circle (8pt);
  \node at (3,1) {$Y$};
  \node at (5,1) {$X$};
 \end{tikzpicture}
\end{equation*}
\end{enumerate}
Moreover, every equivalence class contains a unique pair $X\To{\Phi}Y\To{\Psi}Z$ for which $\feedcc{F}{Y}{Y}$ and $\feeddd{G}{Y}{Y}$ are empty and $\pi_1(\Phi)=0$.

\end{corollary}
\begin{proof}
 Consider a pair $(\Phi,\Psi)\in\Hom_\Cob(X,Y)\times\Hom_\Cob(Y,Z)$.  Define $\feeddd{\ol{H}}{X}{X}$, $\feeddc{H}{X}{Z}$, $\feedcd{H}{Z}{X}$, and $\feedcc{\ol{H}}{Z}{Z}$ as above.  Define the $0$-manifold $Y'=Y'_-\sqcup Y'_+$ where $|Y'_-|=|Y_-|-1$ and $|Y'_+|=|Y_+|-1$.  We define a new pair $(\Phi',\Psi')\in\Hom(X,Y')\times \Hom(Y',Z)$ via...
\end{proof}

\erase{%begin erase
\begin{lemma}[Technical 1]

Define pullbacks:
$$\xymatrix{
\feedcd{\varphi}{Y}{X}\ar[d]&thing_1\ar[l]\ar[d]\urlimit\\
\inp{Y}&\feedcd{\psi}{Z}{Y}\ar[l]
}
\hspace{.8in}
\xymatrix{
\feeddc{\varphi}{X}{Y}\ar[r]&\outp{Y}\\
thing_2\ar[r]\ar[u]\lllimit&\feeddc{\psi}{Y}{Z}\ar[u]
}
$$
and
$$\xymatrix@=15pt{
&thing_3\ar[rr]\ar[dd]\ullimit&&\feeddd{\psi\circ\varphi}{X}{X}\ar[dd]&&thing_4\ar[ll]\ar[dd]\urlimit\\
&&thing_5\ar[ld]\ar'[r][rr]&&\feedcd{\psi\circ\varphi}{Z}{X}\ar[dl]&&thing_6\ar'[l][ll]\ar[dl]\\
&\feeddd{\psi}{Y}{Y}\ar[rr]&&P&&\feedcc{\varphi}{Y}{Y}\ar[ll]\\
thing_7\ar[ur]\ar[rr]&&\feeddc{\psi\circ\varphi}{X}{Z}\ar[ur]&&thing_8\ar[ll]\ar[ur]\\
&thing_9\ar'[u][uu]\ar[rr]\lllimit&&\feedcc{\psi\circ\varphi}{Z}{Z}\ar'[u][uu]&&thing_{10}\ar[uu]\ar[ll]\lrlimit
}
$$

There are bijective maps: 
$$\feeddd{\varphi}{X}{X}\To{\sim}\feeddd{\psi\circ_0\phi}{X}{X}$$
$$\feedcc{\psi}{Z}{Z}\To{\sim}\feedcc{\psi\circ_0\phi}{Z}{Z}$$
$$thing_1\To{\sim}\feedcd{\psi\circ_1\varphi}{Z}{X}$$
$$thing_2\To{\sim}\feeddc{\psi\circ_1\varphi}{X}{Z}.$$
And surjections
\begin{align*}
thing_3\to\bigsqcup_{j\ge1}\feeddd{\psi\circ_{2j}\varphi}{X}{X}
&\hsp\tn{and}\hsp
thing_4\to\bigsqcup_{j\ge1}\feeddd{\psi\circ_{2j}\varphi}{X}{X}\\
thing_5\to\bigsqcup_{j\ge1}\feedcd{\psi\circ_{2j+1}\varphi}{Z}{X}
&\hsp\tn{and}\hsp
thing_6\to\bigsqcup_{j\ge1}\feedcd{\psi\circ_{2j+1}\varphi}{Z}{X}\\
thing_7\to\bigsqcup_{j\ge1}\feeddc{\psi\circ_{2j+1}\varphi}{X}{Z}
&\hsp\tn{and}\hsp
thing_8\to\bigsqcup_{j\ge1}\feeddc{\psi\circ_{2j+1}\varphi}{X}{Z}\\
thing_9\to\bigsqcup_{j\ge1}\feedcc{\psi\circ_{2j}\varphi}{Z}{Z}
&\hsp\tn{and}\hsp
thing_{10}\to\bigsqcup_{j\ge1}\feedcc{\psi\circ_{2j}\varphi}{Z}{Z}
\end{align*}
where the preimage of an element in a set denoted by $\circ_{2j}$ or $\circ_{2j+1}$ contains exactly $j$ elements.

Moreover, the maps
\[thing_3\sqcup thing_5\sqcup thing_7\sqcup thing_9\to \feeddd{\psi}{Y}{Y}\]
\[thing_4\sqcup thing_6\sqcup thing_8\sqcup thing_{10}\to \feedcc{\varphi}{Y}{Y}\]
are injective and there is a bijection between their respective complements.
\end{lemma}

\begin{proof}

**

\end{proof}

These induce decompositions
\[\ol{\feeddd{\psi}{Y}{Y}}=\ol{thing_3}\otimes \ol{thing_5}\otimes \ol{thing_7}\otimes \ol{thing_9}\otimes \ol{outer loop wires}\]
\[\ol{\feedcc{\varphi}{Y}{Y}}=\ol{thing_4}\otimes \ol{thing_6}\otimes \ol{thing_8}\otimes \ol{thing_{10}}\otimes \ol{inner loop wires}\]}%end erase

\begin{proposition}

We have the following combinatorial description of a compact closed monoidal structure on $\Cob/\mcO$, in terms of the combinatorial description given in Proposition \ref{prop:set theoretic cob1}.
\begin{description}
\item [Tensor object] Given objects $X,X'\in\Cob/\mcO$, their tensor product, written $X\oplus X'$, is given by 
$$\inp{(X\oplus X')}=\inp{X}\sqcup\inp{X'}, \hsp\tn{and}\hsp \outp{(X\oplus X')}=\outp{X}\sqcup\outp{X'}.$$
\item [Unit object] The unit object is $(\emptyset,\emptyset)$; it is denoted $\square$.
\item [Dual object] Given any object $X=(\inp{X},\outp{X})$, its dual is denoted and defined as 
$$X^\vee=(\outp{X},\inp{X}).$$
\item [Tensor morphism] Given morphisms $|X|\Too{\domn{\varphi}}|\varphi|\Fromm{\codomn{\varphi}}|Y|$ and $|X'|\Too{\domn{\varphi}\hspace{0pt}'}|\varphi|\Fromm{\codomn{\varphi}'}|Y'|$, their tensor product is given by disjoint union across the board,
$$|X|\sqcup|X'|\Too{\domn{\varphi}\sqcup\domn{\varphi}'}|\varphi|\sqcup|\varphi'|\Fromm{\codomn{\varphi}\sqcup\codomn{\varphi}\hspace{0in}'}|Y|\sqcup|Y'|$$
\item [Unit morphism] Given any object $X$, the unit morphism $\eta_X\taking\square\to X\oplus X^\vee$ is given by letting $|\eta_X|=|X|$, noting that $|X\oplus X|=|X|\sqcup|X|$, and using the fold map 
$$\emptyset\To{!}|X|\From{fold}|X|\sqcup|X|.$$
\item [Counit morphism] Given any object $X$, the unit morphism $\epsilon_X\taking X\oplus X^\vee\to\square$ is given by letting $|\epsilon_X|=|X|$, noting that $|X\oplus X|=|X|\sqcup|X|$, and using the fold map 
$$|X|\sqcup|X|\To{fold}|X|\From{!}\emptyset.$$

\end{description}

\end{proposition}

\begin{proof}

The simplicity in the definition of $\oplus$ and $\square$ make it easy to see that the tensor and unit object give $\Cob/\mcO$ the structure of a symmetric monoidal category. It remains to show that the dual object satisfies the necessary equations in terms of its unit and counit morphisms.

Proving that the necessary equation for the unit morphism holds amounts to showing that the following composite is the identity:
$$|X|\Too{\tn{inl}}|X|\sqcup|X|\Fromm{\left\{\parbox{.1in}{\scriptsize inl\\inr\\inr}\right.}|X|\sqcup|X|\sqcup|X|\Too{\left\{\parbox{.1in}{\scriptsize inl\\inl\\inr}\right.}|X|\sqcup|X|\From{\tn{inr}}|X|.$$
This easily checked; it may help to picture it as follows, where we take the colimit of the middle seven $|X|$'s in the diagram.
$$\xymatrix@R=1pt{
&&|X|\ar[dl]\ar[dr]\\
|X|\ar[r]&|X|&&|X|\\
&&|X|\ar[ur]\ar[dl]\\
&|X|&&|X|&|X|\ar[l]\\
&&|X|\ar[ul]\ar[ur]
}
$$
The equation for the counit is similar.

\end{proof}

Given a typed finite set $s$, we have an object $S:=(s,s)\in\Ob(\Cob/\mcO)$. There is a natural morphism $X\to X\oplus S$ and a natural morphism $X\oplus S\to X$, as shown:
\begin{align}\label{dia:proto loops}
|X|\Too{inl}|X|\sqcup|S|\Fromm{\id}|X\oplus S|.\hspace{.7in}
|X\oplus S|\Too{\id}|X|\sqcup |S|\Fromm{inl}|X|
\end{align}


\begin{lemma}

Let $s$ be a typed finite set and $S=(s,s)$. The composite of the morphisms $X\to X\oplus S\to X$ from (\ref{dia:proto loops}) constitutes a natural transformation 
$$\Loop(s)\taking\id_{\Cob/\mcO}\to\id_{\Cob/\mcO}$$ 
whose $X$-component $\Loop(s)_X\taking X\to X$ is given by the co-relation 
$$|X|\To{inl} |X|\sqcup |S|\From{inl} |X|.$$

\end{lemma}

\begin{proof}

Clearly the composite is given by the co-relation $|X|\to |X|\sqcup |S|\from |X|$, because this is the pushout of the co-relations in (\ref{dia:proto loops}). Thus it remains to show that these co-relations really are morphisms (i.e. that they satisfy the bijective inclusions property), and that these morphisms are natural in $X$.

It is easy to see that these co-relations satisfy the bijective inclusions property, because $\inp{X}\sqcup\outp{(X\oplus S)}=|X|\sqcup|S|=\outp{X}\sqcup\inp{(X\oplus S)}.$ To see that $\Loop(s)$ is natural in $s$, one checks that for any $\varphi\taking X\to Y$, the two co-relations below are identical:
$$
\xymatrix{
|X|\ar[r]^{\domn{\varphi}}&|\varphi|\ar[r]^{inl}&|\varphi|\sqcup|S|\urlimit\\
&|Y|\ar[u]_{\codomn{\varphi}}\ar[r]^-{inl}&|Y|\sqcup|S|\ar[u]_{\codomn{\varphi}\sqcup|S|}\\
&&|Y|\ar[u]_{inl}
}
\hspace{.8in}
\xymatrix{
|X|\ar[r]^-{inl}&|X|\sqcup|S|\ar[r]^{\domn{\varphi}\sqcup|S|}&|\varphi|\sqcup|S|\urlimit\\
&|X|\ar[u]_{inl}\ar[r]^-{\domn{\varphi}}&|\varphi|\ar[u]_{inl}\\
&&|Y|\ar[u]_{\codomn{\varphi}}
}$$
\end{proof}


%%%%%%%%%%%%%%
\section{Tracing bijections}

Let $n\in\NN$ be a natural number and let $\ul{n}:=[1,n]$ be the corresponding finite set. We can write any bijection $\ul{n}\To{\iso}\ul{n}$ in {\em cycle notation}. Suppose $k\in\NN$ is another natural number. Given a bijection $\sigma\taking\ul{n+k}\To{\iso}\ul{n+k}$, we can simply remove all values of $\ul{k}$ from the cycle notation. The result, which we denote $(Rem')^k_{n,n}(\sigma)$, can be considered a bijection $\ul{n}\To{\iso}\ul{n}$.

\begin{example}

Let $n=3$ and $k=4$. Suppose given the bijection $\sigma\taking\ul{7}\to\ul{7}$ with cycle decomposition $(1 3 7)(6 5)(2 4)$. Then after removing $4,5,6,7$ from these cycles, we get $(Rem')^k_{n,n}(\sigma)=(1 3)(2)$. 

Notice that we really seemed to get $(1 3)()(2)$, but the {\em empty cycle} $()$ have been mysteriously dropped.

\end{example}

Instead of starting with a bijection, we could start with a cycle decomposition that also includes an integer $\ell\in\NN$ determining a set $\ul\ell$ of empty cycles. Then, applying the above formula we could opt to not drop the empty cycle, but rather append it to the set $\ul\ell$; we denote that procedure by $Rem^k_{n,n}$.


\subsection{The interaction of the trace on $\mcB$ and a traced monoidal category}

For any set $\mcO$, let $\mcB/\mcO$ be the traced symmetric monoidal category of bijections over $\mcO$ with labeled loops.  The trace of $\mcB/\mcO$ is denoted $Rem$, as above.

Let $\mcM$ be a traced monoidal category, and let $\mcO=\Ob(\mcM)$. Then given any object $X\taking n\to\mcO$ in $\mcB/\mcO$, we denote by $\ol{X}\in\mcM$ the tensor product 
$$\ol{X}:=\bigotimes_{i\in n}X(i)$$
Note that given a morphism $\sigma\taking X\to Y$ in $\mcB/\mcO$, we get a morphism $\ol{\sigma}\taking\ol{X}\to\ol{Y}$ in $\mcM$, where the loops are traced out to dimensions.

\begin{proposition}

Let $\mcM$ be a traced monoidal category, and let $\mcO=\Ob(\mcM)$. Let $X\taking n\to\mcO$, $X'\taking n'\to\mcO$, and $U\taking k\to\mcO$ in $\Ob(\mcB/\mcO)$ be finite sets over $\mcO$. Let $\sigma\taking n+k\to n'+k$ be a morphism in $\mcB/\mcO$. Then we have 
$$Tr^{\ol{U}}_{\ol{X},\ol{X'}}(\ol{\sigma})=\ol{Rem^U_{X,X'}(\sigma)}$$
That is, the functor $\ol{\;\cdot\;}\taking\mcB/\mcO\to\mcM$ is a traced monoidal functor.

\end{proposition}

\begin{proof}

**

\end{proof}

The following result may not belong in this section.

\begin{proposition}

Let $\mcS$ denote the (skeleton of the) category of finite sets and surjections; each object is of the form $\ul{n}=\{1,2,\ldots,n\}$. There is a functor $\mcS\to\Grp\op$ sending $\ul{n}$ to $S_n$, the symmetric group on $n$ letters.

\end{proposition}

\section{Main theorem}

\begin{lemma}

Given a function $f\taking\mcO\to\mcO'$, there is an induced strong monoidal functor $f^*\taking\Cob/\mcO'\to\Cob/\mcO$. This defines a functor 
$$\Cob/-\taking\Set\to\SMC_{st}\op$$
from the category of sets to the opposite of the category of SMCs and strong monoidal functors.

\end{lemma}

\begin{proof}

**

\end{proof}

\begin{lemma}

Let $U,U',X,Y$ be typed finite sets, and let $\sigma\taking U\To{\iso} U'$ be an isomorphism. Then the diagram:
$$\xymatrix{
\Hom(\ol{X}\otimes\ol{U},\ol{Y}\otimes\ol{U})\ar[rr]\ar[dr]_{Tr^{\ol{U}}_{\ol{X},\ol{Y}}}&&\Hom(\ol{X}\otimes\ol{U'},\ol{Y}\otimes\ol{U'})\ar[dl]_{Tr^{\ol{U'}}_{\ol{X},\ol{Y}}}\\
&\Hom(\ol{X},\ol{Y})
}
$$
commutes, where the top morphism is conjugation with $\sigma$.

\end{lemma}

\begin{proof}

This follows from dinaturality of trace.

\end{proof}

\begin{definition}

Define {\em the category of typed 1-cobordism algebras}, denoted $(\Cob/\bullet)\alg$, to have objects 
$$\Ob((\Cob/\bullet)\alg):=\{(\mcO,\mcP)\|\mcO\in\Ob(\Set)\tn{ and } \mcP\in\Ob\big((\Cob/\mcO)\alg\big)\},$$
and to have morphisms
$$\Hom_{(\Cob/\bullet)\alg}((\mcO,\mcP),(\mcO',\mcP')):=\{(f,r)\|f\taking\mcO\to\mcO', r\taking\mcP\to(\Cob/f)^*(\mcP')\}.$$

\end{definition}

\begin{theorem}\label{thm:cobalg trace adjunction}

Let $(\Cob/\bullet)\alg$ be the category of typed 1-cobordism algebras, and let $\TSMC$ denote the category of traced symmetric monoidal categories (and traced lax monoidal functors between them). There is an adjunction
$$\xymatrix{L\taking(\Cob/\bullet)\alg\ar@<.5ex>[r]&\TSMC:\!R\ar@<.5ex>[l]}$$

\end{theorem}

\begin{proof}

Let $\mcM$ be a traced monoidal category with objects $\mcO$. We will define a $\Cob/\mcO$-algebra $\mcP=R(\mcM)\taking\Cob/\mcO\to\Set$ as follows. For an object $X\in\Ob(\Cob/\mcO)$, set 
$$\mcP(X):=\Hom_{\mcM}(\vinp{X},\voutp{X}).$$
We next consider morphisms.


Following Proposition~\ref{prop:set theoretic cob1} a morphism $\Phi\taking X\too Y$ consists of a typed bijection 
$$\varphi\taking\inp{X}\sqcup \outp{Y}\To{\iso}\outp{X}\sqcup \inp{Y},$$ 
together with a typed finite set $S$. Given an element $f\in\mcP(X)$ we must construct $\mcP(\Phi)(f)\in\mcP(Y)$. Let $\dim(\ol{S})=\Trace^{\ol{S}}_{I,I}\big[\id_{\ol{S}}\big]\in\mcS_\mcM$. Then we use the formula
$$\mcP(\Phi)(f):=
\left(\Trace^{\ol{\feeddd{\varphi}{X}{X}}}_{\ol{\feedcd{\varphi}{Y}{X}},\ol{\feeddc{\varphi}{X}{Y}}}\big[f\big]
\otimes\id_{\ol{\feedcc{\varphi}{Y}{Y}}}\right)^{\dim(\ol{S})}
$$
which is more properly written as
\[\tag{$*$}\Big(\ol{\feedac{\varphi}{Y}{X}}\Big)^{-1}\circ\bigg(\Trace^{\ol{\feeddd{\varphi}{X}{X}}}_{\ol{\feedcd{\varphi}{Y}{X}},\ol{\feeddc{\varphi}{X}{Y}}}\bigg[\ol{\feedda{\varphi}{X}{Y}}\circ f\circ\Big(\ol{\feedad{\varphi}{X}{Y}}\Big)^{-1}\bigg]
\otimes\id_{\ol{\feedcc{\varphi}{Y}{Y}}}\bigg)^{\dim(\ol{S})}\circ\ol{\feedca{\varphi}{Y}{X}},\]
%\[
%\tag{$*$}
%\Trace^{\ol{\feeddd{\varphi}{X}{X}}}_{\ol{Y_-},\ol{Y_+}}\Bigg[
%\bigg(\Big(\ol{\feedac{\varphi}{Y}{X}}\Big)^{-1}\otimes\id_{\ol{\feeddd{\varphi}{X}{X}}}\bigg)\circ\bigg(\id_{\ol{\feedcc{\varphi}{Y}{Y}}}\otimes \bigg(\ol{\feedda{\varphi}{X}{Y}}\circ f\circ\Big(\ol{\feedad{\varphi}{X}{Y}}\Big)^{-1}\bigg)\bigg)\circ\bigg(\ol{\feedca{\varphi}{Y}{X}}\otimes\id_{\ol{\feeddd{\varphi}{X}{X}}}\bigg)
%\Bigg]
%\otimes\dim(\ol{S})
%\]
where we used the following isomorphisms which we suppress from the notation as above whenever this will not lead to any confusion:
$$
\ol{\feedad{\varphi}{X}{Y}}:\ol{\inp{X}}\To{\sim}\ol{\feedcd{\varphi}{Y}{X}}\otimes\ol{\feeddd{\varphi}{X}{X}}
\hsp\tn{and}\hsp
\ol{\feedda{\varphi}{X}{Y}}:\ol{\outp{X}}\To{\sim}\ol{\feeddc{\varphi}{X}{Y}}\otimes\ol{\feeddd{\varphi}{X}{X}}
$$
$$
\ol{\feedca{\varphi}{Y}{X}}:\ol{\inp{Y}}\To{\sim}\ol{\feedcc{\varphi}{Y}{Y}}\otimes\ol{\feedcd{\varphi}{Y}{X}}
\hsp\tn{and}\hsp
\ol{\feedac{\varphi}{Y}{X}}:\ol{\outp{Y}}\To{\sim}\ol{\feedcc{\varphi}{Y}{Y}}\otimes\ol{\feeddc{\varphi}{X}{Y}}.
$$

Given also $\Psi\taking[\inp{Y},\outp{Y}]\too[\inp{Z},\outp{Z}]$, we need to show that the following equation holds: 
$$\mcP(\Psi)\circ\mcP(\Phi)(f)=^?\mcP(\Psi\circ\Phi)(f).$$
These are given as follows (with the isomorphisms above suppressed for readability):
\begin{align}
\label{eq:composition 1}
\mcP(\Psi)\circ\mcP(\Phi)(f)&=
\id_{\ol{\feedcc{\psi}{Z}{Z}}}\otimes\Trace^{\ol{\feeddd{\psi}{Y}{Y}}}_{\ol{\feedcd{\psi}{Z}{Y}},\ol{\feeddc{\psi}{Y}{Z}}}
\bigg[
\id_{\ol{\feedcc{\varphi}{Y}{Y}}}\otimes\Trace^{\ol{\feeddd{\varphi}{X}{X}}}_{\ol{\feedcd{\varphi}{Y}{X}},\ol{\feeddc{\varphi}{X}{Y}}}\big[f\big]
\bigg]
\otimes\dim(\ol{S\sqcup T})
\\\nonumber\\
\label{eq:composition 2}
\mcP(\Psi\circ\Phi)(f)&=
\id_{\ol{\feedcc{\psi\circ\varphi}{Z}{Z}}}\otimes \Trace^{\ol{\feeddd{\psi\circ\varphi}{X}{X}}}_{\ol{\feedcd{\psi\circ\varphi}{Z}{X}},\ol{\feeddc{\psi\circ\varphi}{X}{Z}}}\big[f\big]\otimes\dim(\ol{S\sqcup T\sqcup ``newloops"})
\end{align}

We begin with a simplification of \eqref{eq:composition 1} which as in ($*$) can more precisely be written as
\begin{align*}
&\Big(\ol{\feedac{\psi}{Z}{Y}}\Big)^{-1}\circ\Bigg(\id_{\ol{\feedcc{\psi}{Z}{Z}}}\otimes\Trace^{\ol{\feeddd{\psi}{Y}{Y}}}_{\ol{\feedcd{\psi}{Z}{Y}},\ol{\feeddc{\psi}{Y}{Z}}}
\bigg[\ol{\feedda{\psi}{Y}{Z}}\circ\Big(\ol{\feedac{\varphi}{Y}{X}}\Big)^{-1}\circ\\
&\hspace{1in}\circ\bigg(\id_{\ol{\feedcc{\varphi}{Y}{Y}}}\otimes \Trace^{\ol{\feeddd{\varphi}{X}{X}}}_{\ol{\feedcd{\varphi}{Y}{X}},\ol{\feeddc{\varphi}{X}{Y}}}\Big[\ol{\feedda{\varphi}{X}{Y}}\circ f\circ\Big(\ol{\feedad{\varphi}{X}{Y}}\Big)^{-1}\Big]\bigg)\circ\ol{\feedca{\varphi}{Y}{X}}\circ\Big(\ol{\feedad{\psi}{Y}{Z}}\Big)^{-1}\bigg]\Bigg)\circ\ol{\feedca{\psi}{Z}{Y}}.
\end{align*}

To understand the relationship between \eqref{eq:composition 1} and \eqref{eq:composition 2} we introduce more notation.  There are decompositions
\begin{align*}
 \feeddd{\psi\circ\varphi}{X}{X}&=\bigsqcup_{i\ge0}\feeddd{\psi\;\accentset{i}{\circ}\;\varphi}{X}{X}&\text{and}&&\feeddc{\psi\circ\varphi}{X}{Z}&=\bigsqcup_{i\ge0}\feeddc{\psi\;\accentset{i}{\circ}\;\varphi}{X}{Z}\\
 \feedcd{\psi\circ\varphi}{Z}{X}&=\bigsqcup_{i\ge0}\feedcd{\psi\;\accentset{i}{\circ}\;\varphi}{Z}{X}&\text{and}&&\feedcc{\psi\circ\varphi}{Z}{Z}&=\bigsqcup_{i\ge0}\feedcc{\psi\;\accentset{i}{\circ}\;\varphi}{Z}{Z}
\end{align*}
where we write $\accentset{i}{\circ}$ to indicate those wires that cross the border of $Y$ exactly $i$ times as they traverse from their source to their destination.  In particular, we immediately have
\[\feeddd{\varphi}{X}{X}=\feeddd{\psi\;\accentset{0}{\circ}\;\varphi}{X}{X}\quad\text{and}\quad\feedcc{\psi}{Z}{Z}=\feedcc{\psi\;\accentset{0}{\circ}\;\varphi}{Z}{Z}\]
\[\feedcd{\psi}{Z}{Y}=\bigsqcup_{i\ge1}\feedcc{\psi\;\accentset{i}{\circ}\;\varphi}{Z}{Z}\sqcup\bigsqcup_{i\ge1}\feedcd{\psi\;\accentset{i}{\circ}\;\varphi}{Z}{X}\quad\text{and}\quad\feeddc{\psi}{Y}{Z}=\bigsqcup_{i\ge1}\feedcc{\psi\;\accentset{i}{\circ}\;\varphi}{Z}{Z}\sqcup\bigsqcup_{i\ge1}\feeddc{\psi\;\accentset{i}{\circ}\;\varphi}{X}{Z}\]
\[\feedcd{\varphi}{Y}{X}=\bigsqcup_{i\ge1}\feeddd{\psi\;\accentset{i}{\circ}\;\varphi}{X}{X}\sqcup\bigsqcup_{i\ge1}\feedcd{\psi\;\accentset{i}{\circ}\;\varphi}{Z}{X}\quad\text{and}\quad\feeddc{\varphi}{X}{Y}=\bigsqcup_{i\ge1}\feeddd{\psi\;\accentset{i}{\circ}\;\varphi}{X}{X}\sqcup\bigsqcup_{i\ge1}\feeddc{\psi\;\accentset{i}{\circ}\;\varphi}{X}{Z}\]
\[\feedcd{\psi\;\accentset{2j}{\circ}\;\varphi}{Z}{X}=\emptyset=\feeddc{\psi\;\accentset{2j}{\circ}\;\varphi}{X}{Z}\quad\text{and}\quad\feeddd{\psi\;\accentset{2j+1}{\circ}\;\varphi}{X}{X}=\emptyset=\feedcc{\psi\;\accentset{2j+1}{\circ}\;\varphi}{Z}{Z}.\]

With this notation in hand, it then follows from the second vanishing axiom and the superposing axiom of the trace that we may rewrite the composition of \eqref{eq:composition 2} as follows:
\begin{align*}
 &\Trace^{\bigotimes_{i\ge0}\ol{\feeddd{\psi\;\accentset{i}{\circ}\;\varphi}{X}{X}}}_{\ol{Z_-},\ol{Z_+}}\bigg[\id_{\bigotimes_{i\ge0}\ol{\feedcc{\psi\;\accentset{i}{\circ}\;\varphi}{Z}{Z}}}\otimes f\bigg]\otimes\dim(\ol{S\sqcup T\sqcup ``newloops"})\\
 &\hsp=\Trace^{\ol{\feeddd{\varphi}{X}{X}}}_{\ol{Z_-},\ol{Z_+}}\Bigg[
 \id_{\ol{\feedcc{\psi}{Z}{Z}}}\otimes\Trace^{\bigotimes_{i\ge1}\ol{\feeddd{\psi\;\accentset{i}{\circ}\;\varphi}{X}{X}}}_{\ol{\feedcd{\psi}{Z}{Y}}\otimes\ol{\feeddd{\varphi}{X}{X}},\ol{\feeddc{\psi}{Y}{Z}}\otimes\ol{\feeddd{\varphi}{X}{X}}}\bigg[\id_{\bigotimes_{i\ge1}\ol{\feedcc{\psi\;\accentset{i}{\circ}\;\varphi}{Z}{Z}}}\otimes f\bigg]
 \Bigg]\otimes\dim(\ol{S\sqcup T\sqcup ``newloops"}).
\end{align*}
The next result gives an equivalent form of \eqref{eq:composition 1} and is the first step in identifying with the new form of \eqref{eq:composition 2} above.

\begin{lemma}\label{le:combining traces}
\begin{align*}
&\Trace^{\ol{\feeddd{\psi}{Y}{Y}}}_{\ol{\feedcd{\psi}{Z}{Y}},\ol{\feeddc{\psi}{Y}{Z}}}
\bigg[\ol{\feedda{\psi}{Y}{Z}}\circ\Big(\ol{\feedac{\varphi}{Y}{X}}\Big)^{-1}\circ\bigg(\id_{\ol{\feedcc{\varphi}{Y}{Y}}}\otimes \Trace^{\ol{\feeddd{\varphi}{X}{X}}}_{\ol{\feedcd{\varphi}{Y}{X}},\ol{\feeddc{\varphi}{X}{Y}}}\Big[\ol{\feedda{\varphi}{X}{Y}}\circ f\circ\Big(\ol{\feedad{\varphi}{X}{Y}}\Big)^{-1}\Big]\bigg)\circ\ol{\feedca{\varphi}{Y}{X}}\circ\Big(\ol{\feedad{\psi}{Y}{Z}}\Big)^{-1}\bigg]\\
&=\Trace^{\ol{\feeddd{\varphi}{X}{X}}}_{\ol{\feedcd{\psi}{Z}{Y}},\ol{\feeddc{\psi}{Y}{Z}}}\Bigg[
\Trace^{\ol{\feeddd{\psi}{Y}{Y}}}_{\ol{\feedcd{\psi}{Z}{Y}}\otimes\ol{\feeddd{\varphi}{X}{X}},\ol{\feeddc{\psi}{Y}{Z}}\otimes\ol{\feeddd{\varphi}{X}{X}}}\bigg[
\bigg(\id_{\ol{\feeddc{\psi}{Y}{Z}}}\otimes\gamma_{\ol{\feeddd{\psi}{Y}{Y}},\ol{\feeddd{\varphi}{X}{X}}}\bigg)\circ\bigg(\Big(\ol{\feedda{\psi}{Y}{Z}}\circ\Big(\ol{\feedac{\varphi}{Y}{X}}\Big)^{-1}\Big)\otimes\id_{\ol{\feeddd{\varphi}{X}{X}}}\bigg)\circ\\
&\circ\bigg(\id_{\ol{\feedcc{\varphi}{Y}{Y}}}\otimes\Big(\ol{\feedda{\varphi}{X}{Y}}\circ f\circ\Big(\ol{\feedad{\varphi}{X}{Y}}\Big)^{-1}\Big)\bigg)\circ\bigg(\Big(\ol{\feedca{\varphi}{Y}{X}}\circ\Big(\ol{\feedad{\psi}{Y}{Z}}\Big)^{-1}\Big)\otimes\id_{\ol{\feeddd{\varphi}{X}{X}}}\bigg)\circ\bigg(\id_{\ol{\feedcd{\psi}{Z}{Y}}}\otimes\gamma_{\ol{\feeddd{\varphi}{X}{X}},\ol{\feeddd{\psi}{Y}{Y}}}\bigg)\bigg]\Bigg]\\
\end{align*}
\end{lemma}

\begin{proof}
\begin{align*}
&\Trace^{\ol{\feeddd{\psi}{Y}{Y}}}_{\ol{\feedcd{\psi}{Z}{Y}},\ol{\feeddc{\psi}{Y}{Z}}}
\bigg[\ol{\feedda{\psi}{Y}{Z}}\circ\Big(\ol{\feedac{\varphi}{Y}{X}}\Big)^{-1}\circ\bigg(\id_{\ol{\feedcc{\varphi}{Y}{Y}}}\otimes \Trace^{\ol{\feeddd{\varphi}{X}{X}}}_{\ol{\feedcd{\varphi}{Y}{X}},\ol{\feeddc{\varphi}{X}{Y}}}\Big[\ol{\feedda{\varphi}{X}{Y}}\circ f\circ\Big(\ol{\feedad{\varphi}{X}{Y}}\Big)^{-1}\Big]\bigg)\circ\ol{\feedca{\varphi}{Y}{X}}\circ\Big(\ol{\feedad{\psi}{Y}{Z}}\Big)^{-1}\bigg]\\
&=\Trace^{\ol{\feeddd{\psi}{Y}{Y}}}_{\ol{\feedcd{\psi}{Z}{Y}},\ol{\feeddc{\psi}{Y}{Z}}}
\Bigg[\Trace^{\ol{\feeddd{\varphi}{X}{X}}}_{\ol{\feedcd{\psi}{Z}{Y}}\otimes\ol{\feeddd{\psi}{Y}{Y}},\ol{\feeddc{\psi}{Y}{Z}}\otimes\ol{\feeddd{\psi}{Y}{Y}}}\bigg[\bigg(\Big(\ol{\feedda{\psi}{Y}{Z}}\circ\Big(\ol{\feedac{\varphi}{Y}{X}}\Big)^{-1}\Big)\otimes\id_{\ol{\feeddd{\varphi}{X}{X}}}\bigg)\circ\\
&\hspace{1in}\circ\bigg(\id_{\ol{\feedcc{\varphi}{Y}{Y}}}\otimes\Big(\ol{\feedda{\varphi}{X}{Y}}\circ f\circ\Big(\ol{\feedad{\varphi}{X}{Y}}\Big)^{-1}\Big)\bigg)\circ\bigg(\Big(\ol{\feedca{\varphi}{Y}{X}}\circ\Big(\ol{\feedad{\psi}{Y}{Z}}\Big)^{-1}\Big)\otimes\id_{\ol{\feeddd{\varphi}{X}{X}}}\bigg)\bigg]\Bigg]\\
&=\Trace^{\ol{\feeddd{\psi}{Y}{Y}}\otimes\ol{\feeddd{\varphi}{X}{X}}}_{\ol{\feedcd{\psi}{Z}{Y}},\ol{\feeddc{\psi}{Y}{Z}}}\Bigg[
\bigg(\Big(\ol{\feedda{\psi}{Y}{Z}}\circ\Big(\ol{\feedac{\varphi}{Y}{X}}\Big)^{-1}\Big)\otimes\id_{\ol{\feeddd{\varphi}{X}{X}}}\bigg)\circ\\
&\hspace{1in}\circ\bigg(\id_{\ol{\feedcc{\varphi}{Y}{Y}}}\otimes\Big(\ol{\feedda{\varphi}{X}{Y}}\circ f\circ\Big(\ol{\feedad{\varphi}{X}{Y}}\Big)^{-1}\Big)\bigg)\circ\bigg(\Big(\ol{\feedca{\varphi}{Y}{X}}\circ\Big(\ol{\feedad{\psi}{Y}{Z}}\Big)^{-1}\Big)\otimes\id_{\ol{\feeddd{\varphi}{X}{X}}}\bigg)\Bigg]\\
&=\Trace^{\ol{\feeddd{\varphi}{X}{X}}\otimes\ol{\feeddd{\psi}{Y}{Y}}}_{\ol{\feedcd{\psi}{Z}{Y}},\ol{\feeddc{\psi}{Y}{Z}}}\Bigg[
\bigg(\id_{\ol{\feeddc{\psi}{Y}{Z}}}\otimes\gamma_{\ol{\feeddd{\psi}{Y}{Y}},\ol{\feeddd{\varphi}{X}{X}}}\bigg)\circ\bigg(\Big(\ol{\feedda{\psi}{Y}{Z}}\circ\Big(\ol{\feedac{\varphi}{Y}{X}}\Big)^{-1}\Big)\otimes\id_{\ol{\feeddd{\varphi}{X}{X}}}\bigg)\circ\\
&\circ\bigg(\id_{\ol{\feedcc{\varphi}{Y}{Y}}}\otimes\Big(\ol{\feedda{\varphi}{X}{Y}}\circ f\circ\Big(\ol{\feedad{\varphi}{X}{Y}}\Big)^{-1}\Big)\bigg)\circ\bigg(\Big(\ol{\feedca{\varphi}{Y}{X}}\circ\Big(\ol{\feedad{\psi}{Y}{Z}}\Big)^{-1}\Big)\otimes\id_{\ol{\feeddd{\varphi}{X}{X}}}\bigg)\circ\bigg(\id_{\ol{\feedcd{\psi}{Z}{Y}}}\otimes\gamma_{\ol{\feeddd{\varphi}{X}{X}},\ol{\feeddd{\psi}{Y}{Y}}}\bigg)\Bigg]\\
&=\Trace^{\ol{\feeddd{\varphi}{X}{X}}}_{\ol{\feedcd{\psi}{Z}{Y}},\ol{\feeddc{\psi}{Y}{Z}}}\Bigg[
\Trace^{\ol{\feeddd{\psi}{Y}{Y}}}_{\ol{\feedcd{\psi}{Z}{Y}}\otimes\ol{\feeddd{\varphi}{X}{X}},\ol{\feeddc{\psi}{Y}{Z}}\otimes\ol{\feeddd{\varphi}{X}{X}}}\bigg[
\bigg(\id_{\ol{\feeddc{\psi}{Y}{Z}}}\otimes\gamma_{\ol{\feeddd{\psi}{Y}{Y}},\ol{\feeddd{\varphi}{X}{X}}}\bigg)\circ\bigg(\Big(\ol{\feedda{\psi}{Y}{Z}}\circ\Big(\ol{\feedac{\varphi}{Y}{X}}\Big)^{-1}\Big)\otimes\id_{\ol{\feeddd{\varphi}{X}{X}}}\bigg)\circ\\
&\circ\bigg(\id_{\ol{\feedcc{\varphi}{Y}{Y}}}\otimes\Big(\ol{\feedda{\varphi}{X}{Y}}\circ f\circ\Big(\ol{\feedad{\varphi}{X}{Y}}\Big)^{-1}\Big)\bigg)\circ\bigg(\Big(\ol{\feedca{\varphi}{Y}{X}}\circ\Big(\ol{\feedad{\psi}{Y}{Z}}\Big)^{-1}\Big)\otimes\id_{\ol{\feeddd{\varphi}{X}{X}}}\bigg)\circ\bigg(\id_{\ol{\feedcd{\psi}{Z}{Y}}}\otimes\gamma_{\ol{\feeddd{\varphi}{X}{X}},\ol{\feeddd{\psi}{Y}{Y}}}\bigg)\bigg]\Bigg]\\
\end{align*}

Using that $\ol{Y_-}=\ol{\feedcd{\psi}{Z}{Y}}\otimes\ol{\feeddd{\psi}{Y}{Y}}$ and $\ol{Y_+}=\ol{\feeddc{\psi}{Y}{Z}}\otimes\ol{\feeddd{\psi}{Y}{Y}}$ we may apply the second vanishing axiom of the trace to get
\[\Trace^{\ol{\feeddd{\psi}{Y}{Y}}}_{\ol{\feedcd{\psi}{Z}{Y}},\ol{\feeddc{\psi}{Y}{Z}}}
\bigg[\Trace^{\ol{\feeddd{\varphi}{X}{X}}}_{\ol{\inp{Y}},\ol{\outp{Y}}}\Big[f\otimes\id_{\ol{\feedcc{\varphi}{Y}{Y}}}\Big]\bigg]=\Trace^{\ol{\feeddd{\psi}{Y}{Y}}\otimes\ol{\feeddd{\varphi}{X}{X}}}_{\ol{\feedcd{\psi}{Z}{Y}},\ol{\feeddc{\psi}{Y}{Z}}}
\Big[\id_{\ol{\feedcc{\varphi}{Y}{Y}}}\otimes f\Big].\]
By dinaturality using the symmetry $\gamma_{\ol{\feeddd{\varphi}{X}{X}},\ol{\feeddd{\psi}{Y}{Y}}}$ we get
\begin{align*}
&\Trace^{\ol{\feeddd{\psi}{Y}{Y}}\otimes\ol{\feeddd{\varphi}{X}{X}}}_{\ol{\feedcd{\psi}{Z}{Y}},\ol{\feeddc{\psi}{Y}{Z}}}
\Big[\id_{\ol{\feedcc{\varphi}{Y}{Y}}}\otimes f\Big]
=\Trace^{\ol{\feeddd{\varphi}{X}{X}}\otimes\ol{\feeddd{\psi}{Y}{Y}}}_{\ol{\feedcd{\psi}{Z}{Y}},\ol{\feeddc{\psi}{Y}{Z}}}
\Big[
\big(\id_{\ol{\feeddc{\psi}{Y}{Z}}}\otimes\gamma_{\ol{\feeddd{\psi}{Y}{Y}},\ol{\feeddd{\varphi}{X}{X}}}\big)\circ\big(\id_{\ol{\feedcc{\varphi}{Y}{Y}}}\otimes f\big)\circ\\
&\hspace{4in}\circ\big(\id_{\ol{\feedcd{\psi}{Z}{Y}}}\otimes\gamma_{\ol{\feeddd{\varphi}{X}{X}},\ol{\feeddd{\psi}{Y}{Y}}}\big)
\Big].
\end{align*}
Applying the second vanishing axiom again this becomes
\begin{align*}
&\Trace^{\ol{\feeddd{\varphi}{X}{X}}}_{\ol{\feedcd{\psi}{Z}{Y}},\ol{\feeddc{\psi}{Y}{Z}}}
\bigg[
\Trace^{\ol{\feeddd{\psi}{Y}{Y}}}_{\ol{\feedcd{\psi}{Z}{Y}}\otimes\ol{\feeddd{\varphi}{X}{X}},\ol{\feeddc{\psi}{Y}{Z}}\otimes\ol{\feeddd{\varphi}{X}{X}}}
\Big[
\big(\id_{\ol{\feeddc{\psi}{Y}{Z}}}\otimes\gamma_{\ol{\feeddd{\psi}{Y}{Y}},\ol{\feeddd{\varphi}{X}{X}}}\big)\circ\big(\id_{\ol{\feedcc{\varphi}{Y}{Y}}}\otimes f\big)\circ\\
&\hspace{4in}\circ\big(\id_{\ol{\feedcd{\psi}{Z}{Y}}}\otimes\gamma_{\ol{\feeddd{\varphi}{X}{X}},\ol{\feeddd{\psi}{Y}{Y}}}\big)
\Big]
\bigg].
\end{align*}
Once again applying the superposing axiom of the trace we may reincorporate the map $\id_{\ol{\feedcc{\psi}{Z}{Z}}}$ and obtain the result.
\end{proof}

The basic wires of $\feeddd{\psi}{Y}{Y}$ can be partitioned as $A_0\sqcup A_{XX}\sqcup A_{XZ}\sqcup A_{ZX}\sqcup A_{ZZ}$ where: 
\begin{itemize}
 \item a composed wire passing through an element of $A_0$ has no beginning or end;
 \item a composed wire passing through an element of $A_{XX}$ begins at $X$ terminates at $X$;
 \item $A_{XZ}$, $A_{ZX}$, $A_{ZZ}$ are defined similarly.
\end{itemize}
Denote by $\delta_\psi$ the bijection just identified, i.e.
\[\delta_\psi:A_0\sqcup A_{XX}\sqcup A_{XZ}\sqcup A_{ZX}\sqcup A_{ZZ}\Too{\sim}\feeddd{\psi}{Y}{Y}.\]
Write $P_{XX}=\displaystyle\bigsqcup_{i\ge1}\feeddd{\psi\;\accentset{i}{\circ}\;\varphi}{X}{X}$ for the set of composed wires connecting $\outp{X}$ to $\inp{X}$ that truly arise from the composition and note that any such wire must pass through at least one element of $\feeddd{\psi}{Y}{Y}$, thus we have a surjective map $\pi_\psi:A_{XX}\onto P_{XX}$.  Moreover, the fiber of $\pi_\psi$ over any element of $P_{XX}$ has a canonical ordering by following the composed wire from its beginning at $\outp{X}$ to its termination at $\inp{X}$.  There are also canonical identifications of $P_{XX}$ as a subset of each of the sets $\feeddc{\varphi}{X}{Y}$ and $\feedcd{\varphi}{Y}{X}$.

We also identify a partition of $\feedcc{\varphi}{Y}{Y}$ into subsets $B_0\sqcup B_{XX}\sqcup B_{XZ}\sqcup B_{ZX}\sqcup B_{ZZ}$ defined as above, i.e. there is a bijection
\[\delta_\varphi:B_0\sqcup B_{XX}\sqcup B_{XZ}\sqcup B_{ZX}\sqcup B_{ZZ}\Too{\sim}\feedcc{\varphi}{Y}{Y}.\]
Similarly we write $P_{ZZ}=\displaystyle\bigsqcup_{i\ge1}\feedcc{\psi\;\accentset{i}{\circ}\;\varphi}{Z}{Z}$ for the set of composed wires connecting $Z$ to $Z$ that truly arise from the composition and note as above that we have a surjective map $\pi_\varphi:B_{ZZ}\onto P_{ZZ}$, where again the fibers are canonically ordered sets.  Note also that $P_{ZZ}$ is canonically identified as a subset of each of the sets $\feedcd{\psi}{Z}{Y}$ and $\feeddc{\psi}{Y}{Z}$. 

By tracing the wires forward through the boundary of $Y$ we obtain canonical ``feed-forward'' inclusions
\[\vec{\sigma}_{XZ}:A_{XZ}\into B_{XZ},\quad \vec{\sigma}_{ZZ}:A_{ZZ}\into B_{ZZ},\quad \vec{\sigma}'_{XX}:B_{XX}\into A_{XX},\quad \vec{\sigma}'_{ZX}:B_{ZX}\into A_{ZX}.\]
By the same process we also obtain bijections $\sigma_0:A_0\to B_0$ and $\sigma_0':B_0\to A_0$.  Note that $\vec{\sigma}_{XZ}$ and $\vec{\sigma}'_{ZX}$ are actually bijections.  Denote by $\vec{A'}_{XX}$ the complement of $B_{XX}$ inside $A_{XX}$ and by $\vec{B'}_{ZZ}$ the complement of $A_{ZZ}$ inside $B_{ZZ}$.  

Similarly we may trace certain wires backwards through the boundary of $Y$ to obtain ``feed-backward'' inclusions
\[\cev{\sigma}_{ZZ}:A_{ZZ}\into B_{ZZ},\quad \cev{\sigma}'_{XX}:B_{XX}\into A_{XX}\]
and denote by $\cev{A'}_{XX},\cev{B'}_{ZZ}$ the respective complements.

With this notation we may use dinaturality to rewrite the internal trace of Lemma~\ref{le:combining traces} as
\begin{align*}
&\Trace^{\ol{A_0}\otimes\ol{A_{XX}}\otimes\ol{A_{XZ}}\otimes\ol{A_{ZX}}\otimes\ol{A_{ZZ}}}_{\ol{\feedcd{\psi}{Z}{Y}}\otimes\ol{\feeddd{\varphi}{X}{X}},\ol{\feeddc{\psi}{Y}{Z}}\otimes\ol{\feeddd{\varphi}{X}{X}}}\Bigg[\bigg(\id_{\ol{\feeddc{\psi}{Y}{Z}}}\otimes\id_{\ol{\feeddd{\varphi}{X}{X}}}\otimes\ol{\delta_\psi}\bigg)\circ
\bigg(\id_{\ol{\feeddc{\psi}{Y}{Z}}}\otimes\gamma_{\ol{\feeddd{\psi}{Y}{Y}},\ol{\feeddd{\varphi}{X}{X}}}\bigg)\circ\\
&\circ\bigg(\Big(\ol{\feedda{\psi}{Y}{Z}}\circ\Big(\ol{\feedac{\varphi}{Y}{X}}\Big)^{-1}\Big)\otimes\id_{\ol{\feeddd{\varphi}{X}{X}}}\bigg)\circ\bigg(\id_{\ol{\feedcc{\varphi}{Y}{Y}}}\otimes\Big(\ol{\feedda{\varphi}{X}{Y}}\circ f\circ\Big(\ol{\feedad{\varphi}{X}{Y}}\Big)^{-1}\Big)\bigg)\circ\\
&\circ\bigg(\Big(\ol{\feedca{\varphi}{Y}{X}}\circ\Big(\ol{\feedad{\psi}{Y}{Z}}\Big)^{-1}\Big)\otimes\id_{\ol{\feeddd{\varphi}{X}{X}}}\bigg)\circ\bigg(\id_{\ol{\feedcd{\psi}{Z}{Y}}}\otimes\gamma_{\ol{\feeddd{\varphi}{X}{X}},\ol{\feeddd{\psi}{Y}{Y}}}\bigg)\circ\bigg(\id_{\ol{\feedcd{\psi}{Z}{Y}}}\otimes\id_{\ol{\feeddd{\varphi}{X}{X}}}\otimes(\ol{\delta_\psi})^{-1}\bigg)\Bigg]\\
\end{align*}
\begin{align*}
&=\Trace^{\ol{A_0}\otimes\ol{A_{XX}}\otimes\ol{A_{XZ}}\otimes\ol{A_{ZX}}\otimes\ol{A_{ZZ}}}_{\ol{\feedcd{\psi}{Z}{Y}}\otimes\ol{\feeddd{\varphi}{X}{X}},\ol{\feeddc{\psi}{Y}{Z}}\otimes\ol{\feeddd{\varphi}{X}{X}}}\Bigg[
\bigg(\id_{\ol{\feeddc{\psi}{Y}{Z}}}\otimes\gamma_{\ol{A_0}\otimes\ol{A_{XX}}\otimes\ol{A_{XZ}}\otimes\ol{A_{ZX}}\otimes\ol{A_{ZZ}},\ol{\feeddd{\varphi}{X}{X}}}\bigg)\circ\\
&\circ\bigg(\Big(\Big(\id_{\ol{\feeddc{\psi}{Y}{Z}}}\otimes\ol{\delta_\psi}\Big)\circ\ol{\feedda{\psi}{Y}{Z}}\circ\Big(\ol{\feedac{\varphi}{Y}{X}}\Big)^{-1}\Big)\otimes\id_{\ol{\feeddd{\varphi}{X}{X}}}\bigg)\circ\bigg(\id_{\ol{\feedcc{\varphi}{Y}{Y}}}\otimes\Big(\ol{\feedda{\varphi}{X}{Y}}\circ f\circ\Big(\ol{\feedad{\varphi}{X}{Y}}\Big)^{-1}\Big)\bigg)\circ\\
&\circ\bigg(\Big(\ol{\feedca{\varphi}{Y}{X}}\circ\Big(\ol{\feedad{\psi}{Y}{Z}}\Big)^{-1}\circ\Big(\id_{\ol{\feedcd{\psi}{Z}{Y}}}\otimes(\ol{\delta_\psi})^{-1}\Big)\Big)\otimes\id_{\ol{\feeddd{\varphi}{X}{X}}}\bigg)\circ\bigg(\id_{\ol{\feedcd{\psi}{Z}{Y}}}\otimes\gamma_{\ol{\feeddd{\varphi}{X}{X}},\ol{A_0}\otimes\ol{A_{XX}}\otimes\ol{A_{XZ}}\otimes\ol{A_{ZX}}\otimes\ol{A_{ZZ}}}\bigg)\Bigg]
\end{align*}
\begin{align*}
&=\Trace^{\ol{A_0}\otimes\ol{A_{XX}}\otimes\ol{A_{XZ}}\otimes\ol{A_{ZX}}\otimes\ol{A_{ZZ}}}_{\ol{\feedcd{\psi}{Z}{Y}}\otimes\ol{\feeddd{\varphi}{X}{X}},\ol{\feeddc{\psi}{Y}{Z}}\otimes\ol{\feeddd{\varphi}{X}{X}}}\Bigg[
\bigg(\id_{\ol{\feeddc{\psi}{Y}{Z}}}\otimes\gamma_{\ol{A_0}\otimes\ol{A_{XX}}\otimes\ol{A_{XZ}}\otimes\ol{A_{ZX}}\otimes\ol{A_{ZZ}},\ol{\feeddd{\varphi}{X}{X}}}\bigg)\circ\\
&\circ\bigg(\Big(\Big(\id_{\ol{\feeddc{\psi}{Y}{Z}}}\otimes\ol{\delta_\psi}\Big)\circ\ol{\feedda{\psi}{Y}{Z}}\circ\Big(\ol{\feedac{\varphi}{Y}{X}}\Big)^{-1}\Big)\otimes\id_{\ol{\feeddd{\varphi}{X}{X}}}\bigg)\circ\\
&\circ\bigg(\big(\ol{\delta_\varphi}\big)^{-1}\otimes\ol{\feedda{\varphi}{X}{Y}}\bigg)\circ\bigg(\id_{\ol{B_0}}\otimes\id_{\ol{B_{XX}}}\otimes\id_{\ol{B_{XZ}}}\otimes\id_{\ol{B_{ZX}}}\otimes\id_{\ol{B_{ZZ}}}\otimes f\bigg)\circ\bigg(\ol{\delta_\varphi}\otimes\Big(\ol{\feedad{\varphi}{X}{Y}}\Big)^{-1}\bigg)\circ\\
&\circ\bigg(\Big(\ol{\feedca{\varphi}{Y}{X}}\circ\Big(\ol{\feedad{\psi}{Y}{Z}}\Big)^{-1}\circ\Big(\id_{\ol{\feedcd{\psi}{Z}{Y}}}\otimes(\ol{\delta_\psi})^{-1}\Big)\Big)\otimes\id_{\ol{\feeddd{\varphi}{X}{X}}}\bigg)\circ\bigg(\id_{\ol{\feedcd{\psi}{Z}{Y}}}\otimes\gamma_{\ol{\feeddd{\varphi}{X}{X}},\ol{A_0}\otimes\ol{A_{XX}}\otimes\ol{A_{XZ}}\otimes\ol{A_{ZX}}\otimes\ol{A_{ZZ}}}\bigg)\Bigg]
\end{align*}

Our goal is to understand the map
\[\Bigg(\bigg(\Big(\id_{\ol{\feeddc{\psi}{Y}{Z}}}\otimes\ol{\delta_\psi}\Big)\circ\ol{\feedda{\psi}{Y}{Z}}\circ\Big(\ol{\feedac{\varphi}{Y}{X}}\Big)^{-1}\bigg)\otimes\id_{\ol{\feeddd{\varphi}{X}{X}}}\Bigg)\circ\bigg(\big(\ol{\delta_\varphi}\big)^{-1}\otimes\ol{\feedda{\varphi}{X}{Y}}\bigg)\]
which can be visualized via the following diagram in $\Set$:\\

\begin{tikzpicture}
%first column
 \draw (1.5,-1) ellipse (.2 and .5);
 \node at (1.75,-0.35) {$\scriptstyle \feeddd{\varphi}{X}{X}$};
 \draw (1.5,0.25) ellipse (.1 and .25);
 \node at (1.1,0.5) {$\scriptstyle A_{XX}$};
 \draw (1.5,1.25) ellipse (.1 and .25);
 \node at (1.1,1.5) {$\scriptstyle A_{XZ}$};
 \draw (1.5,2.25) ellipse (.1 and .25);
 \node at (1.25,2.5) {$\scriptstyle A_0$};
 \draw (1.5,3.25) ellipse (.1 and .25);
 \node at (1.1,3.5) {$\scriptstyle A_{ZX}$};
 \draw (1.5,4.25) ellipse (.1 and .25);
 \node at (1.1,4.5) {$\scriptstyle A_{ZZ}$};
% \draw (1.5,5.5) ellipse (.2 and .5);
% \node at (1.75,6.2) {$\scriptstyle \feedcd{\psi}{Z}{Y}$};
% \draw (1.5,7) ellipse (.2 and .5);
% \node at (1.75,7.7) {$\scriptstyle \feedcc{\psi}{Z}{Z}$};
 \draw (1.5,6.25) ellipse (.4 and 1);
 \node at (1.9,7.25) {$\scriptstyle \inp{Z}$};
 %first arrows
 \draw[-stealth] (1.5375,-1) -- (2.9625,-1);
 \draw[-stealth] (1.5,0.25) -- (2.9,2);
 \draw[-stealth] (1.5,1.25) -- (2.875,2.125);
 \draw[-stealth] (1.5,2.25) -- (2.875,2.25);
 \draw[-stealth] (1.5,3.25) -- (2.875,2.375);
 \draw[-stealth] (1.5,4.25) -- (2.9,2.5);
% \draw[-stealth] (1.5375,5.5) -- (2.9625,5.5);
% \draw[-stealth] (1.5375,7) -- (2.9625,7);
 \draw[-stealth] (1.5,6.125) -- (2.9625,5.5);
 \draw[-stealth] (1.5,6.375) -- (2.9625,7);
 %second column
 \draw (3,-1) ellipse (.2 and .5);
 \node at (3.25,-0.35) {$\scriptstyle \feeddd{\varphi}{X}{X}$};
 \draw (3,2.25) ellipse (.5 and 1.25);
 \node at (3.25,3.7) {$\scriptstyle \feeddd{\psi}{Y}{Y}$};
 \draw (3,5.5) ellipse (.2 and .5);
 \node at (3.25,6.2) {$\scriptstyle \feedcd{\psi}{Z}{Y}$};
 \draw (3,7) ellipse (.2 and .5);
 \node at (3.25,7.7) {$\scriptstyle \feedcc{\psi}{Z}{Z}$};
 %second arrows
 \draw[-stealth] (3.0375,-1) -- (4.4625,-1);
 \draw[-stealth] (3.0375,2.25) -- (4.4625,2.876);
 \draw[-stealth] (3.0375,5.5) -- (4.4625,3.125);
 \draw[-stealth] (3.0375,7) -- (4.4625,7);
 %third column
 \draw (4.5,-1) ellipse (.2 and .5);
 \node at (4.75,-0.35) {$\scriptstyle \feeddd{\varphi}{X}{X}$};
 \draw (4.5,3) ellipse (.7 and 1.75);
 \node at (4.95,4.75) {$\scriptstyle \inp{Y}$};
 \draw (4.5,7) ellipse (.2 and .5);
 \node at (4.75,7.7) {$\scriptstyle \feedcc{\psi}{Z}{Z}$};
 %third arrows
 \draw[-stealth] (4.5375,-1) -- (5.9625,-1);
 \draw[-stealth] (4.625,2.875) -- (5.9625,0.5);
 \draw[-stealth] (4.625,3.125) -- (5.9625,3.75);
 \draw[-stealth] (4.5375,7) -- (5.9625,7);
 %fourth column
 \draw (6,-1) ellipse (.2 and .5);
 \node at (6.25,-0.35) {$\scriptstyle \feeddd{\varphi}{X}{X}$};
 \draw (6,0.5) ellipse (.2 and .5);
 \node at (6.25,1.2) {$\scriptstyle \feedcd{\varphi}{Y}{X}$};
 \draw (6,3.75) ellipse (.5 and 1.25);
 \node at (6.25,5.2) {$\scriptstyle \feedcc{\varphi}{Y}{Y}$};
 \draw (6,7) ellipse (.2 and .5);
 \node at (6.25,7.7) {$\scriptstyle \feedcc{\psi}{Z}{Z}$};
 %fourth arrows
 \draw[-stealth] (6.0375,-1) -- (7.5,-0.375);
 \draw[-stealth] (6.0375,0.5) -- (7.5,-0.125);
 \draw[-stealth] (6.1,3.5) -- (7.5,1.75);
 \draw[-stealth] (6.125,3.625) -- (7.5,2.75);
 \draw[-stealth] (6.125,3.75) -- (7.5,3.75);
 \draw[-stealth] (6.125,3.875) -- (7.5,4.75);
 \draw[-stealth] (6.1,4) -- (7.5,5.75);
 \draw[-stealth] (6.0375,7) -- (7.4625,7);
%fifth column
 \draw (7.5,-0.25) ellipse (.4 and 1);
 \node at (7.9,0.75) {$\scriptstyle \inp{X}$};
 \draw (7.5,1.75) ellipse (.1 and .25);
 \node at (7.9,2) {$\scriptstyle B_{XX}$};
 \draw (7.5,2.75) ellipse (.1 and .25);
 \node at (7.9,3) {$\scriptstyle B_{XZ}$};
 \draw (7.5,3.75) ellipse (.1 and .25);
 \node at (7.75,4) {$\scriptstyle B_0$};
 \draw (7.5,4.75) ellipse (.1 and .25);
 \node at (7.9,5) {$\scriptstyle B_{ZX}$};
 \draw (7.5,5.75) ellipse (.1 and .25);
 \node at (7.9,6) {$\scriptstyle B_{ZZ}$};
 \draw (7.5,7) ellipse (.2 and .5);
 \node at (7.75,7.7) {$\scriptstyle \feedcc{\psi}{Z}{Z}$};
 %fifth arrows
 \draw[-stealth] (7.5,-0.25) -- (9,-0.25);
 \node at (8.25,-0.1) {$\scriptstyle f$};
 \draw[-stealth] (7.5,1.75) -- (9,1.75);
 \draw[-stealth] (7.5,2.75) -- (9,2.75);
 \draw[-stealth] (7.5,3.75) -- (9,3.75);
 \draw[-stealth] (7.5,4.75) -- (9,4.75);
 \draw[-stealth] (7.5,5.75) -- (9,5.75);
 \draw[-stealth] (7.5375,7) -- (8.9625,7);
%sixth column
 \draw (9,-0.25) ellipse (.4 and 1);
 \node at (8.6,0.75) {$\scriptstyle \outp{X}$};
 \draw (9,1.75) ellipse (.1 and .25);
 \node at (8.6,2) {$\scriptstyle B_{XX}$};
 \draw (9,2.75) ellipse (.1 and .25);
 \node at (8.6,3) {$\scriptstyle B_{XZ}$};
 \draw (9,3.75) ellipse (.1 and .25);
 \node at (8.75,4) {$\scriptstyle B_0$};
 \draw (9,4.75) ellipse (.1 and .25);
 \node at (8.6,5) {$\scriptstyle B_{ZX}$};
 \draw (9,5.75) ellipse (.1 and .25);
 \node at (8.6,6) {$\scriptstyle B_{ZZ}$};
 \draw (9,7) ellipse (.2 and .5);
 \node at (8.75,7.7) {$\scriptstyle \feedcc{\psi}{Z}{Z}$};
 %sixth arrows
 \draw[-stealth] (9,-0.375) -- (10.4625,-1);
 \draw[-stealth] (9,-0.125) -- (10.4625,0.5);
 \draw[-stealth] (9,1.75) -- (10.4,3.5);
 \draw[-stealth] (9,2.75) -- (10.375,3.625);
 \draw[-stealth] (9,3.75) -- (10.375,3.75);
 \draw[-stealth] (9,4.75) -- (10.375,3.875);
 \draw[-stealth] (9,5.75) -- (10.4,4);
 \draw[-stealth] (9.0375,7) -- (10.4625,7);
 %seventh column
 \draw (10.5,-1) ellipse (.2 and .5);
 \node at (10.25,-0.35) {$\scriptstyle \feeddd{\varphi}{X}{X}$};
 \draw (10.5,0.5) ellipse (.2 and .5);
 \node at (10.25,1.2) {$\scriptstyle \feeddc{\varphi}{X}{Y}$};
 \draw (10.5,3.75) ellipse (.5 and 1.25);
 \node at (10.25,5.2) {$\scriptstyle \feedcc{\varphi}{Y}{Y}$};
 \draw (10.5,7) ellipse (.2 and .5);
 \node at (10.25,7.7) {$\scriptstyle \feedcc{\psi}{Z}{Z}$};
 %seventh arrows
 \draw[-stealth] (10.5375,-1) -- (11.9625,-1);
 \draw[-stealth] (10.5375,0.5) -- (11.875,2.875);
 \draw[-stealth] (10.5375,3.75) -- (11.875,3.125);
 \draw[-stealth] (10.5375,7) -- (11.9625,7);
 %eighth column
 \draw (12,-1) ellipse (.2 and .5);
 \node at (11.75,-0.35) {$\scriptstyle \feeddd{\varphi}{X}{X}$};
 \draw (12,3) ellipse (.7 and 1.75);
 \node at (11.55,4.75) {$\scriptstyle \outp{Y}$};
 \draw (12,7) ellipse (.2 and .5);
 \node at (11.75,7.7) {$\scriptstyle \feedcc{\psi}{Z}{Z}$};
 %eighth arrows
 \draw[-stealth] (12.0375,-1) -- (13.4625,-1);
 \draw[-stealth] (12.0375,2.876) -- (13.4625,2.25);
 \draw[-stealth] (12.0375,3.125) -- (13.4625,5.5);
 \draw[-stealth] (12.0375,7) -- (13.4625,7);
 %ninth column
 \draw (13.5,-1) ellipse (.2 and .5);
 \node at (13.25,-0.35) {$\scriptstyle \feeddd{\varphi}{X}{X}$};
 \draw (13.5,2.25) ellipse (.5 and 1.25);
 \node at (13.25,3.7) {$\scriptstyle \feeddd{\psi}{Y}{Y}$};
 \draw (13.5,5.5) ellipse (.2 and .5);
 \node at (13.25,6.2) {$\scriptstyle \feeddc{\psi}{Y}{Z}$};
 \draw (13.5,7) ellipse (.2 and .5);
 \node at (13.25,7.7) {$\scriptstyle \feedcc{\psi}{Z}{Z}$};
 %ninth arrows
 \draw[-stealth] (13.5375,-1) -- (14.9625,-1);
 \draw[-stealth] (13.6,2) -- (15,0.25);
 \draw[-stealth] (13.625,2.125) -- (15,1.25);
 \draw[-stealth] (13.625,2.25) -- (15,2.25);
 \draw[-stealth] (13.625,2.375) -- (15,3.25);
 \draw[-stealth] (13.6,2.5) -- (15,4.25);
% \draw[-stealth] (13.5375,5.5) -- (14.9625,5.5);
% \draw[-stealth] (13.5375,7) -- (14.9625,7);
 \draw[-stealth] (13.5375,5.5) -- (15,6.125);
 \draw[-stealth] (13.5375,7) -- (15,6.375);
 %tenth column
 \draw (15,-1) ellipse (.2 and .5);
 \node at (14.75,-0.35) {$\scriptstyle \feeddd{\varphi}{X}{X}$};
 \draw (15,0.25) ellipse (.1 and .25);
 \node at (15.4,0.5) {$\scriptstyle A_{XX}$};
 \draw (15,1.25) ellipse (.1 and .25);
 \node at (15.4,1.5) {$\scriptstyle A_{XZ}$};
 \draw (15,2.25) ellipse (.1 and .25);
 \node at (15.25,2.5) {$\scriptstyle A_0$};
 \draw (15,3.25) ellipse (.1 and .25);
 \node at (15.4,3.5) {$\scriptstyle A_{ZX}$};
 \draw (15,4.25) ellipse (.1 and .25);
 \node at (15.4,4.5) {$\scriptstyle A_{ZZ}$};
% \draw (15,5.5) ellipse (.2 and .5);
% \node at (14.75,6.2) {$\scriptstyle \feeddc{\psi}{Y}{Z}$};
% \draw (15,7) ellipse (.2 and .5);
% \node at (14.75,7.7) {$\scriptstyle \feedcc{\psi}{Z}{Z}$};
 \draw (15,6.25) ellipse (.4 and 1);
 \node at (14.6,7.25) {$\scriptstyle \outp{Z}$};
\end{tikzpicture}


%Old version
%\begin{align*}
%&\Trace^{\ol{\feeddd{\psi}{Y}{Y}}}_{\ol{\feedcd{\psi}{Z}{Y}}\otimes\ol{\feeddd{\varphi}{X}{X}},\ol{\feeddc{\psi}{Y}{Z}}\otimes\ol{\feeddd{\varphi}{X}{X}}}
%\bigg[
%\Big(\id_{\ol{\feeddc{\psi}{Y}{Z}}}\otimes\gamma_{\ol{\feeddd{\psi}{Y}{Y}},\ol{\feeddd{\varphi}{X}{X}}}\Big)\circ\Big(\id_{\ol{\feedcc{\varphi}{Y}{Y}}}\otimes f\Big)\circ\Big(\id_{\ol{\feedcd{\psi}{Z}{Y}}}\otimes\gamma_{\ol{\feeddd{\varphi}{X}{X}},\ol{\feeddd{\psi}{Y}{Y}}}\Big)
%\bigg]\\
%&\quad=\Trace^{\ol{P_{XX}}\otimes\ol{A_0}\otimes\ol{A_X}\otimes\ol{A_Z}}_{\ol{\feedcd{\psi}{Z}{Y}}\otimes\ol{\feeddd{\varphi}{X}{X}},\ol{\feeddc{\psi}{Y}{Z}}\otimes\ol{\feeddd{\varphi}{X}{X}}}
%\bigg[\Big(\id_{\ol{\feeddc{\psi}{Y}{Z}}\otimes\ol{\feeddd{\varphi}{X}{X}}}\otimes\ol{\delta_\psi}\Big)\circ \Big(\id_{\ol{\feeddc{\psi}{Y}{Z}}}\otimes\gamma_{\ol{\feeddd{\psi}{Y}{Y}},\ol{\feeddd{\varphi}{X}{X}}}\Big)\circ\\
%&\hspace{1.5in}\circ\Big(\id_{\ol{\feedcc{\varphi}{Y}{Y}}}\otimes f\Big)\circ\Big(\id_{\ol{\feedcd{\psi}{Z}{Y}}}\otimes\gamma_{\ol{\feeddd{\varphi}{X}{X}},\ol{\feeddd{\psi}{Y}{Y}}}\Big)\circ\Big(\id_{\ol{\feedcd{\psi}{Z}{Y}}\otimes\ol{\feeddd{\varphi}{X}{X}}}\otimes(\ol{\delta_\psi})^{-1}\Big)\bigg]
%\end{align*}

\newpage

By the second vanishing axiom this is equivalent to $\Trace_{_{\ol{\feedcd{\psi}{Z}{Y}}\otimes\ol{\feeddd{\varphi}{X}{X}},\ol{\feeddc{\psi}{Y}{Z}}\otimes\ol{\feeddd{\varphi}{X}{X}}}}^{\ol{P_{XX}}}$ applied to the following
\begin{align*}
&\Trace^{\ol{A_0}\otimes\ol{A_X}\otimes\ol{A_Z}}_{\ol{\feedcd{\psi}{Z}{Y}}\otimes\ol{\feeddd{\varphi}{X}{X}}\otimes\ol{P_{XX}},\ol{\feeddc{\psi}{Y}{Z}}\otimes\ol{\feeddd{\varphi}{X}{X}}\otimes\ol{P_{XX}}}\Bigg[
\bigg(\id_{\ol{\feeddc{\psi}{Y}{Z}}}\otimes\gamma_{\ol{A_0}\otimes\ol{A_{XX}}\otimes\ol{A_{XZ}}\otimes\ol{A_{ZX}}\otimes\ol{A_{ZZ}},\ol{\feeddd{\varphi}{X}{X}}}\bigg)\circ\\
&\circ\bigg(\Big(\Big(\id_{\ol{\feeddc{\psi}{Y}{Z}}}\otimes\ol{\delta_\psi}\Big)\circ\ol{\feedda{\psi}{Y}{Z}}\circ\Big(\ol{\feedac{\varphi}{Y}{X}}\Big)^{-1}\Big)\otimes\id_{\ol{\feeddd{\varphi}{X}{X}}}\bigg)\circ\bigg(\id_{\ol{\feedcc{\varphi}{Y}{Y}}}\otimes\Big(\ol{\feedda{\varphi}{X}{Y}}\circ f\circ\Big(\ol{\feedad{\varphi}{X}{Y}}\Big)^{-1}\Big)\bigg)\circ\\
&\circ\bigg(\Big(\ol{\feedca{\varphi}{Y}{X}}\circ\Big(\ol{\feedad{\psi}{Y}{Z}}\Big)^{-1}\circ\Big(\id_{\ol{\feedcd{\psi}{Z}{Y}}}\otimes(\ol{\delta_\psi})^{-1}\Big)\Big)\otimes\id_{\ol{\feeddd{\varphi}{X}{X}}}\bigg)\circ\bigg(\id_{\ol{\feedcd{\psi}{Z}{Y}}}\otimes\gamma_{\ol{\feeddd{\varphi}{X}{X}},\ol{A_0}\otimes\ol{A_{XX}}\otimes\ol{A_{XZ}}\otimes\ol{A_{ZX}}\otimes\ol{A_{ZZ}}}\bigg)\Bigg].
\end{align*}

%old version
%\begin{align}
%\label{eq:trace1} &\Trace^{\ol{P_{XX}}}_{\ol{\feedcd{\psi}{Z}{Y}}\otimes\ol{\feeddd{\varphi}{X}{X}},\ol{\feeddc{\psi}{Y}{Z}}\otimes\ol{\feeddd{\varphi}{X}{X}}}
% \Bigg[\Trace^{\ol{A_0}\otimes\ol{A_X}\otimes\ol{A_Z}}_{\ol{\feedcd{\psi}{Z}{Y}}\otimes\ol{\feeddd{\varphi}{X}{X}}\otimes\ol{P_{XX}},\ol{\feeddc{\psi}{Y}{Z}}\otimes\ol{\feeddd{\varphi}{X}{X}}\otimes\ol{P_{XX}}}
%\big[F\big]\Bigg],
%\end{align}
%where we abbreviated
%\begin{align*}
%&F=\Big(\id_{\ol{\feeddc{\psi}{Y}{Z}}}\otimes\gamma_{\ol{P_{XX}}\otimes\ol{A_0}\otimes\ol{A_X}\otimes\ol{A_Z},\ol{\feeddd{\varphi}{X}{X}}}\Big)\circ\Big(\id_{\ol{P_{ZZ}}}\otimes\id_{\ol{B_0}}\otimes\id_{\ol{B_X}}\otimes\id_{\ol{B_Z}}\otimes f\Big)\circ\\
%&\hspace{3.5in}\circ\Big(\id_{\ol{\feedcd{\psi}{Z}{Y}}}\otimes\gamma_{\ol{\feeddd{\varphi}{X}{X}},\ol{P_{XX}}\otimes\ol{A_0}\otimes\ol{A_X}\otimes\ol{A_Z}}\Big).
%\end{align*}
But note that there are natural bijections 
\[\omega_1:\feedcd{\psi}{Z}{Y}\sqcup\feeddd{\varphi}{X}{X}\sqcup P_{XX}\To{\sim}P_{ZZ}\sqcup\inp{X}\quad\text{and}\quad \omega_2:\feeddc{\psi}{Y}{Z}\sqcup\feeddd{\varphi}{X}{X}\sqcup P_{XX}\To{\sim}P_{ZZ}\sqcup\outp{X}\]
so that by naturality the trace above may be rewritten as
\begin{equation}\label{eq:internal trace}
\Trace^{\ol{A_0}\otimes\ol{A_X}\otimes\ol{A_Z}}_{\ol{P_{ZZ}}\otimes\ol{\inp{X}},\ol{P_{ZZ}}\otimes\ol{\outp{X}}}
\big[F'\big].
\end{equation}
where
\begin{align*}
F'
&=\bigg(\big(\ol{\omega_2}\big)^{-1}\otimes\id_{\ol{A_0}}\otimes\id_{\ol{A_X}}\otimes\id_{\ol{A_Z}}\bigg)\circ\bigg(\id_{\ol{\feeddc{\psi}{Y}{Z}}}\otimes\gamma_{\ol{P_{XX}}\otimes\ol{A_0}\otimes\ol{A_X}\otimes\ol{A_Z},\ol{\feeddd{\varphi}{X}{X}}}\bigg)\circ\\
&\circ\bigg(\Big(\Big(\id_{\ol{\feeddc{\psi}{Y}{Z}}}\otimes\ol{\delta_\psi}\Big)\circ\ol{\feedda{\psi}{Y}{Z}}\circ\Big(\ol{\feedac{\varphi}{Y}{X}}\Big)^{-1}\Big)\otimes\id_{\ol{\feeddd{\varphi}{X}{X}}}\bigg)\circ\bigg(\id_{\ol{\feedcc{\varphi}{Y}{Y}}}\otimes\Big(\ol{\feedda{\varphi}{X}{Y}}\circ f\circ\Big(\ol{\feedad{\varphi}{X}{Y}}\Big)^{-1}\Big)\bigg)\circ\\
&\circ\bigg(\Big(\ol{\feedca{\varphi}{Y}{X}}\circ\Big(\ol{\feedad{\psi}{Y}{Z}}\Big)^{-1}\circ\Big(\id_{\ol{\feedcd{\psi}{Z}{Y}}}\otimes(\ol{\delta_\psi})^{-1}\Big)\Big)\otimes\id_{\ol{\feeddd{\varphi}{X}{X}}}\bigg)\circ\bigg(\id_{\ol{\feedcd{\psi}{Z}{Y}}}\otimes\gamma_{\ol{\feeddd{\varphi}{X}{X}},\ol{P_{XX}}\otimes\ol{A_0}\otimes\ol{A_X}\otimes\ol{A_Z}}\bigg)\circ\\
&\circ\bigg(\ol{\omega_1}\otimes\id_{\ol{A_0}}\otimes\id_{\ol{A_X}}\otimes\id_{\ol{A_Z}}\bigg).
\end{align*}
Again using the second vanishing axiom this is equivalent to $\Trace^{\ol{A_0}\otimes\ol{A_X}}_{\ol{P_{ZZ}}\otimes\ol{\inp{X}},\ol{P_{ZZ}}\otimes\ol{\outp{X}}}$ applied to the following
\begin{align*}
\Trace^{\ol{A_Z}}_{\ol{P_{ZZ}}\otimes\ol{\inp{X}}\otimes\ol{A_0}\otimes\ol{A_X},\ol{P_{ZZ}}\otimes\ol{\outp{X}}\otimes\ol{A_0}\otimes\ol{A_X}}\big[F''\big]
\end{align*}
where
\begin{align*}
F''
&=\bigg(\big(\ol{\omega_2}\big)^{-1}\otimes\id_{\ol{A_0}}\otimes\id_{\ol{A_X}}\otimes\id_{\ol{A_Z}}\bigg)\circ\bigg(\id_{\ol{\feeddc{\psi}{Y}{Z}}}\otimes\gamma_{\ol{P_{XX}}\otimes\ol{A_0}\otimes\ol{A_X}\otimes\ol{A_Z},\ol{\feeddd{\varphi}{X}{X}}}\bigg)\circ\\
&\circ\bigg(\Big(\Big(\id_{\ol{\feeddc{\psi}{Y}{Z}}}\otimes\ol{\delta_\psi}\Big)\circ\ol{\feedda{\psi}{Y}{Z}}\circ\Big(\ol{\feedac{\varphi}{Y}{X}}\Big)^{-1}\Big)\otimes\id_{\ol{\feeddd{\varphi}{X}{X}}}\bigg)\circ\\
&\circ\bigg(\big(\ol{\delta_\varphi}\big)^{-1}\otimes\ol{\feedda{\varphi}{X}{Y}}\bigg)\circ\bigg(\id_{\ol{P_{ZZ}}}\otimes\id_{\ol{B_0}}\otimes\id_{\ol{B_X}}\otimes\id_{\ol{B_Z}}\otimes f\bigg)\circ\bigg(\ol{\delta_\varphi}\otimes\Big(\ol{\feedad{\varphi}{X}{Y}}\Big)^{-1}\bigg)\circ\\
&\circ\bigg(\Big(\ol{\feedca{\varphi}{Y}{X}}\circ\Big(\ol{\feedad{\psi}{Y}{Z}}\Big)^{-1}\circ\Big(\id_{\ol{\feedcd{\psi}{Z}{Y}}}\otimes(\ol{\delta_\psi})^{-1}\Big)\Big)\otimes\id_{\ol{\feeddd{\varphi}{X}{X}}}\bigg)\circ\bigg(\id_{\ol{\feedcd{\psi}{Z}{Y}}}\otimes\gamma_{\ol{\feeddd{\varphi}{X}{X}},\ol{P_{XX}}\otimes\ol{A_0}\otimes\ol{A_X}\otimes\ol{A_Z}}\bigg)\circ\\
&\circ\bigg(\ol{\omega_1}\otimes\id_{\ol{A_0}}\otimes\id_{\ol{A_X}}\otimes\id_{\ol{A_Z}}\bigg)\\
\end{align*}


The following claim contains the main technical details of the proof.

\begin{claim}\mbox{}
 \begin{itemize}
  \item we have bijections $\sigma_Z:P_{ZZ}\sqcup A_Z\to P_{ZZ}\sqcup B_Z$ and $\sigma_Z':P_{ZZ}\sqcup B_Z\to P_{ZZ}\sqcup A_Z$ for which
  \[\Trace_{\ol{P_{ZZ}},\ol{P_{ZZ}}}^{A_Z}\Big[\sigma_Z'\circ\big(\id_{\ol{P_{ZZ}}}\otimes\id_{B_Z}\big)\circ\sigma_Z\Big]=\id_{P_{ZZ}}\]
  \item we have bijections $\sigma_X:\inp{X}\sqcup A_X\to \inp{X}\sqcup B_X$ and $\sigma_X':\outp{X}\sqcup B_X\to \outp{X}\sqcup A_X$ for which
  \[\Trace_{\ol{\inp{X}},\ol{\outp{X}}}^{A_X}\Big[\sigma_X'\circ\big(f\otimes\id_{B_X}\big)\circ\sigma_X\Big]=f\]
  \item for the bijections $\sigma_0:A_0\to B_0$ and $\sigma_0':B_0\to A_0$ we have
  \[\Trace_{I,I}^{A_0}[\sigma_0'\circ\id_{B_0}\circ\sigma_0]=\dim(\ol{``newloops"})\]
 \end{itemize}
\end{claim}

Using this we may complete the proof.  Indeed, by the second vanishing axiom and the superposing axiom \eqref{eq:internal trace} becomes
\begin{align*}
&\Trace^{\ol{A_0}\otimes\ol{A_X}\otimes\ol{A_Z}}_{\ol{P_{ZZ}}\otimes\ol{\inp{X}},\ol{P_{ZZ}}\otimes\ol{\outp{X}}}
\Big[\id_{\ol{P_{ZZ}}}\otimes\id_{\ol{B_0}}\otimes\id_{\ol{B_X}}\otimes\id_{\ol{B_Z}}\otimes f\Big]\\
&=\Trace^{\ol{A_0}\otimes\ol{A_X}}_{\ol{P_{ZZ}}\otimes\ol{\inp{X}},\ol{P_{ZZ}}\otimes\ol{\outp{X}}}\bigg[
\Trace^{\ol{A_Z}}_{\ol{P_{ZZ}}\otimes\ol{\inp{X}}\otimes\ol{A_0}\otimes\ol{A_X},\ol{P_{ZZ}}\otimes\ol{\outp{X}}\otimes\ol{A_0}\otimes\ol{A_X}}\Big[\id_{\ol{P_{ZZ}}}\otimes\id_{\ol{B_0}}\otimes\id_{\ol{B_X}}\otimes\id_{\ol{B_Z}}\otimes f\Big]\bigg]\\
&=\Trace^{\ol{A_0}\otimes\ol{A_X}}_{\ol{P_{ZZ}}\otimes\ol{\inp{X}},\ol{P_{ZZ}}\otimes\ol{\outp{X}}}\bigg[
\id_{\ol{B_0}}\otimes\id_{\ol{B_X}}\otimes f\otimes\Trace^{\ol{A_Z}}_{\ol{P_{ZZ}},\ol{P_{ZZ}}}\Big[\sigma_Z'\circ\big(\id_{\ol{P_{ZZ}}}\otimes\id_{B_Z}\big)\circ\sigma_Z\Big]\bigg]\\
&=\Trace^{\ol{A_0}\otimes\ol{A_X}}_{\ol{P_{ZZ}}\otimes\ol{\inp{X}},\ol{P_{ZZ}}\otimes\ol{\outp{X}}}\Big[\id_{\ol{B_0}}\otimes\id_{\ol{B_X}}\otimes f\otimes\id_{\ol{P_{ZZ}}}\Big],
\end{align*}
where the last equality is contained in the claim above.  Then via a symmetry, the superposing axiom, the second vanishing axiom, and again the superposing axiom this becomes
\begin{align*}
&\id_{\ol{P_{ZZ}}}\otimes\Trace^{\ol{A_0}\otimes\ol{A_X}}_{\ol{\inp{X}},\ol{\outp{X}}}\Big[
\id_{\ol{B_0}}\otimes\id_{\ol{B_X}}\otimes f\Big]\\
&=\id_{\ol{P_{ZZ}}}\otimes\Trace^{\ol{A_0}}_{\ol{\inp{X}},\ol{\outp{X}}}\bigg[
\Trace^{\ol{A_X}}_{\ol{\inp{X}}\otimes\ol{A_0},\ol{\outp{X}}\otimes\ol{A_0}}\Big[\id_{\ol{B_0}}\otimes\id_{\ol{B_X}}\otimes f\Big]\bigg]\\
&=\id_{\ol{P_{ZZ}}}\otimes\Trace^{\ol{A_0}}_{\ol{\inp{X}},\ol{\outp{X}}}\bigg[
\id_{\ol{B_0}}\otimes\Trace^{\ol{A_X}}_{\ol{\inp{X}},\ol{\outp{X}}}\Big[\sigma_X'\circ\big(f\otimes\id_{B_X}\big)\circ\sigma_X\Big]\bigg]\\
&=\id_{\ol{P_{ZZ}}}\otimes\Trace^{\ol{A_0}}_{\ol{\inp{X}},\ol{\outp{X}}}\Big[
\id_{\ol{B_0}}\otimes f\Big],
\end{align*}
where the last equality is contained in the claim above.  Finally by a symmetry, the superposing axiom, and the claim above this becomes
\begin{align*}
&\id_{\ol{P_{ZZ}}}\otimes f\otimes\Trace^{\ol{A_0}}_{I,I}\Big[
\sigma_0'\circ\id_{B_0}\circ\sigma_0\Big]=\id_{\ol{P_{ZZ}}}\otimes f\otimes\dim(\ol{``newloops"}).
\end{align*}

Thus combining these observations with \eqref{eq:trace1} we get
\begin{align*}
 &\Trace^{\ol{\feeddd{\psi}{Y}{Y}}}_{\ol{\feedcd{\psi}{Z}{Y}}\otimes\ol{\feeddd{\varphi}{X}{X}},\ol{\feeddc{\psi}{Y}{Z}}\otimes\ol{\feeddd{\varphi}{X}{X}}}
\Big[
\big(\id_{\ol{\feeddc{\psi}{Y}{Z}}}\otimes\gamma_{\ol{\feeddd{\psi}{Y}{Y}},\ol{\feeddd{\varphi}{X}{X}}}\big)\circ\big(\id_{\ol{\feedcc{\varphi}{Y}{Y}}}\otimes f\big)\circ\big(\id_{\ol{\feedcd{\psi}{Z}{Y}}}\otimes\gamma_{\ol{\feeddd{\varphi}{X}{X}},\ol{\feeddd{\psi}{Y}{Y}}}\big)
\Big]\\
 &=\Trace^{\ol{P_{XX}}}_{\ol{\feedcd{\psi}{Z}{Y}}\otimes\ol{\feeddd{\varphi}{X}{X}},\ol{\feeddc{\psi}{Y}{Z}}\otimes\ol{\feeddd{\varphi}{X}{X}}}
 \Big[\id_{\ol{P_{ZZ}}}\otimes f\otimes\dim(\ol{``newloops"})\Big]\\
  &=\Trace^{\bigotimes_{i\ge1}\ol{\feeddd{\psi\;\accentset{i}{\circ}\;\varphi}{X}{X}}}_{\ol{\feedcd{\psi}{Z}{Y}}\otimes\ol{\feeddd{\varphi}{X}{X}},\ol{\feeddc{\psi}{Y}{Z}}\otimes\ol{\feeddd{\varphi}{X}{X}}}\bigg[\id_{\bigotimes_{i\ge1}\ol{\feedcc{\psi\;\accentset{i}{\circ}\;\varphi}{Z}{Z}}}\otimes f\bigg]\otimes\dim(\ol{``newloops"})
 \end{align*}
 as desired.
 
 \newpage

In order to establish the claim we introduce some more convenient notation.  We have decompositions
\[\inp{Y}=\bigsqcup_{i\ge0}\bigsqcup_{j\ge0}(\psi\;\accentset{\,i\ j}{\circ}\;\varphi)\sqcup(\psi\;\accentset{\infty}{\circ}\;\varphi)\hsp\tn{and}\hsp \outp{Y}=\bigsqcup_{i\ge0}\bigsqcup_{j\ge0}(\psi\;\underaccent{\,\ i\ j}{\circ}\;\varphi)\sqcup(\psi\,\underaccent{\;\ \infty}{\circ}\,\varphi)\]
where $\psi\;\accentset{\infty}{\circ}\;\varphi$ and $\psi\,\underaccent{\ \;\infty}{\circ}\,\varphi$ represent points on the border of $Y$ which are contained in loops of $\psi\circ\varphi$ and
\[\psi\;\accentset{\,i\ j}{\circ}\;\varphi=\begin{cases}\feeddc{\psi\;\accentset{i+j+1}{\circ}\;\varphi}{X}{Z} & \text{if $i$ is odd, $j$ is odd;}\\\feeddd{\psi\;\accentset{i+j+1}{\circ}\;\varphi}{X}{X} & \text{if $i$ is odd, $j$ is even;}\\\feedcd{\psi\;\accentset{i+j+1}{\circ}\;\varphi}{Z}{X} & \text{if $i$ is even, $j$ is even;}\\\feedcc{\psi\;\accentset{i+j+1}{\circ}\;\varphi}{Z}{Z} & \text{if $i$ is even, $j$ is odd;}\end{cases}
\quad\tn{and}\quad
\psi\;\underaccent{i\ j}{\circ}\;\varphi=\begin{cases}\feedcd{\psi\;\accentset{i+j+1}{\circ}\;\varphi}{Z}{X} & \text{if $i$ is odd, $j$ is odd;}\\\feedcc{\psi\;\accentset{i+j+1}{\circ}\;\varphi}{Z}{Z} & \text{if $i$ is odd, $j$ is even;}\\\feeddc{\psi\;\accentset{i+j+1}{\circ}\;\varphi}{X}{Z} & \text{if $i$ is even, $j$ is even;}\\\feeddd{\psi\;\accentset{i+j+1}{\circ}\;\varphi}{X}{X} & \text{if $i$ is even, $j$ is odd;}\end{cases}\]
that is, $\ \accentset{\,i\ j}{\circ}\ $ (or $\ \underaccent{\,i\ j}{\circ}\,$ ) indicates those points on the border of $Y$ for which there is a wire of $\psi\circ\phi$ which has crossed the border of $Y$ exactly $i$ times before reaching the current point on the border and crosses the border of $Y$ exactly $j$ times after leaving the current point on the border.

In this notation we may identify 
\begin{align*}
 \feedcc{\varphi}{Y}{Y}
 &=\bigsqcup_{i\ge1}\bigsqcup_{j\ge0}(\psi\;\underaccent{\,\ i\ j}{\circ}\;\varphi)\sqcup(\psi\,\underaccent{\;\ \infty}{\circ}\,\varphi)
 \quad\tn{and}\quad
 \feeddc{\varphi}{X}{Y}=\bigsqcup_{j\ge0}(\psi\;\underaccent{\,\ 0\ j}{\circ}\;\varphi)
 \quad\tn{and}\quad
 \feedcd{\varphi}{Y}{X}=\bigsqcup_{i\ge0}(\psi\;\accentset{\,i\ 0}{\circ}\;\varphi)\\
 \feeddd{\psi}{Y}{Y}
 &=\bigsqcup_{i\ge1}\bigsqcup_{j\ge0}(\psi\;\accentset{\,i\ j}{\circ}\;\varphi)\sqcup(\psi\;\accentset{\infty}{\circ}\;\varphi)
 \quad\quad\!\!\tn{and}\quad
 \feedcd{\psi}{Z}{Y}=\bigsqcup_{j\ge0}(\psi\;\accentset{\,0\ j}{\circ}\;\varphi)
 \ \quad\tn{and}\quad
 \feeddc{\psi}{Y}{Z}=\bigsqcup_{i\ge0}(\psi\;\underaccent{\,\ i\ 0}{\circ}\;\varphi)
\end{align*}
where in general we have the following canonical ``feed-forward identifications" whenever $j>0$:
\[\psi\;\accentset{\,i\ j}{\circ}\;\varphi=\psi\;\underaccent{\,\,\ i+1\ j-1}{\circ}\;\varphi\hsp\tn{and}\hsp\psi\;\underaccent{\,\ i\ j}{\circ}\;\varphi=\psi\;\accentset{\,i+1\ j-1}{\circ}\;\varphi.\]
It will be convenient to isolate the following subsets:
\[\psi\;\underaccent{\,\ \bfe\ \bfe}{\circ}\;\varphi=\bigsqcup_{k\ge1}\bigsqcup_{\ell\ge0}(\psi\;\underaccent{\,\ 2k\ 2\ell}{\circ}\;\varphi)\hsp\tn{and}\hsp\psi\;\accentset{\,\ \bfe\ \bfe}{\circ}\;\varphi=\bigsqcup_{k\ge1}\bigsqcup_{\ell\ge0}(\psi\;\accentset{\,\ 2k\ 2\ell}{\circ}\;\varphi)\]
\[\psi\;\underaccent{\,\ \bfe\ \bfo}{\circ}\;\varphi=\bigsqcup_{k\ge1}\bigsqcup_{\ell\ge0}(\psi\;\underaccent{\,\ 2k\ 2\ell+1}{\circ}\;\varphi)\hsp\tn{and}\hsp\psi\;\accentset{\,\ \bfe\ \bfo}{\circ}\;\varphi=\bigsqcup_{k\ge1}\bigsqcup_{\ell\ge0}(\psi\;\accentset{\,\ 2k\ 2\ell+1}{\circ}\;\varphi)\]
\[\psi\;\underaccent{\,\ \bfo\ \bfo}{\circ}\;\varphi=\bigsqcup_{k\ge0}\bigsqcup_{\ell\ge0}(\psi\;\underaccent{\,\ 2k+1\ 2\ell+1}{\circ}\;\varphi)\hsp\tn{and}\hsp\psi\;\accentset{\,\ \bfo\ \bfo}{\circ}\;\varphi=\bigsqcup_{k\ge0}\bigsqcup_{\ell\ge0}(\psi\;\accentset{\,\ 2k+1\ 2\ell+1}{\circ}\;\varphi)\]
\[\psi\;\underaccent{\,\ \bfo\ \bfe}{\circ}\;\varphi=\bigsqcup_{k\ge0}\bigsqcup_{\ell\ge0}(\psi\;\underaccent{\,\ 2k+1\ 2\ell}{\circ}\;\varphi)\hsp\tn{and}\hsp\psi\;\accentset{\,\ \bfo\ \bfe}{\circ}\;\varphi=\bigsqcup_{k\ge0}\bigsqcup_{\ell\ge0}(\psi\;\accentset{\,\ 2k+1\ 2\ell}{\circ}\;\varphi).\]

\begin{lemma}
 The trace
 \[\Trace^{\ol{\feeddd{\psi}{Y}{Y}}}_{\ol{\feedcd{\psi}{Z}{Y}}\otimes\ol{\feeddd{\varphi}{X}{X}},\ol{\feeddc{\psi}{Y}{Z}}\otimes\ol{\feeddd{\varphi}{X}{X}}}
\big(
(\id_{\ol{\feeddc{\psi}{Y}{Z}}}\otimes\gamma_{\ol{\feeddd{\psi}{Y}{Y}},\ol{\feeddd{\varphi}{X}{X}}})\circ(\id_{\ol{\feedcc{\varphi}{Y}{Y}}}\otimes f)\circ(\id_{\ol{\feedcd{\psi}{Z}{Y}}}\otimes\gamma_{\ol{\feeddd{\varphi}{X}{X}},\ol{\feeddd{\psi}{Y}{Y}}})
\big)\]
simplifies to
\[\Trace^{\bigotimes_{k\ge1}\ol{\feeddd{\psi\;\accentset{2k}{\circ}\;\varphi}{X}{X}}
}_{\ol{\feedcd{\psi}{Z}{Y}}\otimes\ol{\feeddd{\varphi}{X}{X}},\ol{\feeddc{\psi}{Y}{Z}}\otimes\ol{\feeddd{\varphi}{X}{X}}}
\bigg(
\id_{\bigotimes_{\ell\ge1}\ol{\feedcc{\psi\;\accentset{2\ell}{\circ}\;\varphi}{Z}{Z}}
}\otimes f
\bigg)\otimes\dim(\ol{``newloops"})\]
\end{lemma}
\begin{proof} 
 We begin by making substitutions according to the notation just introduced:
 \begin{align*}
  &\Trace^{\ol{\feeddd{\psi}{Y}{Y}}}_{\ol{\feedcd{\psi}{Z}{Y}}\otimes\ol{\feeddd{\varphi}{X}{X}},\ol{\feeddc{\psi}{Y}{Z}}\otimes\ol{\feeddd{\varphi}{X}{X}}}
\bigg(\id_{\ol{\feedcc{\varphi}{Y}{Y}}}\otimes f\bigg)\\
  &\quad=\Trace^{\bigotimes_{\substack{\scriptscriptstyle i\ge1\\ \scriptscriptstyle j\ge0}}\ol{\psi\;\accentset{\,i\ j}{\circ}\;\varphi}\otimes\ol{\psi\;\accentset{\infty}{\circ}\;\varphi}}_{\bigotimes_{j\ge0}\ol{\psi\;\accentset{\,0\ j}{\circ}\;\varphi}\otimes\ol{\feeddd{\varphi}{X}{X}},\bigotimes_{i\ge0}\ol{\psi\;\underaccent{\,i\ 0}{\circ}\;\varphi}\otimes\ol{\feeddd{\varphi}{X}{X}}}
\Bigg(\id_{\ol{\psi\,\underaccent{\,\infty}{\circ}\,\varphi}}\otimes\id_{\bigotimes_{\substack{\scriptscriptstyle i\ge1\\ \scriptscriptstyle j\ge0}}\ol{\psi\;\underaccent{\,i\ j}{\circ}\;\varphi}}\otimes f\Bigg).
\end{align*}

Applying a symmetry we may separate into paths with an even and odd length to their ends to get
 \begin{align*}
  \Trace^{\ol{\psi\;\accentset{\infty}{\circ}\;\varphi}\otimes\ol{\psi\;\accentset{\ \bfe\ \bfe}{\circ}\;\varphi}\otimes\ol{\psi\;\accentset{\ \bfe\ \bfo}{\circ}\;\varphi}\otimes\ol{\psi\;\accentset{\ \bfo\ \bfe}{\circ}\;\varphi}\otimes\ol{\psi\;\accentset{\ \bfo\ \bfo}{\circ}\;\varphi}}_{\bigotimes_{j\ge0}\ol{\psi\;\accentset{\,0\ j}{\circ}\;\varphi}\otimes\ol{\feeddd{\varphi}{X}{X}},\bigotimes_{i\ge0}\ol{\psi\;\underaccent{\,i\ 0}{\circ}\;\varphi}\otimes\ol{\feeddd{\varphi}{X}{X}}}
\Bigg(\id_{\ol{\psi\,\underaccent{\,\infty}{\circ}\,\varphi}}\otimes\id_{\ol{\psi\;\underaccent{\,\ \bfe\ \bfe}{\circ}\;\varphi}}\otimes\id_{\ol{\psi\;\underaccent{\,\ \bfe\ \bfo}{\circ}\;\varphi}}\otimes\id_{\ol{\psi\;\underaccent{\,\ \bfo\ \bfe}{\circ}\;\varphi}}\otimes\id_{\ol{\psi\;\underaccent{\,\ \bfo\ \bfo}{\circ}\;\varphi}}\otimes f\Bigg).
\end{align*}

Feeding-forward (or backwards) all terms to their ends this becomes
 \begin{align*}
  \Trace^{\ol{\psi\;\accentset{\infty}{\circ}\;\varphi}\otimes\bigotimes_{\substack{\scriptscriptstyle i\ge1\\ \scriptscriptstyle k\ge0}}\ol{\psi\;\accentset{i+2k\ 0\ \ \ \,}{\circ}\;\varphi}\otimes\bigotimes_{\substack{\scriptscriptstyle i\ge1\\ \scriptscriptstyle k\ge0}}\quad\ol{\psi\;\underaccent{i+2k+1\ 0\ \ \ \ }{\circ}\;\varphi}}_{\bigotimes_{j\ge0}\ol{\psi\;\accentset{\,0\ j}{\circ}\;\varphi}\otimes\ol{\feeddd{\varphi}{X}{X}},\bigotimes_{i\ge0}\ol{\psi\;\underaccent{\,i\ 0}{\circ}\;\varphi}\otimes\ol{\feeddd{\varphi}{X}{X}}}
\Bigg(\id_{\ol{\psi\,\underaccent{\,\infty}{\circ}\,\varphi}}\otimes\id_{\bigotimes_{\substack{\scriptscriptstyle i\ge1\\ \scriptscriptstyle \ell\ge0}}\ol{\psi\;\underaccent{i+2\ell\ 0}{\circ}\;\varphi}}\otimes \id_{\bigotimes_{\substack{\scriptscriptstyle i\ge1\\ \scriptscriptstyle \ell\ge0}}\quad\ol{\psi\;\accentset{i+2\ell+1\ 0\ \ \ }{\circ}\;\varphi}}\otimes f\Bigg).
\end{align*}

Separating terms now gives
 \begin{align*}
  \Trace^{\ol{\psi\;\accentset{\infty}{\circ}\;\varphi}\otimes\bigotimes_{k\ge0}\quad\ol{\psi\;\accentset{2k+1\ 0\ \ \ }{\circ}\;\varphi}\otimes\bigotimes_{\substack{\scriptscriptstyle i\ge2\\ \scriptscriptstyle k\ge0}}\ol{\psi\;\accentset{i+2k\ 0}{\circ}\;\varphi}\otimes\bigotimes_{\substack{\scriptscriptstyle i\ge1\\ \scriptscriptstyle k\ge0}}\quad\ol{\psi\;\underaccent{i+2k+1\ 0\ \ \ }{\circ}\;\varphi}}_{\bigotimes_{j\ge0}\ol{\psi\;\accentset{\,0\ j}{\circ}\;\varphi}\otimes\ol{\feeddd{\varphi}{X}{X}},\bigotimes_{i\ge0}\ol{\psi\;\underaccent{\,i\ 0}{\circ}\;\varphi}\otimes\ol{\feeddd{\varphi}{X}{X}}}
\Bigg(\id_{\ol{\psi\,\underaccent{\,\infty}{\circ}\,\varphi}}\otimes\id_{\bigotimes_{\ell\ge0}\quad\ol{\psi\;\underaccent{2\ell+1\ 0\ \ \ }{\circ}\;\varphi}}\otimes \id_{\bigotimes_{\substack{\scriptscriptstyle i\ge2\\ \scriptscriptstyle \ell\ge0}}\ol{\psi\;\underaccent{i+2\ell\ 0}{\circ}\;\varphi}}\otimes \id_{\bigotimes_{\substack{\scriptscriptstyle i\ge1\\ \scriptscriptstyle \ell\ge0}}\quad\ol{\psi\;\accentset{i+2\ell+1\ 0\ \ \ }{\circ}\;\varphi}}\otimes f\Bigg)
\end{align*}
which via a symmetry is equivalent to
 \begin{align*}
  \Trace^{\ol{\psi\;\accentset{\infty}{\circ}\;\varphi}\otimes\bigotimes_{k\ge0}\quad\ol{\psi\;\accentset{2k+1\ 0\ \ \ }{\circ}\;\varphi}\otimes\bigotimes_{\substack{\scriptscriptstyle i\ge2\\ \scriptscriptstyle k\ge0}}\ol{\psi\;\accentset{i+2k\ 0}{\circ}\;\varphi}\otimes\bigotimes_{\substack{\scriptscriptstyle i\ge1\\ \scriptscriptstyle k\ge0}}\quad\ol{\psi\;\underaccent{i+2k+1\ 0\ \ \ }{\circ}\;\varphi}}_{\bigotimes_{j\ge0}\ol{\psi\;\accentset{\,0\ j}{\circ}\;\varphi}\otimes\ol{\feeddd{\varphi}{X}{X}},\bigotimes_{i\ge0}\ol{\psi\;\underaccent{\,i\ 0}{\circ}\;\varphi}\otimes\ol{\feeddd{\varphi}{X}{X}}}
\Bigg(\id_{\ol{\psi\,\underaccent{\,\infty}{\circ}\,\varphi}}\otimes\id_{\bigotimes_{\ell\ge0}\quad\ol{\psi\;\underaccent{2\ell+1\ 0\ \ \ }{\circ}\;\varphi}}\otimes f\otimes \id_{\bigotimes_{\substack{\scriptscriptstyle i\ge1\\ \scriptscriptstyle \ell\ge0}}\quad\ol{\psi\;\accentset{i+2\ell+1\ 0\ \ \ }{\circ}\;\varphi}}\otimes \id_{\bigotimes_{\substack{\scriptscriptstyle i\ge2\\ \scriptscriptstyle \ell\ge0}}\ol{\psi\;\underaccent{i+2\ell\ 0}{\circ}\;\varphi}}\Bigg).
\end{align*}

Now by the second vanishing axiom we may replace this by
\begin{align*}
&\Trace^{\ol{\psi\;\accentset{\infty}{\circ}\;\varphi}\otimes\bigotimes_{k\ge0}\quad\ol{\psi\;\accentset{2k+1\ 0\ \ \ }{\circ}\;\varphi}}_{\bigotimes_{j\ge0}\ol{\psi\;\accentset{\,0\ j}{\circ}\;\varphi}\otimes\ol{\feeddd{\varphi}{X}{X}},\bigotimes_{i\ge0}\ol{\psi\;\underaccent{\,i\ 0}{\circ}\;\varphi}\otimes\ol{\feeddd{\varphi}{X}{X}}}
\Bigg(\\
&\hspace{1in}\Trace^{\bigotimes_{\substack{\scriptscriptstyle i\ge2\\ \scriptscriptstyle k\ge0}}\ol{\psi\;\accentset{i+2k\ 0}{\circ}\;\varphi}\otimes\bigotimes_{\substack{\scriptscriptstyle i\ge1\\ \scriptscriptstyle k\ge0}}\quad\ol{\psi\;\underaccent{i+2k+1\ 0\ \ \ }{\circ}\;\varphi}}_{\bigotimes_{j\ge0}\ol{\psi\;\accentset{\,0\ j}{\circ}\;\varphi}\otimes\ol{\feeddd{\varphi}{X}{X}}\otimes\ol{\psi\;\accentset{\infty}{\circ}\;\varphi}\otimes\bigotimes_{k\ge0}\quad\ol{\psi\;\accentset{2k+1\ 0\ \ \ }{\circ}\;\varphi},\bigotimes_{i\ge0}\ol{\psi\;\underaccent{\,i\ 0}{\circ}\;\varphi}\otimes\ol{\feeddd{\varphi}{X}{X}}\otimes\ol{\psi\;\accentset{\infty}{\circ}\;\varphi}\otimes\bigotimes_{k\ge0}\quad\ol{\psi\;\accentset{2k+1\ 0\ \ \ }{\circ}\;\varphi}}
\Big(F\Big)\Bigg)
\end{align*}
where we abbreviate
\[F=\id_{\ol{\psi\,\underaccent{\,\infty}{\circ}\,\varphi}}\otimes\id_{\bigotimes_{\ell\ge0}\quad\ol{\psi\;\underaccent{2\ell+1\ 0\ \ \ }{\circ}\;\varphi}}\otimes f\otimes \id_{\bigotimes_{\substack{\scriptscriptstyle i\ge1\\ \scriptscriptstyle \ell\ge0}}\quad\ol{\psi\;\accentset{i+2\ell+1\ 0\ \ \ }{\circ}\;\varphi}}\otimes \id_{\bigotimes_{\substack{\scriptscriptstyle i\ge2\\ \scriptscriptstyle \ell\ge0}}\ol{\psi\;\underaccent{i+2\ell\ 0}{\circ}\;\varphi}}.\]

Superposing then gives
\[\]

Finally tracing out (yanking) common terms gives 
\[\Trace^{\bigotimes_{k\ge0}\ol{\psi\;\accentset{2k+1\ 0\ \ \ \,}{\circ}\;\varphi}}_{\ol{\feedcd{\psi}{Z}{Y}}\otimes\ol{\feeddd{\varphi}{X}{X}},\ol{\feeddc{\psi}{Y}{Z}}\otimes\ol{\feeddd{\varphi}{X}{X}}}
\bigg(
\id_{\bigotimes_{\ell\ge0}\quad\ol{\psi\;\underaccent{2\ell+1\ 0\ \ \ }{\circ}\;\varphi}}\otimes f
\bigg)\otimes\dim(\ol{``newloops"})\]
which identifies with the result.
\end{proof}

Combining the results of the preceding lemmas we get
\begin{align*}
&\mcP(\Psi)\circ\mcP(\Phi)(f)\\
&=
\Trace^{\ol{\feeddd{\psi}{Y}{Y}}}_{\ol{Z_-},\ol{Z_+}}
\Big(
\id_{\ol{\feedcc{\psi}{Z}{Z}}}\otimes
\Trace^{\ol{\feeddd{\varphi}{X}{X}}}_{\ol{Y_-},\ol{Y_+}}(\id_{\ol{\feedcc{\varphi}{Y}{Y}}}\otimes f)\otimes\dim(\ol{S})
\Big)
\otimes\dim(\ol{T})\\
&=\Trace^{\ol{\feeddd{\varphi}{X}{X}}}_{\ol{\inp{Z}},\ol{\outp{Z}}}
\Bigg(
\id_{\ol{\feedcc{\psi}{Z}{Z}}}\otimes\Trace^{\bigotimes_{k\ge1}\ol{\feeddd{\psi\;\accentset{2k}{\circ}\;\varphi}{X}{X}}
}_{\ol{\feedcd{\psi}{Z}{Y}}\otimes\ol{\feeddd{\varphi}{X}{X}},\ol{\feeddc{\psi}{Y}{Z}}\otimes\ol{\feeddd{\varphi}{X}{X}}}
\bigg(
\id_{\bigotimes_{\ell\ge1}\ol{\feedcc{\psi\;\accentset{2\ell}{\circ}\;\varphi}{Z}{Z}}
}\otimes f
\bigg)
\Bigg)\otimes\dim(\ol{S\sqcup T\sqcup ``newloops"})\\
&=\Trace^{\ol{\feeddd{\psi\circ\varphi}{X}{X}}}_{\ol{Z_-},\ol{Z_+}}(\id_{\ol{\feedcc{\psi\circ\varphi}{Z}{Z}}}\otimes f)\otimes\dim(\ol{S\sqcup T\sqcup ``newloops"})\\
&=\mcP(\Psi\circ\Phi)(f)
\end{align*}
as desired.
\end{proof}

\begin{theorem}

The adjunction of Theorem \ref{thm:cobalg trace adjunction} is monadic. 

\end{theorem}

Every category has an underlying reflexive graph, which includes its objects and morphisms (including identities), but not its composition formula. By {\em symmetric monoidal reflexive graph}, we mean a reflexive graph $\mcG$ together with a vertex $I\in \mcG$ and a graph homomorphism $\otimes\taking\mcG\times\mcG\to\mcG$ that is commutative, associative, and unital with respect to $I$, all up to coherent isomorphisms. By {\em traced symmetric monoidal reflexive graph} we mean a symmetric monoidal reflexive graph with a family of trace functions $Tr$ that satisfy superposition, vanishing, and yanking (all the properties of traced SMCs that don't refer to the composition formula).

\begin{conjecture}
The category of traced symmetric monoidal categories is equivalent to the category of traced symmetric monoidal relfexive graphs.
\end{conjecture}

\begin{conjecture}
Let $\mcM$ be a symmetric monoidal category and let $(\Cob/\bullet)\alg_\mcM$ be the category of algebras in $\mcM$ (e.g. whose objects are pairs $(\mcO,\mcP)$, where $\mcO$ is a set and $\mcP\taking\Cob/\mcO\to\mcM$ is a lax symmetric monoidal functor). Then there is a monadic adjunction 
$$\xymatrix{L\taking(\Cob/\bullet)\alg_\mcM\ar@<.5ex>[r]&\mcM-\TSMC:\!R,\ar@<.5ex>[l]}$$
where $\mcM-\TSMC$ is the category of traced SMCs enriched in $\mcM$.
\end{conjecture}

\section{Yanking Compositions of Wiring Diagrams}
We have some technique ,``yanking", by which we trade $\phi,\psi$ in for $\phi',\psi'$ such that there's an equivalence and two nice facts:
$$\xymatrix@R=10pt{
&Y\ar[dr]^\psi\ar@{}[dd]|{\Bigg\Updownarrow}\\
X\ar[ur]^\phi\ar[dr]_{\phi'}&&Z\\
&Y'\ar[ur]_{\psi'}
}
\hsp\tn{where}
\hspace{.2in}
\xymatrix{
\emptyset\ar[r]\ar[d]\ullimit&\feedcc{\phi'}{Y}{Y}\ar[d]\\
\feeddd{\psi'}{Y}{Y}\ar[r]&\inp{Y'}
}
\hspace{.7in}
\xymatrix{
\emptyset\ar[r]\ar[d]\ullimit&\feedcc{\phi'}{Y}{Y}\ar[d]\\
\feeddd{\psi'}{Y}{Y}\ar[r]&\outp{Y'}
}
$$

Then we have $\feeddd{\psi'\circ\phi'}{X}{X}=\feeddd{\phi'}{X}{X}+\feeddd{\psi'}{Y}{Y}$ and $\feedcc{\psi'\circ\phi'}{Z}{Z}=\feedcc{\phi'}{Y}{Y}+\feedcc{\psi'}{Z}{Z}$.  And so compatibility of trace and compositions is immediate by comparing equation \eqref{eq:composition 2} and Lemma~\ref{le:combining traces}.

\section{Old stuff}

Suppose given a typed bijection $\phi\taking\inp{X}\sqcup \outp{Y}\To{\iso}\outp{X}\sqcup \inp{Y}$. We'll define $\feeddd{\varphi}{X}{X}, \feeddc{\varphi}{X}{Y}, \feedcd{\varphi}{Y}{X},$ and $\feedcc{\varphi}{Y}{Y}$ below; roughly they are ``$X$'s feed $X$'s", ``$X$'s feed $Y$'s", ``$Y$'s feed $X$'s", and ``$Y$'s feed $Y$'s", respectively. They are given by the following fiber products in $\TFS$.
$$
\xymatrix@=35pt{
\feeddd{\varphi}{X}{X}\ar[r]\ar[d]\ullimit&\inp{X}\ar[d]^{\phi\big|_{\inp{X}}}&\feedcd{\varphi}{Y}{X}\ar[l]\ar[d]\urlimit\\
\outp{X}\ar[r]&\outp{X}\sqcup \inp{Y}&\inp{Y}\ar[l]\\
\feeddc{\varphi}{X}{Y}\ar[r]\ar[u]\lllimit&\outp{Y}\ar[u]_{\phi\big|_{\outp{Y}}}&\feedcc{\varphi}{Y}{Y}\ar[l]\ar[u]\lrlimit
}
$$
Note that every arrow above is an injection and we have the following four bijections, 
\begin{align*}
\feeddd{\varphi}{X}{X}\sqcup \feeddc{\varphi}{X}{Y}\iso\outp{X},
&\hsp
\feedcd{\varphi}{Y}{X}\sqcup \feedcc{\varphi}{Y}{Y}\iso\inp{Y},
\\
\feeddd{\varphi}{X}{X}\sqcup \feedcd{\varphi}{Y}{X}\iso\inp{X},
&\hsp
\feeddc{\varphi}{X}{Y}\sqcup \feedcc{\varphi}{Y}{Y}\iso\outp{Y}.
\end{align*}

\section{General Lemmas on TSMC's}

\begin{definition}
 Let $f:A'\to B'$, $a:A'\To{\sim} A\otimes U$, $b:B'\To{\sim} B\otimes U$.  Then we write $\Trace_{a,b}^U\big[f\big]:=\Trace_{A,B}^U\big[b\circ f\circ a^{-1}\big]$.
\end{definition}
\begin{lemma}
 Let $f:A'\to B'$, $a:A'\To{\sim} A\otimes U$, $b:B'\To{\sim} B\otimes U$, $c:A\To{\sim} C\otimes V$, $d:B\To{\sim} D\otimes V$.  Then we have
 \begin{equation}
  \Trace_{c,d}^V\Big[\Trace_{a,b}^U\big[f\big]\Big]=\Trace_{C,D}^U\Big[\Trace_{(C\otimes\gamma_{V,U})\circ(c\otimes U)\circ a,(D\otimes\gamma_{V,U})\circ(d\otimes U)\circ b}^V\big[f\big]\Big].
 \end{equation}
\end{lemma}
\begin{proof}
 By naturality we may rewrite $\Trace_{c,d}^V\Big[\Trace_{a,b}^U\big[f\big]\Big]$ as
 \[\Trace_{C,D}^V\Big[\Trace_{C\otimes V,D\otimes V}^U\big[(d\otimes U)\circ b\circ f\circ a^{-1}\circ (c^{-1}\otimes U)\big]\Big]\]
 which by the second vanishing axiom is equivalent to
 \[\Trace_{C,D}^{V\otimes U}\Big[(d\otimes U)\circ b\circ f\circ a^{-1}\circ (c^{-1}\otimes U)\Big].\]
 By dinaturality on either side we may write this as
 \[\Trace_{C,D}^{U\otimes V}\Big[(D\otimes\gamma_{V,U})\circ(d\otimes U)\circ b\circ f\circ a^{-1}\circ (c^{-1}\otimes U)\circ(C\otimes\gamma_{U,V})\Big]\]
 which by the second vanishing axiom is equivalent to
 \[\Trace_{C,D}^U\Big[\Trace_{C\otimes U,D\otimes U}^V\big[(D\otimes\gamma_{V,U})\circ(d\otimes U)\circ b\circ f\circ a^{-1}\circ (c^{-1}\otimes U)\circ(C\otimes\gamma_{U,V})\big]\Big]\]
 and noting that $\gamma_{U,V}^{-1}=\gamma_{V,U}$ completes the proof.
\end{proof}

\end{document}