\documentclass{amsart}

\usepackage{savesym}
\usepackage{txfonts,stmaryrd}
\savesymbol{lrcorner}\savesymbol{ulcorner}
\usepackage{amssymb, amscd,setspace,mathtools,makecell}
\usepackage{enumerate}
\usepackage[usenames,dvipsnames]{xcolor}
\usepackage[bookmarks=true,colorlinks=true, linkcolor=MidnightBlue, citecolor=cyan]{hyperref}
\usepackage{lmodern}
\usepackage{graphicx,float}
\restoresymbol{txfonts}{lrcorner}\restoresymbol{txfonts}{ulcorner}
\usepackage{tensor}

\usepackage{tikz}
\usetikzlibrary{arrows,calc,chains,matrix,positioning,scopes,snakes}

%Begin tikz macros
\def\blackbox#1#2#3#4#5{%(width,height), number inputs, number outputs, label, arrow length
  \pgfgetlastxy{\llx}{\lly}%assumes path has been set to a point representing the lower left corner of the box
  \path #1;
  \pgfgetlastxy{\w}{\h}
  \pgfmathsetlengthmacro{\urx}{\llx+\w}
  \pgfmathsetlengthmacro{\ury}{\lly+\h}
  \draw (\llx,\lly) rectangle (\urx,\ury);
  \pgfmathsetlengthmacro{\xave}{(\llx+\urx)/2}
  \pgfmathsetlengthmacro{\yave}{\ury-8}
  \node at (\xave,\yave) {#4};
  \pgfmathsetlengthmacro{\ydiff}{\ury-\lly}
  \pgfmathsetlengthmacro{\lstep}{\ydiff/(#2+1)}
  \pgfmathsetlengthmacro{\rstep}{\ydiff/(#3+1)}
  \ifnum #2=0{}\else{ 
   \foreach \l in {1,...,#2}{
    \draw [->] ($(\llx,\lly)+(-#5/2,0)+\l*(0,\lstep)$) -- ($(\llx,\lly)+(#5/2,0)+\l*(0,\lstep)$);}}\fi
  \ifnum #3=0{}\else{
   \foreach \r in {1,...,#3}{
    \draw [->] ($(\urx,\ury)+(-#5/2,0)-\r*(0,\rstep)$) -- ($(\urx,\ury)+(#5/2,0)-\r*(0,\rstep)$);}}\fi
}

\def\blackboxinners#1#2#3#4#5{%(width,height), number inputs, number outputs, label, arrow length
  \pgfgetlastxy{\llx}{\lly}%assumes path has been set to a point representing the lower left corner of the box
  \path #1;
  \pgfgetlastxy{\w}{\h}
  \pgfmathsetlengthmacro{\urx}{\llx+\w}
  \pgfmathsetlengthmacro{\ury}{\lly+\h}
  \draw (\llx,\lly) rectangle (\urx,\ury);
  \pgfmathsetlengthmacro{\xave}{(\llx+\urx)/2}
  \pgfmathsetlengthmacro{\yave}{\ury-8}
  \node at (\xave,\yave) {#4};
  \pgfmathsetlengthmacro{\ydiff}{\ury-\lly}
  \pgfmathsetlengthmacro{\lstep}{\ydiff/(#2+1)}
  \pgfmathsetlengthmacro{\rstep}{\ydiff/(#3+1)}
  \ifnum #2=0{}\else{ 
   \foreach \l in {1,...,#2}{
    \pgfmathsetlengthmacro{\newx}{\llx+#5*28.45274/2}
    \pgfmathsetlengthmacro{\newy}{\lly+\l*\lstep}
    \node at ($(\newx,\newy)+(-1.5,\l*12-\l*\lstep)$) {\tiny$(\pgfmathparse{\newx/28.45274}\pgfmathresult cm,\pgfmathparse{\newy/28.45274}\pgfmathresult cm)$};
    \draw [->] ($(\llx,\lly)+(-#5/2,0)+\l*(0,\lstep)$) -- ($(\llx,\lly)+(#5/2,0)+\l*(0,\lstep)$);}}\fi
  \ifnum #3=0{}\else{
   \foreach \r in {1,...,#3}{
    \pgfmathsetlengthmacro{\newx}{\urx-#5*28.45274/2}
    \pgfmathsetlengthmacro{\newy}{\ury-\r*\rstep}
    \node at ($(\newx,\newy)+(1.5,-\r*12+\r*\rstep)$) {\tiny $(\pgfmathparse{\newx/28.45274}\pgfmathresult cm,\pgfmathparse{\newy/28.45274}\pgfmathresult cm)$};
    \draw [->] ($(\urx,\ury)+(-#5/2,0)-\r*(0,\rstep)$) -- ($(\urx,\ury)+(#5/2,0)-\r*(0,\rstep)$);}}\fi
}

\def\blackboxouters#1#2#3#4#5{%(width,height), number inputs, number outputs, label, arrow length
  \pgfgetlastxy{\llx}{\lly}%assumes path has been set to a point representing the lower left corner of the box
  \path #1;
  \pgfgetlastxy{\w}{\h}
  \pgfmathsetlengthmacro{\urx}{\llx+\w}
  \pgfmathsetlengthmacro{\ury}{\lly+\h}
  \draw (\llx,\lly) rectangle (\urx,\ury);
  \pgfmathsetlengthmacro{\xave}{(\llx+\urx)/2}
  \pgfmathsetlengthmacro{\yave}{\ury-8}
  \node at (\xave,\yave) {#4};
  \pgfmathsetlengthmacro{\ydiff}{\ury-\lly}
  \pgfmathsetlengthmacro{\lstep}{\ydiff/(#2+1)}
  \pgfmathsetlengthmacro{\rstep}{\ydiff/(#3+1)}
  \ifnum #2=0{}\else{ 
   \foreach \l in {1,...,#2}{
    \pgfmathsetlengthmacro{\newx}{\llx-#5*28.45274/2}
    \pgfmathsetlengthmacro{\newy}{\lly+\l*\lstep}
    \node at ($(\newx,\newy)+(-1.5,\l*12-\l*\lstep)$) {\tiny$(\pgfmathparse{\newx/28.45274}\pgfmathresult cm,\pgfmathparse{\newy/28.45274}\pgfmathresult cm)$};
    \draw [->] ($(\llx,\lly)+(-#5/2,0)+\l*(0,\lstep)$) -- ($(\llx,\lly)+(#5/2,0)+\l*(0,\lstep)$);}}\fi
  \ifnum #3=0{}\else{
   \foreach \r in {1,...,#3}{
    \pgfmathsetlengthmacro{\newx}{\urx+#5*28.45274/2}
    \pgfmathsetlengthmacro{\newy}{\ury-\r*\rstep}
    \node at ($(\newx,\newy)+(1.5,-\r*12+\r*\rstep)$) {\tiny $(\pgfmathparse{\newx/28.45274}\pgfmathresult cm,\pgfmathparse{\newy/28.45274}\pgfmathresult cm)$};
    \draw [->] ($(\urx,\ury)+(-#5/2,0)-\r*(0,\rstep)$) -- ($(\urx,\ury)+(#5/2,0)-\r*(0,\rstep)$);}}\fi
}

\def\dashbox#1#2#3#4#5{%(width,height), number inputs, number outputs, label, arrow length
  \pgfgetlastxy{\llx}{\lly}%assumes path has been set to a point representing the lower left corner of the box
  \path #1;
  \pgfgetlastxy{\w}{\h}
  \pgfmathsetlengthmacro{\urx}{\llx+\w}
  \pgfmathsetlengthmacro{\ury}{\lly+\h}
  \draw [dashed] (\llx,\lly) rectangle (\urx,\ury);
  \pgfmathsetlengthmacro{\xave}{(\llx+\urx)/2}
  \pgfmathsetlengthmacro{\yave}{\ury-8}
  \node at (\xave,\yave) {#4};
  \pgfmathsetlengthmacro{\ydiff}{\ury-\lly}
  \pgfmathsetlengthmacro{\lstep}{\ydiff/(#2+1)}
  \pgfmathsetlengthmacro{\rstep}{\ydiff/(#3+1)}
  \ifnum #2=0{}\else{ 
   \foreach \l in {1,...,#2}{
    \draw [->] ($(\llx,\lly)+(-#5/2,0)+\l*(0,\lstep)$) -- ($(\llx,\lly)+(#5/2,0)+\l*(0,\lstep)$);}}\fi
  \ifnum #3=0{}\else{
   \foreach \r in {1,...,#3}{
    \draw [->] ($(\urx,\ury)+(-#5/2,0)-\r*(0,\rstep)$) -- ($(\urx,\ury)+(#5/2,0)-\r*(0,\rstep)$);}}\fi
}

\def\fillbox#1#2#3#4#5#6{%(width,height), number inputs, number outputs, label, arrow length, fill strength
  \pgfgetlastxy{\llx}{\lly}%assumes path has been set to a point representing the lower left corner of the box
  \path #1;
  \pgfgetlastxy{\w}{\h}
  \pgfmathsetlengthmacro{\urx}{\llx+\w}
  \pgfmathsetlengthmacro{\ury}{\lly+\h}
  \filldraw [fill=gray!#6] (\llx,\lly) rectangle (\urx,\ury);
  \pgfmathsetlengthmacro{\xave}{(\llx+\urx)/2}
  \pgfmathsetlengthmacro{\yave}{\ury-8}
  \node at (\xave,\yave) {#4};
  \pgfmathsetlengthmacro{\ydiff}{\ury-\lly}
  \pgfmathsetlengthmacro{\lstep}{\ydiff/(#2+1)}
  \pgfmathsetlengthmacro{\rstep}{\ydiff/(#3+1)}
  \ifnum #2=0{}\else{ 
   \foreach \l in {1,...,#2}{
    \draw [->] ($(\llx,\lly)+(-#5/2,0)+\l*(0,\lstep)$) -- ($(\llx,\lly)+(#5/2,0)+\l*(0,\lstep)$);}}\fi
  \ifnum #3=0{}\else{
   \foreach \r in {1,...,#3}{
    \draw [->] ($(\urx,\ury)+(-#5/2,0)-\r*(0,\rstep)$) -- ($(\urx,\ury)+(#5/2,0)-\r*(0,\rstep)$);}}\fi
}


\def\delaynode#1#2#3{%(x-coord,y-coord), size, arrow length
  \path #1;
  \pgfgetlastxy{\lx}{\ly}
  \pgfmathsetlengthmacro{\rx}{\lx+#2*3+#3}
  \pgfmathsetlengthmacro{\ry}{\ly}
  \filldraw (\lx,\ly) circle (#2 pt);
  \draw [->] (\lx,\ly) -- (\rx,\ry);
}

\def\delaynodeouters#1#2#3{%(x-coord,y-coord), size, arrow length
  \path #1;
  \pgfgetlastxy{\lx}{\ly}
  \pgfmathsetlengthmacro{\rx}{\lx+#2*3+#3}
  \pgfmathsetlengthmacro{\ry}{\ly}
  \filldraw (\lx,\ly) circle (#2 pt);
  \draw [->] (\lx,\ly) -- (\rx,\ry);
   \node at (\rx+15,\ry+15){\tiny $(\pgfmathparse{\rx/28.45274}\pgfmathresult cm,\pgfmathparse{\ry/28.45274}\pgfmathresult cm)$};
}


\def\directarc#1#2{%left endpoint, right endpoint
  \path #1;
  \pgfgetlastxy{\lx}{\ly}
  \path #2;
  \pgfgetlastxy{\rx}{\ry}
  \pgfmathsetlengthmacro{\xave}{(\lx+\rx)/2}
  \draw #1 .. controls (\xave,\ly) and (\xave,\ry) .. #2;
}

\def\backarc#1#2{%left coordinate, right coordinate
  \path #1;
  \pgfgetlastxy{\lx}{\ly}
  \path #2;
  \pgfgetlastxy{\rx}{\ry}
  \pgfmathsetlengthmacro{\xave}{(\lx+\rx)/2}
  \draw #1 .. controls (\xave,\ry) and (\xave,\ly) .. #2;
}

\def\loopright#1#2#3{%upper coordinate, lower coordinate, stretch width
  \path #1;
  \pgfgetlastxy{\ux}{\uy}
  \path #2;
  \pgfgetlastxy{\lx}{\ly}
  \pgfmathsetlengthmacro{\maxx}{max(\ux,\lx)}
  \pgfmathsetlengthmacro{\farx}{\maxx+#3}
  \draw #1 .. controls (\farx,\uy) and (\farx,\ly) .. #2;
}

\def\loopleft#1#2#3{%upper coordinate, lower coordinate, stretch width
  \path #1;
  \pgfgetlastxy{\ux}{\uy}
  \path #2;
  \pgfgetlastxy{\lx}{\ly}
  \pgfmathsetlengthmacro{\minx}{min(\ux,\lx)}
  \pgfmathsetlengthmacro{\farx}{\minx-#3}
  \draw #1 .. controls (\farx,\uy) and (\farx,\ly) .. #2;
}

\def\fancyarc#1#2#3#4{%upper coordinate, lower coordinate, stretch width, max height adjust
  \path #1;
  \pgfgetlastxy{\ux}{\uy}
  \path #2;
  \pgfgetlastxy{\lx}{\ly}
  \pgfmathsetlengthmacro{\xave}{(\lx+\ux)/2}
%  \node at (\lx,\ly+20){\tiny $\pgfmathparse{\lx/28.45274}\pgfmathresult cm$,\hsp$\pgfmathparse{\ux/28.45274}\pgfmathresult cm$};
%  \node at (\xave,\ly+50){\tiny $\pgfmathparse{\xave/28.45274}\pgfmathresult cm$};
  \pgfmathsetlengthmacro{\yave}{(\ly+\uy)/2+#4}
  \loopleft{#1}{(\xave,\yave)}{#3}
  \loopright{#2}{(\xave,\yave)}{#3}
}

\def\activetikz#1{$$\begin{tikzpicture}#1\end{tikzpicture}$$}
\def\inactivetikz#1{}
%End tikz macros


\input xy
\xyoption{all} \xyoption{poly} \xyoption{knot}\xyoption{curve}
\usepackage{xy,color}
\newcommand{\comment}[1]{}

\newcommand{\longnote}[2][4.9in]{\fcolorbox{black}{yellow}{\parbox{#1}{\color{black} #2}}}
\newcommand{\shortnote}[1]{\fcolorbox{black}{yellow}{\color{black} #1}}
\newcommand{\start}[1]{\shortnote{Start here: #1.}}
\newcommand{\q}[1]{\begin{question}#1\end{question}}
\newcommand{\g}[1]{\begin{guess}#1\end{guess}}

\def\tn{\textnormal}
\def\mf{\mathfrak}
\def\mc{\mathcal}

\def\ZZ{{\mathbb Z}}
\def\QQ{{\mathbb Q}}
\def\RR{{\mathbb R}}
\def\CC{{\mathbb C}}
\def\AA{{\mathbb A}}
\def\PP{{\mathbb P}}
\def\NN{{\mathbb N}}
\def\SS{{\mathbb S}}
\def\HH{{\mathbb H}}

\def\acts{\lefttorightarrow}
\def\Hom{\tn{Hom}}
\def\iHom{\Rightarrow}
\def\Aut{\tn{Aut}}
\def\Sub{\tn{Sub}}
\def\Mor{\tn{Mor}}
\def\Fun{\tn{Fun}}
\def\Path{\tn{Path}}
\def\im{\tn{im}}
\def\Ob{\tn{Ob}}
\def\dim{\tn{dim}}
\def\Trace{\tn{Tr}}
\def\Op{\tn{Op}}
\def\SEL*{\tn{SEL*}}
\def\Res{\tn{Res}}
\def\hsp{\hspace{.3in}}
\newcommand{\hsps}[1]{{\hspace{2mm} #1\hspace{2mm}}}
\newcommand{\tin}[1]{\text{\tiny #1}}

\def\singleton{{\{*\}}}
\newcommand{\boxtitle}[1]{\begin{center}#1\end{center}}
\def\Loop{{\mcL oop}}
\def\LoopSchema{{\parbox{.5in}{\fbox{\xymatrix{\LMO{s}\ar@(l,u)[]^f}}}}}
\def\Wks{{\mcW ks}}
\def\lcone{^\triangleleft}
\def\rcone{^\triangleright}
\def\to{\rightarrow}
\def\from{\leftarrow}
\def\down{\downarrrow}
\def\Down{\Downarrow}
\def\Up{\Uparrow}
\def\taking{\colon}
\def\pls{``\!+\!"}
\newcommand{\pathto}[1]{\stackrel{#1}\leadsto}
\def\inj{\hookrightarrow}
\def\surj{\twoheadrightarrow}
\def\surjj{\longtwoheadrightarrow}
\def\pfunc{\rightharpoonup}
\def\Pfunc{\xrightharpoonup}
\def\too{\longrightarrow}
\def\fromm{\longleftarrow}
\def\tooo{\longlongrightarrow}
\def\tto{\rightrightarrows}
\def\ttto{\equiv\!\!>}
\newcommand{\xyto}[2]{\xymatrix@=1pt{\ar[rr]^-{#1}&\hspace{#2}&}}
\newcommand{\xyequals}[1]{\xymatrix@=1pt{\ar@{=}[rr]&\hspace{#1}&}}
\def\ss{\subseteq}
\def\superset{\supseteq}
\def\iso{\cong}
\def\down{\downarrow}
\def\|{{\;|\;}}
\def\m1{{-1}}
\def\op{^\tn{op}}
\def\la{\langle}
\def\ra{\rangle}
\def\wt{\widetilde}
\def\wh{\widehat}
\def\we{\simeq}
\def\ol{\overline}
\def\ul{\underline}
\def\vect{\overrightarrow}
\def\qeq{\mathop{=}^?}

\def\rr{\raggedright}

%\newcommand{\LMO}[1]{\bullet^{#1}}
%\newcommand{\LTO}[1]{\bullet^{\tn{#1}}}
\newcommand{\LMO}[1]{\stackrel{#1}{\bullet}}
\newcommand{\LTO}[1]{\stackrel{\tt{#1}}{\bullet}}
\newcommand{\LA}[2]{\ar[#1]^-{\tn {#2}}}
\newcommand{\LAL}[2]{\ar[#1]_-{\tn {#2}}}
\newcommand{\obox}[3]{\stackrel{#1}{\fbox{\parbox{#2}{#3}}}}
\newcommand{\labox}[2]{\obox{#1}{1.6in}{#2}}
\newcommand{\mebox}[2]{\obox{#1}{1in}{#2}}
\newcommand{\smbox}[2]{\stackrel{#1}{\fbox{#2}}}
\newcommand{\fakebox}[1]{\tn{$\ulcorner$#1$\urcorner$}}
\newcommand{\sq}[4]{\xymatrix{#1\ar[r]\ar[d]&#2\ar[d]\\#3\ar[r]&#4}}
\newcommand{\namecat}[1]{\begin{center}$#1:=$\end{center}}


\def\ullimit{\ar@{}[rd]|(.3)*+{\lrcorner}}
\def\urlimit{\ar@{}[ld]|(.3)*+{\llcorner}}
\def\lllimit{\ar@{}[ru]|(.3)*+{\urcorner}}
\def\lrlimit{\ar@{}[lu]|(.25)*+{\ulcorner}}
\def\ulhlimit{\ar@{}[rd]|(.3)*+{\diamond}}
\def\urhlimit{\ar@{}[ld]|(.3)*+{\diamond}}
\def\llhlimit{\ar@{}[ru]|(.3)*+{\diamond}}
\def\lrhlimit{\ar@{}[lu]|(.3)*+{\diamond}}
\newcommand{\clabel}[1]{\ar@{}[rd]|(.5)*+{#1}}
\newcommand{\TriRight}[7]{\xymatrix{#1\ar[dr]_{#2}\ar[rr]^{#3}&&#4\ar[dl]^{#5}\\&#6\ar@{}[u] |{\Longrightarrow}\ar@{}[u]|>>>>{#7}}}
\newcommand{\TriLeft}[7]{\xymatrix{#1\ar[dr]_{#2}\ar[rr]^{#3}&&#4\ar[dl]^{#5}\\&#6\ar@{}[u] |{\Longleftarrow}\ar@{}[u]|>>>>{#7}}}
\newcommand{\TriIso}[7]{\xymatrix{#1\ar[dr]_{#2}\ar[rr]^{#3}&&#4\ar[dl]^{#5}\\&#6\ar@{}[u] |{\Longleftrightarrow}\ar@{}[u]|>>>>{#7}}}


\newcommand{\arr}[1]{\ar@<.5ex>[#1]\ar@<-.5ex>[#1]}
\newcommand{\arrr}[1]{\ar@<.7ex>[#1]\ar@<0ex>[#1]\ar@<-.7ex>[#1]}
\newcommand{\arrrr}[1]{\ar@<.9ex>[#1]\ar@<.3ex>[#1]\ar@<-.3ex>[#1]\ar@<-.9ex>[#1]}
\newcommand{\arrrrr}[1]{\ar@<1ex>[#1]\ar@<.5ex>[#1]\ar[#1]\ar@<-.5ex>[#1]\ar@<-1ex>[#1]}

\newcommand{\To}[1]{\xrightarrow{#1}}
\newcommand{\Too}[1]{\xrightarrow{\ \ #1\ \ }}
\newcommand{\From}[1]{\xleftarrow{#1}}
\newcommand{\Fromm}[1]{\xleftarrow{\ \ #1\ \ }}
\def\qeq{\stackrel{?}{=}}

\newcommand{\Adjoint}[4]{\xymatrix@1{{#2}\ar@<.5ex>[r]^-{#1} &{#3} \ar@<.5ex>[l]^-{#4}}}

\def\id{\tn{id}}
\def\dom{\tn{dom}}
\def\cod{\tn{cod}}
\def\Top{{\bf Top}}
\def\Kls{{\bf Kls}}
\def\Cat{{\bf Cat}}
\def\Oprd{{\bf Oprd}}
\def\LH{{\bf LH}}
\def\Monad{{\bf Monad}}
\def\Mon{{\bf Mon}}
\def\CMon{{\bf CMon}}
\def\cpo{{\bf cpo}}
\def\Vect{\text{Vect}}
\def\Rep{{\bf Rep}}
\def\HCat{{\bf HCat}}
\def\Cnst{{\bf Cnst}}
\def\Str{\tn{Str}}
\def\List{\tn{List}}
\def\Exc{\tn{Exc}}
\def\Sets{{\bf Sets}}
\def\Cob{{\bf Cob}}
\def\Grph{{\bf Grph}}
\def\SGrph{{\bf SGrph}}
\def\Rel{\mcR\tn{el}}
\def\JRel{J\mcR\tn{el}}
\def\Inst{{\bf Inst}}
\def\Type{{\bf Type}}
\def\Set{{\bf Set}}
\def\TFS{{\bf TFS}}
\def\Qry{{\bf Qry}}
\def\set{{\text \textendash}{\bf Set}}
\def\sets{{\text \textendash}{\bf Alg}}
\def\alg{{\text \textendash}{\bf Alg}}
\def\rel{{\text \textendash}{\bf Rel}}
\def\inst{{{\text \textendash}\bf \Inst}}
\def\sSet{{\bf sSet}}
\def\sSets{{\bf sSets}}
\def\Grpd{{\bf Grpd}}
\def\Pre{{\bf Pre}}
\def\Shv{{\bf Shv}}
\def\Rings{{\bf Rings}}
\def\bD{{\bf \Delta}}
\def\dispInt{\parbox{.1in}{$\int$}}
\def\bhline{\Xhline{2\arrayrulewidth}}
\def\bbhline{\Xhline{2.5\arrayrulewidth}}


\def\Comp{\tn{Comp}}
\def\Supp{\tn{Supp}}
\def\Dmnd{\tn{Dmnd}}


\def\colim{\mathop{\tn{colim}}}
\def\hocolim{\mathop{\tn{hocolim}}}
\def\undsc{\rule{2mm}{0.4pt}}


\def\mcA{\mc{A}}
\def\mcB{\mc{B}}
\def\mcC{\mc{C}}
\def\mcD{\mc{D}}
\def\mcE{\mc{E}}
\def\mcF{\mc{F}}
\def\mcG{\mc{G}}
\def\mcH{\mc{H}}
\def\mcI{\mc{I}}
\def\mcJ{\mc{J}}
\def\mcK{\mc{K}}
\def\mcL{\mc{L}}
\def\mcM{\mc{M}}
\def\mcN{\mc{N}}
\def\mcO{\mc{O}}
\def\mcP{\mc{P}}
\def\mcQ{\mc{Q}}
\def\mcR{\mc{R}}
\def\mcS{\mc{S}}
\def\mcT{\mc{T}}
\def\mcU{\mc{U}}
\def\mcV{\mc{V}}
\def\mcW{\mc{W}}
\def\mcX{\mc{X}}
\def\mcY{\mc{Y}}
\def\mcZ{\mc{Z}}

\def\bfS{{\bf S}}\def\bfSs{{\bf Ss}}
\def\bfT{{\bf T}}\def\bfTs{{\bf Ts}}
\def\bfW{{\bf W}}

\def\tnN{\tn{N}}


\def\bE{\bar{E}}
\def\bF{\bar{F}}
\def\bG{\bar{G}}
\def\bH{\bar{H}}
\def\bX{\bar{X}}
\def\bY{\bar{Y}}

\newcommand{\subsub}[1]{\setcounter{subsubsection}{\value{theorem}}\subsubsection{#1}\addtocounter{theorem}{1}}

\def\Finm{{\bf Fin_{m}}}
\def\Bag{\tn{Bag}}
\newcommand{\back}[1]{\stackrel{\from}{#1}\!}
%\newcommand{\kls}[1]{{\text \textendash}\wt{\bf Kls}({#1})}
\def\Dist{\text{Dist}}
\def\Dst{{\bf Dst}}
\def\WkFlw{{\bf WkFlw}}
\def\monOb{{\blacktriangle}}
\def\Infl{{\bf Infl}}
\def\Tur{\tn{Tur}}
\def\Halt{\{\text{Halt}\}}
\def\Tape{{T\!ape}}
\def\Prb{{\bf Prb}}
\def\Prbs{{\wt{\bf Prb}}}
\def\El{{\bf El}}
\def\Gr{{\bf Gr}}
\def\DT{{\bf DT}}
\def\DB{{\bf DB}}
\def\Tables{{\bf Tables}}
\def\Sch{{\bf Sch}}
\def\Fin{{\bf Fin}}
\def\PrO{{\bf PrO}}
\def\PrOs{{\bf PrOs}}
\def\JLat{{\bf JLat}}
\def\JLats{{\bf JLats}}
\def\P{{\bf P}}
\def\SC{{\bf SC}}
\def\ND{{\bf ND}}
\def\Poset{{\bf Poset}}
\def\Bool{\tn{Bool}}
\newcommand{\labelDisp}[2]{\begin{align}\label{#1}\text{#2}\end{align}}

\newcommand{\inp}[1]{{#1_-}}
\newcommand{\outp}[1]{{#1_+}}
\newcommand{\vset}[1]{#1_{\tt type}}
\newcommand{\loc}[1]{{\tt loc}(#1)}
\newcommand{\extr}[1]{{\tt ext}(#1)}
\newcommand{\domn}[1]{{\overset{\bullet}{#1}}}
\newcommand{\codomn}[1]{{\underset{\bullet}{#1}}}
\newcommand{\inpm}[1]{{{\scriptstyle\bullet}#1}}
\newcommand{\outpm}[1]{{#1{\scriptstyle\bullet}}}
%\newcommand{\domn}[1]{{{\scriptstyle\bullet}#1}}
%\newcommand{\codomn}[1]{{#1{\scriptstyle\bullet}}}
%\newcommand{\inpm}[1]{{\underset{\bullet}{#1}}}
%\newcommand{\outpm}[1]{{\overset{\bullet}{#1}}}
\newcommand{\feeddd}[3]{{\tensor*[^{#2}_{\color{white}{!}}]{{#1}}{^{#3}}}}%the color thing is to get overlines to be the same height.
\newcommand{\feeddc}[3]{{\tensor*[^{#2}]{{#1}}{_{#3}}}}
\newcommand{\feedcd}[3]{{\tensor*[_{#2}]{{#1}}{^{#3}}}}
\newcommand{\feedcc}[3]{{\tensor*[^{\color{white}{!}}_{#2}]{{#1}}{_{#3}}}}
\newcommand{\feeddb}[2]{{\tensor*[^{#2}]{{#1}}{}}}
\newcommand{\feedbc}[2]{{\tensor*{{#1}}{^~_{#2}}}}
\newcommand{\feedcb}[2]{{\tensor*[^~_{#2}]{{#1}}{}}}
\newcommand{\feedbd}[2]{{\tensor*{{#1}}{^{#2}}}}
%\newcommand{\feeddd}[3]{{\tensor*[^{#2}_{#3}]{#1}{}}}
%\newcommand{\feeddc}[3]{{\tensor*[^{#2}]{#1}{_{#3}}}}
%\newcommand{\feedcd}[3]{{\tensor*[_{#3}]{#1}{^{#2}}}}
%\newcommand{\feedcc}[3]{{\tensor*{#1}{^{#2}_{#3}}}}
\newcommand{\vLst}[1]{\ol{#1}}
\newcommand{\Strm}[1]{{\tn{Strm}(#1)}}
\newcommand{\SP}[2]{{\tn{SP}(#1,#2)}}
\newcommand{\LP}[2]{{\tn{LP}(#1,#2)}}
\newcommand{\LPP}[2]{{\tn{LP}'(#1,#2)}}
\newcommand{\Hist}{\tn{Hist}}
\newcommand{\xleadsto}[1]{\stackrel{#1}{\leadsto}}
\newcommand{\strst}[1]{\big|_{[1,#1]}} %"stream restriction"
\newcommand{\Del}[1]{DN_{#1}}
\newcommand{\Dem}[1]{{Dm_{#1}}}
\newcommand{\Sup}[1]{{Sp_{#1}}}
%\newcommand{\Con}[1]{{Con_{#1}}}
\newcommand{\inDem}[1]{{in\Dem{#1}}}
\newcommand{\inSup}[1]{{in\Sup{#1}}}
\newcommand{\vinSup}[1]{\vLst{\inSup{#1}}}
\newcommand{\vinDem}[1]{\vLst{\inDem{#1}}}
\newcommand{\vSup}[1]{\vLst{\Sup{#1}}}
\newcommand{\vDem}[1]{\vLst{\Dem{#1}}}
\newcommand{\vDel}[1]{\vLst{\Del{#1}}}
\newcommand{\vinp}[1]{\vLst{\inp{#1}}}
\newcommand{\vintr}[1]{\vLst{\loc{#1}}}
\newcommand{\voutp}[1]{\vLst{\outp{#1}}}
\def\zipwith{\;\raisebox{3pt}{${}_\varcurlyvee$}\;}
\def\tbzipwith{\!\;\raisebox{2pt}{${}_\varcurlyvee$}\;\!}
\newcommand{\ffootnote}[2]{\hspace{#1}\footnote{#2}}



\def\lin{\ell\tn{In}}
\def\lout{\ell\tn{Out}}
\def\gin{g\tn{In}}
\def\gout{g\tn{Out}}
\def\min{m^{in}}
\def\mout{m^{out}}
\def\sin{s^{in}}
\def\sout{s^{out}}
\newcommand{\disc}[1]{{\ul{#1}}}

\newcommand{\Wir}[1]{\bfW_{#1}}
\def\SMC{{\bf SMC}}
\def\TSMC{{\bf TSMC}}
\newcommand{\FS}[1]{\tn{FS}_{/\mathcal{#1}}}
\def\TFSO{\text{\bf TSMC}_{\Ob(\FS{O})}}
\newcommand{\Bij}[1]{\mcB_{\mathcal{#1}}}
\def\Int{\tn{Int}}

%\newcommand{\strm}[1]{{\left(#1\right)^\NN}}

\makeatletter\let\c@figure\c@equation\makeatother %Aligns figure numbering and equation numbering.
\newtheorem{theorem}[subsubsection]{Theorem}
\newtheorem{lemma}[subsubsection]{Lemma}
\newtheorem{proposition}[subsubsection]{Proposition}
\newtheorem{corollary}[subsubsection]{Corollary}
\newtheorem{fact}[subsubsection]{Fact}

\theoremstyle{remark}
\newtheorem{remark}[subsubsection]{Remark}
\newtheorem{example}[subsubsection]{Example}
\newtheorem{application}[subsubsection]{Application}
\newtheorem{warning}[subsubsection]{Warning}
\newtheorem{question}[subsubsection]{Question}
\newtheorem{guess}[subsubsection]{Guess}
\newtheorem{answer}[subsubsection]{Answer}
\newtheorem{claim}[subsubsection]{Claim}

\theoremstyle{definition}
\newtheorem{definition}[subsubsection]{Definition}
\newtheorem{notation}[subsubsection]{Notation}
\newtheorem{conjecture}[subsubsection]{Conjecture}
\newtheorem{postulate}[subsubsection]{Postulate}
\newtheorem{construction}[subsubsection]{Construction}
\newtheorem{ann}[subsubsection]{Announcement}
\newenvironment{announcement}{\begin{ann}}{\hspace*{\fill}$\lozenge$\end{ann}}


\setcounter{tocdepth}{1}
\setcounter{secnumdepth}{2}


%Standard geometry (DO NOT CHANGE) seems to be: \newgeometry{left=1.76in,right=1.76in,top=1.6in,bottom=1.3in}

\usepackage[paperwidth=8.5in,paperheight=11in,text={5.7in,8in},centering]{geometry}
%\newgeometry{left=1.55in,right=1.55in,top=1.6in,bottom=1.3in}
\usepackage{lscape}

\begin{document}

\title{String diagrams for traced monoidal categories as Cob(1)-algebras}

\author{Dylan Rupel}
\address{Northeastern University\\360 Huntington Ave.\\Boston, MA 02115}
\email{dylanrupel@gmail.com}

\author{David I. Spivak}
\address{Massachusetts Institute of Technology\\77 Massachusetts Ave.\\Cambridge, MA 02139}
\email{dspivak@gmail.com}

\thanks{Spivak acknowledges support by ONR grant N000141310260 and AFOSR grant FA9550-14-1-0031.}


\maketitle

\tableofcontents

\section{First ideas}

By manifold, we always mean manifold with boundary; we must say {\em closed manifold} if we mean to say that the boundary is empty. When a manifold $Y$ is oriented, we denote the same manifold with opposite orientation by $Y^\vee$.

Recall that an {\em oriented finite 0-manifold} consists of a pair of finite sets $\inp{X},\outp{X}$, where $\inp{X}$ has negative orientation and $\outp{X}$ has positive orientation. Given a manifold $\mcO$, an {\em oriented 0-manifold over $\mcO$} is an oriented $0$-manifold $X$ together with a function $\tau_X\taking\inp{X}\sqcup \outp{X}\to\mcO$. 

If $X$ and $Y$ are oriented $0$-manifolds, an {\em oriented cobordism from $X$ to $Y$} consists of an oriented manifold $W$ whose boundary $\partial W$ is identified with $X\sqcup Y^\vee$. We will always be working with oriented manifolds and cobordisms, so we often speak of them as simply manifolds and cobordisms. 

Given a cobordism $W$ from $X$ to $Y$ and a cobordism $W'$ from $Y$ to $Z$, we can glue them together along a thickening of $Y$ to obtain a cobordism from $X$ to $Z$. If $X$ and $Y$ are manifolds over $\mcO$, when we speak of a cobordism between them, we mean that our manifold $W$ is over $\mcO$.

We denote the category whose objects are oriented $0$-manifolds and oriented cobordisms between them by $\Cob$ (because we will only have occasion to speak of $0$-manifolds in this paper), and the category of 0-manifolds and cobordisms over $\mcO$ by $\Cob/\mcO$.

\begin{proposition}\label{prop:set theoretic cob1}
We have the following combinatorial description of the category $\Cob/\mcO$.

\begin{description}
\item [Objects]An object in $X\in\Ob(\Cob/\mcO)$ can be identified with a pair $X=(\inp{X},\outp{X})$ of finite sets, together with a function $\inp{X}\sqcup\outp{X}\to\mcO$. We write $|X|$ to denote the set $\inp{X}\sqcup\outp{X}$.
\item[Morphisms]
A morphism $\Phi\taking X\to Y$ in $\Cob/\mcO$ can be identified with a pair $(|\varphi|,\varphi)$, where $|\varphi|$ is a typed finite set, called the set of {\em components of $\Phi$}, and $\varphi\taking |X|\sqcup|Y|\to |\varphi|$ is a typed function, satisfying the following condition, which we call the {\em bijective inclusions property}:
	\begin{quote}\tn{(Bijective inclusions property:)}
	The restrictions $\inpm{\varphi}:=\varphi\big|_{\inp{X}\sqcup\outp{Y}}$ and $\outpm{\varphi}:=\varphi\big|_{\outp{X}\sqcup\inp{Y}}$ are monomorphisms, and they are isomorphic as such; i.e. there exists a typed bijection 
	$$\varphi'\taking\inp{X}\sqcup\outp{Y}\Too{\iso}\outp{X}\sqcup\inp{Y}$$
	such that the following diagram commutes:
	$$\xymatrix@C=12pt{
	\inp{X}\sqcup\outp{Y}\ar[rr]^{\varphi'}\ar@{_(->}[dr]_{\inpm{\varphi}}&&\outp{X}\sqcup\inp{Y}\ar@{^(->}[dl]^{\outpm{\varphi}}\\
	&|\varphi|
	}
	$$
	Note that $\varphi'$ is unique if it exists, so it can be recovered from $\varphi$.
	\end{quote}
It is often convenient to denote the restrictions $\domn{\varphi}=\varphi\big|_{|X|}$ and $\codomn{\varphi}=\varphi\big|_{|Y|}$, which together define the morphism as a co-relation,
$$|X|\Too{\domn{\varphi}}|\varphi|\Fromm{\codomn{\varphi}}|Y|$$

The two typed functions $\inpm{\varphi}$ and $\outpm{\varphi}$ have the same image in $|\varphi|$; we denote the complement of this image as $|\varphi(\Loop)|$. Thus an equivalent (but less useful) definition for a morphism $X\to Y$ in $\Cob/\mcO$ is: a set $|\varphi(\Loop)|$ and a typed bijection $\inp{X}\sqcup\outp{Y}\To{\iso}\outp{X}\sqcup\inp{Y}$. 
\item [Identity] Let $X=(\inp{X},\outp{X})\in\Ob(\Cob/\mcO)$. The identity morphism on $X$,  is given by the co-relation $|X|\Too{\id}|X|\Fromm{\id}|X|$.
\item [Composition] Given composable morphisms $X\To{\Phi}Y\To{\Psi}Z$, their composite is given by taking the pushout of their co-relations:
$$
\xymatrix{
|X|\ar[r]^{\domn{\varphi}}&|\phi|\ar[r]&|\psi\circ\phi|\urlimit\\
&|Y|\ar[r]^{\domn{\psi}}\ar[u]_{\codomn{\varphi}}&|\psi|\ar[u]\\
&&|Z|\ar[u]_{\codomn{\psi}}
}
$$
\end{description}

\end{proposition}

\begin{proof}

**

$$\xymatrix{
\inp{X}\ar[rr]\ar[d]&&\inp{X}\sqcup\outp{Z}\ar@{-->}@/_3pc/[dddd]&&\outp{Z}\ar[ll]\ar[d]\\
\inp{X}\sqcup\outp{Y}\ar[dd]\ar[dr]&\outp{Y}\ar[l]\ar[dr]&&\inp{Y}\ar[r]\ar[dl]&\inp{Y}\sqcup\outp{Z}\ar[dd]\ar[dl]\\
&|\phi|&|Y|\ar[l]\ar[r]&|\psi|\\
\outp{X}\sqcup\inp{Y}\ar[ur]&\inp{Y}\ar[l]\ar[ur]&&\outp{Y}\ar[ul]\ar[r]&\outp{Y}\sqcup\inp{Z}\ar[ul]\\
\outp{X}\ar[u]\ar[rr]&&\outp{X}\sqcup\inp{Z}&&\inp{Z}\ar[ll]\ar[u]
}
$$
$$\xymatrix{
\inp{X}\ar[rr]\ar[d]&&\inp{X}\sqcup\outp{Z}\ar[dd]\ar@{-->}@/_3pc/[dddd]&&\outp{Z}\ar[ll]\ar[d]\\
\inp{X}\sqcup\outp{Y}\ar[dd]\ar[drr]&\outp{Y}\ar[l]&&\inp{Y}\ar[r]&\inp{Y}\sqcup\outp{Z}\ar[dd]\ar[dll]\\
&&|\phi|\sqcup_{|Y|}|\psi|\\
\outp{X}\sqcup\inp{Y}\ar[urr]&\inp{Y}\ar[l]&&\outp{Y}\ar[r]&\outp{Y}\sqcup\inp{Z}\ar[ull]\\
\outp{X}\ar[u]\ar[rr]&&\outp{X}\sqcup\inp{Z}\ar[uu]&&\inp{Z}\ar[ll]\ar[u]
}
$$
The two $\inp{Y}$'s are identified over $|\phi|\sqcup_{|Y|}|\psi|$ and so are the two $\outp{Y}$'s. We can show that the dotted arrow is an isomorphism by induction on the cardinality of $|Y|$.

\end{proof}

\subsection{Feeds}
The contravariant variable is on top, the covariant variable is on the bottom; the feeding variable is to the left, the fed variable is to the right. So given $\varphi\taking X\to Y$, we have the following.

$$\xymatrix@=30pt{
\feeddd{\varphi}{X}{X}\ar[r]\ar[d]\ullimit&\inp{X}\ar[d]^{\feedbd{\varphi}{X}}&\feedcd{\varphi}{Y}{X}\ar[l]\ar[d]\urlimit\\
\outp{X}\ar[r]^{\feeddb{\varphi}{X}}&|\varphi|&\inp{Y}\ar[l]^{\feedcb{\varphi}{Y}}\\
\feeddc{\varphi}{X}{Y}\ar[u]\ar[r]\lllimit&\outp{Y}\ar[u]^{\feedbc{\varphi}{Y}}&\feedcc{\varphi}{Y}{Y}\ar[l]\ar[u]\lrlimit
}
$$
%It's not too bad of an abuse to redraw this as follows:
%$$\xymatrix@=30pt{
%\feeddd{\varphi}{X}{X}\ar[r]\ar[d]\ullimit&\feedbd{\varphi}{X}\ar[d]&\feedcd{\varphi}{Y}{X}\ar[l]\ar[d]\urlimit\\
%\feeddb{\varphi}{X}\ar[r]&|\varphi|&\feedcb{\varphi}{Y}\ar[l]\\
%\feeddc{\varphi}{X}{Y}\ar[u]\ar[r]\lllimit&\feedbc{\varphi}{Y}\ar[u]&\feedcc{\varphi}{Y}{Y}\ar[l]\ar[u]\lrlimit
%}
%$$

\begin{proposition}[Composition wires]

Suppose given $\Phi\taking X\to Y$ and $\Psi\taking Y\to Z$. Let $P$ be the colimit of the following diagram:
$$\xymatrix{
\feeddd{\varphi}{X}{X}\ar[r]\ar[d]&\inp{X}&\feedcd{\varphi}{Y}{X}\ar[l]\ar[d]\\
\outp{X}&\feeddd{\psi}{Y}{Y}\ar[r]\ar[d]&\inp{Y}&\feedcd{\psi}{Z}{Y}\ar[l]\ar[d]\\
\feeddc{\varphi}{X}{Y}\ar[u]\ar[r]&\outp{Y}&\feedcc{\varphi}{Y}{Y}\ar[u]\ar[l]&\inp{Z}\\
&\feeddc{\psi}{Y}{Z}\ar[u]\ar[r]&\outp{Z}&\feedcc{\psi}{Z}{Z}\ar[l]\ar[u]
}
$$
Then we have typed functions as to the left below.
$$
\xymatrix{
&\inp{X}\ar[d]\\
\outp{X}\ar[r]&P&\inp{Z}\ar[l]\\
&\outp{Z}\ar[u]
}
\hspace{.8in}
\xymatrix{
\feeddd{(\psi\circ\phi)}{X}{X}\ar[r]\ar[d]\ullimit&\inp{X}\ar[d]&\feedcd{(\psi\circ\phi)}{Z}{X}\ar[l]\ar[d]\urlimit\\
\outp{X}\ar[r]&P&\inp{Z}\ar[l]\\
\feeddc{(\psi\circ\phi)}{X}{Z}\ar[r]\ar[u]\lllimit&\outp{Z}\ar[u]&\feedcc{(\psi\circ\phi)}{Z}{Z}\ar[u]\ar[l]\lrlimit
}
$$
The pullbacks in the righthand diagram are the indicated wires of the composition $\psi\circ\phi$.

\end{proposition}

\begin{proof}

**

\end{proof}

\begin{lemma}[Technical 1]

Define pullbacks:
$$\xymatrix{
\feedcd{\varphi}{Y}{X}\ar[d]&thing_1\ar[l]\ar[d]\urlimit\\
\inp{Y}&\feedcd{\psi}{Z}{Y}\ar[l]
}
\hspace{.8in}
\xymatrix{
\feeddc{\varphi}{X}{Y}\ar[r]&\outp{Y}\\
thing_2\ar[r]\ar[u]\lllimit&\feeddc{\psi}{Y}{Z}\ar[u]
}
$$
and
$$\xymatrix@=15pt{
&thing_3\ar[rr]\ar[dd]\ullimit&&\feeddd{(\psi\circ\varphi)}{X}{X}\ar[dd]&&thing_4\ar[ll]\ar[dd]\urlimit\\
&&thing_5\ar[ld]\ar'[r][rr]&&\feedcd{(\psi\circ\varphi)}{Z}{X}\ar[dl]&&thing_6\ar'[l][ll]\ar[dl]\\
&\feeddd{\psi}{Y}{Y}\ar[rr]&&P&&\feedcc{\varphi}{Y}{Y}\ar[ll]\\
thing_7\ar[ur]\ar[rr]&&\feeddc{(\psi\circ\varphi)}{X}{Z}\ar[ur]&&thing_8\ar[ll]\ar[ur]\\
&thing_9\ar'[u][uu]\ar[rr]\lllimit&&\feedcc{(\psi\circ\varphi)}{Z}{Z}\ar'[u][uu]&&thing_{10}\ar[uu]\ar[ll]\lrlimit
}
$$

There are injective maps: 
$$\feeddd{\varphi}{X}{X}\to\feeddd{(\psi\circ\phi)}{X}{X}$$
$$\feedcc{\psi}{Z}{Z}\to\feedcc{(\psi\circ\phi)}{Z}{Z}$$
$$thing_1\to\feedcd{(\psi\circ\varphi)}{Z}{X}$$
$$thing_2\to\feeddc{(\psi\circ\varphi)}{X}{Z}$$
And surjections
\begin{align*}
\feeddd{\varphi}{X}{X}\sqcup thing_3\to\feeddd{(\psi\circ\phi)}{X}{X}
&\hsp\tn{and}\hsp
\feeddd{\varphi}{X}{X}\sqcup thing_4\to\feeddd{(\psi\circ\phi)}{X}{X}\\
\feedcc{\psi}{Z}{Z}\sqcup thing_9\to\feedcc{(\psi\circ\phi)}{Z}{Z}
&\hsp\tn{and}\hsp
\feedcc{\psi}{Z}{Z}\sqcup thing_{10}\to\feedcc{(\psi\circ\phi)}{Z}{Z}\\
thing_1\sqcup thing_5\to\feedcd{(\psi\circ\varphi)}{Z}{X}
&\hsp\tn{and}\hsp
thing_1\sqcup thing_6\to\feedcd{(\psi\circ\varphi)}{Z}{X}\\
thing_2\sqcup thing_7\to\feeddc{(\psi\circ\varphi)}{X}{Z}
&\hsp\tn{and}\hsp
thing_2\sqcup thing_8\to\feeddc{(\psi\circ\varphi)}{X}{Z}
\end{align*}

\end{lemma}

\begin{proposition}

We have the following combinatorial description of a compact closed monoidal structure on $\Cob/\mcO$, in terms of the combinatorial description given in Proposition \ref{prop:set theoretic cob1}.
\begin{description}
\item [Tensor object] Given objects $X,X'\in\Cob/\mcO$, their tensor product, written $X\oplus X'$, is given by 
$$\inp{(X\oplus X')}=\inp{X}\sqcup\inp{X'}, \hsp\tn{and}\hsp \outp{(X\oplus X')}=\outp{X}\sqcup\outp{X'}.$$
\item [Unit object] The unit object is $(\emptyset,\emptyset)$; it is denoted $\square$.
\item [Dual object] Given any object $X=(\inp{X},\outp{X})$, its dual is denoted and defined as 
$$X^\vee=(\outp{X},\inp{X}).$$
\item [Tensor morphism] Given morphisms $|X|\Too{\domn{\varphi}}|\varphi|\Fromm{\codomn{\varphi}}|Y|$ and $|X'|\Too{\domn{\varphi}\hspace{0pt}'}|\varphi|\Fromm{\codomn{\varphi}'}|Y'|$, their tensor product is given by disjoint union across the board,
$$|X|\sqcup|X'|\Too{\domn{\varphi}\sqcup\domn{\varphi}'}|\varphi|\sqcup|\varphi'|\Fromm{\codomn{\varphi}\sqcup\codomn{\varphi}\hspace{0in}'}|Y|\sqcup|Y'|$$
\item [Unit morphism] Given any object $X$, the unit morphism $\eta_X\taking\square\to X\oplus X^\vee$ is given by letting $|\eta_X|=|X|$, noting that $|X\oplus X|=|X|\sqcup|X|$, and using the fold map 
$$\emptyset\To{!}|X|\From{fold}|X|\sqcup|X|.$$
\item [Counit morphism] Given any object $X$, the unit morphism $\epsilon_X\taking X\oplus X^\vee\to\square$ is given by letting $|\epsilon_X|=|X|$, noting that $|X\oplus X|=|X|\sqcup|X|$, and using the fold map 
$$|X|\sqcup|X|\To{fold}|X|\From{!}\emptyset.$$

\end{description}

\end{proposition}

\begin{proof}

The simplicity in the definition of $\oplus$ and $\square$ make it easy to see that the tensor and unit object give $\Cob/\mcO$ the structure of a symmetric monoidal category. It remains to show that the dual object satisfies the necessary equations in terms of its unit and counit morphisms.

Proving that the necessary equation for the unit morphism holds amounts to showing that the following composite is the identity:
$$|X|\Too{\tn{inl}}|X|\sqcup|X|\Fromm{\left\{\parbox{.1in}{\scriptsize inl\\inr\\inr}\right.}|X|\sqcup|X|\sqcup|X|\Too{\left\{\parbox{.1in}{\scriptsize inl\\inl\\inr}\right.}|X|\sqcup|X|\From{\tn{inr}}|X|.$$
This easily checked; it may help to picture it as follows, where we take the colimit of the middle seven $|X|$'s in the diagram.
$$\xymatrix@R=1pt{
&&|X|\ar[dl]\ar[dr]\\
|X|\ar[r]&|X|&&|X|\\
&&|X|\ar[ur]\ar[dl]\\
&|X|&&|X|&|X|\ar[l]\\
&&|X|\ar[ul]\ar[ur]
}
$$
The equation for the counit is similar.

\end{proof}

Given a typed finite set $s$, we have an object $S:=(s,s)\in\Ob(\Cob/\mcO)$. There is a natural morphism $X\to X\oplus S$ and a natural morphism $X\oplus S\to X$, as shown:
\begin{align}\label{dia:proto loops}
|X|\Too{inl}|X|\sqcup|S|\Fromm{\id}|X\oplus S|.\hspace{.7in}
|X\oplus S|\Too{\id}|X|\sqcup |S|\Fromm{inl}|X|
\end{align}


\begin{lemma}

Let $s$ be a typed finite set and $S=(s,s)$. The composite of the morphisms $X\to X\oplus S\to X$ from (\ref{dia:proto loops}) constitutes a natural transformation 
$$\Loop(s)\taking\id_{\Cob/\mcO}\to\id_{\Cob/\mcO}$$ 
whose $X$-component $\Loop(s)_X\taking X\to X$ is given by the co-relation 
$$|X|\To{inl} |X|\sqcup |S|\From{inl} |X|.$$

\end{lemma}

\begin{proof}

Clearly the composite is given by the co-relation $|X|\to |X|\sqcup |S|\from |X|$, because this is the pushout of the co-relations in (\ref{dia:proto loops}). Thus it remains to show that these co-relations really are morphisms (i.e. that they satisfy the bijective inclusions property), and that these morphisms are natural in $X$.

It is easy to see that these co-relations satisfy the bijective inclusions property, because $\inp{X}\sqcup\outp{(X\oplus S)}=|X|\sqcup|S|=\outp{X}\sqcup\inp{(X\oplus S)}.$ To see that $\Loop(s)$ is natural in $s$, one checks that for any $\varphi\taking X\to Y$, the two co-relations below are identical:
$$
\xymatrix{
|X|\ar[r]^{\domn{\varphi}}&|\varphi|\ar[r]^{inl}&|\varphi|\sqcup|S|\urlimit\\
&|Y|\ar[u]_{\codomn{\varphi}}\ar[r]^-{inl}&|Y|\sqcup|S|\ar[u]_{\codomn{\varphi}\sqcup|S|}\\
&&|Y|\ar[u]_{inl}
}
\hspace{.8in}
\xymatrix{
|X|\ar[r]^-{inl}&|X|\sqcup|S|\ar[r]^{\domn{\varphi}\sqcup|S|}&|\varphi|\sqcup|S|\urlimit\\
&|X|\ar[u]_{inl}\ar[r]^-{\domn{\varphi}}&|\varphi|\ar[u]_{inl}\\
&&|Y|\ar[u]_{\codomn{\varphi}}
}$$
\end{proof}



\section{Main theorem}

\begin{lemma}

Given a function $f\taking\mcO\to\mcO'$, there is an induced strong monoidal functor $f^*\taking\Cob/\mcO'\to\Cob/\mcO$. This defines a functor 
$$\Cob/-\taking\Set\to\SMC_{st}\op$$
from the category of sets to the opposite of the category of SMCs and strong monoidal functors.

\end{lemma}

\begin{proof}

**

\end{proof}

\begin{definition}

Define {\em the category of wiring diagram algebras}, denoted $(\Cob/\bullet)\alg$, to have objects 
$$\Ob((\Cob/\bullet)\alg):=\{(\mcO,\mcP)\|\mcO\in\Ob(\Set)\tn{ and } \mcP\in\Ob\big((\Cob/\mcO)\alg\big)\},$$
and to have morphisms
$$\Hom_{(\Cob/\bullet)\alg}((\mcO,\mcP),(\mcO',\mcP')):=\{(f,r)\|f\taking\mcO\to\mcO', r\taking\mcP\to(\Cob/f)^*(\mcP')\}.$$

\end{definition}

\begin{theorem}\label{thm:cobalg trace adjunction}

Let $(\Cob/\bullet)\alg$ be the category of wiring diagram algebras, and let $\TSMC$ denote the category of traced symmetric monoidal categories (and lax monoidal functors between them). There is an adjunction
$$\xymatrix{L\taking(\Cob/\bullet)\alg\ar@<.5ex>[r]&\TSMC:\!R\ar@<.5ex>[l]}$$

\end{theorem}

\begin{proof}

Let $\mcM$ be a traced monoidal category with objects $\mcO$. We will define a $\Cob/\mcO$-algebra $\mcP=R(\mcM)\taking\Cob/\mcO\to\Set$ as follows. On an object $[\inp{X},\outp{X}]\in\Ob(\Cob/\mcO)$, put 
$$\mcP[\inp{X},\outp{X}]:=\Hom_{\mcM}(\vinp{X},\voutp{X}).$$
We next consider morphisms.


A morphism $\Phi\taking[\inp{X},\outp{X}]\too[\inp{Y},\outp{Y}]$ consists of a typed bijection 
$$\phi\taking\inp{X}\sqcup \outp{Y}\To{\iso}\outp{X}\sqcup \inp{Y},$$ 
together with a typed finite set $S$. Given an element $f\in\mcP[\inp{X},\outp{X}]$ we must construct $\mcP(\Phi)(f)\in\mcP[\inp{Y},\outp{Y}]$. Let $\dim(\ol{S})=\Trace^{\ol{S}}_{I,I}(\id_{\ol{S}})$. Then we use the formula
$$\mcP(\Phi)(f):=
\Trace^{\ol{\feeddd{\varphi}{X}{X}}}_{\ol{Y_-},\ol{Y_+}}\Big(
\id_{\ol{\feedcc{\varphi}{Y}{Y}}}\otimes f
\Big)
\otimes\dim(\ol{S})
$$
To be pedantic, 
$$\mcP(\Phi)(f):=
\Trace^{\ol{\feeddd{\varphi}{X}{X}}}_{\ol{Y_-},\ol{Y_+}}\Big(
(\ol{\varphi_+}^{-1}\otimes\id_{\ol{\feeddd{\varphi}{X}{X}}})\circ\big(\id_{\ol{\feedcc{\varphi}{Y}{Y}}}\otimes (\ol{\varphi^+}\circ f\circ\ol{\varphi^-}^{-1})\big)\circ(\ol{\varphi_-}\otimes\id_{\ol{\feeddd{\varphi}{X}{X}}})
\Big)
\otimes\dim(\ol{S})
$$
%$$\mcP(\Phi)(f):=
%\Trace^{\ol{\feeddd{\varphi}{X}{X}}}_{\ol{Y_-},\ol{Y_+}}\Big(
%(\id_{\ol{\feeddc{\varphi}{X}{Y}}}\otimes\gamma^{~}_{\ol{\feeddd{\varphi}{X}{X}},\ol{\feedcc{\varphi}{Y}{Y}}})
%\circ(f\otimes\id_{\ol{\feedcc{\varphi}{Y}{Y}}})
%\circ(\id_{\ol{\feedcd{\varphi}{Y}{X}}}\otimes\gamma^{~}_{\ol{\feedcc{\varphi}{Y}{Y}},\ol{\feeddd{\varphi}{X}{X}}})
%\Big)
%\otimes\dim(\ol{S})
%$$
Note, for $\varphi\taking X\to Y$, we're thinking of the domain guys, $\inp{X}$, $\outp{X}$ as follows
$$
\ol{\varphi^-}:\ol{\inp{X}}\To{\sim}\ol{\feedcd{\varphi}{Y}{X}}\otimes\ol{\feeddd{\varphi}{X}{X}}
\hsp\tn{and}\hsp
\ol{\varphi^+}:\ol{\outp{X}}\To{\sim}\ol{\feeddc{\varphi}{X}{Y}}\otimes\ol{\feeddd{\varphi}{X}{X}}
$$
while we're thinking of codomain guys $\inp{Y}$, and $\outp{Y}$ as follows:
$$
\ol{\varphi_-}:\ol{\inp{Y}}\To{\sim}\ol{\feedcc{\varphi}{Y}{Y}}\otimes\ol{\feedcd{\varphi}{Y}{X}}
\hsp\tn{and}\hsp
\ol{\varphi_+}:\ol{\outp{Y}}\To{\sim}\ol{\feedcc{\varphi}{Y}{Y}}\otimes\ol{\feeddc{\varphi}{X}{Y}}
$$

Given also $\Psi\taking[\inp{Y},\outp{Y}]\too[\inp{Z},\outp{Z}]$, we need to show that the following equation holds: 
$$\mcP(\Psi)\circ\mcP(\Phi)(f)=^?\mcP(\Psi\circ\Phi)(f).$$
These are:
\begin{align}
\label{eq:composition 1}
\mcP(\Psi)\circ\mcP(\Phi)(f)&=
\Trace^{\ol{\feeddd{\psi}{Y}{Y}}}_{\ol{Z_-},\ol{Z_+}}
\Big(
\id_{\ol{\feedcc{\psi}{Z}{Z}}}\otimes
\Trace^{\ol{\feeddd{\varphi}{X}{X}}}_{\ol{Y_-},\ol{Y_+}}(\id_{\ol{\feedcc{\varphi}{Y}{Y}}}\otimes f)\otimes\dim(\ol{S})
\Big)
\otimes\dim(\ol{T})
\\\nonumber\\
\label{eq:composition 2}
\mcP(\Psi\circ\Phi)(f)&=
\Trace^{\ol{\feeddd{(\psi\circ\varphi)}{X}{X}}}_{\ol{Z_-},\ol{Z_+}}(\id_{\ol{\feedcc{(\psi\circ\varphi)}{Z}{Z}}}\otimes f)\otimes\dim(\ol{S\sqcup T\sqcup ``newloops"})
\end{align}

We begin by simplifying (2).  Note that $\dim(\ol{S})$ is an automorphism of the trivial object $I$ and thus may be pulled out of the trace $\Trace^{\ol{\feeddd{\psi}{Y}{Y}}}_{\ol{Z_-},\ol{Z_+}}$ (more precise statement needed).  Thus we may restrict our attention to $\Trace^{\ol{\feeddd{\psi}{Y}{Y}}}_{\ol{Z_-},\ol{Z_+}}\Big(\id_{\ol{\feedcc{\psi}{Z}{Z}}}\otimes\Trace^{\ol{\feeddd{\varphi}{X}{X}}}_{\ol{Y_-},\ol{Y_+}}(\id_{\ol{\feedcc{\varphi}{Y}{Y}}}\otimes f)\Big)$.

\begin{lemma}\label{le:combining traces}
 \begin{align*}
&\Trace^{\ol{\feeddd{\psi}{Y}{Y}}}_{\ol{Z_-},\ol{Z_+}}
\Big(
\id_{\ol{\feedcc{\psi}{Z}{Z}}}\otimes
\Trace^{\ol{\feeddd{\varphi}{X}{X}}}_{\ol{Y_-},\ol{Y_+}}(\id_{\ol{\feedcc{\varphi}{Y}{Y}}}\otimes f)
\Big)\\
&=\Trace^{\ol{\feeddd{\varphi}{X}{X}}}_{\ol{\inp{Z}},\ol{\outp{Z}}}
\Big(
\id_{\ol{\feedcc{\psi}{Z}{Z}}}\otimes\Trace^{\ol{\feeddd{\psi}{Y}{Y}}}_{\ol{\feedcd{\psi}{Z}{Y}}\otimes\ol{\feeddd{\varphi}{X}{X}},\ol{\feeddc{\psi}{Y}{Z}}\otimes\ol{\feeddd{\varphi}{X}{X}}}
\big(
(\id_{\ol{\feedcd{\psi}{Z}{Y}}}\otimes\gamma_{\ol{\feeddd{\psi}{Y}{Y}},\ol{\feeddd{\varphi}{X}{X}}})\circ(\id_{\ol{\feedcc{\varphi}{Y}{Y}}}\otimes f)\circ(\id_{\ol{\feedcd{\psi}{Z}{Y}}}\otimes\gamma_{\ol{\feeddd{\varphi}{X}{X}},\ol{\feeddd{\psi}{Y}{Y}}})
\big)
\Big).
\end{align*}
\end{lemma}
\begin{proof}
Since $\ol{Z_-}=\ol{\feedcc{\psi}{Z}{Z}}\otimes\ol{\feedcd{\psi}{Z}{Y}}$ and $\ol{Z_+}=\ol{\feedcc{\psi}{Z}{Z}}\otimes\ol{\feeddc{\psi}{Y}{Z}}$ we may apply the superposing axiom of the trace for the map $\id_{\ol{\feedcc{\psi}{Z}{Z}}}$ to get
\[\Trace^{\ol{\feeddd{\psi}{Y}{Y}}}_{\ol{Z_-},\ol{Z_+}}
\Big(
\id_{\ol{\feedcc{\psi}{Z}{Z}}}\otimes\Trace^{\ol{\feeddd{\varphi}{X}{X}}}_{\ol{Y_-},\ol{Y_+}}(\id_{\ol{\feedcc{\varphi}{Y}{Y}}}\otimes f)
\Big)=\id_{\ol{\feedcc{\psi}{Z}{Z}}}\otimes\Trace^{\ol{\feeddd{\psi}{Y}{Y}}}_{\ol{\feedcd{\psi}{Z}{Y}},\ol{\feeddc{\psi}{Y}{Z}}}
\Big(
\Trace^{\ol{\feeddd{\varphi}{X}{X}}}_{\ol{Y_-},\ol{Y_+}}(\id_{\ol{\feedcc{\varphi}{Y}{Y}}}\otimes f)
\Big).\]
Using that $\ol{Y_-}=\ol{\feedcd{\psi}{Z}{Y}}\otimes\ol{\feeddd{\psi}{Y}{Y}}$ and $\ol{Y_+}=\ol{\feeddc{\psi}{Y}{Z}}\otimes\ol{\feeddd{\psi}{Y}{Y}}$ we may apply the second vanishing axiom of the trace to get
\[\Trace^{\ol{\feeddd{\psi}{Y}{Y}}}_{\ol{\feedcd{\psi}{Z}{Y}},\ol{\feeddc{\psi}{Y}{Z}}}
\Big(\Trace^{\ol{\feeddd{\varphi}{X}{X}}}_{\ol{\inp{Y}},\ol{\outp{Y}}}(f\otimes\id_{\ol{\feedcc{\varphi}{Y}{Y}}})\Big)=\Trace^{\ol{\feeddd{\psi}{Y}{Y}}\otimes\ol{\feeddd{\varphi}{X}{X}}}_{\ol{\feedcd{\psi}{Z}{Y}},\ol{\feeddc{\psi}{Y}{Z}}}
(\id_{\ol{\feedcc{\varphi}{Y}{Y}}}\otimes f).\]
By dinaturality using the symmetry $\gamma_{\ol{\feeddd{\varphi}{X}{X}},\ol{\feeddd{\psi}{Y}{Y}}}$ we get
\[\Trace^{\ol{\feeddd{\psi}{Y}{Y}}\otimes\ol{\feeddd{\varphi}{X}{X}}}_{\ol{\feedcd{\psi}{Z}{Y}},\ol{\feeddc{\psi}{Y}{Z}}}
(\id_{\ol{\feedcc{\varphi}{Y}{Y}}}\otimes f)
=\Trace^{\ol{\feeddd{\varphi}{X}{X}}\otimes\ol{\feeddd{\psi}{Y}{Y}}}_{\ol{\feedcd{\psi}{Z}{Y}},\ol{\feeddc{\psi}{Y}{Z}}}
\big(
(\id_{\ol{\feedcd{\psi}{Z}{Y}}}\otimes\gamma_{\ol{\feeddd{\psi}{Y}{Y}},\ol{\feeddd{\varphi}{X}{X}}})\circ(\id_{\ol{\feedcc{\varphi}{Y}{Y}}}\otimes f)\circ(\id_{\ol{\feedcd{\psi}{Z}{Y}}}\otimes\gamma_{\ol{\feeddd{\varphi}{X}{X}},\ol{\feeddd{\psi}{Y}{Y}}})
\big).\]
Applying the second vanishing axiom again gives
\begin{align*}
&\Trace^{\ol{\feeddd{\varphi}{X}{X}}\otimes\ol{\feeddd{\psi}{Y}{Y}}}_{\ol{\feedcd{\psi}{Z}{Y}},\ol{\feeddc{\psi}{Y}{Z}}}
\big(
(\id_{\ol{\feedcd{\psi}{Z}{Y}}}\otimes\gamma_{\ol{\feeddd{\psi}{Y}{Y}},\ol{\feeddd{\varphi}{X}{X}}})\circ(\id_{\ol{\feedcc{\varphi}{Y}{Y}}}\otimes f)\circ(\id_{\ol{\feedcd{\psi}{Z}{Y}}}\otimes\gamma_{\ol{\feeddd{\varphi}{X}{X}},\ol{\feeddd{\psi}{Y}{Y}}})
\big)\\
&=\Trace^{\ol{\feeddd{\varphi}{X}{X}}}_{\ol{\feedcd{\psi}{Z}{Y}},\ol{\feeddc{\psi}{Y}{Z}}}
\Big(
\Trace^{\ol{\feeddd{\psi}{Y}{Y}}}_{\ol{\feedcd{\psi}{Z}{Y}}\otimes\ol{\feeddd{\varphi}{X}{X}},\ol{\feeddc{\psi}{Y}{Z}}\otimes\ol{\feeddd{\varphi}{X}{X}}}
\big(
(\id_{\ol{\feedcd{\psi}{Z}{Y}}}\otimes\gamma_{\ol{\feeddd{\psi}{Y}{Y}},\ol{\feeddd{\varphi}{X}{X}}})\circ(\id_{\ol{\feedcc{\varphi}{Y}{Y}}}\otimes f)\circ(\id_{\ol{\feedcd{\psi}{Z}{Y}}}\otimes\gamma_{\ol{\feeddd{\varphi}{X}{X}},\ol{\feeddd{\psi}{Y}{Y}}})
\big)
\Big).
\end{align*}
Once again applying the superposing axiom of the trace for the map $\id_{\ol{\feedcc{\psi}{Z}{Z}}}$ gives
\begin{align*}
&\id_{\ol{\feedcc{\psi}{Z}{Z}}}\otimes\Trace^{\ol{\feeddd{\varphi}{X}{X}}}_{\ol{\feedcd{\psi}{Z}{Y}},\ol{\feeddc{\psi}{Y}{Z}}}
\Big(
\Trace^{\ol{\feeddd{\psi}{Y}{Y}}}_{\ol{\feedcd{\psi}{Z}{Y}}\otimes\ol{\feeddd{\varphi}{X}{X}},\ol{\feeddc{\psi}{Y}{Z}}\otimes\ol{\feeddd{\varphi}{X}{X}}}
\big(
(\id_{\ol{\feedcd{\psi}{Z}{Y}}}\otimes\gamma_{\ol{\feeddd{\psi}{Y}{Y}},\ol{\feeddd{\varphi}{X}{X}}})\circ(\id_{\ol{\feedcc{\varphi}{Y}{Y}}}\otimes f)\circ(\id_{\ol{\feedcd{\psi}{Z}{Y}}}\otimes\gamma_{\ol{\feeddd{\varphi}{X}{X}},\ol{\feeddd{\psi}{Y}{Y}}})
\big)
\Big)\\
&=\Trace^{\ol{\feeddd{\varphi}{X}{X}}}_{\ol{\inp{Z}},\ol{\outp{Z}}}
\Big(
\id_{\ol{\feedcc{\psi}{Z}{Z}}}\otimes\Trace^{\ol{\feeddd{\psi}{Y}{Y}}}_{\ol{\feedcd{\psi}{Z}{Y}}\otimes\ol{\feeddd{\varphi}{X}{X}},\ol{\feeddc{\psi}{Y}{Z}}\otimes\ol{\feeddd{\varphi}{X}{X}}}
\big(
(\id_{\ol{\feedcd{\psi}{Z}{Y}}}\otimes\gamma_{\ol{\feeddd{\psi}{Y}{Y}},\ol{\feeddd{\varphi}{X}{X}}})\circ(\id_{\ol{\feedcc{\varphi}{Y}{Y}}}\otimes f)\circ(\id_{\ol{\feedcd{\psi}{Z}{Y}}}\otimes\gamma_{\ol{\feeddd{\varphi}{X}{X}},\ol{\feeddd{\psi}{Y}{Y}}})
\big)
\Big).
\end{align*}
Combining these observations completes the proof.
\end{proof}

\begin{lemma}
 The trace
 \[\Trace^{\ol{\feeddd{\psi}{Y}{Y}}}_{\ol{\feedcd{\psi}{Z}{Y}}\otimes\ol{\feeddd{\varphi}{X}{X}},\ol{\feeddc{\psi}{Y}{Z}}\otimes\ol{\feeddd{\varphi}{X}{X}}}
\big(
(\id_{\ol{\feedcd{\psi}{Z}{Y}}}\otimes\gamma_{\ol{\feeddd{\psi}{Y}{Y}},\ol{\feeddd{\varphi}{X}{X}}})\circ(\id_{\ol{\feedcc{\varphi}{Y}{Y}}}\otimes f)\circ(\id_{\ol{\feedcd{\psi}{Z}{Y}}}\otimes\gamma_{\ol{\feeddd{\varphi}{X}{X}},\ol{\feeddd{\psi}{Y}{Y}}})
\big)\]
simplifies to...

\end{lemma}

Possibly useful identities for (2):
\[\Trace^{\ol{\feeddd{(\psi\circ\varphi)}{X}{X}}}_{\ol{\inp{Z}},\ol{\outp{Z}}}(f\otimes\id_{\ol{\feedcc{(\psi\circ\varphi)}{Z}{Z}}})=\Trace^{\ol{P_X^X}\otimes\ol{\feeddd{\varphi}{X}{X}}}_{\ol{\inp{Z}},\ol{\outp{Z}}}(f\otimes\id_{\ol{\feedcc{(\psi\circ\varphi)}{Z}{Z}}})\]
\[\Trace^{\ol{P_X^X}\otimes\ol{\feeddd{\varphi}{X}{X}}}_{\ol{\inp{Z}},\ol{\outp{Z}}}(f\otimes\id_{\ol{\feedcc{(\psi\circ\varphi)}{Z}{Z}}})=\Trace^{\ol{P_X^X}\otimes\ol{\feeddd{\varphi}{X}{X}}}_{\ol{\inp{Z}},\ol{\outp{Z}}}(f\otimes\id_{\ol{\feedcc{\psi}{Z}{Z}}}\otimes\id_{\ol{P_Z^Z}})\]
\shortnote{The result will follow if we can show:}
\[\Trace^{\ol{\feeddd{\psi}{Y}{Y}}\otimes\ol{\feeddd{\varphi}{X}{X}}}_{\ol{\inp{Z}},\ol{\outp{Z}}}
(f\otimes\id_{\ol{\feedcc{\varphi}{Y}{Y}}}\otimes\id_{\ol{\feedcc{\psi}{Z}{Z}}})=\Trace^{\ol{\feeddd{\varphi}{X}{X}}\otimes\ol{P_X^X}}_{\ol{\inp{Z}},\ol{\outp{Z}}}(f\otimes\id_{\ol{\feedcc{\psi}{Z}{Z}}}\otimes\id_{\ol{P_Z^Z}})\otimes\dim(\ol{``newloops"})\]
Using that 
\[\inp{Z}=\feedcd{(\psi\circ\varphi)}{Z}{X}\sqcup \feedcc{(\psi\circ\varphi)}{Z}{Z}=P_X^Z\sqcup P_Z^Z\sqcup \feedcc{\psi}{Z}{Z}\]
\[\outp{Z}=\feeddc{(\psi\circ\varphi)}{X}{Z}\sqcup \feedcc{(\psi\circ\varphi)}{Z}{Z}=P_Z^X\sqcup P_Z^Z\sqcup \feedcc{\psi}{Z}{Z},\]
where each of the final equalities above follow from Lemma~\ref{le:composed wires}, and the superposition axiom of the trace we get
\[\Trace^{\ol{\feeddd{\psi}{Y}{Y}}\otimes\ol{\feeddd{\varphi}{X}{X}}}_{\ol{\inp{Z}},\ol{\outp{Z}}}
(f\otimes\id_{\ol{\feedcc{\varphi}{Y}{Y}}}\otimes\id_{\ol{\feedcc{\psi}{Z}{Z}}})=\Trace^{\ol{\feeddd{\psi}{Y}{Y}}\otimes\ol{\feeddd{\varphi}{X}{X}}}_{\ol{P_X^Z}\otimes\ol{P_Z^Z},\ol{P_Z^X}\otimes\ol{P_Z^Z}}
(f\otimes\id_{\ol{\feedcc{\varphi}{Y}{Y}}})\otimes\id_{\ol{\feedcc{\psi}{Z}{Z}}}\]
\shortnote{how to eliminate $\ol{P_Z^Z}$ above?} $\feedcc{\varphi}{Y}{Y}$ should decompose as those wires which map to $P_L$ and those wires that map to its complement in $P_Y$, with similar decompositions for $\feeddd{\psi}{Y}{Y}$, this should allow to split off the needed ``newloops'' factor.  A similar statement should hold for the elements of $\feedcc{\varphi}{Y}{Y}$ and $\feeddd{\psi}{Y}{Y}$ that map to $P_Z^Z$.  But what about the remaining elements, i.e. those mapping to $P_X^X\sqcup P_X^Z\sqcup P_Z^X$?

and
\[\Trace^{\ol{\feeddd{\varphi}{X}{X}}\otimes\ol{P_X^X}}_{\ol{\inp{Z}},\ol{\outp{Z}}}(f\otimes\id_{\ol{\feedcc{\psi}{Z}{Z}}}\otimes\id_{\ol{P_Z^Z}})=\Trace^{\ol{\feeddd{\varphi}{X}{X}}\otimes\ol{P_X^X}}_{\ol{P_X^Z},\ol{P_Z^X}}(f)\otimes\id_{\ol{\feedcc{\psi}{Z}{Z}}}\otimes\id_{\ol{P_Z^Z}}.\]

By dinaturality in the upper factor for the map $r\otimes\id_{\ol{\feeddd{\varphi}{X}{X}}}:\ol{\feeddd{\psi}{Y}{Y}}\otimes\ol{\feeddd{\varphi}{X}{X}}\to\ol{P_X^X}\otimes\ol{\feeddd{\varphi}{X}{X}}$....

\bigskip

Here's a picture that may be useful.
\inactivetikz{
	%outer box Z
	\path(0,0);\blackbox{(10,8)}{3}{3}{$Z$}{.8}
	%mid box Y
	\path(2.5,1.5);\blackbox{(5,5)}{4}{4}{$Y$}{.6}
	%inner box X
	\path(4,3);\blackbox{(2,2)}{3}{3}{$X$}{.4}
	%outer morphism
	% Z feeds Z
	\fancyarc{(.4,6)}{(9.6,6)}{-30}{30}\node at (5,7.1){\tiny$\feedcc{\psi}{Z}{Z}$};
	% Z feeds Y
	\directarc{(.4,4)}{(2.2,5.5)}\node at (1.5,5.1){\tiny$\feedcd{\psi}{Z}{Y}$};
	\directarc{(.4,2)}{(2.2,4.5)}\node at (1.5,4.1){\tiny$\feedcd{\psi}{Z}{Y}$};
	% Y feeds Z
	\directarc{(7.8,5.5)}{(9.6,4)}\node at (8.5,5.1){\tiny$\feeddc{\psi}{Y}{Z}$};
	\directarc{(7.8,4.5)}{(9.6,2)}\node at (8.5,4.1){\tiny$\feeddc{\psi}{Y}{Z}$};
	% Y feeds Y
	\fancyarc{(2.2,3.5)}{(7.8,3.5)}{40}{-90}\node at (5,.8){\tiny$\feeddd{\psi}{Y}{Y}$};
	\fancyarc{(2.2,2.5)}{(7.8,2.5)}{20}{-50}\node at (5,.4){\tiny$\feeddd{\psi}{Y}{Y}$};
	%circles
	\draw(5,1.1) circle(5pt);\node at (5.2,1.1){\tiny$\psi_{loop}$};
	%inner morphism
	%Y feeds Y
	\directarc{(2.8,5.5)}{(7.2,5.5)}\node at (5,5.5){\tiny$\feedcc{\varphi}{Y}{Y}$};
	\directarc{(2.8,2.5)}{(7.2,2.5)}\node at (5,2.5){\tiny$\feedcc{\varphi}{Y}{Y}$};
	%Y feeds X
	\directarc{(2.8,4.5)}{(3.8,4.5)}\node at (3.5,4.5){\tiny$\feedcd{\varphi}{Y}{X}$};
	\directarc{(2.8,3.5)}{(3.8,4)}\node at (3.5,3.8){\tiny$\feedcd{\varphi}{Y}{X}$};
	%X feeds Y
	\directarc{(6.2,4.5)}{(7.2,4.5)}\node at (6.5,4.5){\tiny$\feeddc{\varphi}{X}{Y}$};
	\directarc{(6.2,4)}{(7.2,3.5)}\node at (6.5,3.8){\tiny$\feeddc{\varphi}{X}{Y}$};
	%X feeds X 
	\fancyarc{(3.8,3.5)}{(6.2,3.5)}{10}{-23}\node at (5,2.7){\tiny$\feeddd{\varphi}{X}{X}$};
	%circles
	\draw(5,2) circle(5pt);\node at (5.2,2){\tiny$\phi_{loop}$};
}

\inactivetikz{
	%outer box Z
	\path(0,0);\blackbox{(10,8)}{3}{3}{$Z$}{.8}
	%inner box X
	\path(4,1.5);\blackbox{(2,2)}{3}{3}{$X$}{.4}
	% Z feeds Z
	\directarc{(.4,6)}{(9.6,6)}
	\directarc{(.4,4)}{(9.6,4)}
	% Z feeds X
	\directarc{(.4,2)}{(3.8,3)}
	% X feeds Z
	\directarc{(6.2,3)}{(9.6,2)}
	% X feeds X
	\fancyarc{(3.8,2.5)}{(6.2,2.5)}{20}{-60}
	\fancyarc{(3.8,2)}{(6.2,2)}{10}{-30}
	%circles
	\draw(2,1) circle(5pt);
	\draw(2.5,1.2) circle(5pt);
	\draw(3,.7) circle(5pt);
}

\end{proof}

\begin{theorem}

The adjunction of Theorem \ref{thm:cobalg trace adjunction} is monadic. 

\end{theorem}

Every category has an underlying reflexive graph, which includes its objects and morphisms (including identities), but not its composition formula. By {\em symmetric monoidal reflexive graph}, we mean a reflexive graph $\mcG$ together with a vertex $I\in \mcG$ and a graph homomorphism $\otimes\taking\mcG\times\mcG\to\mcG$ that is commutative, associative, and unital with respect to $I$, all up to coherent isomorphisms. By {\em traced symmetric monoidal reflexive graph} we mean a symmetric monoidal reflexive graph with a family of trace functions $Tr$ that satisfy superposition, vanishing, and yanking (all the properties of traced SMCs that don't refer to the composition formula).

\begin{conjecture}
The category of traced symmetric monoidal categories is equivalent to the category of traced symmetric monoidal relfexive graphs.
\end{conjecture}

\begin{conjecture}
Let $\mcM$ be a symmetric monoidal category and let $(\Cob/\bullet)\alg_\mcM$ be the category of algebras in $\mcM$ (e.g. whose objects are pairs $(\mcO,\mcP)$, where $\mcO$ is a set and $\mcP\taking\Cob/\mcO\to\mcM$ is a lax symmetric monoidal functor). Then there is a monadic adjunction 
$$\xymatrix{L\taking(\Cob/\bullet)\alg_\mcM\ar@<.5ex>[r]&\mcM-\TSMC:\!R,\ar@<.5ex>[l]}$$
where $\mcM-\TSMC$ is the category of traced SMCs enriched in $\mcM$.
\end{conjecture}

\section{Yanking Compositions of Wiring Diagrams}
We have some technique ,``yanking", by which we trade $\phi,\psi$ in for $\phi',\psi'$ such that there's an equivalence and two nice facts:
$$\xymatrix@R=10pt{
&Y\ar[dr]^\psi\ar@{}[dd]|{\Bigg\Updownarrow}\\
X\ar[ur]^\phi\ar[dr]_{\phi'}&&Z\\
&Y'\ar[ur]_{\psi'}
}
\hsp\tn{where}
\hspace{.2in}
\xymatrix{
\emptyset\ar[r]\ar[d]\ullimit&\feedcc{\phi'}{Y}{Y}\ar[d]\\
\feeddd{\psi'}{Y}{Y}\ar[r]&\inp{Y'}
}
\hspace{.7in}
\xymatrix{
\emptyset\ar[r]\ar[d]\ullimit&\feedcc{\phi'}{Y}{Y}\ar[d]\\
\feeddd{\psi'}{Y}{Y}\ar[r]&\outp{Y'}
}
$$

Then we have $\feeddd{(\psi'\circ\phi')}{X}{X}=\feeddd{\phi'}{X}{X}+\feeddd{\psi'}{Y}{Y}$ and $\feedcc{(\psi'\circ\phi')}{Z}{Z}=\feedcc{\phi'}{Y}{Y}+\feedcc{\psi'}{Z}{Z}$.  And so compatibility of trace and compositions is immediate by comparing equation \eqref{eq:composition 2} and Lemma~\ref{le:combining traces}.

\section{Old stuff}

Suppose given a typed bijection $\phi\taking\inp{X}\sqcup \outp{Y}\To{\iso}\outp{X}\sqcup \inp{Y}$. We'll define $\feeddd{\varphi}{X}{X}, \feeddc{\varphi}{X}{Y}, \feedcd{\varphi}{Y}{X},$ and $\feedcc{\varphi}{Y}{Y}$ below; roughly they are ``$X$'s feed $X$'s", ``$X$'s feed $Y$'s", ``$Y$'s feed $X$'s", and ``$Y$'s feed $Y$'s", respectively. They are given by the following fiber products in $\TFS$.
$$
\xymatrix@=35pt{
\feeddd{\varphi}{X}{X}\ar[r]\ar[d]\ullimit&\inp{X}\ar[d]^{\phi\big|_{\inp{X}}}&\feedcd{\varphi}{Y}{X}\ar[l]\ar[d]\urlimit\\
\outp{X}\ar[r]&\outp{X}\sqcup \inp{Y}&\inp{Y}\ar[l]\\
\feeddc{\varphi}{X}{Y}\ar[r]\ar[u]\lllimit&\outp{Y}\ar[u]_{\phi\big|_{\outp{Y}}}&\feedcc{\varphi}{Y}{Y}\ar[l]\ar[u]\lrlimit
}
$$
Note that every arrow above is an injection and we have the following four bijections, 
\begin{align*}
\feeddd{\varphi}{X}{X}\sqcup \feeddc{\varphi}{X}{Y}\iso\outp{X},
&\hsp
\feedcd{\varphi}{Y}{X}\sqcup \feedcc{\varphi}{Y}{Y}\iso\inp{Y},
\\
\feeddd{\varphi}{X}{X}\sqcup \feedcd{\varphi}{Y}{X}\iso\inp{X},
&\hsp
\feeddc{\varphi}{X}{Y}\sqcup \feedcc{\varphi}{Y}{Y}\iso\outp{Y}.
\end{align*}


\end{document}